\documentclass[12pt]{amsart}
\usepackage{amssymb,amsmath,amsthm,mathrsfs,graphics,hyperref}
\numberwithin{equation}{section}
\raggedbottom
\oddsidemargin=0in
\evensidemargin=0in
\textwidth=6.5in
\textheight=9in
\topmargin=0.25in
\headheight=0in
\headsep=0in
\footskip=0in
\parskip=10bp
\parindent=0bp
\newcommand{\x}{\times}
\newcommand{\C}{\ensuremath{\clubsuit}}
\newcommand{\D}{\ensuremath{\diamondsuit}}
\renewcommand{\H}{\ensuremath{\heartsuit}}
\renewcommand{\S}{\ensuremath{\spadesuit}}

\newcommand{\hand}[4]{\quad\S~#1\quad\H~#2\quad\D~#3\quad\C~#4}

\pagestyle{empty}
\begin{document}
\thispagestyle{empty}


\textbf{\LARGE Probabilities and Yarboroughs}

\textbf{Question 1: What is the probability of being dealt a 
Yarborough?}

As so often in mathematics, in order to answer the original question, 
we start by asking a different question.

\textbf{Question 2: How many possible Yarboroughs are there?}

If we know this, then we can divide the answer by the number of possible 
bridge hands (which we now know is 635013559600) to get the answer to 
the original Question~1.

On the other hand, Question~2 can be answered by the same techniques 
that we used to count all possible hands.  What is a Yarborough, after 
all?  It's a hand with no card higher than a nine.  That is, it is a 
hand of thirteen cards selected from a set of thirty-two cards (namely 
the deuce through nine of each suit: 8 cards per suit $\x$ 4 suits = 32 
cards.)  Therefore,
 \boldmath
 \begin{align*}
 \textbf{number of possible Yarboroughs} &~=~ \binom{32}{13} = 
 \frac{32!}{13!\x19!} = 347373600.
 \end{align*}
 \unboldmath

That takes care of Question 1.  As for Question 2, the answer is
 \boldmath
 \begin{multline*}
 \textbf{probability of being dealt a Yarborough} ~=~ 
 \frac{\textbf{number of possible Yarboroughs}}{\textbf{number of 
 possible hands}}\\\\
 ~=~ \binom{32}{13} ~/~ \binom{52}{13} ~=~ 
 \frac{347373600}{635013559600} ~\approx~ 0.00054703 = 0.054703\%
 \end{multline*}
 \unboldmath
or about 1 in 1828.  So you can expect to be dealt a Yarborough about 
one of every 1828 hands you play. But that doesn't mean you can't have 
four Yarboroughs in a row, or go fifteen years without being dealt a 
Yarborough (lucky you!).  It just means that \emph{on average}, one out 
of every 1828 or so hands you get will be a Yarborough.

\textbf{Question 3: What is the probability of being dealt at least one 
Yarborough in a 24-board session?}

The naive answer is to take the number 0.00054703 and multiply it by 24.  
Actually, that's not mathematically precise (although you do happen, in 
this case to get a number that is reasonably close to the correct 
answer).  Here is an analogy.  You flip a coin.  The chance that it 
comes up heads is 50\%.  Does that mean that if you flip it twice, the 
chance of getting at least one heads is 2$\x$50\% = 100\%?  Of course 
not!  The chance of getting tails twice is 25\%, so the chance of that 
\emph{not} happening --- that is, of getting heads at least once --- is 
75\%.

The same reasoning can be used to answer Question~3.  The chance of 
\emph{not} being dealt a Yarborough on any particular deal is 
approximately
  $$1 - 0.00054703 = 0.99945297$$
and so the chance of not being dealt a Yarborough on any of 24 separate 
deals is approximately
  $$(0.99945297)^{24} \approx 0.9869535$$
and so the chance of being dealt a Yarborough (i.e., not not being dealt 
a Yarborough) on at least one of 24 deals is approximately
  $$1-0.9869535=0.0130465=1.30465\%$$
or about 1 in 77.

\pagebreak

\textbf{Question 4: What is the probability of being dealt at least one 
Yarborough in each of two 24-board sessions?}

(This was Virginia Seaver's question that led to this whole project!)

Using the answer to Question 3, and the same logic, the answer is 
  $$(0.0130465)^2\approx 0.0001702122=0.01702122\%$$
or about 1 in 5875.  Hardly an everyday occurrence, but not that much 
less frequent than being dealt a Yarborough in the first place (which, 
remember, was about 1/1828, or roughly three times 1/5875.)

\pagebreak
\end{document}
