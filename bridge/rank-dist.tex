\documentclass[12pt]{amsart}
\usepackage{amssymb,amsmath,amsthm,mathrsfs,graphics,hyperref}
\numberwithin{equation}{section}
\raggedbottom
\oddsidemargin=0in
\evensidemargin=0in
\textwidth=6.5in
\textheight=9in
\topmargin=0.25in
\headheight=0in
\headsep=0in
\footskip=0in
\parskip=10bp
\parindent=0bp
\newcommand{\x}{\times}
\newcommand{\C}{\ensuremath{\clubsuit}}
\newcommand{\D}{\ensuremath{\diamondsuit}}
\renewcommand{\H}{\ensuremath{\heartsuit}}
\renewcommand{\S}{\ensuremath{\spadesuit}}

\newcommand{\hand}[4]{\quad\S~#1\quad\H~#2\quad\D~#3\quad\C~#4}

\pagestyle{empty}
\begin{document}
\thispagestyle{empty}


\textbf{\LARGE Rank Distributions}

I picked up this hand at the club yesterday (October 9, 2011):

\hand{543}{853}{K43}{K854}

For bridge, not so good (I believe the opponents bid 3NT and made five), 
but boy, is that a nice hand for gin rummy.

How rare is it to pick up a hand with cards of only five different ranks?

Quick notation: Let's describe the \textbf{rank distribution} of a hand
by listing the number of ranks that contribute four, three, two, and one
card to the hand, respectively.  The hand I picked up had rank distribution $[0,3,2,0]$ --- three ``tripletons''
(fives, fours and threes) and three ``doubletons'' (kings and eights).
To get the total number of different ranks, you just add up these numbers.
Notice that that total has to be at least 4 (because three ranks can together
only account for twelve cards) and is at most 13.

Let's try four different ranks (rather than five) first, since the calculation is easier.  
The possibilities are $[3,0,0,1]$, $[2,1,1,0]$, and $[1,3,0,0]$.

For $[3,0,0,1]$: we need to choose three ranks to have all four cards; then 
choose a fourth rank (out of ten possibilities), then one card in it. 
Total:
  \[
  \binom{13}{3}\cdot10\cdot4 = 286\cdot10\cdot4 = 11440.
  \]

For $[2,1,1,0]$: choose two ranks to have all four cards; choose a third rank 
and three cards from it; choose a fourth rank and two cards from it. 
Total:
  \[
  \binom{13}{2}\cdot11\binom{4}{3}\cdot10\binom{4}{2} =
  78\cdot11\cdot4\cdot10\cdot6 = 205920.
  \]

For $[1,3,0,0]$: choose one rank to have all four cards; choose three other 
ranks and three cards from each of them. Total:
  \[
  13\cdot\binom{12}{3}\cdot\binom{4}{3}^3
  = 13\cdot220\cdot4^3 = 183040.
  \]

Grand total: 11440 + 205920 + 183040 = 400400 (a rather attractive 
number, isn't it?)

Remember that the number of bridge hands you could be dealt is 
635013559600, so the probability of picking up a four-rank hand is 
400400/635013559600, which is about .000000630537717, or about 1/1600000 
(after rounding off rather crudely).  So on average, you'll pick 
up one of these things every 1,600,000 deals.

How about the original question?  I think it's time to automate this. I 
wrote the following short computer program in Maple (a commercial 
computer algebra package).

\pagebreak

\begin{verbatim}
with(combinat):
C := binomial:
L := []:
for four from 0 to 3 do
for three from 0 to 4 do
for two from 0 to 6 do
for one from 0 to 13 do
  if 4*four + 3*three + 2*two + one = 13 then
    numranks := four+three+two+one:
    numhands := C(13,four) * C(13-four,three)
        * C(13-four-three,two) * C(13-four-three-two,one) 
        * 4^three * 6^two * 4^one:
    L := [op(L), [numranks, [four,three,two,one], numhands]]:
fi od od od od:
\end{verbatim}

Here \texttt{four} is the number of ranks with four cards in the hand; 
notice that four can be any number between 0 and 3.  The program looks 
at all possible values for \texttt{four}, \texttt{three}, \texttt{two} 
and \texttt{one}, but only considers a particular quadruple of values if 
it accounts for exactly 13 cards.  Adding up those four variables gives 
you the number of ranks in the hand; the longer calculation is the 
number of hands with those rank numbers.

The complete output, sorted by most frequent to least frequent, is on the next page.
Some observations about it:
\begin{itemize}
\item The most common single hand-type is $[0,0,3,7]$, i.e., three doubletons and seven singletons,
(so, ten ranks total) which happens about 19\% of the time.

\item On the other hand, the most common number of ranks in a hand is nine,
which is about a 37\% chance.

\item It is interesting that $[0,1,1,8]$ and $[0,0,2,9]$ are exactly equally likely.
There's probably no clever reason why this is the case; the numbers just happen
to be equal.

\item A hand ``without a singleton'' (i.e., \texttt{one} = 0) occurs about once every 
1700 deals.

\item A hand with one card of every suit (i.e., \texttt{one} = 13) occurs about once 
every 10,000 deals.

\item Five ranks (like the hand I was dealt), on the average, will happen once 
in every 6600 deals.  If you play 30 deals of bridge a week on average
then this is about four years.  Maybe I was due.
\end{itemize}
\pagebreak

\begin{center}
\begin{tabular}{c|cccc|c}
\textbf{Ranks} & \texttt{one} & \texttt{two} & \texttt{three} & \texttt{four} & \textbf{Probability}\\ \hline
10 & 0 & 0 & 3 & 7 & 0.191266086\\
9 & 0 & 0 & 4 & 5 & 0.188277553\\
9 & 0 & 1 & 2 & 6 & 0.167357825\\
8 & 0 & 1 & 3 & 4 & 0.125518369\\
10 & 0 & 1 & 1 & 8 & 0.063755362\\
11 & 0 & 0 & 2 & 9 & 0.063755362\\
8 & 0 & 0 & 5 & 3 & 0.056483266\\
8 & 0 & 2 & 1 & 5 & 0.033471565\\
7 & 0 & 1 & 4 & 2 & 0.023534694\\
7 & 0 & 2 & 2 & 3 & 0.020919728\\
8 & 1 & 0 & 2 & 5 & 0.012551837\\
9 & 0 & 2 & 0 & 7 & 0.010625894\\
9 & 1 & 0 & 1 & 7 & 0.00796942\\
12 & 0 & 0 & 1 & 11 & 0.006182338\\
11 & 0 & 1 & 0 & 10 & 0.005667143\\
7 & 1 & 0 & 3 & 3 & 0.005229932\\
7 & 0 & 0 & 6 & 1 & 0.003530204\\
7 & 1 & 1 & 1 & 4 & 0.003486621\\
6 & 0 & 2 & 3 & 1 & 0.002241399\\
8 & 1 & 1 & 0 & 6 & 0.001859531\\
7 & 0 & 3 & 0 & 4 & 0.001549609\\
10 & 1 & 0 & 0 & 9 & 0.001180655\\
6 & 1 & 1 & 2 & 2 & 0.0011207\\
6 & 0 & 3 & 1 & 2 & 0.000996178\\
6 & 0 & 1 & 5 & 0 & 0.000504315\\
6 & 1 & 0 & 4 & 1 & 0.000420262\\
6 & 1 & 2 & 0 & 3 & 0.00016603\\
13 & 0 & 0 & 0 & 13 & 0.000105681\\
6 & 2 & 0 & 1 & 3 & 6.22611E-05\\
7 & 2 & 0 & 0 & 5 & 5.81104E-05\\
5 & 0 & 3 & 2 & 0 & 4.66958E-05\\
5 & 1 & 2 & 1 & 1 & 4.66958E-05\\
5 & 1 & 1 & 3 & 0 & 3.50219E-05\\
5 & 0 & 4 & 0 & 1 & 1.03768E-05\\
5 & 2 & 0 & 2 & 1 & 8.75547E-06\\
5 & 2 & 1 & 0 & 2 & 3.89132E-06\\
4 & 2 & 1 & 1 & 0 & 3.24277E-07\\
4 & 1 & 3 & 0 & 0 & 2.88246E-07\\
4 & 3 & 0 & 0 & 1 & 1.80154E-08
\end{tabular}
\end{center}

(The E's are scientific notation: e.g., 6.22611E-05 = $6.22611\x10^{-5}=.0000622611$.
In this case, just think ``really small number.'')
\end{document}