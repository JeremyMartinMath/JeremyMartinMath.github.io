\documentclass{amsart}
\usepackage{amssymb,amsmath,amsthm,mathrsfs,graphics,hyperref}
\numberwithin{equation}{section}
\raggedbottom
\oddsidemargin=0in
\evensidemargin=0in
\textwidth=6.5in
\textheight=9in
\topmargin=0.25in
\headheight=0in
\headsep=0in
\footskip=0in
\parskip=10bp
\parindent=0bp
\newcommand{\x}{\times}
\newcommand{\C}{$\clubsuit$}
\newcommand{\D}{$\diamondsuit$}
\renewcommand{\H}{$\heartsuit$}
\renewcommand{\S}{$\spadesuit$}
\pagestyle{empty}
\begin{document}
\thispagestyle{empty}

\textbf{1. How many bridge hands are there?  That is, how many ways are there
of selecting 13 cards from a deck of 52?}

No problem --- just start listing all the possible hands!  Just kidding.
That would require a ludicrously huge amount of
time and scratch paper.  Even computers have better things to do with their time.
There must be a better way to count hands without listing every single one.

Imagine dealing a single bridge hand.  For simplicity, we won't worry about the other 
three players; we'll just deal North the top thirteen cards from the top of 
the deck.

\begin{itemize}
\item There are 52 possibilities for the first card.  Let's say it's the \D7.
\item There are 51 possibilities for the second card (because it can be anything \emph{except} the \D7).
Let's say it's the \S K.
\item There are 50 possibilities for the third card (because it can be anything \emph{except} the \D7 or \S K).
Let's say it's the \C4.
\item \dots
\item There are 40 possibilities for the thirteenth card (because it can be anything except the \D7, \S K, \C4, or any of the
nine other cards already dealt).%  Let's say it's the \H J.
\end{itemize}

That means that the number of possible hands is $52\x51\x50\x\cdots\x42\x41\x40 = 3954242643911239680000$.
Right?

Wrong.   The reason is that \emph{the order in which the cards were dealt is irrelevant.}  If, say, the \D7 and \S K had switched places,
or even if the first thirteen cards had been shuffled arbitrarily while leaving the rest of the deck alone, then the hand we dealt
wouldn't have changed.  Therefore, we need to divide 3954242643911239680000 by the number of ways to shuffle a 13-card hand,
which is $13\x12\x11\x\cdots\x3\x2\x1$ (for the same logic as before).  This number, by the way, is written $13!$ (pronounced
``thirteen factorial'') --- if you see an exclamation point in mathematics, that's almost certainly what it means.

\boldmath
\begin{align*}
\textbf{number of possible bridge hands} ~&=~ \frac{52\x51\x50\x\cdots\x42\x41\x40}{13\x12\x11\x\cdots\x3\x2\x1}\\\\
&=~ \frac{3954242643911239680000}{6227020800}\\\\
&=~ \mathbf{635013559600}.
\end{align*}
\unboldmath
Also, by the way, observe that $52\x51\x\cdots\x40=52!/39!$ --- because if you wrote $52!/39!$ out in its full glory, then
you'd be able to cancel 39/39 and 38/38 and 37/37 and a lot more stuff, and all that survived would be $52\x51\x\cdots\x40$.
That means that we can write down a more concise formula:
$$635013559600 =  \frac{52!}{39! \x 13!}.$$

This kind of calculation (``How many ways are there to choose a subset of $k$ things from a set of $n$ things?'') is extremely
common; common enough, in fact, that it has a special symbol: $\binom{n}{k}$.  (You might also see $C(n,k)$ or ${}_nC_k$,
but the notation $\binom{n}{k}$ is what most mathematicians would use.)  By the same logic as above, these numbers can be
calculated by the formula
$$\binom{n}{k} = \frac{n(n-1)(n-2)\cdots(n-k+2)(n-k+1)}{k(k-1)(k-2)\cdots(2)(1)} = \frac{n!}{k!(n-k)!}$$

\pagebreak

\textbf{2. What is the probability of being dealt a Yarborough?}

To answer this, we first ask a different question:

\textbf{2a. How many possible Yarboroughs are there?}

If we know this, then we can divide the answer by the number of possible bridge hands (which, remember,
we now know) to get the answer to the original question 2.

On the other hand, Question~2a can be answered by simple counting.  What is a Yarborough, after all?  It's a hand
with no card higher than a nine.  That is, it is a hand of thirteen cards selected from a set of thirty-two cards
(namely the deuce through nine of each suit: 8 cards per suit $\x$ 4 suits = 32 cards.)  Therefore,
\boldmath
\begin{align*}
\textbf{number of possible Yarboroughs} &~=~ \binom{32}{13} = \frac{32!}{13!\x19!} = 347373600.
\end{align*}
\unboldmath

That takes care of Question 2a.  As for Question 2, the answer is
\boldmath
\begin{multline*}
\textbf{probability of being dealt a Yarborough} ~=~ 
\frac{\textbf{number of possible Yarboroughs}}{\textbf{number of possible hands}}\\\\
~=~ \binom{32}{13} ~/~ \binom{52}{13} ~=~ \frac{347373600}{635013559600} ~\approx~ 0.00054703 = 0.054703\%
\end{multline*}
\unboldmath

or about 1 in 1828.  So you can expect to be dealt a Yarborough about one of every 1828 hands you play.
But that doesn't mean you can't have four Yarboroughs in a row, or go fifteen years without being dealt a Yarborough (lucky you!),
just that \emph{on average} one out of every 1828 or so hands will be a Yarborough.

\textbf{2b. What is the probability of being dealt at least one Yarborough in a 24-board session?}

The naive answer is to take the number 0.00054703 and multiply it by 24.  Actually, that's not mathematically
precise (although you do happen, in this case to get a number that is reasonably close
to the correct answer).  Here is an analogy.  You flip a coin.  The chance that it comes up heads is 50\%.  Does
that mean that if you flip it twice, the chance of getting at least one heads is 2$\x$50\% = 100\%?  Of course
not!  The chance of getting tails twice is 25\%, so the chance of that \emph{not} happening --- that is,
of getting heads at least once --- is 75\%.  

The same reasoning can be used to answer Question~2b.  The chance of \emph{not} being dealt a Yarborough is
approximately
$$1 - 0.00054703 = 0.99945297$$
and so the chance of not being dealt a Yarborough on any of 24 separate deals is approximately
$$(0.99945297)^{24} \approx 0.9869535$$
and so the chance of being dealt a Yarborough (i.e., not not being dealt a Yarborough) on at least one of 24 deals is approximately
$$1-0.9869535=0.0130465=1.30465\%$$
or about 1 in 77.

\textbf{2c. What is the probability of being dealt at least one Yarborough in each of two 24-board sessions?}

With what we know so far, this is now a relatively easy one: it's approximately $(0.0130465)^2\approx
0.0001702122=0.01702122\%$, or about 1 in 5875.  Indeed, hardly an everyday occurrence, but
not that much less frequent than being dealt a Yarborough in the first place (which, remember, was about
1/1828, or roughly three times 1/5875.)

\pagebreak
\end{document}