\documentclass{amsart}
\usepackage{amssymb,amsmath,amsthm,mathrsfs,graphics,hyperref,stmaryrd,psfrag,arcs,xypic}
\usepackage[enableskew]{youngtab}
\numberwithin{equation}{section}
\raggedbottom
\oddsidemargin=0in
\evensidemargin=0in
\textwidth=6.5in
\textheight=8.8in
\topmargin=0.25in
\headheight=0in
\headsep=0.2in
\footskip=0in
\parskip=10bp
\parindent=0bp

\newcommand{\excise}[1]{}
\newcommand{\Latex}[1]{\textbackslash\texttt{#1}}

\newcommand{\bigpad}{\rule[-14mm]{0mm}{30mm}}
\newcommand{\smallpad}{\rule[-1.5mm]{0mm}{5mm}}
\newcommand{\pad}{\rule[-3mm]{0mm}{8mm}}
\newcommand{\padup}{\rule{0mm}{5mm}}
\newcommand{\paddown}{\rule[-3mm]{0mm}{2mm}}
\newcommand{\blank}{\rule{1.25in}{0.25mm}}
\newcommand{\commentout}[1]{}
\newcommand{\yell}[1]{\fbox{\rule[-1mm]{0mm}{4mm} \large\bf #1 }}
\newcommand{\bang}{$\bullet$\quad}
\newcommand{\indnt}{\phantom{.}\qquad}
\newcommand{\littleline}{\begin{center}\rule{4in}{0.5bp}\end{center}}

\newcommand{\includefigure}[3]{{
  \begin{center}
  \resizebox{#1}{#2}{\includegraphics{{figs/#3}}}
  \end{center}}}
\newcommand{\includefigurewithinmath}[3]{{
  \resizebox{#1}{#2}{\includegraphics{{figs/#3}}}}}

%\newcommand{\defterm}[1]{\underline{\textbf{#1}}}
\newcommand{\defterm}[1]{\textbf{#1}}

\DeclareMathOperator{\ch}{\mathbf{ch}}
\DeclareMathOperator{\colspace}{colspace}
\DeclareMathOperator{\corank}{corank}
\DeclareMathOperator{\CST}{CST}
\DeclareMathOperator{\deln}{del}
\DeclareMathOperator{\diag}{diag}
\DeclareMathOperator{\ess}{ess}
\DeclareMathOperator{\Gr}{Gr}
\DeclareMathOperator{\Hom}{Hom}
\DeclareMathOperator{\im}{im}
\DeclareMathOperator{\Irr}{Irr}
\DeclareMathOperator{\Ind}{Ind}
\DeclareMathOperator{\Int}{Int}
\DeclareMathOperator{\lcm}{lcm}
\DeclareMathOperator{\link}{lk}
\DeclareMathOperator{\nullity}{nullity}
\DeclareMathOperator{\nullspace}{nullspace}
\DeclareMathOperator{\Poin}{Poin}
\DeclareMathOperator{\proj}{proj}
\DeclareMathOperator{\rank}{rank}
\DeclareMathOperator{\Res}{Res}
\DeclareMathOperator{\Span}{span}
\DeclareMathOperator{\supp}{supp}
\DeclareMathOperator{\row}{row}
\DeclareMathOperator{\rowspace}{rowspace}
\DeclareMathOperator{\sh}{sh}
\DeclareMathOperator{\tr}{tr}
\DeclareMathOperator{\wt}{wt}

\newtheorem{theorem}{Theorem}[section]
\newtheorem{proposition}[theorem]{Proposition}
\newtheorem{lemma}[theorem]{Lemma}
\newtheorem{corollary}[theorem]{Corollary}
\theoremstyle{definition}
\newtheorem{definition}[theorem]{Definition}
\newtheorem{example}[theorem]{Example}
\newtheorem{remark}[theorem]{Remark}
\newtheorem{problem}[theorem]{Problem}


\newcommand{\cor}{{\bf Corollary: }}
\newcommand{\defn}{{\bf Definition: }}
\newcommand{\defns}{{\bf Definitions: }}
\newcommand{\exa}{{\bf Example: }}
\newcommand{\fact}{{\bf Fact: }}
\newcommand{\lem}{{\bf Lemma: }}
\newcommand{\notn}{{\bf Notation: }}
\newcommand{\obs}{{\bf Observation: }}
\newcommand{\note}{{\bf Note: }}
\newcommand{\prop}{{\bf Proposition: }}
\newcommand{\rmk}{{\bf Remark: }}
\newcommand{\thm}{{\bf Theorem: }}

\newcommand{\basecase}{\emph{Base case: }}
\newcommand{\indstep}{\emph{Inductive step: }}
\newcommand{\skpr}{\emph{Sketch of proof: }}

\newcommand{\0}{\emptyset}
\newcommand{\Alt}{\mathfrak{A}}
\newcommand{\Braid}{Br}
\newcommand{\CHI}{\chi^{\phantom{*}}}
\newcommand{\Cl}{C\ell}
\newcommand{\covers}{\gtrdot}
\newcommand{\coveredby}{\lessdot}
\newcommand{\dedge}[1]{\overrightarrow{{#1}}}
\newcommand{\dom}{\rhd}
\newcommand{\domeq}{\unrhd}
\newcommand{\domby}{\lhd}
\newcommand{\dombyeq}{\unlhd}
\newcommand{\Fspan}{\Ff\text{-span}}
\newcommand{\isom}{\cong}
\newcommand{\join}{\vee}
\renewcommand{\Join}{\bigvee}
\newcommand{\Laff}{L^{\text{aff}}}
\newcommand{\lin}[1]{\overleftrightarrow{{#1}}}
\newcommand{\meet}{\wedge}
\newcommand{\Meet}{\bigwedge}
\newcommand{\ov}[1]{\overline{{#1}}}
\newcommand{\partn}{\vdash}
\newcommand{\qqandqq}{\qquad\text{and}\qquad}
\newcommand{\qandq}{\quad\text{and}\quad}
\newcommand{\qand}{\quad\text{and}}
\newcommand{\qbin}[2]{{\begin{bmatrix}#1\\#2\end{bmatrix}_q}}
\newcommand{\sd}{\triangle} % symmetric difference
\newcommand{\simK}{\underset{K}{\sim}} % Knuth equivalence
\newcommand{\simJ}{\underset{J}{\sim}} % jeu de taquin equivalence
\newcommand{\sm}{\setminus}
\newcommand{\st}{~|~}
\newcommand{\soln}{\textit{Solution:\ }}
\newcommand{\Sym}{\mathfrak{S}}
\newcommand{\un}[1]{\underset{*}{#1}}
\newcommand{\unA}{\un{A}}
\newcommand{\unB}{\un{B}}
\newcommand{\unw}{\un{w}}
\newcommand{\unx}{\un{x}}
\newcommand{\uny}{\un{y}}
\newcommand{\unz}{\un{z}}
\newcommand{\x}{\times}

\renewcommand{\aa}{\mathbf{a}}
\newcommand{\bb}{\mathbf{b}}
\newcommand{\nn}{\mathbf{n}}
\newcommand{\pp}{\mathbf{p}}
\newcommand{\qq}{\mathbf{q}}
\newcommand{\xx}{\mathbf{x}}
\newcommand{\yy}{\mathbf{y}}
\newcommand{\zz}{\mathbf{z}}
 
\newcommand{\A}{\mathcal{A}}
\newcommand{\B}{\mathcal{B}}
\newcommand{\C}{\mathcal{C}}
\newcommand{\M}{\mathcal{M}}
\renewcommand{\P}{\mathcal{P}}

\newcommand{\BB}{\mathscr{B}}  %% use these for fancy script fonts -- requires mathrsfs package
\newcommand{\CC}{\mathscr{C}}
\newcommand{\FF}{\mathscr{F}}
\newcommand{\II}{\mathscr{I}}
\newcommand{\LL}{\mathscr{L}}
\newcommand{\PP}{\mathscr{P}}
\renewcommand{\SS}{\mathscr{S}}
\newcommand{\XX}{\mathscr{X}}

\newcommand{\TT}{\tilde{T}}

\newcommand{\Aa}{\mathbb{A}}
\newcommand{\Cc}{\mathbb{C}}
\newcommand{\Ff}{\mathbb{F}}
\newcommand{\Nn}{\mathbb{N}}
\newcommand{\Pp}{\mathbb{P}}
\newcommand{\Qq}{\mathbb{Q}}
\newcommand{\Rr}{\mathbb{R}}
\newcommand{\Zz}{\mathbb{Z}}

\newcommand{\rhodef}{\rho^{\phantom{*}}_{{\rm def}}}
\newcommand{\rhotriv}{\rho^{\phantom{*}}_{{\rm triv}}}
\newcommand{\rhosign}{\rho^{\phantom{*}}_{{\rm sign}}}
\newcommand{\rhoreg}{\rho^{\phantom{*}}_{{\rm reg}}}
\newcommand{\chidef}{\chi^{\phantom{*}}_{{\rm def}}}
\newcommand{\chitriv}{\chi^{\phantom{*}}_{{\rm triv}}}
\newcommand{\chisign}{\chi^{\phantom{*}}_{{\rm sign}}}
\newcommand{\chireg}{\chi^{\phantom{*}}_{{\rm reg}}}
\newcommand{\scp}[2]{\left\langle #1,\:#2\right\rangle_G}
\newcommand{\scpH}[2]{\left\langle #1,\:#2\right\rangle_H}

\newcounter{probno}
\setcounter{probno}{0}
\newcounter{partno}
\setcounter{partno}{0}
%% versions that don't print the number of points
\newcommand{\prob}{
  \vskip10bp%
  \setcounter{partno}{0}%
  \addtocounter{probno}{1}%
  {\bf Problem~\#{\arabic{probno}}}\quad}
\newcommand{\probpart}{%\rule{0in}{0in}\\ \phantom{xxx}
  \addtocounter{partno}{1}%
  {\bf (\#\arabic{probno}\alph{partno})}\ \ }
\newcommand{\probcont}{%
  {\bf Problem~\#{\arabic{probno}}}~(\emph{continued})}
\newcommand{\probo}{
  \setcounter{partno}{0}%
  \addtocounter{probno}{1}%
  {\bf (\#\arabic{probno})}\ \ }
%% versions that do print the number of points
\newcommand{\Prob}[1]{
  \vskip10bp%
  \setcounter{partno}{0}%
  \addtocounter{probno}{1}%
  {\bf Problem~\#{\arabic{probno}}~[{#1}~pts]}\quad}
\newcommand{\Probpart}[1]{%\rule{0in}{0in}\\ \phantom{xxx}
  \addtocounter{partno}{1}%
  {\bf (\#\arabic{probno}\alph{partno})~[{#1}~pts]}\ \ }

%\renewcommand{\thefootnote}{\fnsymbol{footnote}}

\usepackage{youngtab}
\begin{document}
\thispagestyle{empty}
{\bf Math 821, Spring 2014\\
Solution Set \#2\\
Due date: Friday, February 14}

\yellprob{
Show that a homotopy equivalence $f:X\to Y$ induces a bijection 
between the set of path-components of~$X$ and the set of path-components 
of~$Y$, and that~$f$ restricts to a homotopy equivalence from each 
path-component of~$X$ to the corresponding path-component of~$Y$. Prove 
also the corresponding statements with components instead of 
path-components. Deduce that if the components of a space~$X$ coincide 
with its path-components, then the same holds for any space~$Y$ homotopy 
equivalent to $X$.}

\newcommand{\cpt}[1]{[{#1}]}
\newcommand{\pcpt}[1]{\langle{#1}\rangle}

\soln In general, if $p$ is a point in a topological space,
let's write $\cpt{p}$ for the \underline{component} of
that space containing $p$, and $\pcpt{p}$ for the \underline{path-component} of
that space containing $p$.

\underline{First,} we want to prove that
\[\pcpt{x}=\pcpt{x'} \quad\iff\quad \pcpt{f(x)}=\pcpt{f(x')}.\]
The $\implies$ direction is easy: if $\phi$ is an $x,x'$-path in $X$,
then $f\circ\phi$ is an $f(x),f(x')$-path in $Y$.

For the reverse direction, if $\psi$ is an $f(x),f(x')$-path in $Y$,
then $g\circ\psi$  is a $g(f(x)),g(f(x'))$-path in $X$.
On the other hand, $\pcpt{x}=\pcpt{g(f(x))}$
because, by definition of homotopy equivalence,
there is a homotopy $h_t:X\to X$ with $h_0=\IdMap$
and $h_1=g\circ f$; the function $\gamma:I\to X$
given by $\gamma(t)=h_t(x)$ therefore defines a path from $x$ to $g(f(x))$.
Similarly, we can construct a path from $g(f(x'))$ to $x'$.
Concatenating these paths with $g\circ\psi$ gives an $x,x'$-path
in $X$ and establishes the $\impliedby$ direction.

\underline{Second,} we want to prove that $\cpt{x}=\cpt{x'}$
$\iff$ $\cpt{f(x)}=\cpt{f(x')}$.  The $\implies$ direction
follows from the fact that the continuous image of a connected
space is connected.

We now want to show that if $\cpt{f(x)}=\cpt{f(x')}$, then
$\cpt{x}=\cpt{x'}$.
By the $\implies$ direction, the hypothesis implies that 
$\cpt{g(f(x))}=\cpt{g(f(x'))}$.  Again, consider the homotopy
$H:X\x I\to X$ with $H(z,0)=z$ and $H(z,1)=g(f(z))$.
Let $U=\cpt{H(x,1)}$ and $V=X\sm U$;
then the clopen decomposition $X=U\dju V$
pulls back to a clopen decomposition
$$X\x I = H^{-1}(U) \dju H^{-1}(V)$$
but both $(x,1)$ and $(x',1)$ lie in the same piece
of this decomposition because
$H$ maps them into the same component of~$X$.
That piece must $H^{-1}(U)$,
On the other hand, path-components are contained in
connected components, and $X$ has paths
from $(x,0)$ to $(x,1)$ and from $(x',0)$ to $(x',1)$, namely
$$\phi(t)=H(x,t),\qquad \phi'(t)=H(x',t),$$
so $(x,0)$ and $(x',0)$ belong to $H^{-1}(U)$ as well,
which is to say that $x,x'\in U$ as desired.

\underline{Third,} once we know that $f$ maps (path-)components to (path-)components, we know that for every component $C$ of $X$, the composition $g\circ f$ maps $C$ to some (path-)component $C'$ of $X$.  On the other hand, if $\{h_t\}$ is a homotopy between $g\circ f$ and $\IdMap_X$, then for every $p\in C$ there is a path $t\mapsto h_t(p)$ from $p$ to $g(f(p))$.  This is only possible if $C'=C$.  Now $\{h_t|_C\}$ is a homotopy between $(g\circ f)|_C = (g|_D)\circ(f|_C)$ (where $D$ is the (path-)component of $Y$ containing $f(C)$) and $\IdMap_C$.
 
The ``deduce that\dots'' part is immediate.  This problem is another
justification that homotopy equivalence is a sensible thing
to consider.

Some students (at least in 2011) argued that since the continuous image of a (path-)connected
space is (path-)connected, the maps $f$ and $g$ induce surjections $P_X\to P_Y$
and $P_Y\to P_X$ respectively, where $P_X$ means the set of (path-)connected components
of $X$; therefore, $|P_X|\geq|P_Y|\geq|P(X)|$ and equality holds throughout.  However,
these numbers may be infinite, when the statement ``$a\leq b\leq a\implies a=b$'' is
the (highly nontrivial) Bernstein-Schr\"oder-Cantor theorem of set theory.  In the
context of topology, I think it's more natural to show directly that $f$ and $g$ induce
bijections.

%------------------------------------------------------------------------------------------------------------------------

\yellprob{Let $p,q$ be distinct points on $S^2$, and let $X$ be the space obtained by gluing
them together.  Determine the homotopy type of $X$.}

\soln $X\htop S^1\vee S^2$.  This is Example~0.11 in Hatcher (p.13),
so I only assigned 5 points for it.  But it is helpful to be able to
see the homotopy equivalence for yourself.  Here is my description of it.

To see this, start with $S^1\vee S^2$, which looks like a sphere glued to a circle
at a point $p$.  Let's put the circle inside the sphere (figure; left)
Think of the circle as a closed loop from $p$ to itself.
Move the other one endpoint of the loop to another point $q\neq p$ (figure; center);
this move is a homotopy equivalence by
Proposition 0.18.  Then collapse the resulting non-closed path to a point (figure; right);
this is a homotopy equivalence by Prop.~0.17, and the resulting space is $X$.

%------------------------------------------------------------------------------------------------------------------------

\yellprob{For $k\geq 1$, let $T_n$ denote the $n$-holed torus.  Construct a cell complex structure on $T_n$.}

\soln Again, this is an example in Hatcher (p.5), so only 5 points.
There are lots of ways to impose a cell complex structure.  Here's one, which stems from the observation that $T_n=T_1\#T_{n-1}$,
where $\#$ is the operation of \emph{connected sum}: cut out a small disk from each of the operands,
then glue their boundaries together.\footnote{This is not a purely topological operation, in the sense that
you need to know the dimensions of two spaces in order to build their connected sum.}  This basically
reduces the problem to finding a fine enough cell structure on the torus that is compatible with this operation.
For example, start with the cell structure on the 1-hole torus shown below.  Make $n$ photocopies, delete the cells
$X_1,Y_2,X_2,\dots,X_{n-1},Y_n$ (here, e.g., $Y_2$ means ``the cell labeled $Y$ in the 2nd photocopy'')
and identify the boundaries: $\bd X_1=\bd Y_2$, \dots, $\bd X_{n-1}=\bd Y_n$.
\includefigure{2in}{2in}{torus4x4}

%------------------------------------------------------------------------------------------------------------------------

\yellprob{Let $X$ be a finite graph lying in a half-plane $P\subset\Rr^3$ and intersecting the
edge $e$ of $P$ in a subset of its vertices.  Describe the homotopy type of the ``surface of revolution''
obtained by rotating $X$ about $e$.}

\soln Let $R(X)$ be the surface of revolution thus obtained.
First, we can assume that $X$ is connected --- if it has multiple connected components $X_\alpha$,
then $R(X)$ is the disjoint union of the $R(X_\alpha)$.

Second, observe that if $a$ is an edge of $X$ that has at least one endpoint not in $e$,
then $R(X)\htop R(X/a)$; this corresponds to deformation-retracting an annulus
(specifically, the piece of $R(X)$ corresponding to $a$) onto a circle.  (We just need to
make sure that no other point hits the boundary during the contraction process.  For example,
if such a contraction produces an edge between two vertices on $e$, then that edge needs to
be some kind of an arc rather than a line segment.)

Let
\begin{align*}
n &= \text{total number of vertices of $X$,}\\
b &= \text{number of vertices of $X$ lying on $e$},\\
m &= \text{number of edges of $X$.}
\end{align*}

If $b=0$, then we can eventually replace $X$ with a graph $Y$ with one vertex and $m-n+1$ loops, while preserving the homotopy type of $R(X)$.  We then have
\[R(X) \htop R(Y) = S^1\x X = S^1 \x (S^1)^{(m-n+1)}\]
(which is the infamous ``not-a-torus'' space).

If $b>1$, then we will eventually end up with a graph $Y$ all of whose vertices lie on $e$,
and such that $R(Y)\htop R(X)$.
Then $Y$ has $b$ vertices and $m-(n-b)$ edges.

Let $T$ be a spanning tree of $R(Y)$.  Then $T$ has $b-1$ edges, each of which gets rotated into
a 2-sphere, and so $R(T)\htop (S^2)^{\vee(b-1)}$.  (This notation means
``the wedge sum of $b-1$ copies of $S^2$''.)  For each
additional edge $a\not\in T$, the revolution $R(a)$ is either a sphere attached to $R(T)$ at two points
(if $a$ is not a loop) or a sphere with two of its points identified, then attached to $R(T)$ (if $a$
is a loop).  In either case, the edge $a$ contributes an additional $S^1\vee S^2$ to the homotopy type.
We conclude that
\begin{align*}
R(X) &\htop R(Y) \htop (S^2)^{\vee(b-1)} ~\vee~ (S^1\vee S^2)^{\vee(m-(n-b)-(b-1))}\\
  &= (S^2)^{\vee(b-1)} ~\vee~ (S^1\vee S^2)^{\vee(m-n+1)}\\
  &= \boxed{(S^2)^{\vee(m-n+b)} ~\vee~ (S^1)^{\vee(m-n+1)}.}
\end{align*}
(The graph pictured has
 $n=9$, $b=4$, $m=15$, $m-n+b=10$, $m-n+1=7$.)

A comment: I should have required that each component of $X$ have at least one vertex on $e$.  (Otherwise, if for example $X$ consists of one vertex off $e$ and a loop attached to it, then $R(X)$ is a torus rather than a wedge of spheres.)

%------------------------------------------------------------------------------------------------------------------------

\yellprob{Let $0\leq k\leq n$.
Recall from class that the \emph{Grassmannian} $G(k,n)$
is defined as the space of $k$-dimensional subspaces $V\subset\Rr^n$,
so that in particular, $G(1,\Rr^n)=\Rr P^{n-1}$.  (Fact: Everything in
this problem works the same way if you change $\Rr$ to $\Cc$, except
that the dimensions of all the cells get doubled.)

\probpart Work out an explicit cell decomposition for $G(2,4)$
as a finite CW-complex.  That is,
describe how to decompose the set $G(2,4)$ into pieces, each of which is
isomorphic to a $\Rr$-vector space.
If you do this correctly (hint: row-reduced echelon form),
then the isomorphisms should be straightforward from the construction.}

Any $V\in G(2,4)$ can be expressed as the column span
of a $k\x n$ matrix $M$.  Performing elementary column operations
on the matrix doesn't change the span, and we know that
we can eventually put $M$ into a unique reduced column-echelon form,
i.e., one of the following things:
$$
\begin{bmatrix} 1 & 0\\ 0 & 1\\ * & *\\ * & *\end{bmatrix},\qquad
\begin{bmatrix} 1 & 0\\ * & 0\\ 0 & 1\\ * & *\end{bmatrix},\qquad
\begin{bmatrix} 1 & 0\\ * & 0\\ * & 0\\ 0 & 1\end{bmatrix},\qquad
\begin{bmatrix} 0 & 0\\ 1 & 0\\ 0 & 1\\ * & *\end{bmatrix},\qquad
\begin{bmatrix} 0 & 0\\ 1 & 0\\ * & 0\\ 0 & 1\end{bmatrix},\qquad
\begin{bmatrix} 0 & 0\\ 0 & 0\\ 1 & 0\\ 0 & 1\end{bmatrix}
$$
This gives a cell structure with $f$-polynomial
\[f(X,q) := \sum_{e_\alpha\in X} q^{\dim e_\alpha} = q^4+q^3+2q^2+q+1.\]
Specifically, the cells classify points in $G(2,4)$
by the locations of the pivots in its reduced column-echelon form.
Moving the pivots upwards gives a bigger cell; specifically,
if we write $e_{ij}$ for the cell whose pivots are in rows $i,j$ with $i<j$,
then $e_{ij}\subseteq\overline{e_{i'j'}}$ iff $i\leq i'$ and $j\leq j'$.

\yellprobpart{Describe the attaching poset of $G(2,4)$.  (Recall that this
is the partially ordered
set whose elements are the cells $e_\alpha$, and whose order
relation is given by $e_\alpha\geq e_\beta$ if
$\overline{e_\alpha}\supseteq e_\beta$).}

It looks like this:
\[\xymatrix{
&e_{12}\ar@{-}[d]\\
&e_{13}\ar@{-}[dl]\ar@{-}[dr]\\
e_{14}\ar@{-}[dr]&&e_{23}\ar@{-}[dl]\\
&e_{24}\ar@{-}[d]\\
&e_{34}
}\]

\yellprobpart{Describe the attaching poset of $G(2,5)$.}

The cells can be labeled $\{e_{ij} \st 1\leq i<j\leq 5\}$.
The order relation is given combinatorially by
\[\ov{e_{ij}}\supset e_{i'j'}\qquad\iff\qquad i\leq i' \qandq j\leq j'.\]
Here is the whole partially ordered set:
\[\xymatrix{
&&e_{12}\ar@{-}[d]\\
&&e_{13}\ar@{-}[dl]\ar@{-}[dr]\\
&e_{14}\ar@{-}[dl]\ar@{-}[dr]&&e_{23}\ar@{-}[dl]\\
e_{15}\ar@{-}[dr]&&e_{24}\ar@{-}[dl]\ar@{-}[dr]\\
&e_{25}\ar@{-}[dr]&&e_{34}\ar@{-}[dl]\\
&&e_{35}\ar@{-}[d]\\
&&e_{45}
}\]

\yellprobpart{Write out the poset $P(2,3)$ of all Ferrers diagrams with at most two 
rows and at most three columns, ordered by containment (as sets of 
squares).  Compare it to your previous answer.}

\[\xymatrix{
&&{\yng(3,3)} \ar@{-}[d]\\
&&{\yng(3,2)}\ar@{-}[dl]\ar@{-}[dr]\\
&{\yng(2,2)}\ar@{-}[dl]\ar@{-}[dr]&&{\yng(3,1)}\ar@{-}[dl]\\
{\yng(2,1)}\ar@{-}[dr]&&{\yng(3)}\ar@{-}[dl]\ar@{-}[dr]\\
&{\yng(1,1)}\ar@{-}[dr]&&{\yng(2)}\ar@{-}[dl]\\
&&{\yng(1)}\ar@{-}[d]\\
&&\text{\huge$\0$}
}\]

The two posets are isomorphic.  This is true in general --- the attaching poset of the Schubert cell decomposition of $G(k,n)$ is isomorphic to the lattice of partitions that fit inside a $(n-k)\x k$ rectangle.  The number of squares in a partition equals the dimension of the corresponding Schubert cell.  By the way, the general formula for the number of Schubert cells of each dimension is very nice:

\[\sum_{e_\alpha\in G(k,n)} q^{\dim e_\alpha} = 
\frac{(1-q^n)(1-q^{n-1})\cdots(1-q^{n-k+1})}{(1-q^k)(1-q^{k-1})\cdots(1-q)}.\]
This is called a \emph{$q$-binomial coefficient}.  It isn't even obvious that this expression is a polynomial --- but in fact it is.

\vfill

\yellprob{{\bf [Extra credit; Hatcher p.19, \#20]} Show that the subspace $X\subset\Rr^3$ formed by a Klein bottle intersecting itself in a circle, as shown in the figure on p.19 of Hatcher, is homotopy equivalent to $S^1\vee S^1\vee S^2$.}

\soln I'm not going to try to draw the picture in LaTeX, but here's the idea -- try to see it in your head.  Look at the disk where the cylinder intersects itself.  Squash that to a point --- this is a homotopy equivalence.  What is left looks like a sphere in which three points have been identified.  By an argument just like that of Problem \#2 above, this space is homotopy-equivalent to $S^2\vee S^1\vee S^1$.

\end{document}
