\documentclass{amsart}
\usepackage{amssymb,amsmath,amsthm,mathrsfs,graphics,hyperref,stmaryrd,psfrag,arcs,xypic}
\usepackage[enableskew]{youngtab}
\numberwithin{equation}{section}
\raggedbottom
\oddsidemargin=0in
\evensidemargin=0in
\textwidth=6.5in
\textheight=8.8in
\topmargin=0.25in
\headheight=0in
\headsep=0.2in
\footskip=0in
\parskip=10bp
\parindent=0bp

\newcommand{\excise}[1]{}
\newcommand{\Latex}[1]{\textbackslash\texttt{#1}}

\newcommand{\bigpad}{\rule[-14mm]{0mm}{30mm}}
\newcommand{\smallpad}{\rule[-1.5mm]{0mm}{5mm}}
\newcommand{\pad}{\rule[-3mm]{0mm}{8mm}}
\newcommand{\padup}{\rule{0mm}{5mm}}
\newcommand{\paddown}{\rule[-3mm]{0mm}{2mm}}
\newcommand{\blank}{\rule{1.25in}{0.25mm}}
\newcommand{\commentout}[1]{}
\newcommand{\yell}[1]{\fbox{\rule[-1mm]{0mm}{4mm} \large\bf #1 }}
\newcommand{\bang}{$\bullet$\quad}
\newcommand{\indnt}{\phantom{.}\qquad}
\newcommand{\littleline}{\begin{center}\rule{4in}{0.5bp}\end{center}}

\newcommand{\includefigure}[3]{{
  \begin{center}
  \resizebox{#1}{#2}{\includegraphics{{figs/#3}}}
  \end{center}}}
\newcommand{\includefigurewithinmath}[3]{{
  \resizebox{#1}{#2}{\includegraphics{{figs/#3}}}}}

%\newcommand{\defterm}[1]{\underline{\textbf{#1}}}
\newcommand{\defterm}[1]{\textbf{#1}}

\DeclareMathOperator{\ch}{\mathbf{ch}}
\DeclareMathOperator{\colspace}{colspace}
\DeclareMathOperator{\corank}{corank}
\DeclareMathOperator{\CST}{CST}
\DeclareMathOperator{\deln}{del}
\DeclareMathOperator{\diag}{diag}
\DeclareMathOperator{\ess}{ess}
\DeclareMathOperator{\Gr}{Gr}
\DeclareMathOperator{\Hom}{Hom}
\DeclareMathOperator{\im}{im}
\DeclareMathOperator{\Irr}{Irr}
\DeclareMathOperator{\Ind}{Ind}
\DeclareMathOperator{\Int}{Int}
\DeclareMathOperator{\lcm}{lcm}
\DeclareMathOperator{\link}{lk}
\DeclareMathOperator{\nullity}{nullity}
\DeclareMathOperator{\nullspace}{nullspace}
\DeclareMathOperator{\Poin}{Poin}
\DeclareMathOperator{\proj}{proj}
\DeclareMathOperator{\rank}{rank}
\DeclareMathOperator{\Res}{Res}
\DeclareMathOperator{\Span}{span}
\DeclareMathOperator{\supp}{supp}
\DeclareMathOperator{\row}{row}
\DeclareMathOperator{\rowspace}{rowspace}
\DeclareMathOperator{\sh}{sh}
\DeclareMathOperator{\tr}{tr}
\DeclareMathOperator{\wt}{wt}

\newtheorem{theorem}{Theorem}[section]
\newtheorem{proposition}[theorem]{Proposition}
\newtheorem{lemma}[theorem]{Lemma}
\newtheorem{corollary}[theorem]{Corollary}
\theoremstyle{definition}
\newtheorem{definition}[theorem]{Definition}
\newtheorem{example}[theorem]{Example}
\newtheorem{remark}[theorem]{Remark}
\newtheorem{problem}[theorem]{Problem}


\newcommand{\cor}{{\bf Corollary: }}
\newcommand{\defn}{{\bf Definition: }}
\newcommand{\defns}{{\bf Definitions: }}
\newcommand{\exa}{{\bf Example: }}
\newcommand{\fact}{{\bf Fact: }}
\newcommand{\lem}{{\bf Lemma: }}
\newcommand{\notn}{{\bf Notation: }}
\newcommand{\obs}{{\bf Observation: }}
\newcommand{\note}{{\bf Note: }}
\newcommand{\prop}{{\bf Proposition: }}
\newcommand{\rmk}{{\bf Remark: }}
\newcommand{\thm}{{\bf Theorem: }}

\newcommand{\basecase}{\emph{Base case: }}
\newcommand{\indstep}{\emph{Inductive step: }}
\newcommand{\skpr}{\emph{Sketch of proof: }}

\newcommand{\0}{\emptyset}
\newcommand{\Alt}{\mathfrak{A}}
\newcommand{\Braid}{Br}
\newcommand{\CHI}{\chi^{\phantom{*}}}
\newcommand{\Cl}{C\ell}
\newcommand{\covers}{\gtrdot}
\newcommand{\coveredby}{\lessdot}
\newcommand{\dedge}[1]{\overrightarrow{{#1}}}
\newcommand{\dom}{\rhd}
\newcommand{\domeq}{\unrhd}
\newcommand{\domby}{\lhd}
\newcommand{\dombyeq}{\unlhd}
\newcommand{\Fspan}{\Ff\text{-span}}
\newcommand{\isom}{\cong}
\newcommand{\join}{\vee}
\renewcommand{\Join}{\bigvee}
\newcommand{\Laff}{L^{\text{aff}}}
\newcommand{\lin}[1]{\overleftrightarrow{{#1}}}
\newcommand{\meet}{\wedge}
\newcommand{\Meet}{\bigwedge}
\newcommand{\ov}[1]{\overline{{#1}}}
\newcommand{\partn}{\vdash}
\newcommand{\qqandqq}{\qquad\text{and}\qquad}
\newcommand{\qandq}{\quad\text{and}\quad}
\newcommand{\qand}{\quad\text{and}}
\newcommand{\qbin}[2]{{\begin{bmatrix}#1\\#2\end{bmatrix}_q}}
\newcommand{\sd}{\triangle} % symmetric difference
\newcommand{\simK}{\underset{K}{\sim}} % Knuth equivalence
\newcommand{\simJ}{\underset{J}{\sim}} % jeu de taquin equivalence
\newcommand{\sm}{\setminus}
\newcommand{\st}{~|~}
\newcommand{\soln}{\textit{Solution:\ }}
\newcommand{\Sym}{\mathfrak{S}}
\newcommand{\un}[1]{\underset{*}{#1}}
\newcommand{\unA}{\un{A}}
\newcommand{\unB}{\un{B}}
\newcommand{\unw}{\un{w}}
\newcommand{\unx}{\un{x}}
\newcommand{\uny}{\un{y}}
\newcommand{\unz}{\un{z}}
\newcommand{\x}{\times}

\renewcommand{\aa}{\mathbf{a}}
\newcommand{\bb}{\mathbf{b}}
\newcommand{\nn}{\mathbf{n}}
\newcommand{\pp}{\mathbf{p}}
\newcommand{\qq}{\mathbf{q}}
\newcommand{\xx}{\mathbf{x}}
\newcommand{\yy}{\mathbf{y}}
\newcommand{\zz}{\mathbf{z}}
 
\newcommand{\A}{\mathcal{A}}
\newcommand{\B}{\mathcal{B}}
\newcommand{\C}{\mathcal{C}}
\newcommand{\M}{\mathcal{M}}
\renewcommand{\P}{\mathcal{P}}

\newcommand{\BB}{\mathscr{B}}  %% use these for fancy script fonts -- requires mathrsfs package
\newcommand{\CC}{\mathscr{C}}
\newcommand{\FF}{\mathscr{F}}
\newcommand{\II}{\mathscr{I}}
\newcommand{\LL}{\mathscr{L}}
\newcommand{\PP}{\mathscr{P}}
\renewcommand{\SS}{\mathscr{S}}
\newcommand{\XX}{\mathscr{X}}

\newcommand{\TT}{\tilde{T}}

\newcommand{\Aa}{\mathbb{A}}
\newcommand{\Cc}{\mathbb{C}}
\newcommand{\Ff}{\mathbb{F}}
\newcommand{\Nn}{\mathbb{N}}
\newcommand{\Pp}{\mathbb{P}}
\newcommand{\Qq}{\mathbb{Q}}
\newcommand{\Rr}{\mathbb{R}}
\newcommand{\Zz}{\mathbb{Z}}

\newcommand{\rhodef}{\rho^{\phantom{*}}_{{\rm def}}}
\newcommand{\rhotriv}{\rho^{\phantom{*}}_{{\rm triv}}}
\newcommand{\rhosign}{\rho^{\phantom{*}}_{{\rm sign}}}
\newcommand{\rhoreg}{\rho^{\phantom{*}}_{{\rm reg}}}
\newcommand{\chidef}{\chi^{\phantom{*}}_{{\rm def}}}
\newcommand{\chitriv}{\chi^{\phantom{*}}_{{\rm triv}}}
\newcommand{\chisign}{\chi^{\phantom{*}}_{{\rm sign}}}
\newcommand{\chireg}{\chi^{\phantom{*}}_{{\rm reg}}}
\newcommand{\scp}[2]{\left\langle #1,\:#2\right\rangle_G}
\newcommand{\scpH}[2]{\left\langle #1,\:#2\right\rangle_H}

\newcounter{probno}
\setcounter{probno}{0}
\newcounter{partno}
\setcounter{partno}{0}
%% versions that don't print the number of points
\newcommand{\prob}{
  \vskip10bp%
  \setcounter{partno}{0}%
  \addtocounter{probno}{1}%
  {\bf Problem~\#{\arabic{probno}}}\quad}
\newcommand{\probpart}{%\rule{0in}{0in}\\ \phantom{xxx}
  \addtocounter{partno}{1}%
  {\bf (\#\arabic{probno}\alph{partno})}\ \ }
\newcommand{\probcont}{%
  {\bf Problem~\#{\arabic{probno}}}~(\emph{continued})}
\newcommand{\probo}{
  \setcounter{partno}{0}%
  \addtocounter{probno}{1}%
  {\bf (\#\arabic{probno})}\ \ }
%% versions that do print the number of points
\newcommand{\Prob}[1]{
  \vskip10bp%
  \setcounter{partno}{0}%
  \addtocounter{probno}{1}%
  {\bf Problem~\#{\arabic{probno}}~[{#1}~pts]}\quad}
\newcommand{\Probpart}[1]{%\rule{0in}{0in}\\ \phantom{xxx}
  \addtocounter{partno}{1}%
  {\bf (\#\arabic{probno}\alph{partno})~[{#1}~pts]}\ \ }

%\renewcommand{\thefootnote}{\fnsymbol{footnote}}

\usepackage{youngtab}
\begin{document}
\thispagestyle{empty}
{\bf Math 821 Problem Set \#7\\
Due date: Friday, May 2}
\bigskip


%%%%%%%%%%%%%%%%%%%%%%%%%%%%%%%%%%%%%%%%%%%%%%%%%%%%%%%%%%
\yellprob{[Hatcher p.156 \#9abc]  Compute the homology groups of the following spaces:

(a) The quotient of $\Ss^2$ obtained by identifying the north and south poles to a point.

(b) $\Ss^1\x(\Ss^1\vee \Ss^1)$.

(c) The space obtained from $D^2$ by first deleting the interiors of two disjoint subdisks, and then identifying all three resulting circles together via homomorphisms preserving clockwise orientations of these circles.}

\soln (a) Let $A$ consist of the north and south poles in $\Ss^2$ and let $X=\Ss^2/A$.  The pair $(\Ss^2,A)$ is good (because $A$ is a subcomplex in the standard cell structure on $\Ss^2$), so $\HH(X=H(\Ss^2,A)$.  The interesting part of the LES is
\[\underbrace{\HH_2(A)}_0\to\underbrace{\HH_2(\Ss^2)}_{\Zz}\to\HH_2(X)\to\underbrace{\HH_1(A)}_0\to\underbrace{\HH_1(\Ss^2)}_0\to\HH_1(X)\to\underbrace{\HH_0(A)}_{\Zz}\to\underbrace{\HH_0(\Ss^2)}_0\]
from which we get
\[\HH_2(X)=\Zz, \quad \HH_1(X)=\Zz,\quad \HH_0(X)=0\]
(the last because $X$ is path-connected).

(b) Let's use cellular homology.  The space $Y=\Ss^1\x(\Ss^1\vee\Ss^1)$ can be regarded as two tori joined along their meridional circles, so the cell structure looks like this:
\includefigure{3in}{1.625in}{two-tori}
All the cellular chain maps are zero, so the unreduced homology groups are free of order equal to the number of cells of that dimension.  Therefore
\[\HH_2(Y)=\Zz^2, \quad \HH_1(Y)=\Zz^2,\quad \HH_0(X)=0.\]

(c) Again let's use cellular homology.  Call this space $W$.  It can be endowed with the cell structure below, with $E_2=\{P\}$, $E_1=\{a,b,c,d\}$, $E_0=\{x,y\}$ and orientations as indicated:

\includefigure{6in}{3in}{hatcher-p156no9c}

Starting at the right-hand vertex and moving down, we see that $\bd P=b-c-b-d-b-a+d-a+c+a=-b-a$.
Thus $\bd P = a+e-a+d-a+c=-a+c+d+e$ and $\bd Q=e+b+d+b+c-b=e+d+c+b$, so the full cellular chain complex $E_\bullet(W)$ is
\[
\Zz E^2\xrightarrow[\bordermatrix{&P\cr a&-1\cr b&-1\cr c&0\cr d&0}]{d_2}
\Zz E^1\xrightarrow[\bordermatrix{&a&b&c&d\cr x&-1&1&0&-1\cr y&1&-1&0&1}]{d_1}
\Zz E^0 \to 0
\]
We can then compute homology in Macaulay2, e.g.:
\begin{verbatim}
d2 = matrix {{-1}, {-1}, {0}, {0}};
d1 = matrix {{-1,1,0,-1}, {1,-1,0,1}};
C  = chainComplex(d1,d2);
prune HH C
\end{verbatim}
The end result:
\[\HH_2(W)=\Zz, \qquad \HH_1(W)=\Zz^2, \qquad \HH_0(W)=0.\]

%%%%%%%%%%%%%%%%%%%%%%%%%%%%%%%%%%%%%%%%%%%%%%%%%%%%%%%%%%
\yellprob{[Hatcher p.157 \#19] Compute $H_i(\RP^n/\RP^m)$ for $m<n$ by cellular homology, using the standard CW structure on $\RP^n$ with $\RP^m$ as its $m$-skeleton.}

\soln The relative cellular chain complex $E_\bullet(\RP^n,\RP^m)$ is the cokernel of the injection $E_\bullet(\RP^m)\into E_\bullet(\RP^n)$.  Thus its groups are

\[E_k(\RP^n,\RP^m) = \begin{cases} \Zz & \text{ if } m+1\leq k\leq n,\\ 0 & \text{ otherwise.} \end{cases}\]

and its maps are the same as for $\RP^n$: i.e., the map $d_k:E_k\to E_{k-1}$ is multiplication by 2 if $k$ is even, or zero if $k$ is odd.  Therefore,

\[\boxed{H_k(\RP^n,\RP^m) = \begin{cases}
\Zz & \text{ if $k=n$ and $n$ is odd,} \\
\Zz_2 & \text{ if $m+1\leq k<n$ and $k$ is odd,}\\
0 & \text{ otherwise.}
\end{cases}}\]

%%%%%%%%%%%%%%%%%%%%%%%%%%%%%%%%%%%%%%%%%%%%%%%%%%%%%%%%%%
\yellprob{[Hatcher p.156 \#11] Let $K$ be the 3-dimensional cell complex obtained from the cube $I^3$ by identifying each pair of opposite faces via a one-quarter twist.  (See exercise \#14 on p.54.)  Compute the homology groups $\HH_n(K;\Zz)$ and $\HH_n(K;\Zz_2)$ for $n>0$.}

\soln
Call this complex $K$.  We start with the oriented labeled cube as shown.
\includefigure{6in}{3in}{twistedcube}
Identifying opposite faces with the twist brings us down to two vertices $x,y$ (indicated by their colors in the diagram) and forces the edge identifications
\[(b,-a,-d,c)\sim(k,j,-i,-\ell),\quad (e,-a,-f,i)\sim(-c,-h,k,g),\quad (\ell,f,-d,-h)\sim(e,-b,-g,j).\]
Calling the top, left and front 2-faces respectively $A,B,C$ and the solid cube $Q$., we get the cellular chain complex
\[
\Zz E^3\xrightarrow[\bordermatrix{&Q\cr A&0\cr B&0\cr C&0\cr}]{d_3}
\Zz E^2\xrightarrow[\bordermatrix{&A&B&C\cr
a&-1&1&1\cr
b&1&-1&1\cr
c&1&1&1\cr
d&-1&-1&1}]{d_2}
\Zz E^1\xrightarrow[\bordermatrix{&a&b&c&d\cr 
x&-1&-1&1&1\cr y&1&1&-1&-1}]{d_1}
\Zz E^0 \to 0.
\]
(It is perhaps not clear geometrically whether the coefficients of $\bd Q$ should be 0, 2 or $-2$, but it is clear from the algebra --- the condition $d_2d_1=0$ says that $d_2$ must be orthogonal to every row of $d_1$, and $d_1$ has full row rank so the only possibility for $d_2$ is the zero vector.)  We can compute the homology groups in Macaulay:
\begin{verbatim}
d3 = matrix {{0}, {0}, {0}};
d2 = matrix {{-1,1,1}, {1,-1,1}, {1,1,1}, {-1,-1,1}};
d1 = matrix {{-1,-1,1,1},{1,1,-1,-1}};
C  = chainComplex(d1,d2,d3);
prune HH C
\end{verbatim}
The result is
\[\boxed{\HH_3(K)=\Zz, \quad \HH_2(K)=0, \quad \HH_1(K)=\Zz_2^2.}\]

Now for homology with $\Zz_2$-coefficients.  Note that it is \textbf{not} sufficient simply to tensor these groups with $\Zz_2$.  Instead, we have to tensor the chain complex with $\Zz_2$ and then recompute homology.  The Macaulay symbol for tensor product is $**$, so all you need to do is this:
\begin{verbatim}
C2  = C ** ZZ/2
prune HH C2
\end{verbatim}
The result is
\[\boxed{\HH_3(K;\Zz_2)=\Zz_2, \quad \HH_2(K;\Zz_2)=\Zz_2^2, \quad \HH_1(K)=\Zz_2^2.}\]
Indeed, note that $\HH_2(K;\Zz_2)\neq\HH_2(K;\Zz)\otimes\Zz$.

It is also possible to use the Universal Coefficient Theorem for Homology [Hatcher, p.264], which says that
there are split short exact sequences
$0\to H_n(K)\otimes_{\Zz}\Zz_2 \to H_n(K;\Zz_2) \to \Tor(H_{n-1}(K),\Zz_2) \to 0$.
Using general facts about tensor product --- in particular, tensoring a module with itself or the ground ring does nothing to it, and tensor is distributive over direct sum --- we get
\begin{align*}
0&\to \Zz_2 \to H_3(K;\Zz_2) \to \Tor(0,\Zz_2) \to 0,\\
0&\to H_2(K;\Zz_2) \to \Tor(\Zz_2^2,\Zz_2) \to 0,\\
0&\to \Zz_2^2 \to H_1(K;\Zz_2) \to \Tor(\Zz,\Zz_2) \to 0.
\end{align*}

Using standard properties of Tor [Hatcher, Prop.~3A.5, p.265], we can rewrite these short exact sequences:
\begin{align*}
0&\to \Zz_2 \to H_3(K;\Zz_2) \to 0,\\
0&\to H_2(K;\Zz_2) \to \Zz_2^2 \to 0,\\
0&\to \Zz_2^2 \to H_1(K;\Zz_2) \to 0
\end{align*}
so the middle maps are all isomorphisms.

%%%%%%%%%%%%%%%%%%%%%%%%%%%%%%%%%%%%%%%%%%%%%%%%%%%%%%%%%%
\yellprob{[Hatcher p.157 \#20,22]  In this problem $\rchi$ denotes Euler characteristic.

(a) Let $X,Y$ be finite CW-complexes.  Show that $\rchi(X\x Y)=\rchi(X)\cdot\rchi(Y)$.

(b) Let $X$ be a finite CW complex and let $\tilde X\xrightarrow{p} X$ be an $n$-sheeted covering space.  Show that
$\rchi(\tilde X)=n\cdot\rchi(X)$.}

Note: Part (a) is easy, but part (b) is much harder and relies on facts about Lebesgue numbers of coverings, which I pushed under the rug when I talked about subdivision.  So I awarded full credit for doing (a) correctly and saying something intelligent about (b).

\soln
(a) Define the $f$-polynomial of $X$ to be $f(X,q)=\sum_{e_\alpha\in X} q^{\dim e_\alpha}$.  Then
$\rchi(X)=f(X,-1)$ and $f(X\x Y,q)=f(X,q)\cdot f(Y,q)$ (because the product of a $k$-cell and an $\ell$-cell is a
$(k+\ell)$-cell), so
\[\rchi(X\x Y)=f(X\x Y,-1)=f(X,-1)f(Y,-1)=\rchi(X)\cdot\rchi(Y).\]

(b) Let $\mathcal{U}=\{U_i\}_{i\in\mathscr{I}}$ be an open cover of $X$ such that each $p^{-1}(U_i)$ is a disjoint union of $n$ homeomorphic copies of $U_i$; we may assume that $\mathscr{I}$ is finite, say $\{1,\dots,m\}$, since $X$ is compact.  Also, abbreviate $\tilde U_i=p^{-1}(U_i)$.

Repeatedly barycentrically subdivide the cells of $X$ until every cell is contained in some $U_i$.  (This is possible by a Lebesgue number argument; see p.123 of Hatcher.)  Call the resulting $\Delta$-complex $\Delta$.  We can now endow $\tilde X$ with the structure of a $\Delta$-complex $\tilde\Delta$ whose simplices are the components of the preimages of the simplices in $\Delta$.  We then have $f(\tilde X,q)=n\cdot f(X,q)$ and so
\[\rchi(\tilde X)=\rchi(\tilde\Delta)=f(\tilde\Delta,-1)=n\cdot f(\Delta,q)=n\cdot\rchi(X).\]

{\bf Note:}
The reason this subdivision argument is necessary is that there is no guarantee that $p^{-1}(\sigma)$ is homeomorphic to $n$ copies of $\sigma$ for every cell $\sigma$.  Another possible avenue of attack, which I cannot get to work, is as follows.  The covering map $p$ induces a map of chain complexes
\[
\xymatrix{
\cdots\rto&
\bigoplus_{i=1}^m C_{k+1}(\tilde U_i) \rto^\bd \dto_{p_\sharp}&
\bigoplus_{i=1}^m C_{k}(\tilde U_i) \rto^\bd \dto_{p_\sharp}&
\bigoplus_{i=1}^m C_{k-1}(\tilde U_i) \dto_{p_\sharp}\rto&\cdots\\
\cdots\rto&
\bigoplus_{i=1}^m C_{k+1}(U_i) \rto^\bd &
\bigoplus_{i=1}^m C_{k}(U_i) \rto^\bd &
\bigoplus_{i=1}^m C_{k-1}(U_i)\rto&\cdots\\
}\]
where the horizontal maps are the direct sums of the boundary maps on the summands.  Since $\{\tilde U_i\}$ and $\{U_i\}$ are open covers of $\tilde X$ and $X$ respectively, the Subdivision Lemma says that
the rows compute their homology groups.  Certainly $C_k(\tilde U_i)=C_k(U_i)^{\oplus n}$ and
$H_k(\tilde U_i)=H_k(U_i)^{\oplus n}$.  However, while the \defterm{groups} in this diagram break as direct sums, the \defterm{chain complexes} don't --- if $U_i\cap U_j\neq\0$ then the map $\bd:C_k(U_i)\to C_{k-1}(U_j)$ is not zero.  Indeed, it is not true that $H_k(\tilde X)\isom H_k(X)$ in general; for example, take $X=\Ss^1$ and $\tilde X$ to be its connected $n$-fold covering space, which is also homeomorphic to $\Ss^1$.


%%%%%%%%%%%%%%%%%%%%%%%%%%%%%%%%%%%%%%%%%%%%%%%%%%%%%%%%%%
\yellprob{[Hatcher p.155 \#2, modified] Given a map $f:\Ss^{2n}\to \Ss^{2n}$, show that there is some point $x\in\Ss^{2n}$
with either $f(x)=x$ or $f(x)=-x$.  Deduce that every map $\RP^{2n}\to\RP^{2n}$ has a fixed point.  (Hint: Use the fact that $\Ss^{2n}$ is a covering space of $\RP^{2n}$.)}

\soln For the first part, suppose that there exists a function~$f$ without the given property.  Then we can construct
a nonzero tangent vector field $T(x)$ on $\Ss^{2n}$ as follows: let $g(x)$ be the
projection of $f(x)$ onto the tangent hyperplane to $\Ss^{2n}$ at $x$, and
let $T(x)$ be the vector from $x$ to $g(x)$.  Note that $f(x)\not\in\{x,-x\}$
implies that $g(x)\neq x$, so $T(x)\neq 0$.  We have constructed an everywhere-nonzero
tangent vector field on $\Ss^{2n}$, which is impossible by the Hairy Bowling Ball Theorem.

(An equally good solution is to incorporate the argument of the Hairy Bowling Ball Theorem
explicitly.  Consider
$$F(x,t)=\frac{x\cos\,t+f(x)\sin\,t}{\|x\cos\,t+f(x)\sin\,t\|}$$
which is well-defined because $x$ and $f(x)$ are linearly independent.
Then $F(x,0)=x/\|x\|=x$ and $F(x,\pi)=-x/\|x\|=-x$, so $F$ is a homotopy 
between the identity and antipodal maps, a contradiction because their degrees
are 1 and $-1$ respectively.  This is essentially the argument for property (g) of
degree (Hatcher, p.134); you could even just cite that property.)

For the second part, let $p:\Ss^{2n}\to\RP^{2n}$ be the usual two-to-one covering space map
that identifies antipodal points of $\Ss^{2n}$, so that $p(x)=p(-x)$.  For any $g:\RP^{2n}\to\RP^{2n}$, we have a commutative
diagram
$$\xymatrix{
\Ss^{2n} \ar[r]^f\ar[d]^p\ar[dr]^{g\circ p} & \Ss^{2n} \ar[d]^p\\
\RP^{2n} \ar[r]^g & \RP^{2n}}$$
where $f$ is obtained by lifting $g\circ p$ (using the Lifting Criterion,
Prop.~1.33, p.61 of Hatcher).  (Note that we do not have to construct $f$ explicitly!)
By the first part of the problem, there is a point
$x\in \Ss^{2n}$ with $f(x)=x$ or $-x$, so $p(x)$ is a fixed point of $g$.

%%%%%%%%%%%%%%%%%%%%%%%%%%%%%%%%%%%%%%%%%%%%%%%%%%%%%%%%%%
\yellprob{[Hatcher p.157 \#28(a), modified] (a) Use a Mayer-Vietoris sequence to compute the homology groups of the space $X$ obtained from a torus $T=\Ss^1\x \Ss^1$ by attaching a M\"obius band $M$ via a homeomorphism from the boundary circle $C$ of $M$ to the circle $\Ss^1\x\{x_0\}$ in the torus.

(b) How does the answer change if $C$ is attached to a closed loop that wraps $k$ times around the first circle (i.e., via the path $f:I\to\Ss^1\x\Ss^1$ given by $f(t)=(e^{2\pi i kt},e^{2\pi it})$)?}

{\color{red}{\bf  Mea culpa!} Part (b) is inconsistent.  The formula should be $f(t)=(e^{2\pi kit},1)$ (or replace 1 with any complex number on $\Ss^1\subset\Cc$).  This caused some of you to write $\gamma(1)=(k,1,2)$ in part (b), in which case the value of $k$ is irrelevant --- this vector is always part of a $\Zz$-basis.}

\soln (a)
Certainly $H_0(X)=\Zz$ because $X$ is path-connected.  We know the homology groups of $T\cap M=\Ss^1$, of $M\htop \Ss^1$, and of $T$, so the nonzero part of the reduced Mayer-Vietoris sequence is

\[0\to \Zz\xrightarrow{~\alpha~} H_2(X)\xrightarrow{~\beta~} \underbrace{H_1(T\cap M)}_\Zz\xrightarrow{~\gamma~} \underbrace{H_1(T)\oplus H_1(M)}_{\Zz^3}\xrightarrow{~\delta~}H_1(X) \to 0.\]

The map $\gamma$ is given by $\gamma(1)=(1,0,2)$ (since $C$ is a double cover of the central circle, which generates $H_1(M)$).  Certainly $\gamma$ is one-to-one,
so $\beta=0$, hence $\alpha$ is an isomorphism.  That is, $H_2(X)\isom\Zz$.  The rest of the diagram is a short exact sequence
\[
 0\to \underbrace{H_1(T\cap M)}_\Zz\xrightarrow{~\gamma~} \underbrace{H_1(T)\oplus H_1(M)}_{\Zz^3}\xrightarrow{~\delta~}H_1(X)\to 0.\]
 Evidently $H_1(X)$ has free rank 2.
The vector $\gamma(1)$ is part of a basis for $\Zz^3$ (say $\{(1,0,2),(0,1,0),(0,0,1)\}$.  Therefore $H_1(X)=\coker \gamma$ is torsion-free and $H_1(X)=\Zz^2$.  In conclusion,

\[\boxed{\pad\quad H_2(X)=\Zz, \quad H_1(X)=\Zz^2.\quad}\]

(b) In this case we have $\gamma(1)=(k,0,2)$.  If $k$ is odd then nothing changes.  If $k$ is even, then $\gamma(1)$ is not part of a basis but $\gamma(1)/2$ is, and we will wind up with $H_1(X)=\Zz^2\oplus\Zz_2$.  In short,

\[\boxed{\pad\quad H_2(X)=\Zz, \qquad H_1(X)=\begin{cases} \Zz^2 & \text{ if $k$ is odd,}\\ \Zz^2\oplus\Zz_2 & \text{ if $k$ is even}.\end{cases}\quad}\]

%%%%%%%%%%%%%%%%%%%%%%%%%%%%%%%%%%%%%%%%%%%%%%%%%%%%%%%%%%
\yellprob{[Hatcher p.158 \#29] The surface $M_g$ of genus $g$, embedded in $\Rr^3$ in the standard way, bounds a compact region $R$.  Two copies of $R$, glued together by the identity map between their boundary surfaces $M_g$, form a compact closed 3-manifold $X$.  Compute the homology groups of $X$ using a Mayer-Vietoris sequence.  Also compute the relative groups  $H_i(R,M_g)$.}

\soln From Example 2.36, we have $\HH_1(M_g)=\Zz^{2g}$,
$\HH_2(M_g)=\Zz$, and $\HH_n(M_g)=0$ for $n\geq 3$.
$$
\HH_2(M_g)=\Zz\{M\},\qquad
\HH_1(M_g)=\Zz\{a_1,\dots,a_g,b_1,\dots,b_g\}
$$
where the $a_i$ are the longitudinal circles and the $b_i$ are the
meridional circles.

Meanwhile, $R$ is homotopy equivalent to the wedge of the $g$
longitudinal circles, so $\HH_1(R)=\Zz^g$
and $\HH_n(R)=0$ for $n\geq 2$.

For $n\geq 3$, the Mayer-Vietoris sequence includes the piece
$$\HH_n(R)^2 = 0 \to \HH_n(X) \to \HH_{n-1}(M_g) \to 0 = \HH_{n-1}(R)^2$$
whence $\HH_n(X)\isom \HH_{n-1}(M_g)$; this is zero for $n\geq 4$ and $\Zz$ for $n=3$.

That leaves the end of the sequence, which is
$$\HH_2(R)^2 = 0 \to \HH_2(X) \to \underbrace{\HH_1(M_g)}_{\Zz^{2g}}
\xrightarrow{\bd} \underbrace{\HH_1(R)^2}_{\Zz^{2g}} \to\HH_1(X)\to 0$$
The map $\bd$ kills all the meridional circles and maps
the longitudinal circles $a_i$ to $(a_i,a_i)\in\HH_1(R)^2$;
in particular $\HH_2(X)=\ker\bd\isom\Zz^g$ and $\HH_1(X)=\coker\bd\isom\Zz^g$.

In summary:
\[\boxed{\quad\pad\HH_3(X)=\Zz,\qquad
\HH_2(X)=\Zz^g,\qquad
\HH_1(X)=\Zz^g,\qquad
\HH_0(X)=\Zz\quad}\]
and $\HH_n(X)=0$ for $n>3$.

\littleline

Now for relative homology.  Observe that $R$ deformation-retracts onto the wedge sum of $g$ circles, so $\HH_1(R)\isom\Zz^g$ and $\HH_k(R)=0$ for $k\neq1$.  Also, the inclusion $M_g\into R$ is good (since we can certainly cellulate in such a way that $M_g$ is a CW-subcomplex of~$R$).  So the long exact sequence of reduced relative homology breaks into pieces
\[H_3(R)=0 \to H_3(R,M_g) \to H_2(M_g)=\Zz \to 0\]
and
\[H_2(R)=0 \to H_2(R,M_g) \to H_1(M_g)=\Zz^{2g} \xrightarrow{~\psi~} \HH_1(R)=\Zz^g \to \HH_1(R,M_g)\to \HH_0(M_g)=0.\]
The first of these gives immediately $H_3(R,M_g)=\Zz$.

For the second sequence, regard $M_g$ as the connected sum of $g$ tori.  The longitudinal circles of these tori are mapped by $\psi$ to the generators of $H_1(R)$, while their meridional circles are all mapped to 0.  Therefore,
\[\HH_2(R,M_g) \isom \ker\psi \isom \Zz^g\]
and since $\psi$ is surjective, the arrow following it is the zero map.  But then the group $\HH_1(R,M_g)$ is pinched by two zero maps, so it itself must be zero.

\end{document}