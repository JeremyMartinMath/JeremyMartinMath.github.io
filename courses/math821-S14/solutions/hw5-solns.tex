\documentclass{amsart}
\usepackage{amssymb,amsmath,amsthm,mathrsfs,graphics,hyperref,ifthen,framed,cancel,fullpage,color,ytableau,tcolorbox,bm,tikz}
\usepackage[enableskew]{youngtab}
\raggedbottom
\parskip=10bp
\parindent=0bp
\raggedbottom

%%%%%%%%%%%%%%%%   Colors for TikZ and text  %%%%%%%%%%%%%%%%%

\definecolor{light}{gray}{.75}
\definecolor{med}{gray}{.5}
\definecolor{dark}{gray}{.25}
\newcommand{\Red}[1]{{\color{red}{#1}}}
\newcommand{\RED}[1]{{\color{red}{\boldmath\textbf{#1}\unboldmath}}}
\newcommand{\Blue}[1]{{\color{blue}{#1}}}
\newcommand{\BLUE}[1]{{\color{blue}{\boldmath\textbf{#1}\unboldmath}}}

% Hyperlinks
\hypersetup{colorlinks, citecolor=red, filecolor=black, linkcolor=blue, urlcolor=blue}
\newcommand{\hreftext}[2]{\href{#1}{\Blue{#2}}} % for text anchors
\newcommand{\hrefurl}[2]{\href{#1}{\Blue{\tt #2}}} % if you want to actually write out the URL in text

% Spacing and commands
\newcommand{\bigpad}{\rule[-14mm]{0mm}{30mm}}
\newcommand{\smallpad}{\rule[-1.5mm]{0mm}{5mm}}
\newcommand{\pad}{\rule[-3mm]{0mm}{8mm}}
\newcommand{\padup}{\rule{0mm}{5mm}}
\newcommand{\paddown}{\rule[-3mm]{0mm}{2mm}}
\newcommand{\blank}{\rule{1.25in}{0.25mm}}
\newcommand{\yell}[1]{\fbox{\rule[-1mm]{0mm}{4mm} \large\bf #1 }}
\newcommand{\indnt}{\phantom{.}\qquad}
\newcommand{\littleline}{\begin{center}\rule{4in}{0.5bp}\end{center}}

% macros for inserting figures
\newcommand{\includefigure}[3]{\begin{center}\resizebox{#1}{#2}{\includegraphics{#3}}\end{center}}
\newcommand{\includefigurewithinmath}[3]{\resizebox{#1}{#2}{\includegraphics{#3}}}
% E.g., to insert a standalone figure of width 3" and height 1.5", use
% \includefigure{3in}{1.5in}{foo.pdf}

% math operators
\DeclareMathOperator{\Comp}{Comp}
\DeclareMathOperator{\Fix}{Fix}
\DeclareMathOperator{\id}{id}
\DeclareMathOperator{\im}{im}
\DeclareMathOperator{\lcm}{lcm}
\DeclareMathOperator{\Par}{Par}
\DeclareMathOperator{\rank}{rank}
\DeclareMathOperator{\sh}{sh}
\DeclareMathOperator{\tr}{tr}
\DeclareMathOperator{\wt}{wt}

% theorem environments, automatically numbered
\newtheorem{theorem}{Theorem}[section]
\newtheorem{proposition}[theorem]{Proposition}
\newtheorem{lemma}[theorem]{Lemma}
\newtheorem{corollary}[theorem]{Corollary}
\theoremstyle{definition}
\newtheorem{definition}[theorem]{Definition}
\newtheorem{example}[theorem]{Example}
\newtheorem{remark}[theorem]{Remark}
\newtheorem{problem}[theorem]{Problem}

\newcommand{\skpr}{\emph{Sketch of proof: }}
\newcommand{\soln}{\textit{Solution:\ }}

% Generally useful macros
\newcommand{\excise}[1]{} % useful for commenting out large chunks
\newcommand{\0}{\emptyset}
\newcommand{\compn}{\models} % compositions
\newcommand{\dju}{\mathaccent\cdot\cup} % disjoint union
\newcommand{\dsum}{\displaystyle\sum}
\newcommand{\fallfac}[2]{{#1}^{\underline{#2}}} % falling factorial
\newcommand{\isom}{\cong} % isomorphism symbol
\newcommand{\partn}{\vdash} % partition symbol
\newcommand{\qqandqq}{\qquad\text{and}\qquad}
\newcommand{\qandq}{\quad\text{and}\quad}
\newcommand{\qand}{\quad\text{and}}
\newcommand{\qbin}[2]{{\begin{bmatrix}#1\\#2\end{bmatrix}_q}} % q-binomial coefficient
\newcommand{\risefac}[2]{{#1}^{\overline{#2}}} % rising factorial
\newcommand{\sd}{\triangle} % symmetric difference
\newcommand{\sm}{\setminus} % don't use a minus sign for this
\newcommand{\st}{\colon} % "such that"
\newcommand{\surj}{\twoheadrightarrow}
\newcommand{\x}{\times}

% blackboard bold fonts for sets of numbers
\newcommand{\Cc}{\mathbb{C}} % complex numbers
\newcommand{\Ff}{\mathbb{F}} % finite field
\newcommand{\Nn}{\mathbb{N}} % natural numbers
\newcommand{\Qq}{\mathbb{Q}}
\newcommand{\Rr}{\mathbb{R}}
\newcommand{\Zz}{\mathbb{Z}}

% miscellaneous
\newcommand{\TwoCases}[4]{\begin{cases}{#1}&\text{ #2}\\{#3}&\text{ #4}\end{cases}}
\newcommand{\ThreeCases}[6]{\begin{cases}{#1}&\text{ #2}\\{#3}&\text{ #4}\\{#5}&\text{ #6}\end{cases}}
\newcommand{\bridgehand}[4]{\spadesuit\ {\textsf{#1}}\ \ \heartsuit\ {\textsf{#2}}\ \ \diamondsuit\ {\textsf{#3}}\ \ \clubsuit\ {\textsf{#4}}}

% macros for automatic problem numbering --- students don't have to use these
\newcounter{probno}
\setcounter{probno}{0}
\newcounter{partno}
\setcounter{partno}{0}
%% versions that don't print the number of points
\newcommand{\prob}{
  \vskip10bp%
  \setcounter{partno}{0}%
  \addtocounter{probno}{1}%
  {\bf Problem~\#{\arabic{probno}}}\quad}

% No initial whitespace to make framing look nicer
\newcommand{\probns}{
  \setcounter{partno}{0}%
  \addtocounter{probno}{1}%
  {\bf Problem~\#{\arabic{probno}}}\quad}

\newcommand{\probpart}{%
  \addtocounter{partno}{1}%
  {\bf (\#\arabic{probno}\alph{partno})}\ \ }
\newcommand{\probcont}{%
  {\bf Problem~\#{\arabic{probno}}}~(\emph{continued})}
\newcommand{\probo}{
  \setcounter{partno}{0}%
  \addtocounter{probno}{1}%
  {\bf (\#\arabic{probno}}\ \ }

%% versions that do print the number of points
\newcommand{\Prob}[1]{
  \vskip10bp%
  \setcounter{partno}{0}%
  \addtocounter{probno}{1}%
  {\bf Problem~\#{\arabic{probno}}~[{#1}~pts]}\quad}

%no initial whitespace
\newcommand{\Probns}[1]{
  \setcounter{partno}{0}%
  \addtocounter{probno}{1}%
  {\bf Problem~\#{\arabic{probno}}~[{#1}~pts]}\quad}
\newcommand{\Probpart}[1]{%\rule{0in}{0in}\\ \phantom{xxx}
  \addtocounter{partno}{1}%
  {\bf (\#\arabic{probno}\alph{partno})~[{#1}~pts]}\ \ }

\newboolean{answers}
\newcommand{\Answer}[1]{\ifthenelse{\boolean{answers}}{{\bf Answer:}\ #1}{}\bigskip}

\usepackage{youngtab}
\begin{document}
\thispagestyle{empty}
{\bf Math 821 Problem Set \#5\\
Due date: Friday, April 4}
\bigskip

\yellprob{Verify that the simplicial boundary map defined by
\[\bd_n[v_0,\dots,v_n] = \sum_{i=0}^n (-1)^i [v_0,\dots,\widehat{v_i},\dots,v_n]\]
satisfies the equation $\bd_{n-1}\circ\bd_n=0$ for all $n$.  (Yes, this
calculation is done explicitly in Hatcher.
But it is so important that everyone should do it for themselves at least once.)}

\soln The summands in $\bd_{n-1} \bd_n [v_0,\dots,v_n]$ all have the form
$\pm  [v_0,\dots,\widehat{v_i},\dots,\widehat{v_j},\dots,v_n] $ for $i<j$.  Each such summand
arises twice; we need to check that the signs are opposite.  If $v_j$ is removed first, then the sign contribution is $(-1)^j (-1)^i$, because $i$ is the $i^{th}$ leftmost element of the list $[v_0,\dots,v_i,\dots,\widehat{v_j},\dots,v_n]$.
On the other hand, if $v_i$ is removed first, then the sign contribution is $(-1)^i (-1)^{j-1}$, because $j$ is the $(j-1)^{th}$ (not the $j^{th}$!) leftmost element of the list $[v_0,\dots,\widehat{v_i},\dots,v_j,\dots,v_n]$.  Therefore all summands cancel.

\yellprob{Let $X$ be an abstract simplicial complex on vertex set $[n]$ and let $|X|$ be a geometric realization of $X$
(not necessarily the standard one --- it doesn't matter).
What invariant of $|X|$ corresponds to the rank of $H^\Delta_0(X)$? }

\soln We have $H^\Delta_0(X)=\Delta^0(X)/\im\bd_1$.  The group $\Delta_1(X)$ is free abelian on the 0-simplices, i.e., the vertices.  The image of $\bd_1$ is generated by 0-chains $[v]-[w]$  whenever $vw$ is an edge.  If two vertices $v_0,v_n$ are in the same component of $X$, then there is a path $v_0,v_1,\dots,v_n$ in the 1-skeleton, so
\[[v_0]-[v_n] = ([v_0]-[v_1])+([v_1]-[v_2])+\cdots+([v_{n-1}-v_n]) \in \im\bd_1.\]
In other words, any two 0-chains representing vertices in the same component are equal modulo $\im\bd_1$.  On the other hand, the chain $\bd[v,w]=[v]-[w]$ has the property that  the sum of coefficients of vertices in any given component is even (because $v,w$ are certainly in the same component by virtue of the existence of the 1-simplex $[v,w]$), and this property extends $\Zz$-linearly to all of $\im\bd_1$.  Therefore no two vertices in different components are equal modulo $\im\bd_1$.  We conclude that $H^\Delta_0(X)\isom\Zz^c$, where $c$ is the number of components, and any selection of one vertex from each component gives a natural basis for $H^\Delta_0(X)$.

(Note: I should have written ``rank'' rather than ``dimension,'' since
$H_0(X)$ is a $\Zz$-module, not a vector space.)

\yellprob{Consider the matrix
$$M=\begin{bmatrix} 1&1&0\\1&0&1\\0&1&1\end{bmatrix}.$$
Describe $\coker M$
(i) if $M$ is regarded as a linear transformation over $\Qq$;
(ii) if $M$ is regarded as a linear transformation over $\Zz$;
(iii) if $M$ is regarded as a linear transformation over $\Ff_q$
(the finite field with $q$ elements).}

\soln (i) Over $\Qq$, the matrix is nonsingular, hence represents an isomorphism $\Qq^3\to\Qq^3$.  Therefore
$\coker M=\Qq^3/\im M=0$.

(ii) Over $\Zz$, the matrix is still nonsingular, but is not invertible.  Since $\det M=2$, the cockerel must be an abelian group of order 2, so must be $\Zz_2$.  More explicitly, performing $\Zz$-invertible column operations (replacing the first column with the sum of all three) gives the matrix
\[
\underbrace{\begin{bmatrix} 1&1&0\\1&0&1\\0&1&1\end{bmatrix}}_M
\underbrace{\begin{bmatrix} 1&0&0\\1&1&0\\1&0&1\end{bmatrix}}_B =
\begin{bmatrix} 2&1&0\\2&0&1\\2&1&1\end{bmatrix} =
\underbrace{\begin{bmatrix} 1&1&0\\1&0&1\\1&1&1\end{bmatrix}}_A
\underbrace{\begin{bmatrix} 2&0&0\\0&1&0\\0&0&1\end{bmatrix}}_S
\]
with $A$ also $\Zz$-invertible.  The matrix $S$ is the Smith normal form of $M$, from which we can read off $\coker M\isom\Zz_2$.

(iii) Let $q=p^a$.  If $p\neq 2$, then $\det M=2$ is a unit in $\Ff_q$, so the transformation is invertible and $\coker M=0$ just as in (i).

If $p=2$ then the matrix is singular.  The rank is still 2 (since any two columns are linearly independent) so $\coker M=\Ff_q$.  (Note that the cokernel  must be a vector space, so the only invariant we need is its rank.)

\yellprob{[Hatcher p.131 \#4]  Compute \underline{by hand} the simplicial homology groups of the ``triangular parachute'' obtained from $\Delta^2$ by identifying its vertices to a single point.}

Call the complex $P$ (for ``parachute'').  Call the triangle $T$ and the edges $a,b,c$.  It doesn't matter how we orient them --- say $\bd T=a+b+c$.  There is only one vertex $v$, so all edges are loops.  So the simplicial chain complex is

\[\Delta_2=\Zz\{T\} \xrightarrow[\begin{bmatrix}1\\1\\1\end{bmatrix}]{\bd_2}
\Delta_1=\Zz\{a,b,c\} \xrightarrow[\begin{bmatrix}0&0&0\end{bmatrix}]{\bd_1}
\Delta_0=\Zz\{v\}\]
and
\begin{align*}
H^\Delta_2(P) &= \ker\bd_2 &&= 0,\\
H^\Delta_1(P) &= \ker\bd_1/\im\bd_2 = \Zz\{a,b,c\}/\Zz\{a+b+c\}&&\isom\Zz^2,\\
H^\Delta_0(P) &= \Delta_0/\im\bd_1 &&\isom \Zz \quad\text{(or cite Problem 2)}.
\end{align*}

\yellprob{[Hatcher p.131 \#5]  Compute \underline{by hand} the simplicial homology groups of the Klein bottle using the $\Delta$-complex structure on p.102 (with two triangles).}

Using Hatcher's labeling of the simplices, the simplicial chain complex is

\[\Delta_2=\Zz\{U,L\} \xrightarrow[\begin{bmatrix}1&1\\1&-1\\-1&1\end{bmatrix}]{\bd_2}
\Delta_1=\Zz\{a,b,c\} \xrightarrow[\begin{bmatrix}0&0&0\end{bmatrix}]{\bd_1}
\Delta_0=\Zz\{v\}\]

The columns of $\bd_2$ are linearly independent, so $H^\Delta_2(K)=0$, and yet again $H^\Delta_0(K)=\Zz$.
To calculate $H^\Delta_1(K)$, observe that 
\[\{v_1=(1,1,-1), \ v_2=(1,0,0), \ v_3=(0,1,0)\}\]
generates $\Zz^3$ as a $\Zz$-module, and that
\[\im\bd_2 ~=~ \Zz\{(1,1,-1),\ (1,-1,1)\}
~=~ \Zz\{(1,1,-1),\ (1,-1,1)+(1,1,-1\}
~=~ \Zz\{(1,1,-1),\ (2,0,0)\}
~=~ \Zz\{v_1,2v_2\}.\]
Therefore $H^\Delta_1(K) = \Zz^3/\im\bd_2 = \Zz\oplus\Zz_2.$

\yellprob{Check your answers on the last two problems using {\tt Macaulay2} or your favorite computer algebra system.}

Here is one efficient way of doing it:

\begin{verbatim}
D2 = matrix{{1},{1},{1}}; D1 = matrix{{0,0,0}};
Parachute = chainComplex (D1,D2);
prune HH Parachute

D2 = matrix{{1,1},{1,-1},{-1,1}}; D1 = matrix{{0,0,0}};
Klein = chainComplex (D1,D2);
prune HH Klein
\end{verbatim}

\yellprob{Let $\Delta^{n,d}$ denote the $d$-skeleton of the $n$-simplex.  As an abstract simplicial complex, $\Delta$ is generated by all $(d+1)$-element subsets of $\{0,\dots,n\}$.  Use {\tt Macaulay2} (or another computer algebra system) to compute the homology groups of $\Delta^{n,d}$ for various values of $n$ and $d$.  Conjecture a general formula for $H_k(\Delta^{n,d})$ in terms of $n$, $d$ and $k$.  (Prove it, if you want.)}

The answer is
\[\HH_k(\Delta^{n,d}) = \begin{cases}
\Zz^{\binom{n}{d+1}}  & \text{ if $k=d$},\\
0 & \text{ if $k<d$}.
\end{cases}\]

I gave full credit for making the correct conjecture.  With the tools we have available, one probably needs an induction argument (e.g., using the fact that the chain complexes of $\Delta^{n,d}$ and $\Delta^{n,d+1}$ are identical except in dimension $d+1$).  This problem will appear at a later date.

%Consider the subcomplex $\Gamma$ of $\Delta^{n,d}$ generated by all simplices containing vertex 0.  This is called the \emph{star} of vertex~0, and it is a cone, so it is contractible and acyclic.  Contracting $\Gamma$ produces a cell complex homotopy-equivalent to $\Delta^{n,d}$ that is a wedge of $d$-spheres corresponding to the $d$-simplices not containing 0 (i.e., the only faces not in $\Gamma$).  Each such face is given by a choice of $d+1$ vertices from $[n]$, so there are $\binom{n}{d+1}$ of them.  

\end{document}

