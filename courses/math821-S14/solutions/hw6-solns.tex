\documentclass{amsart}
\usepackage{amssymb,amsmath,amsthm,mathrsfs,graphics,hyperref,stmaryrd,psfrag,arcs,xypic}
\usepackage[enableskew]{youngtab}
\numberwithin{equation}{section}
\raggedbottom
\oddsidemargin=0in
\evensidemargin=0in
\textwidth=6.5in
\textheight=8.8in
\topmargin=0.25in
\headheight=0in
\headsep=0.2in
\footskip=0in
\parskip=10bp
\parindent=0bp

\newcommand{\excise}[1]{}
\newcommand{\Latex}[1]{\textbackslash\texttt{#1}}

\newcommand{\bigpad}{\rule[-14mm]{0mm}{30mm}}
\newcommand{\smallpad}{\rule[-1.5mm]{0mm}{5mm}}
\newcommand{\pad}{\rule[-3mm]{0mm}{8mm}}
\newcommand{\padup}{\rule{0mm}{5mm}}
\newcommand{\paddown}{\rule[-3mm]{0mm}{2mm}}
\newcommand{\blank}{\rule{1.25in}{0.25mm}}
\newcommand{\commentout}[1]{}
\newcommand{\yell}[1]{\fbox{\rule[-1mm]{0mm}{4mm} \large\bf #1 }}
\newcommand{\bang}{$\bullet$\quad}
\newcommand{\indnt}{\phantom{.}\qquad}
\newcommand{\littleline}{\begin{center}\rule{4in}{0.5bp}\end{center}}

\newcommand{\includefigure}[3]{{
  \begin{center}
  \resizebox{#1}{#2}{\includegraphics{{figs/#3}}}
  \end{center}}}
\newcommand{\includefigurewithinmath}[3]{{
  \resizebox{#1}{#2}{\includegraphics{{figs/#3}}}}}

%\newcommand{\defterm}[1]{\underline{\textbf{#1}}}
\newcommand{\defterm}[1]{\textbf{#1}}

\DeclareMathOperator{\ch}{\mathbf{ch}}
\DeclareMathOperator{\colspace}{colspace}
\DeclareMathOperator{\corank}{corank}
\DeclareMathOperator{\CST}{CST}
\DeclareMathOperator{\deln}{del}
\DeclareMathOperator{\diag}{diag}
\DeclareMathOperator{\ess}{ess}
\DeclareMathOperator{\Gr}{Gr}
\DeclareMathOperator{\Hom}{Hom}
\DeclareMathOperator{\im}{im}
\DeclareMathOperator{\Irr}{Irr}
\DeclareMathOperator{\Ind}{Ind}
\DeclareMathOperator{\Int}{Int}
\DeclareMathOperator{\lcm}{lcm}
\DeclareMathOperator{\link}{lk}
\DeclareMathOperator{\nullity}{nullity}
\DeclareMathOperator{\nullspace}{nullspace}
\DeclareMathOperator{\Poin}{Poin}
\DeclareMathOperator{\proj}{proj}
\DeclareMathOperator{\rank}{rank}
\DeclareMathOperator{\Res}{Res}
\DeclareMathOperator{\Span}{span}
\DeclareMathOperator{\supp}{supp}
\DeclareMathOperator{\row}{row}
\DeclareMathOperator{\rowspace}{rowspace}
\DeclareMathOperator{\sh}{sh}
\DeclareMathOperator{\tr}{tr}
\DeclareMathOperator{\wt}{wt}

\newtheorem{theorem}{Theorem}[section]
\newtheorem{proposition}[theorem]{Proposition}
\newtheorem{lemma}[theorem]{Lemma}
\newtheorem{corollary}[theorem]{Corollary}
\theoremstyle{definition}
\newtheorem{definition}[theorem]{Definition}
\newtheorem{example}[theorem]{Example}
\newtheorem{remark}[theorem]{Remark}
\newtheorem{problem}[theorem]{Problem}


\newcommand{\cor}{{\bf Corollary: }}
\newcommand{\defn}{{\bf Definition: }}
\newcommand{\defns}{{\bf Definitions: }}
\newcommand{\exa}{{\bf Example: }}
\newcommand{\fact}{{\bf Fact: }}
\newcommand{\lem}{{\bf Lemma: }}
\newcommand{\notn}{{\bf Notation: }}
\newcommand{\obs}{{\bf Observation: }}
\newcommand{\note}{{\bf Note: }}
\newcommand{\prop}{{\bf Proposition: }}
\newcommand{\rmk}{{\bf Remark: }}
\newcommand{\thm}{{\bf Theorem: }}

\newcommand{\basecase}{\emph{Base case: }}
\newcommand{\indstep}{\emph{Inductive step: }}
\newcommand{\skpr}{\emph{Sketch of proof: }}

\newcommand{\0}{\emptyset}
\newcommand{\Alt}{\mathfrak{A}}
\newcommand{\Braid}{Br}
\newcommand{\CHI}{\chi^{\phantom{*}}}
\newcommand{\Cl}{C\ell}
\newcommand{\covers}{\gtrdot}
\newcommand{\coveredby}{\lessdot}
\newcommand{\dedge}[1]{\overrightarrow{{#1}}}
\newcommand{\dom}{\rhd}
\newcommand{\domeq}{\unrhd}
\newcommand{\domby}{\lhd}
\newcommand{\dombyeq}{\unlhd}
\newcommand{\Fspan}{\Ff\text{-span}}
\newcommand{\isom}{\cong}
\newcommand{\join}{\vee}
\renewcommand{\Join}{\bigvee}
\newcommand{\Laff}{L^{\text{aff}}}
\newcommand{\lin}[1]{\overleftrightarrow{{#1}}}
\newcommand{\meet}{\wedge}
\newcommand{\Meet}{\bigwedge}
\newcommand{\ov}[1]{\overline{{#1}}}
\newcommand{\partn}{\vdash}
\newcommand{\qqandqq}{\qquad\text{and}\qquad}
\newcommand{\qandq}{\quad\text{and}\quad}
\newcommand{\qand}{\quad\text{and}}
\newcommand{\qbin}[2]{{\begin{bmatrix}#1\\#2\end{bmatrix}_q}}
\newcommand{\sd}{\triangle} % symmetric difference
\newcommand{\simK}{\underset{K}{\sim}} % Knuth equivalence
\newcommand{\simJ}{\underset{J}{\sim}} % jeu de taquin equivalence
\newcommand{\sm}{\setminus}
\newcommand{\st}{~|~}
\newcommand{\soln}{\textit{Solution:\ }}
\newcommand{\Sym}{\mathfrak{S}}
\newcommand{\un}[1]{\underset{*}{#1}}
\newcommand{\unA}{\un{A}}
\newcommand{\unB}{\un{B}}
\newcommand{\unw}{\un{w}}
\newcommand{\unx}{\un{x}}
\newcommand{\uny}{\un{y}}
\newcommand{\unz}{\un{z}}
\newcommand{\x}{\times}

\renewcommand{\aa}{\mathbf{a}}
\newcommand{\bb}{\mathbf{b}}
\newcommand{\nn}{\mathbf{n}}
\newcommand{\pp}{\mathbf{p}}
\newcommand{\qq}{\mathbf{q}}
\newcommand{\xx}{\mathbf{x}}
\newcommand{\yy}{\mathbf{y}}
\newcommand{\zz}{\mathbf{z}}
 
\newcommand{\A}{\mathcal{A}}
\newcommand{\B}{\mathcal{B}}
\newcommand{\C}{\mathcal{C}}
\newcommand{\M}{\mathcal{M}}
\renewcommand{\P}{\mathcal{P}}

\newcommand{\BB}{\mathscr{B}}  %% use these for fancy script fonts -- requires mathrsfs package
\newcommand{\CC}{\mathscr{C}}
\newcommand{\FF}{\mathscr{F}}
\newcommand{\II}{\mathscr{I}}
\newcommand{\LL}{\mathscr{L}}
\newcommand{\PP}{\mathscr{P}}
\renewcommand{\SS}{\mathscr{S}}
\newcommand{\XX}{\mathscr{X}}

\newcommand{\TT}{\tilde{T}}

\newcommand{\Aa}{\mathbb{A}}
\newcommand{\Cc}{\mathbb{C}}
\newcommand{\Ff}{\mathbb{F}}
\newcommand{\Nn}{\mathbb{N}}
\newcommand{\Pp}{\mathbb{P}}
\newcommand{\Qq}{\mathbb{Q}}
\newcommand{\Rr}{\mathbb{R}}
\newcommand{\Zz}{\mathbb{Z}}

\newcommand{\rhodef}{\rho^{\phantom{*}}_{{\rm def}}}
\newcommand{\rhotriv}{\rho^{\phantom{*}}_{{\rm triv}}}
\newcommand{\rhosign}{\rho^{\phantom{*}}_{{\rm sign}}}
\newcommand{\rhoreg}{\rho^{\phantom{*}}_{{\rm reg}}}
\newcommand{\chidef}{\chi^{\phantom{*}}_{{\rm def}}}
\newcommand{\chitriv}{\chi^{\phantom{*}}_{{\rm triv}}}
\newcommand{\chisign}{\chi^{\phantom{*}}_{{\rm sign}}}
\newcommand{\chireg}{\chi^{\phantom{*}}_{{\rm reg}}}
\newcommand{\scp}[2]{\left\langle #1,\:#2\right\rangle_G}
\newcommand{\scpH}[2]{\left\langle #1,\:#2\right\rangle_H}

\newcounter{probno}
\setcounter{probno}{0}
\newcounter{partno}
\setcounter{partno}{0}
%% versions that don't print the number of points
\newcommand{\prob}{
  \vskip10bp%
  \setcounter{partno}{0}%
  \addtocounter{probno}{1}%
  {\bf Problem~\#{\arabic{probno}}}\quad}
\newcommand{\probpart}{%\rule{0in}{0in}\\ \phantom{xxx}
  \addtocounter{partno}{1}%
  {\bf (\#\arabic{probno}\alph{partno})}\ \ }
\newcommand{\probcont}{%
  {\bf Problem~\#{\arabic{probno}}}~(\emph{continued})}
\newcommand{\probo}{
  \setcounter{partno}{0}%
  \addtocounter{probno}{1}%
  {\bf (\#\arabic{probno})}\ \ }
%% versions that do print the number of points
\newcommand{\Prob}[1]{
  \vskip10bp%
  \setcounter{partno}{0}%
  \addtocounter{probno}{1}%
  {\bf Problem~\#{\arabic{probno}}~[{#1}~pts]}\quad}
\newcommand{\Probpart}[1]{%\rule{0in}{0in}\\ \phantom{xxx}
  \addtocounter{partno}{1}%
  {\bf (\#\arabic{probno}\alph{partno})~[{#1}~pts]}\ \ }

%\renewcommand{\thefootnote}{\fnsymbol{footnote}}

\usepackage{youngtab}
\begin{document}
\thispagestyle{empty}
{\bf Math 821 Problem Set \#6\\
Due date: Friday, April 18}
\bigskip

\yellprob{[Hatcher p.131 \#11] Show that if $A\subset X$ and there is a retraction $X\to A$,
then the map $H_n(A)\to H_n(X)$ induced by the inclusion $A\into X$ is
injective.}

\soln Call the inclusion $i$.  A retraction is by definition a function $r:X\to A$ with $r|_A=r\circ i=\IdMap_A$.
By functoriality  we have a commutative diagram of spaces and continuous maps that induces a commutative diagram of homology groups and homomorphisms:
\[\xymatrix{A \rto^-{i}  \ar@/_1.5pc/[rr]_{\IdMap_A} & X \rto^-{r} & H_n(A) &&
H_n(A) \rto^-{i_*}  \ar@/_1.5pc/[rr]_{\IdMap} & H_n(X) \rto^-{r_*} & H_n(A)}\]
From this we see that $i_*$ is injective (and that $r_*$ is surjective).

%======================================================================

\yellprob{(a) [Hatcher p.132 \#15]  A homological algebra warmup:
Prove that if 
\[A\xrightarrow{f}B\xrightarrow{g}C\xrightarrow{h}D\xrightarrow{j}E\]
is exact with $f$ surjective and $j$ injective, then $C=0$.}

\soln We have $\im f=B=\ker g$, so $g$ is the zero map.  Similarly, $\ker j=\im h=0$, so $h$ is also the zero map.
It follows that $\im g=0=\ker h=C$, so $c=0$.

%======================================================================

\yell{\bf\boldmath(b) Prove the \emph{Snake Lemma}: if the commutative diagram
\[\xymatrix{
0\ar@{-->}[r] & A\rto^d\dto_f & B\rto^e\dto_g & C\rto\dto_h & 0\\
0\rto & A'\rto^{d'} & B'\rto^{e'} & C'\ar@{-->}[r] & 0
}\]
has exact rows, then there is an exact sequence
\[0\dashrightarrow \ker f\xrightarrow{\alpha} \ker g\xrightarrow{\beta} \ker h\xrightarrow{\gamma} \coker f\xrightarrow{\delta}\coker g\xrightarrow{\varepsilon}\coker h\dashrightarrow 0.\]\unboldmath

(The dashed arrows can be either included or omitted --- both versions are often referred to as the Snake Lemma.  In your solution, prove the version without the dashed arrows and then observe what happens if the arrows are included.)}

\soln First, we construct the maps and check that they are well-defined.  Second, we check that the sequence is exact.

\begin{enumerate}
\item For $a\in\ker f$, define $\alpha(a)=da$.  We have $gda=d'fa=0$, so $\alpha(a)\in\ker g$.
\\
\item For $b\in\ker g$, define $\beta(b)=eb$.  We have $heb=e'gb=0$, so $\beta(b)\in\ker h$.
\\
\item For $c\in\ker h$, we can write $c=e(b+\tilde{b})$ for some $b\in B$ and $\tilde{b}\in\ker e$.  (All preimages of $c$ can be obtained in this way by fixing $b$ and letting $\tilde{b}$ vary over $\ker e$.)  Since $\ker e=\im d$ and $d$ is injective, there is a unique $a\in A$ such that $\tilde{b}=da$.  So
\[c=e(b+\tilde{b})=e(b+da) \qqandqq
e'g(b+da)=he(b+da)=hc=0\]
so $g(b+da)\in\ker e'=\im d'$.  Since $d'$ is injective,
there is a unique $a'\in A'$ such that $d'a'=g(b+da)$.  Since
\[d'a'=g(b+da)=gb+gda=gb+d'fa \quad\therefore\qquad
gb=d'a'-d'fa=d'(a'-fa).
\]
That is, $a'$ is determined by $c$ modulo $\im f$, which means that $[a']\in\coker f$ is uniquely determined by~$c$.  Therefore $\gamma(c)=[a']$ is a well-defined function $\ker h\to\coker f$, and the choice of $\tilde b$ was irrelevant (so we might as well have taken $\tilde b=0$ and hence $a=0$).  This lets us rewrite the construction of~$\gamma(c)$ more conveniently for future use:
\begin{equation} \label{construct-gamma}
\text{find $b\in B$ such that $c=eb$, find $a'\in A'$ such that $d'a'=gb$, and put $\gamma(c)=[a']\in\coker f$.}
\end{equation}
\\
\item Let $[a']\in\coker f$, i.e., $[a']=a'+\im f$ for some $a'\in A'$.  Define
\[\delta[a'] = [d'a']=d'a'+\im d'f=d'a'+\im gd=d'a'+g(\im d)\]
which is a well-defined element of $\coker g=B'/\im g$.
\\
\item Let $[b']\in\coker g$, i.e., $[b']=b'+\im g$ for some $b'\in B'$.  Define
\[\varepsilon[b'] = [e'b']=e'b'+\im e'g=e'b'+\im he=e'b'+h(\im e)\]
which is a well-defined element of $\coker h=C'/\im h$.
\end{enumerate}

Now that we have all the maps defined, we check exactness.

\begin{enumerate}
\item Exactness at $\ker f$ (in the dashed-arrow case): If $d$ is injective then so is $\alpha$.
\\
\item Exactness at $\ker g$:
\begin{align*}
\im\alpha ~&=~ \alpha(\ker f) ~=~ d(\ker f) ~=~ \{da\st fa=0\}\\
~&=~ \{da\st d'fa=0\} && \text{by injectivity of $d'$}\\
~&=~ \{da\st gda=0\} && \text{by commutativity}\\
~&=~ \ker g\cap \im d ~=~ \ker g\cap\ker e && \text{by exactness of the top row}\\
~&=~ \ker\beta.
\end{align*}
\item Exactness at $\ker h$: observe that
\begin{align}
\im\beta ~&=~ \beta(\ker g) ~=~ e(\ker g) ~=~ \{eb:\ gb=0\} \label{imbeta} \\
\intertext{and}
\ker\gamma ~&=~ \{c\in\ker h\st \exists b\in B:\ c=eb, \exists a'\in A':\ d'a'=gb,\ \exists a\in A: a'=fa\}\notag\\
~&=~ \{c\in\ker h\st \exists b\in B:\ c=eb, \exists a\in A:\ d'fa=gb\}\notag\\
~&=~ \{eb\st b\in B,\ heb=0,\ \exists a\in A:\ gda=gb\}\label{kergamma}.
\end{align}
Comparing \eqref{imbeta} with \eqref{kergamma} shows that $\im\beta\subset\ker\gamma$, because if $c=eb\in\im\beta$,
then $gb=0$ and we may take $a=0$ in \eqref{kergamma}.  On the other hand, if $c=eb\in\ker\gamma$, then let $\tilde b=b-da$.
Then $c=e\tilde b$ (since $ed=0$) and $\tilde b\in\ker g$ by \eqref{kergamma}, which by \eqref{imbeta} says that $c\in\im\beta$.
\\
\item Exactness at $\coker f$: Let $c\in\ker h$.  Adopting the notation of the construction of $\gamma$ \eqref{construct-gamma}, we have 
\[\delta\gamma(c)=\delta a'=[d'a']=[gb]=0\in\coker g\]
so $\im\gamma\subseteq\ker\delta$.  On the other hand, if $[a']\in\ker\delta$ then $d'a'\in\im g$, i.e., there exists $b\in B$ such that $d'a'=gb$.  But then $\gamma(eb)=[a']$, so $[a']\in\im\gamma$.
\\
\item Exactness at $\coker g$: First, observe that
\begin{align*}
\im\delta ~&=~ \{\delta[a']\st a'\in A'\} ~=~ \{[d'a']\st a'\in A'\}\\
~&=~ \{[b']\st b'\in\im d'\}\\
~&=~ \{[b']\st b'\in\ker e'\} ~\subset~eq \ker\varepsilon.
\end{align*}
On the other hand, if $\varepsilon[b']=0$ then $e'b'=hc$ for some $c\in C$.  Since $e$ is surjective, there exists $b\in B$ such that $eb=c$,
whence $e'b'=heb=e'gb$, i.e., $b'-gb\in\ker e'=\im d'$.  That is, there is some $a'\in A'$ such that
\[b'-gb=d'a']\]
whence $\delta[a']=[d'a']=[b'-gb]=[b']\in\coker g$.  So we have shown that $\ker\varepsilon\subseteq\im\delta$.
\\
\item Exactness at $\coker h$ (in the dashed-arrow case): Assume $e'$ is onto.  For every $[c']\in\coker h$, we can find $b'\in B'$ such that $e'b'=c'$, and then $\varepsilon[b']=[e'b']=[c']$.  So $\varepsilon$ is onto as well.
\end{enumerate}

%======================================================================

\yellprob{Recall that the \emph{torsion subgroup} $T(G)$ of an abelian group is the subgroup consisting of all elements of finite order.  Let $0\to A\xrightarrow{f} B\xrightarrow{g} \Zz^n\to 0$ be a short exact sequence of finitely generated $\Zz$-modules.  Show that $T(A)\isom T(B)$.  In particular, if $A$ is free abelian then so is $B$.

Let $0\to A\xrightarrow{f} B\xrightarrow{g} C\to 0$ be a short exact sequence of finitely generated $\Zz$-modules.  Show that if $C$ is free abelian, then $T(A)\isom T(B)$, but that $A$ free abelian does not necessarily imply that $T(B)=T(C)$.}

\soln Suppose $C$ is free.  If $x\in A$, $k\in\Zz$, and $kx=0$, then $kf(x)=f(kx)=0$, so $f(x)$ is torsion.  Therefore, $f$ restricts to a one-to-one map $T(A)\to T(B)$.  On the other hand, if $b\in B$ is torsion then it is certainly in the kernel of $g$ (because its image is torsion in $C$, hence zero), hence in the image of $f$ by exactness.  So the map $T(A)\to T(B)$ is an isomorphism.

For the other direction, the short exact sequence $0\to\Zz\xrightarrow{2}\Zz\to\Zz_2\to 0$ is a counterexample.

\pagebreak
%======================================================================

\yellprob{[Hatcher p.132 \#17] (a) Compute the homology groups $H_n(X,A)$ when $X$ is $\Ss^2$ or $\Ss^1\x \Ss^1$ and $A$ is a set of $k$ points in $X$ with $k<\infty$.  You may use the computation of the homology groups of $X$ from \S2.1.

(b) Compute the groups $H_n(X,A)$ and $H_n(X,B)$, where $X$ is a closed orientable surface of genus two with $A$ and $B$ the circles shown.  (What are $X/A$ and $X/B$?)
\includefigure{2in}{0.75in}{Hatcher-p132-prob17.pdf}}

\soln (a) Recall that
\begin{align*}
H_2(\Ss^2)&=\Zz, & 
H_2(\Ss^1\x \Ss^1)&=\Zz, & 
H_2(A)&=0, \\
H_1(\Ss^2)&= 0, & 
H_1(\Ss^1\x \Ss^1)&= \Zz^2, & 
H_1(A)&= 0, \\ 
H_0(\Ss^2)&=\Zz, & 
H_0(\Ss^1\x \Ss^1)&=\Zz, & 
H_0(A)&=\Zz^k
\end{align*}
with everything vanishing in dimension $\geq3$.
So the long exact sequence for relative homology
\[0\to H_2(A) \to H_2(X)\to H_2(X,A)\to H_1(A)\to H_1(X)\to H_1(X,A)\to H_0(A)\to H_0(X)\to H_0(X,A)\to 0\]
becomes
\[0\to \Zz\to H_2(X,A)\to 0\to H_1(X)\to H_1(X,A)\to \Zz^k\to \Zz=H_0(X)\to H_0(X,A)\to 0\]
Split up the sequence at the 0.  The first piece tells us that \boldmath\underline{$H_2(X,A)=\Zz$.}\unboldmath

Now focus on the second piece, which is
\[0\to H_1(X)\xrightarrow{~j~} H_1(X,A)\xrightarrow{~\bd~} \Zz^k\xrightarrow{~i~} \Zz=H_0(X)\xrightarrow{~j'~} H_0(X,A)\to 0.\]
The map $i$ is surjective (it maps the class of any point in $A$ to the class of a point in $X$).  Therefore
$\im i=H_0(X)=\ker j'$, so $j'$ is the zero map.  It follows that  \boldmath\underline{$H_0(X,A)=0$. }\unboldmath

Since $i$ is surjective, its kernel must be a copy of $\Zz^{k-1}$.  Replacing the target of $\bd$ with $\im\bd=\ker i$ gives a  short exact sequence
\[0\to H_1(X)\xrightarrow{~j~} H_1(X,A)\xrightarrow{~\bd~} \Zz^{k-1}\to 0.\]
If $X=\Ss^2$ then $j=0$ and so $\bd$ is an isomorphism, and we get  \boldmath\underline{$H_1(\Ss^2,A)=\Zz^{k-1}$. }\unboldmath

If $X=\Ss^1\x \Ss^1$ then the exact sequence is
\[0\to \Zz^2\xrightarrow{~j~} H_1(X,A)\xrightarrow{~\bd~} \Zz^{k-1}\to 0\]
which by the result of a previous problem says that  \boldmath\underline{$H_1(\Ss^2,A)=\Zz^{k+1}$. }\unboldmath

By the way, note that $\Ss^2/A$ is $\Rr^2$ with $k$ points removed, which is homotopy-equivalent to the wedge of $k-1$ circles.  In summary,

\[\boxed{\begin{array}{l|cccc}
& \multicolumn{4}{c}{H_k(X,A)}\paddown \\
& k>2 & k=2 & k=1 & k=0 \\ \hline
\pad X=\Ss^2 & 0 & \Zz & \Zz^{k-1} & 0\\
X=\Ss^1\x\Ss^1 & 0 & \Zz & \Zz^{k+1} & 0
\end{array}}\]

\soln (b) These pairs are good, so $H_n(X,A)=\HH_n(X/A)$ and $H_n(X,B)=\HH_n(X/B)$.   
Contracting $A$ gives $X/A\isom(\Ss^1\x\Ss^1)\wedge(\Ss^1\x\Ss^1)$ (with the contracted point as the wedge point).  Meanwhile, $X/B$ is homotopy-equivalent to $( \Ss^1\x\Ss^1)\wedge \Ss^1$ by the following figure.
\includefigure{3in}{6in}{p132n17b.pdf}
Therefore
\yell{\begin{align*}
H_2(X,A)&=\Zz^2, & H_1(X,A)&=\Zz^4, & H_n(X,A)&=0 \quad (n\neq 1,2),\\
H_2(X,A)&=\Zz, & H_1(X,A)&=\Zz^3, & H_n(X,A)&=0 \quad (n\neq 1,2).
\end{align*}}
\pagebreak
%======================================================================

\yellprob{[Hatcher p.132 \#20] The \textbf{suspension} $SX$ of a space $X$ is obtained by taking two copies of the cone
$CX=X\x[0,1]/X\x\{1\}$ and attaching them along their bases.  Equivalently, take a prism over $X$ and contract each of the top and bottom faces to points:
\[SX = X \x [0,1] ~/~ X\x\{0\} ~/~ X\x\{1\}.\]
For example, the suspension of $\Ss^n$ is $\Ss^{n+1}$.

Prove that $\HH_n(SX)\isom \HH_{n-1}(X)$ for all $n>0$.  More generally, for any integer $k$, compute the reduced homology groups of the union of $k$ copies of $CX$ with their bases identified.  (The suspension is the case $k=2$.)}

\soln First we handle the suspension $SX$.  Observe that $SX=CX/X\x\{0\}$, so the long exact sequence for the pair $(CX,X\x\{0\})$ is
\[\cdots\to\HH_n(CX)\to\HH_n(CX,X\x\{0\})\to\HH_{n-1}(X\x\{0\})\to\HH_{n-1}(CX)\to\cdots\]
but the two outer terms are zero because $CX$ is contractible (recall that it deformation-retracts onto the cone point).  So the middle arrow
is the desired isomorphism.

More generally, let $X^{[k]}$ denote the union of $k$ copies of $CX$ with their bases identified (so in particular $X^{(1)}=CX$ and $X^{(2)}=SX$).  For $k>1$, we can form $X^{[k]}$ from $X^{[k-1]}$ by a two-step process: first attach a cylinder $X\x I$ to the base of $X^{[k-1]}$ along $X\x\{0\}$ to get a space $Y$, then contract $X\x\{1\}$ to a point.  This identifies $X^{[k]}$ with $Y/X\x\{1\}$, and meanwhile $Y$ deformation-retracts to $X^{[k-1]}$ by shrinking $X\x[0,1]$ to $X\x\{0\}$.  Therefore, the inclusion and quotient
\[X\x\{1\} \xrightarrow{~i~} Y \xrightarrow{~j~} Y/(X\x\{1\})\]
gives rise to a long exact sequence
\[\cdots\to H_n(X\x\{1\}) \xrightarrow{~i_*~} H_n(Y) \xrightarrow{~j_*~} H_n(Y/(X\x\{1\})) \xrightarrow{~\bd~} H_{n-1}(X\x\{1\})\to\cdots\]
The map $i_*$ is zero, since any map $\Delta^n\to X\x\{1\}$ is homotopic in $Y$ to a map to the cone point.  Therefore the LES breaks up into short exact sequences
\[0\to H_n(Y) \xrightarrow{~j_*~} H_n(Y/(X\x\{1\})) \xrightarrow{~\bd~} H_{n-1}(X\x\{1\})\to 0.\]
Replacing these spaces with their homotopy equivalents, we can rewrite this as
\begin{equation} \label{suspension-SES}
\boxed{0 \to H_n(X^{[k-1]}) \xrightarrow{~j_*~} H_n(X^{[k]}) \xrightarrow{~\bd~} H_{n-1}(X)\to 0.}
\end{equation}

Note: There are other way to obtain \eqref{suspension-SES} is to consider the good pair $(X^{[k]},X^{[k-1]})$.  The quotient space $X^{[k]}/X^{[k-1]}$ can be identified with $SX$, and the long exact sequence for the pair, together with the isomorphism $\HH_n(SX)\isom \HH_{n-1}(X)$, consists of the sequences \eqref{suspension-SES}, spliced together (although one still has to argue that the connecting homomorphism is zero).  One can also use Corollary 2.24, or a Mayer-Vietoris sequence (which you will learn soon).

\textbf{Obtaining \eqref{suspension-SES} was as far as you had to get to obtain full credit.}  To complete the problem, we show that the sequence \emph{splits}, i.e., that $H_n(X^{[k]}) \isom H_n(X^{[k-1]}) \oplus H_{n-1}(X)$.  By the Splitting Lemma (Hatcher, p.147), we can do this by constructing a homomorphism
\[p: H_n(X^{[k]})\to H_n(X^{[k-1]})=:Q\]
such that $p\circ j_*=\IdMap_Q$.

Regard $X^{[k]}$ as the union of $k$ copies $Z_1,\dots,Z_k$ of the cone $CX$, identified along their bases $X_i\x\{0\}$, and let $U_i$ be an open deformation of $Z_i$ in $X^{[k]}$: e.g., $U_i=Z_i\cup\bigcup_{j\neq i}X_j\x[0,1/2)$.  Note that $U_2\cup\cdots\cup U_k$ deformation-retracts to $X^{[k-1]}$.  Let $\mathcal{U}=\{U_1,\dots,U_k\}$ and $\mathcal{U}'=\{U_2,\dots,U_k\}$.
By the Subdivision Lemma (Hatcher, Prop.~2.21, p.119) we have
\[C^{\mathcal{U}}_\bullet\htop C_\bullet(X^{[k]}) \qqandqq
C^{\mathcal{U}'}_\bullet\htop C_\bullet(X^{[k-1]}). \]
An element of $C^{\mathcal{U}}_n$ is a chain of the form $\tau=\sum_{i=1}^n\sigma_i$, where $\sigma_i\in C_n(U_i)$.  The obvious homeomorphism $U_1\to U_2$ induces an isomorphism $\phi:C_n(U_1)\to C_n(U_2)$, so replacing $\sigma_1$ with $\phi(\sigma_1)$ gives a map $C^{\mathcal{U}}_n\to C^{\mathcal{U}'}_n$, which induces a map on homology:
\[p:H_n(X^{[k]})\to H_n(X^{[k-1]}).\]
On the other hand, the image of $j_*$ consists of chains of the form $\tau$ with $\sigma_1=0$, so it follows that $p\circ j_*$ is the identity on $H_n(X^{[k-1]})$ as desired.  This completes the proof that the short exact sequence \eqref{suspension-SES} splits.  Therefore
\[H_n(X^{[k]}) \isom H_n(X^{[k-1]}) \oplus H_{n-1}(X)\]
and by induction we now know the homology groups of a generalized suspension:

\boldmath\[\boxed{H_n(X^{[k]}) \isom H_{n-1}(X)^{\oplus(k-1)}.} \eqno(*)\]\unboldmath

{\bf A better solution, submitted by Nick and others:}  Consider the effect of starting with the space $X^{[k]}$, regarded as the union of $k$ copies $Z_1,\dots,Z_k$ of the cone $CX$ identified along their bases, and then contracting $Z_k$.  This is a deformation retraction since $Z_k$ is contractible.  On the other hand, each $Z_i$ ($i<k$) turns into a copy of $SX$, and the $Z_i$'s are wedged together at the point coming from the base.  Therefore
\[X^{[k]} \htop (SX)^{\wedge(k-1)}.\]
Using the first part of the problem, together with there fact that  wedge sum is additive on reduced homology, we get the formula $(*)$
%\[\HH_n(X^{[k]}) \isom]\HH_n((SX)^{\wedge(k-1)}) = \HH_n(SX)^{\oplus(k-1)} =  \HH_{n-1}(X)^{\oplus(k-1)}\]
much more easily.

\yellprob{Let $n\leq d\geq 0$ and let $X=\Delta^{n,d}$ denote the $d$-skeleton of the $n$-dimensional simplex (whose vertices are $v_0,v_1,\dots,v_n$).  Most of you conjectured last time that the reduced homology groups of $X$ are given by
\[\HH_k(X) = \begin{cases}
\Zz^{\binom{n}{d}} & \text{ if $k=d$,}\\
0 & \text{ if $k<d$}.
\end{cases}\]
This conjecture is correct.  Prove it without writing down any explicit simplicial boundary matrices.}

\soln If $k<d$, then $\HH_k(X)=\HH_k(\Delta^n)=0$ because $\Delta^n$ is contractible, hence acyclic.
Otherwise, consider the subcomplex
\[\Gamma=\langle\sigma\in X\st 0\in\sigma.\]
This complex is contractible, because it is the cone with apex $v_0$ and base
\[\Lambda=\left\{\sigma\in X\st 0\not\in\sigma,\ \sigma\cup\{v_0\}\in X\right\}.\]
(The complex $\Gamma$ is known as the \emph{star} of $v_0$, and $\Lambda$ is its \emph{link}.)
Since $\Gamma$ is contractible, we have $X\htop X/\Gamma$.  But this latter space is a CW-complex with a single vertex and $\binom{n}{d+1}$ cells of dimension $d$ (corresponding to the simplices supported on $d+1$ of the vertices $v_1,\dots,v_n)$.  That is, $X/\Gamma$ is the wedge of $\binom{n}{d+1}$ copies of $\Ss^n$, hence has the desired homology.

\end{document}

\vfill
\pagebreak

\textbf{\Large Some LaTeX tips}

\textbf{1. Matrices with borders}

The $\backslash$\texttt{bordermatrix} command
can be used for matrices whose columns and rows
you want to label.  This can be useful for bookkeeping
in a simplicial homology calculation.
For example, the boundary map $\bd_2$ of the standard 3-simplex
is
$$
\bordermatrix{
   & 123 & 124 & 134 & 234 \cr
12 &  1  &  1  &  0  &  0\cr
13 & -1  &  0  &  1  &  0\cr
14 &  0  & -1  & -1  &  0\cr
23 &  1  &  0  &  0  &  1\cr
24 &  0  &  1  &  0  & -1\cr
34 &  0  &  0  &  1  &  1
}
$$
which can be produced as follows:
\begin{verbatim}
$$\bordermatrix{
   & 123 & 124 & 134 & 234 \cr
12 &  1  &  1  &  0  &  0\cr
13 & -1  &  0  &  1  &  0\cr
14 &  0  & -1  & -1  &  0\cr
23 &  1  &  0  &  0  &  1\cr
24 &  0  &  1  &  0  & -1\cr
34 &  0  &  0  &  1  &  1}$$
\end{verbatim}

\textbf{2. Commutative diagrams}

The \texttt{xypic} package provides a way to typeset commutative diagrams
in LaTeX.  For instance, consider the following
diagram, which arises in the proof of Theorem 2.10 in Hatcher:
$$
\xymatrix{
\cdots\ar[r] & C_{n+1}(X) \ar[r]^{\bd}\ar[d]^{i_\#} & C_{n}(X) \ar[r]^{\bd} \ar[d]^{i_\#} \ar[dl]^{P} & C_{n-1}(X) \ar[r] \ar[d]^{i_\#} \ar[dl]^{P} &\cdots\\
\cdots\ar[r] & C_{n+1}(Y) \ar[r]_{\bd} & C_{n}(Y) \ar[r]_{\bd} & C_{n-1}(Y)\ar[r] &\cdots\\
}$$
It can be typeset as follows:
\begin{verbatim}
$$\xymatrix{
\cdots\ar[r]
  & C_{n+1}(X) \ar[r]^{\bd} \ar[d]^{i_\#}
  & C_{n}(X)   \ar[r]^{\bd} \ar[d]^{i_\#} \ar[dl]^{P}
  & C_{n-1}(X) \ar[r]       \ar[d]^{i_\#} \ar[dl]^{P}
  &\cdots\\
\cdots\ar[r]
  & C_{n+1}(Y) \ar[r]_{\bd}
  & C_{n}(Y)   \ar[r]_{\bd}
  & C_{n-1}(Y) \ar[r]
  &\cdots}$$
\end{verbatim}
This is like a \texttt{tabular} or \texttt{array} environment:
the \& symbols are delimiters between columns.  The $\backslash$\texttt{ar}
commands create arrows emanating from the current cell in the table, with
the code in [square brackets] specifying where the arrow should point;
e.g., \texttt{$\backslash$ar[dl]} makes an arrow pointing towards the
cell one row down and one column left of the current cell.

\end{document}

