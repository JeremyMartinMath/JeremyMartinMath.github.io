\documentclass{amsart}
\usepackage{amssymb,amsmath,amsthm,mathrsfs,graphics,hyperref,ifthen,framed,cancel,fullpage,color,ytableau,tcolorbox,bm,tikz}
\usepackage[enableskew]{youngtab}
\raggedbottom
\parskip=10bp
\parindent=0bp
\raggedbottom

%%%%%%%%%%%%%%%%   Colors for TikZ and text  %%%%%%%%%%%%%%%%%

\definecolor{light}{gray}{.75}
\definecolor{med}{gray}{.5}
\definecolor{dark}{gray}{.25}
\newcommand{\Red}[1]{{\color{red}{#1}}}
\newcommand{\RED}[1]{{\color{red}{\boldmath\textbf{#1}\unboldmath}}}
\newcommand{\Blue}[1]{{\color{blue}{#1}}}
\newcommand{\BLUE}[1]{{\color{blue}{\boldmath\textbf{#1}\unboldmath}}}

% Hyperlinks
\hypersetup{colorlinks, citecolor=red, filecolor=black, linkcolor=blue, urlcolor=blue}
\newcommand{\hreftext}[2]{\href{#1}{\Blue{#2}}} % for text anchors
\newcommand{\hrefurl}[2]{\href{#1}{\Blue{\tt #2}}} % if you want to actually write out the URL in text

% Spacing and commands
\newcommand{\bigpad}{\rule[-14mm]{0mm}{30mm}}
\newcommand{\smallpad}{\rule[-1.5mm]{0mm}{5mm}}
\newcommand{\pad}{\rule[-3mm]{0mm}{8mm}}
\newcommand{\padup}{\rule{0mm}{5mm}}
\newcommand{\paddown}{\rule[-3mm]{0mm}{2mm}}
\newcommand{\blank}{\rule{1.25in}{0.25mm}}
\newcommand{\yell}[1]{\fbox{\rule[-1mm]{0mm}{4mm} \large\bf #1 }}
\newcommand{\indnt}{\phantom{.}\qquad}
\newcommand{\littleline}{\begin{center}\rule{4in}{0.5bp}\end{center}}

% macros for inserting figures
\newcommand{\includefigure}[3]{\begin{center}\resizebox{#1}{#2}{\includegraphics{#3}}\end{center}}
\newcommand{\includefigurewithinmath}[3]{\resizebox{#1}{#2}{\includegraphics{#3}}}
% E.g., to insert a standalone figure of width 3" and height 1.5", use
% \includefigure{3in}{1.5in}{foo.pdf}

% math operators
\DeclareMathOperator{\Comp}{Comp}
\DeclareMathOperator{\Fix}{Fix}
\DeclareMathOperator{\id}{id}
\DeclareMathOperator{\im}{im}
\DeclareMathOperator{\lcm}{lcm}
\DeclareMathOperator{\Par}{Par}
\DeclareMathOperator{\rank}{rank}
\DeclareMathOperator{\sh}{sh}
\DeclareMathOperator{\tr}{tr}
\DeclareMathOperator{\wt}{wt}

% theorem environments, automatically numbered
\newtheorem{theorem}{Theorem}[section]
\newtheorem{proposition}[theorem]{Proposition}
\newtheorem{lemma}[theorem]{Lemma}
\newtheorem{corollary}[theorem]{Corollary}
\theoremstyle{definition}
\newtheorem{definition}[theorem]{Definition}
\newtheorem{example}[theorem]{Example}
\newtheorem{remark}[theorem]{Remark}
\newtheorem{problem}[theorem]{Problem}

\newcommand{\skpr}{\emph{Sketch of proof: }}
\newcommand{\soln}{\textit{Solution:\ }}

% Generally useful macros
\newcommand{\excise}[1]{} % useful for commenting out large chunks
\newcommand{\0}{\emptyset}
\newcommand{\compn}{\models} % compositions
\newcommand{\dju}{\mathaccent\cdot\cup} % disjoint union
\newcommand{\dsum}{\displaystyle\sum}
\newcommand{\fallfac}[2]{{#1}^{\underline{#2}}} % falling factorial
\newcommand{\isom}{\cong} % isomorphism symbol
\newcommand{\partn}{\vdash} % partition symbol
\newcommand{\qqandqq}{\qquad\text{and}\qquad}
\newcommand{\qandq}{\quad\text{and}\quad}
\newcommand{\qand}{\quad\text{and}}
\newcommand{\qbin}[2]{{\begin{bmatrix}#1\\#2\end{bmatrix}_q}} % q-binomial coefficient
\newcommand{\risefac}[2]{{#1}^{\overline{#2}}} % rising factorial
\newcommand{\sd}{\triangle} % symmetric difference
\newcommand{\sm}{\setminus} % don't use a minus sign for this
\newcommand{\st}{\colon} % "such that"
\newcommand{\surj}{\twoheadrightarrow}
\newcommand{\x}{\times}

% blackboard bold fonts for sets of numbers
\newcommand{\Cc}{\mathbb{C}} % complex numbers
\newcommand{\Ff}{\mathbb{F}} % finite field
\newcommand{\Nn}{\mathbb{N}} % natural numbers
\newcommand{\Qq}{\mathbb{Q}}
\newcommand{\Rr}{\mathbb{R}}
\newcommand{\Zz}{\mathbb{Z}}

% miscellaneous
\newcommand{\TwoCases}[4]{\begin{cases}{#1}&\text{ #2}\\{#3}&\text{ #4}\end{cases}}
\newcommand{\ThreeCases}[6]{\begin{cases}{#1}&\text{ #2}\\{#3}&\text{ #4}\\{#5}&\text{ #6}\end{cases}}
\newcommand{\bridgehand}[4]{\spadesuit\ {\textsf{#1}}\ \ \heartsuit\ {\textsf{#2}}\ \ \diamondsuit\ {\textsf{#3}}\ \ \clubsuit\ {\textsf{#4}}}

% macros for automatic problem numbering --- students don't have to use these
\newcounter{probno}
\setcounter{probno}{0}
\newcounter{partno}
\setcounter{partno}{0}
%% versions that don't print the number of points
\newcommand{\prob}{
  \vskip10bp%
  \setcounter{partno}{0}%
  \addtocounter{probno}{1}%
  {\bf Problem~\#{\arabic{probno}}}\quad}

% No initial whitespace to make framing look nicer
\newcommand{\probns}{
  \setcounter{partno}{0}%
  \addtocounter{probno}{1}%
  {\bf Problem~\#{\arabic{probno}}}\quad}

\newcommand{\probpart}{%
  \addtocounter{partno}{1}%
  {\bf (\#\arabic{probno}\alph{partno})}\ \ }
\newcommand{\probcont}{%
  {\bf Problem~\#{\arabic{probno}}}~(\emph{continued})}
\newcommand{\probo}{
  \setcounter{partno}{0}%
  \addtocounter{probno}{1}%
  {\bf (\#\arabic{probno}}\ \ }

%% versions that do print the number of points
\newcommand{\Prob}[1]{
  \vskip10bp%
  \setcounter{partno}{0}%
  \addtocounter{probno}{1}%
  {\bf Problem~\#{\arabic{probno}}~[{#1}~pts]}\quad}

%no initial whitespace
\newcommand{\Probns}[1]{
  \setcounter{partno}{0}%
  \addtocounter{probno}{1}%
  {\bf Problem~\#{\arabic{probno}}~[{#1}~pts]}\quad}
\newcommand{\Probpart}[1]{%\rule{0in}{0in}\\ \phantom{xxx}
  \addtocounter{partno}{1}%
  {\bf (\#\arabic{probno}\alph{partno})~[{#1}~pts]}\ \ }

\newboolean{answers}
\newcommand{\Answer}[1]{\ifthenelse{\boolean{answers}}{{\bf Answer:}\ #1}{}\bigskip}

\usepackage{youngtab}
\begin{document}
\thispagestyle{empty}
{\bf Math 821 Problem Set \#4\\
Problem Set \#4\\
Due date: Friday, March 14}
\bigskip


\prob The \emph{dunce hat} is the space $D$ obtained from a triangle
by identifying all three edges with each other, with the orientations
indicated below.

(a) Prove that $D$ is simply-connected using Van Kampen's theorem.\\

(b) Find a different, one-line proof that $D$ is simply-connected.

\includefigure{1.5in}{1.5in}{dunce}

\soln

(a) Decompose $D$ into two pieces $A,B$ as follows: $A$ is the
interior of the 2-cell, and $B=D\sm\{p\}$, where $p\in A$.

Then:
\begin{itemize}
\item $A$ is an open disk, hence contractible.
\item $B$ deformation-retracts onto, hence is homotopy-equivalent
to, the boundary triangle, which is just a circle (the
edge $a$ becomes one loop around the circle).
\item $A\cap B$ is an (open) annulus, whose fundamental group is
generated by a path $\gamma$ winding once around $p$.
Note that $\gamma\htop aaa^{-1}$ in $B$.
\end{itemize}
Now,
since $A$ is contractible, Van Kampen's Theorem says that
  $$\pi_1(D) = \pi_1(B) / i_*\pi_1(A\cap B)$$
and
  $$i_*\gamma=aaa^{-1}=a$$
so this quotient is in fact trivial.

(b) By our theorems on 2-dimensional cell complexes, $\pi_1(D)=\langle g \st ggg^{-1}\rangle=\langle g|g\rangle=0$.

\vfill\hrule\vfill\pagebreak

\prob Consider the standard picture of the torus $T=S^1\x S^1$
as a quotient space of the square.  Explain what is wrong with the following ``proof''
(whose conclusion is certainly false):

\begin{quotation}
Consider the open cover $A_\alpha\cup A_\beta\cup A_\gamma$ shown below.
Each one is path-connected and simply-connected, and their intersection is path-connected.
Therefore, by Van Kampen's theorem, the torus is simply-connected.
\end{quotation}

\soln It is true that the sets $A_\alpha,A_\beta,A_\gamma$ are all simply-connected.
However, the intersection $A_\alpha\cap A_\beta\cap A_\gamma$ is not path-connected.
The picture is misleading (which was the idea of the problem); it actually must look something like this,
and the two yellow splotches denote different components of $A_\alpha\cap A_\beta\cap A_\gamma$.

\includefigure{4in}{3in}{torus-decomp2} % for solution

\vfill\hrule\vfill

\prob (Hatcher, p.53, \#4, modified) Let $n\geq 1$ be an integer, and 
let $X\subset\Rr^3$ be the union of $n$ distinct rays emanating from the 
origin.  Compute $\pi_1(\Rr^3\sm X)$.

\soln The map $f_t(\xx)=(1-t)\xx+t\frac{\xx}{\|\xx\|}$ gives a deformation retraction from
$\Rr^3\sm X$ to the unit sphere minus $n$ points.  We can regard the
deleting the first point as giving a copy of $\Rr^2$, so we now have $\Rr^2$ minus $n-1$ points.
This space deformation-retracts to the wedge of $n-1$ squares ($\isom (S^1)^{\vee(n-1)}$), as shown below.
Therefore $\pi_1(X)\isom\pi_1((S^1)^{\vee(n-1)})$
is free on $n-1$ generators.
\includefigure{6in}{2in}{rayscomplement} 


\prob Let $a_1,\dots,a_n$ be nonzero integers.  
Construct a cell complex $X$ from $S^1$ as follows:
For each $j=1,\dots,n$, attach a $2$-cell to $S^1$ by wrapping
it around the circle $a_j$ times.  Compute $\pi_1(X)$.

\soln The theorem on 2-dimensional cell complexes
(from pp.50--52 of Hatcher) implies that
\[\pi_1(X) ~=~ \langle g ~|~ g^{a_1},g^{a_2},\dots,g^{a_n}\rangle = \langle g ~|~ g^k\rangle ~=~ \Zz_k\]
where $k=\gcd(a_1,\dots,a_k)$.

\vfill\hrule\vfill

\prob (Hatcher, p.53, \#6, modified) Let $X$ be a path-connected cell 
complex, and let $Y$ be a cell complex obtained from $X$ by attaching an 
$n$-cell for some $n\geq 3$.  Show that the inclusion $X\inj Y$ induces 
an isomorphism $\pi_1(X)\isom\pi_1(Y)$.

\soln The proof of Prop.~1.26 goes through, changing $e_\alpha^2$ to
$e_\alpha^n$.  At the very end, we have that $A_\alpha$
deformation-retracts onto a circle in $e^n_\alpha\sm\{y_\alpha\}$,
i.e., an $n$-ball minus a point.  But such a thing is simply-connected
(as we know, it is homotopy-equivalent to $S^{n-1}$) and therefore
$\pi_1(A_\alpha)=0$, and the group $N$ in the statement of the proposition
is trivial.

Another argument uses Van Kampen's theorem.
Let $e$ be the $n$-cell that gets attached
(so $e\isom D^n$) and let $f:\bd e=S^{n-1}\to X$ be the attaching map.
Write $Y=X\cup Z$ where $Z$ is obtained by fattening $\bd e$ slightly
into an open set that contains,
and deformation-retracts onto, it.  (This is a mapping cylinder neighborhood
in the sense of Example 0.15.)  Then $Z$ is contractible,
hence simply-connected
(since it deformation-retracts onto an $n$-ball)
and $X\cap Z$ is  simply-connected (because it deformation-retracts onto
the simply-connected $(n-1)$-sphere $\bd e$).  Now applying
Van Kampen's
theorem to the decomposition $Y=X\cup Z$
gives a surjection $\pi_1(X)\to\pi_1(Y)$ whose kernel is zero.\\

\vfill\hrule\vfill


\prob (Hatcher p.79, \#2) Show that if $p_1:\tilde X_1\to X_1$ and $p_2:\tilde X_2\to X_2$ are covering spaces, then so is their product $p_1\x p_2:\tilde X_1\x\tilde X_2\to X_1\x X_2$.

\soln For $i=1,2$, let $\{U^i_\alpha\}$ be an open cover of $X_i$ that is ``good'', i.e., every component of $p_i^{-1}(U^i_\alpha)$ is mapped homeomorphically to $U^i_\alpha$ by $p_i$.  The components $V$ of $(p_1\x p_2)^{-1}(U^1_\alpha\x U^2_\alpha)$ are just the products $V_1\x V_2$, where $V_i$ is a component of $p_i^{-1}(U^i_\alpha)$, and since $p_i|_{U_i}$ are homeomorphisms, so is $(p_1\x p_2)|_V$.  So $\{U^1_\alpha\x U^2_\alpha\}$ is a good cover of $X_1\x X_2$.  By the way, this argument implies that the number of sheets of a covering space is multiplicative under direct product.\\

\vfill\hrule\vfill

\prob (Hatcher p.80, \#12) Let $a$ and $b$ be the generators of $\pi_1(S^1\vee S^1,x_0)$ corresponding to the two copies of $S^1$, with $x_0$ their common point.  Draw a picture of the covering space $\tilde X$ of $S^1\vee S^1$ corresponding to the normal subgroup of $\pi_1(S^1\vee S^1)$ generated by $a^2$, $b^2$, and $(ab)^4$, and prove that this covering space is indeed the correct one.  (I.e., this group should be $p_*\pi_1(\tilde X,\tilde x_0)$.) 

\soln The space $\tilde X$ is a necklace of eight circles:

\includefigure{3.6in}{1.8in}{hatcher-p80n12}

The number of sheets of the covering is 8 (the cardinality of the preimage of any point in $X$, for example $x_0$).  If we take
Take $\tilde x_0\in \tilde X_0$ to be the highlighted point.  Then the colored loops in $\tilde X_0$ given by
\begin{enumerate}
\item the blue circle containing $\tilde x_0$, 
\item the red circle containing $\tilde x_0$, and
\item walking all the way around the ``outer arcs'' of the necklace
\end{enumerate}
map via $p_*$ to the loops $a^2$, $b^2$ and $(ab)^4$ in $G=\pi_1(X)$.

{\bf Claim 1:} $p_*\pi_1(\tilde X)$ is normal in $\pi_1(X)$.  This is true by Prop.~1.39 in Hatcher, but we didn't get to this theorem until Friday 3/14, so here's a more elementary proof.  By the symmetry of $\tilde X$, the same colored loops are available at every basepoint, so 
the group $p_*\pi_1(\tilde X)$ is independent of the choice of basepoint.  Given $g\in\pi_1(X)$ and $h\in p_*\pi_1(\tilde X)$,  lift $g$ to a path $\tilde g$ in $\tilde X$ from $\tilde x_0$ to $\tilde x_1$ and lift $h$ to a loop $\tilde h$ at $\tilde x_1$.  Then $\tilde g\cdot\tilde h\cdot\ov{\tilde h}$ is a loop at $\tilde x_0$ that maps via $p_*$ to $gh\bar g$.  It follows that $p_*\pi_1(\tilde X)$ is closed under conjugation in $G$, proving Claim~1.  Moreover,
\[H:=\Langle a^2,b^2,(ab)^4\Rangle \ \ \subseteq \ \ p_*\pi_1(\tilde X) \ \ \trianglelefteq \ \ G\]
(recall that the notation $\Langle\dots\Rangle$ means ``normal subgroup generated by'') and therefore
\[G/H = \langle a,b ~|~ a^2,b^2,(ab)^4\rangle.\]

{\bf Claim 2:} The set
\[C := \{e,\ a,\ b,\ ab,\ ba,\ aba,\ bab,\ abab\} \ \subset\ G\]
contains a set of coset representatives for $G/H$.
Indeed, any word with two consecutive instances of the same letter can be replaced with a shorter word that is equivalent modulo $H$ (i.e., in the same coset of $H$), so we can pick a set of coset representatives consisting of words alternating between $a$'s and $b$'s (cf.\ the discussion of $\Zz_2*\Zz_2$ on p.42 of Hatcher).  The defining relations of $G/H$ say that $abab$ is its own inverse; on the other hand $(abab)(baba)$ reduces to the empty word, so $(abab)^{-1}=abab=baba$.  The relations also say that $ababa=bab$ and $babab=aba$, so
any alternating word of length 5 or more can be reduced modulo $H$ to an element of $C$.  This proves Claim~2.

We have shown that
\begin{align*}
[G:p_*\pi_1(\tilde X)] &\leq [G:H] && \text{(because $H\subseteq p_*\pi_1(\tilde X)$)}\\
&\leq 8 && \text{(by the construction of coset representatives)}\\
&= [G:p_*\pi_1(\tilde X)] && \text{(since $\tilde X$ is an 8-sheeted covering).}
\end{align*}
Therefore, equality must hold throughout, and it follows that $p_*(\pi_1\tilde X)=H$ as desired,  By the way, $G/H$ is isomorphic to the dihedral group $D_4$ --- which is easily seen to be the set of deck transformations, i.e., the symmetries of the colored octagon.

\end{document}
