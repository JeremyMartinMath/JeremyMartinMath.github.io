\documentclass{amsart}
\usepackage{amssymb,amsmath,amsthm,mathrsfs,graphics,hyperref,ifthen,framed,cancel,fullpage,color,ytableau,tcolorbox,bm,tikz}
\usepackage[enableskew]{youngtab}
\raggedbottom
\parskip=10bp
\parindent=0bp
\raggedbottom

%%%%%%%%%%%%%%%%   Colors for TikZ and text  %%%%%%%%%%%%%%%%%

\definecolor{light}{gray}{.75}
\definecolor{med}{gray}{.5}
\definecolor{dark}{gray}{.25}
\newcommand{\Red}[1]{{\color{red}{#1}}}
\newcommand{\RED}[1]{{\color{red}{\boldmath\textbf{#1}\unboldmath}}}
\newcommand{\Blue}[1]{{\color{blue}{#1}}}
\newcommand{\BLUE}[1]{{\color{blue}{\boldmath\textbf{#1}\unboldmath}}}

% Hyperlinks
\hypersetup{colorlinks, citecolor=red, filecolor=black, linkcolor=blue, urlcolor=blue}
\newcommand{\hreftext}[2]{\href{#1}{\Blue{#2}}} % for text anchors
\newcommand{\hrefurl}[2]{\href{#1}{\Blue{\tt #2}}} % if you want to actually write out the URL in text

% Spacing and commands
\newcommand{\bigpad}{\rule[-14mm]{0mm}{30mm}}
\newcommand{\smallpad}{\rule[-1.5mm]{0mm}{5mm}}
\newcommand{\pad}{\rule[-3mm]{0mm}{8mm}}
\newcommand{\padup}{\rule{0mm}{5mm}}
\newcommand{\paddown}{\rule[-3mm]{0mm}{2mm}}
\newcommand{\blank}{\rule{1.25in}{0.25mm}}
\newcommand{\yell}[1]{\fbox{\rule[-1mm]{0mm}{4mm} \large\bf #1 }}
\newcommand{\indnt}{\phantom{.}\qquad}
\newcommand{\littleline}{\begin{center}\rule{4in}{0.5bp}\end{center}}

% macros for inserting figures
\newcommand{\includefigure}[3]{\begin{center}\resizebox{#1}{#2}{\includegraphics{#3}}\end{center}}
\newcommand{\includefigurewithinmath}[3]{\resizebox{#1}{#2}{\includegraphics{#3}}}
% E.g., to insert a standalone figure of width 3" and height 1.5", use
% \includefigure{3in}{1.5in}{foo.pdf}

% math operators
\DeclareMathOperator{\Comp}{Comp}
\DeclareMathOperator{\Fix}{Fix}
\DeclareMathOperator{\id}{id}
\DeclareMathOperator{\im}{im}
\DeclareMathOperator{\lcm}{lcm}
\DeclareMathOperator{\Par}{Par}
\DeclareMathOperator{\rank}{rank}
\DeclareMathOperator{\sh}{sh}
\DeclareMathOperator{\tr}{tr}
\DeclareMathOperator{\wt}{wt}

% theorem environments, automatically numbered
\newtheorem{theorem}{Theorem}[section]
\newtheorem{proposition}[theorem]{Proposition}
\newtheorem{lemma}[theorem]{Lemma}
\newtheorem{corollary}[theorem]{Corollary}
\theoremstyle{definition}
\newtheorem{definition}[theorem]{Definition}
\newtheorem{example}[theorem]{Example}
\newtheorem{remark}[theorem]{Remark}
\newtheorem{problem}[theorem]{Problem}

\newcommand{\skpr}{\emph{Sketch of proof: }}
\newcommand{\soln}{\textit{Solution:\ }}

% Generally useful macros
\newcommand{\excise}[1]{} % useful for commenting out large chunks
\newcommand{\0}{\emptyset}
\newcommand{\compn}{\models} % compositions
\newcommand{\dju}{\mathaccent\cdot\cup} % disjoint union
\newcommand{\dsum}{\displaystyle\sum}
\newcommand{\fallfac}[2]{{#1}^{\underline{#2}}} % falling factorial
\newcommand{\isom}{\cong} % isomorphism symbol
\newcommand{\partn}{\vdash} % partition symbol
\newcommand{\qqandqq}{\qquad\text{and}\qquad}
\newcommand{\qandq}{\quad\text{and}\quad}
\newcommand{\qand}{\quad\text{and}}
\newcommand{\qbin}[2]{{\begin{bmatrix}#1\\#2\end{bmatrix}_q}} % q-binomial coefficient
\newcommand{\risefac}[2]{{#1}^{\overline{#2}}} % rising factorial
\newcommand{\sd}{\triangle} % symmetric difference
\newcommand{\sm}{\setminus} % don't use a minus sign for this
\newcommand{\st}{\colon} % "such that"
\newcommand{\surj}{\twoheadrightarrow}
\newcommand{\x}{\times}

% blackboard bold fonts for sets of numbers
\newcommand{\Cc}{\mathbb{C}} % complex numbers
\newcommand{\Ff}{\mathbb{F}} % finite field
\newcommand{\Nn}{\mathbb{N}} % natural numbers
\newcommand{\Qq}{\mathbb{Q}}
\newcommand{\Rr}{\mathbb{R}}
\newcommand{\Zz}{\mathbb{Z}}

% miscellaneous
\newcommand{\TwoCases}[4]{\begin{cases}{#1}&\text{ #2}\\{#3}&\text{ #4}\end{cases}}
\newcommand{\ThreeCases}[6]{\begin{cases}{#1}&\text{ #2}\\{#3}&\text{ #4}\\{#5}&\text{ #6}\end{cases}}
\newcommand{\bridgehand}[4]{\spadesuit\ {\textsf{#1}}\ \ \heartsuit\ {\textsf{#2}}\ \ \diamondsuit\ {\textsf{#3}}\ \ \clubsuit\ {\textsf{#4}}}

% macros for automatic problem numbering --- students don't have to use these
\newcounter{probno}
\setcounter{probno}{0}
\newcounter{partno}
\setcounter{partno}{0}
%% versions that don't print the number of points
\newcommand{\prob}{
  \vskip10bp%
  \setcounter{partno}{0}%
  \addtocounter{probno}{1}%
  {\bf Problem~\#{\arabic{probno}}}\quad}

% No initial whitespace to make framing look nicer
\newcommand{\probns}{
  \setcounter{partno}{0}%
  \addtocounter{probno}{1}%
  {\bf Problem~\#{\arabic{probno}}}\quad}

\newcommand{\probpart}{%
  \addtocounter{partno}{1}%
  {\bf (\#\arabic{probno}\alph{partno})}\ \ }
\newcommand{\probcont}{%
  {\bf Problem~\#{\arabic{probno}}}~(\emph{continued})}
\newcommand{\probo}{
  \setcounter{partno}{0}%
  \addtocounter{probno}{1}%
  {\bf (\#\arabic{probno}}\ \ }

%% versions that do print the number of points
\newcommand{\Prob}[1]{
  \vskip10bp%
  \setcounter{partno}{0}%
  \addtocounter{probno}{1}%
  {\bf Problem~\#{\arabic{probno}}~[{#1}~pts]}\quad}

%no initial whitespace
\newcommand{\Probns}[1]{
  \setcounter{partno}{0}%
  \addtocounter{probno}{1}%
  {\bf Problem~\#{\arabic{probno}}~[{#1}~pts]}\quad}
\newcommand{\Probpart}[1]{%\rule{0in}{0in}\\ \phantom{xxx}
  \addtocounter{partno}{1}%
  {\bf (\#\arabic{probno}\alph{partno})~[{#1}~pts]}\ \ }

\newboolean{answers}
\newcommand{\Answer}[1]{\ifthenelse{\boolean{answers}}{{\bf Answer:}\ #1}{}\bigskip}

\usepackage{youngtab}
\begin{document}
\thispagestyle{empty}
{\bf Math 821 Problem Set \#3\\
Posted: Friday 2/25/11\\
Due date: Monday 3/7/11}

\yellprob{Recall that for a space $X$ and base point $p\in X$, we have defined $\pi_1(X,p)$ to be
the set of homotopy classes of $p,p$-paths on $X$ --- or equivalently of continuous functions
$S^1\to X$.  Recall also that $S^0$ consists of two points (let's call them $a$ and $b$) with the discrete topology.
Accordingly, we could define $\pi_0(X,p)$ to be the set of homotopy classes of continuous functions $f:S^0\to X$
such that $f(a)=p$.

Describe the set $\pi_0(X,p)$ intrinsically in terms of $X$.  Is there a natural way to endow it with a group structure?}

\soln The homotopy type of such a thing is determined by the path-connected
component of~$X$ containing $f(b)$.  Therefore, a reasonable interpretation
for $\pi_0(X,x)$ is as the set of connected components.  This set cannot naturally be made into a group.

\yellprob{(Hatcher, p.38, \#2) Show that the change-of-basepoint homomorphism $\beta_h$ (see p.28)
depends only on the homotopy class of the path $h$.}

\soln Recall the setup: $x_0,x_1\in X$; $h$ is a $x_0,x_1$-path in $X$; and $\beta_h$ is the map
$\pi_1(X,x_1)\to\pi_1(X,x_0)$ given by $[f]\mapsto[h\cdot f\cdot\bar h]$.

Suppose that $h_t$ is a homotopy of $x_0,x_1$-paths.  Then $h_t\cdot f\cdot \overline{h_t}$
is a homotopy of $x_0$-loops.  In particular, if $h\htop h'$, then $\beta_h[f]\htop\beta_{h'}[f]$.

\yellprob{(Hatcher, p.38, \#7) Define $f:S^1\x I\to S^1\x I$ by $f(\theta,s)=(\theta+2\pi s,s)$,
so $f$ restricts to the identity on the two boundary circles of $S^1\x I$.   Show that $f$ is homotopic
to the identity by a homotopy $f_t$ that is stationary on \emph{one} of the boundary circles, but not
by any homotopy $f_t$ that is stationary on \emph{both} boundary circles.}

\soln Visualize $S^1\x I$ as a cylinder made of rubber, and
$f$ as a full twist of the cylinder.  (Imagine opening a jar full of extremely old rubber cement.)

(i) Define $f_t:S^1\x I\to S^1\x I$ by
  $$f_t(\theta,s) = (\theta+2\pi ts,s).$$
This is evidently a homotopy (it is continuous in each of
$\theta,s,t$); $f_0$ is the identity map; and $f_1$ is the given map
$f$.  Moreover, $f_t$ is stationary on the circle $S^1\x\{0\}$, i.e.,
$f_t(\theta,0)=(\theta_0)$.

(ii) Suppose that $f_t$ is a homotopy that is stationary on both
boundary circles.  That is, $f_t:S^1\x I\to S^1\x I$ with
\begin{align*}
f_0(\theta,s) &= (\theta,s),         & f_t(\theta,0) = (\theta,0),\\
f_1(\theta,s) &= (\theta+2\pi s, s), & f_t(\theta,1) = (\theta,1).
\end{align*}

We want to derive a contradiction. The idea is to draw a line down
the side of the cylinder, so that
twisting by $f$ wraps the line around the outside in a spiral.
Projecting
these two paths from $S^1\x I$ to $S^1$ will give two closed loops in $S^1$,
one trivial and one that winds once around the circle --- so they cannot
be homotopic.

Here is a precise argument.
Fix some basepoint $\theta_0\in S^1$.  Let $g_t$ be the loop at
$\theta_0$ obtained from $f_t$ by restricting its domain to
$\{\theta_0\}\x I$, then projecting onto the $S^1$ factor.  That is,
  $$g_t(s)=p(f_t(\theta_0,s))$$
where $p$ is the projection map $S^1\x I\to S^1$.  I claim that
$\{g_t\}$ is a path homotopy.  It certainly is a continuously varying
family of functions $I\to S^1$, and
\begin{align*}
g_t(0) &= p(f_t(\theta_0,0)) = (\theta_0,0) = \theta_0,\\
g_t(1) &= p(f_t(\theta_0,1)) = (\theta_0,1) = \theta_0,
\end{align*}
which says that each $g_t$ defines a closed loop with basepoint $\theta_0$.

We then have
\begin{align*}
g_0(s) &= p(f_0(\theta_0,s))\\
       &= p(\theta_0,s)\\
       &= \theta_0,\\
g_1(s) &= p(f_1(\theta_0,s))\\
       &= p(\theta_0+2\pi s,s))\\
       &= \theta_0+2\pi s.
\end{align*}
But these paths have winding numbers 0 and 1 respectively (since they lift
to $\widetilde{g_0}(s)=0$ and $\widetilde{g_1}(s)=2\pi s$ respectively;
recall that the winding number of a loop can be computed from any lift).
Therefore, by what we
know about $\pi_1(S^1)$, they cannot be homotopic.  This is a contradiction and says
that no such homotopy $f_t$ can exist.

\yellprob{[Hatcher p.38 \#8] Does the Borsuk-Ulam theorem hold for the torus?    In other words, for every map $f:S^1\x S^1\to\Rr^2$ must there exist $(x,y)\in S^1\x S^1$ such that $f(x,y)=f(-x,-y)$?  Why or why not?}

No.  If we parametrize the torus $S^1\x S^1$ as $\{f(s,t)=(e^{is},e^{it}):\ s,t\in[0,2\pi]\}$, then the antipode of $f(s,t)$ is the point $f(s+\pi,t+\pi)$.  We can naturally embed the torus in $\Rr^3$ as a donut by, e.g.,
\[(e^{is},e^{it}) \mapsto (5\cos s+\cos t\cos s,\ 5\sin s+\cos t\sin s,\ \sin t).\]
Then the map $P:S^1\x S^1\to\Rr^2$ given by projection onto the $xy$-plane satisfies $P(q)=-P(-q)\neq(0,0)$ for all points $q$ on the torus.

\yellprob{[Hatcher p.39 \#9] Use the 2-dimensional case of the Borsuk-Ulam theorem (Hatcher, Thm.~1.10, p.32) to prove the ``Ham and Cheese Sandwich Theorem'': if $A_1,A_2,A_3$ are compact measurable sets in $\Rr^3$, then there is a plane in $\Rr^3$ that simultaneously divides each $A_i$ into two pieces of equal measure.}

\soln WLOG (scaling if necessary), assume that $A_1,A_2,A_3\subset D^3$.

For $\vv\in S^2$, let $L_\vv$ be the line through $\vv$ and $-v$.  For $t\in[-1,1]$, let $P(t,\vv)$ be the plane parallel to $L_\vv$ that meets it at the point $t\vv$.  Let $B_3$ be the part of $A_3$ that is on the same 
Let $f_\vv(t)$ be the fraction of the volume of $A_3$ that is on the same side of $P(t,\vv)$ as $-2\vv$ is.  Thus $f_\vv(t)$ increases continuously and monotonically from 0 to 1 as $t$ increases from $-1$ to $1$.  Therefore $f_\vv^{-1}(1/2)$ is some nonempty closed connected set, i.e., an interval of the form $[a_\vv,b_\vv]$ (where $a_\vv,b_\vv$ also depend continuously on $\vv$).  Let $Q(\vv)=P((a_\vv+b_\vv)/2,t)$.  Thus $Q_\vv$ is a plane parallel to $L_\vv$ that depends continuously on $\vv$ and, for every $\vv$, splits $A_3$ into two equal-volume pieces.  Note also that $Q(\vv)=Q(-\vv)$.

Now, for $\vv\in S^2$ and $i=1,2$, let $f_i(\vv)$ be the fraction of the volume of $A_i$ that is on the same side of $Q(v)$ as $\vv$ itself is.  By the Borsuk-Ulam theorem, there is some pair of antipodal points $\pm\vv$ such that $f_i(\vv)=f_i(-\vv)$ for $i=1,2$.  Since $f_i(-\vv)=1-f_i(\vv)$, we have $f_i(\vv)=1/2$, so the plane $Q(\vv)$ splits each of $A_1$ and $A_2$ into two equal pieces as well.

\yellprob{[Hatcher p.39 \#12] Fix $p\in S^1$.  Show that every homomorphism $\pi_1(S^1,p)\to\pi(S^1,p)$ can be realized as the induced homomorphism $\phi_*$ for some $\phi:S^1\to S^1$.}

Regard $S^1$ as the unit circle in $\Cc$ and let $p=1$.  The path $f:I\to S^1$ given by $s\mapsto\exp(2\pi is)$ generates the infinite cyclic group $\pi_1(S^1,p)\isom\Zz$.  Every homomorphism $\alpha:\Zz\to\Zz$ is specified by the number $n=\alpha(1)$.

Meanwhile, for any $n\in\Zz$, the complex function $\phi(z)=z^n$ maps $S^1$ to $S^1$, and the path $\phi_* f$ has winding number $n$ because $\phi\circ f(s) = f(s)^n = \exp(2\pi ins)$ lifts to the map $I\to\Rr$ given by $s\mapsto ns$.  

\yellprob{[Hatcher, p.52, \#1]  Recall that the \defterm{center} of a group $G$ is defined as $Z(G)=\{g\in G:\ gh=hg \ \forall h\in G\}$.

\probpart Show that the free product $G*H$ of nontrivial groups $G$ and $H$ has trivial center.}

\soln Any non-identity element $w\in G*H$ can be written uniquely
as a product $w=w_1\cdots w_n$ of non-identity elements of $G$ and $H$, with
letters $w_i$ alternating between $G$ and $H$ (p.42).
If $w_1\in H$ then $w$ does not commute with any non-identity element
of $G$, while if $w_1\in G$ then $w$ does not commute with any non-identity element
of $H$.  Therefore, the only element of the center is the word of length 0, namely $e$.

\yellprobpart{Show that the only elements of $G*H$ of finite order are the conjugates of finite-order elements in~$G\cup H$.}

Suppose that $w\in G*H$ and $w^n=e$.
Write $w$ in reduced form: $w=g_1\cdots g_k$ where the letters alternate between $G$ and $H$.
$$w^n=(g_1\cdots g_k)(g_1\cdots g_k)\ \cdots\ (g_1\cdots g_k)=e.$$
We need to be able to somehow cancel this expression using only relations
within $G$ and $H$.  The only possibility is that $g_k$ and $g_1$ either both belong to $G$ or both to $H$, and that  $g_k=g_1^{-1}$.  Note that this implies that $k$ is odd, say $k=2K+1$.
Canceling gives
$$w^n=(g_2\cdots g_{k-1})(g_2\cdots g_{k-1})\ \cdots \ (g_2\cdots g_k)=e.$$
Now the only possibility for cancellation is that $g_{k-1}=g_2^{-1}$.   Continuing in this way, we eventually
find that
  $$g_k=g_1^{-1}, \quad g_{k-1}=g_2^{-1},\quad \dots,\quad g_{K+2}=g_K^{-1}.$$
But this says that $w=xyx^{-1}$, where $x=g_1\cdots g_K$ and $y=g_{K+1}$.
Moreover, $y$ belongs to either $G$ or $H$ (because it is a single letter), and
  $$y^n = (x^{-1}wx)^n = x^{-1}w^nx = x^{-1}x = e$$
so $y$ has finite order.  So we have shown that every finite-order element of $G*H$
is a conjugate of a finite-order element of one of $G$ or $H$.

\end{document}
