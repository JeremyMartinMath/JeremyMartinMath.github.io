\documentclass{amsart}
\usepackage{amssymb,amsmath,amsthm,mathrsfs,graphics,hyperref,stmaryrd,psfrag,arcs,xypic}
\usepackage[enableskew]{youngtab}
\numberwithin{equation}{section}
\raggedbottom
\oddsidemargin=0in
\evensidemargin=0in
\textwidth=6.5in
\textheight=8.8in
\topmargin=0.25in
\headheight=0in
\headsep=0.2in
\footskip=0in
\parskip=10bp
\parindent=0bp

\newcommand{\excise}[1]{}
\newcommand{\Latex}[1]{\textbackslash\texttt{#1}}

\newcommand{\bigpad}{\rule[-14mm]{0mm}{30mm}}
\newcommand{\smallpad}{\rule[-1.5mm]{0mm}{5mm}}
\newcommand{\pad}{\rule[-3mm]{0mm}{8mm}}
\newcommand{\padup}{\rule{0mm}{5mm}}
\newcommand{\paddown}{\rule[-3mm]{0mm}{2mm}}
\newcommand{\blank}{\rule{1.25in}{0.25mm}}
\newcommand{\commentout}[1]{}
\newcommand{\yell}[1]{\fbox{\rule[-1mm]{0mm}{4mm} \large\bf #1 }}
\newcommand{\bang}{$\bullet$\quad}
\newcommand{\indnt}{\phantom{.}\qquad}
\newcommand{\littleline}{\begin{center}\rule{4in}{0.5bp}\end{center}}

\newcommand{\includefigure}[3]{{
  \begin{center}
  \resizebox{#1}{#2}{\includegraphics{{figs/#3}}}
  \end{center}}}
\newcommand{\includefigurewithinmath}[3]{{
  \resizebox{#1}{#2}{\includegraphics{{figs/#3}}}}}

%\newcommand{\defterm}[1]{\underline{\textbf{#1}}}
\newcommand{\defterm}[1]{\textbf{#1}}

\DeclareMathOperator{\ch}{\mathbf{ch}}
\DeclareMathOperator{\colspace}{colspace}
\DeclareMathOperator{\corank}{corank}
\DeclareMathOperator{\CST}{CST}
\DeclareMathOperator{\deln}{del}
\DeclareMathOperator{\diag}{diag}
\DeclareMathOperator{\ess}{ess}
\DeclareMathOperator{\Gr}{Gr}
\DeclareMathOperator{\Hom}{Hom}
\DeclareMathOperator{\im}{im}
\DeclareMathOperator{\Irr}{Irr}
\DeclareMathOperator{\Ind}{Ind}
\DeclareMathOperator{\Int}{Int}
\DeclareMathOperator{\lcm}{lcm}
\DeclareMathOperator{\link}{lk}
\DeclareMathOperator{\nullity}{nullity}
\DeclareMathOperator{\nullspace}{nullspace}
\DeclareMathOperator{\Poin}{Poin}
\DeclareMathOperator{\proj}{proj}
\DeclareMathOperator{\rank}{rank}
\DeclareMathOperator{\Res}{Res}
\DeclareMathOperator{\Span}{span}
\DeclareMathOperator{\supp}{supp}
\DeclareMathOperator{\row}{row}
\DeclareMathOperator{\rowspace}{rowspace}
\DeclareMathOperator{\sh}{sh}
\DeclareMathOperator{\tr}{tr}
\DeclareMathOperator{\wt}{wt}

\newtheorem{theorem}{Theorem}[section]
\newtheorem{proposition}[theorem]{Proposition}
\newtheorem{lemma}[theorem]{Lemma}
\newtheorem{corollary}[theorem]{Corollary}
\theoremstyle{definition}
\newtheorem{definition}[theorem]{Definition}
\newtheorem{example}[theorem]{Example}
\newtheorem{remark}[theorem]{Remark}
\newtheorem{problem}[theorem]{Problem}


\newcommand{\cor}{{\bf Corollary: }}
\newcommand{\defn}{{\bf Definition: }}
\newcommand{\defns}{{\bf Definitions: }}
\newcommand{\exa}{{\bf Example: }}
\newcommand{\fact}{{\bf Fact: }}
\newcommand{\lem}{{\bf Lemma: }}
\newcommand{\notn}{{\bf Notation: }}
\newcommand{\obs}{{\bf Observation: }}
\newcommand{\note}{{\bf Note: }}
\newcommand{\prop}{{\bf Proposition: }}
\newcommand{\rmk}{{\bf Remark: }}
\newcommand{\thm}{{\bf Theorem: }}

\newcommand{\basecase}{\emph{Base case: }}
\newcommand{\indstep}{\emph{Inductive step: }}
\newcommand{\skpr}{\emph{Sketch of proof: }}

\newcommand{\0}{\emptyset}
\newcommand{\Alt}{\mathfrak{A}}
\newcommand{\Braid}{Br}
\newcommand{\CHI}{\chi^{\phantom{*}}}
\newcommand{\Cl}{C\ell}
\newcommand{\covers}{\gtrdot}
\newcommand{\coveredby}{\lessdot}
\newcommand{\dedge}[1]{\overrightarrow{{#1}}}
\newcommand{\dom}{\rhd}
\newcommand{\domeq}{\unrhd}
\newcommand{\domby}{\lhd}
\newcommand{\dombyeq}{\unlhd}
\newcommand{\Fspan}{\Ff\text{-span}}
\newcommand{\isom}{\cong}
\newcommand{\join}{\vee}
\renewcommand{\Join}{\bigvee}
\newcommand{\Laff}{L^{\text{aff}}}
\newcommand{\lin}[1]{\overleftrightarrow{{#1}}}
\newcommand{\meet}{\wedge}
\newcommand{\Meet}{\bigwedge}
\newcommand{\ov}[1]{\overline{{#1}}}
\newcommand{\partn}{\vdash}
\newcommand{\qqandqq}{\qquad\text{and}\qquad}
\newcommand{\qandq}{\quad\text{and}\quad}
\newcommand{\qand}{\quad\text{and}}
\newcommand{\qbin}[2]{{\begin{bmatrix}#1\\#2\end{bmatrix}_q}}
\newcommand{\sd}{\triangle} % symmetric difference
\newcommand{\simK}{\underset{K}{\sim}} % Knuth equivalence
\newcommand{\simJ}{\underset{J}{\sim}} % jeu de taquin equivalence
\newcommand{\sm}{\setminus}
\newcommand{\st}{~|~}
\newcommand{\soln}{\textit{Solution:\ }}
\newcommand{\Sym}{\mathfrak{S}}
\newcommand{\un}[1]{\underset{*}{#1}}
\newcommand{\unA}{\un{A}}
\newcommand{\unB}{\un{B}}
\newcommand{\unw}{\un{w}}
\newcommand{\unx}{\un{x}}
\newcommand{\uny}{\un{y}}
\newcommand{\unz}{\un{z}}
\newcommand{\x}{\times}

\renewcommand{\aa}{\mathbf{a}}
\newcommand{\bb}{\mathbf{b}}
\newcommand{\nn}{\mathbf{n}}
\newcommand{\pp}{\mathbf{p}}
\newcommand{\qq}{\mathbf{q}}
\newcommand{\xx}{\mathbf{x}}
\newcommand{\yy}{\mathbf{y}}
\newcommand{\zz}{\mathbf{z}}
 
\newcommand{\A}{\mathcal{A}}
\newcommand{\B}{\mathcal{B}}
\newcommand{\C}{\mathcal{C}}
\newcommand{\M}{\mathcal{M}}
\renewcommand{\P}{\mathcal{P}}

\newcommand{\BB}{\mathscr{B}}  %% use these for fancy script fonts -- requires mathrsfs package
\newcommand{\CC}{\mathscr{C}}
\newcommand{\FF}{\mathscr{F}}
\newcommand{\II}{\mathscr{I}}
\newcommand{\LL}{\mathscr{L}}
\newcommand{\PP}{\mathscr{P}}
\renewcommand{\SS}{\mathscr{S}}
\newcommand{\XX}{\mathscr{X}}

\newcommand{\TT}{\tilde{T}}

\newcommand{\Aa}{\mathbb{A}}
\newcommand{\Cc}{\mathbb{C}}
\newcommand{\Ff}{\mathbb{F}}
\newcommand{\Nn}{\mathbb{N}}
\newcommand{\Pp}{\mathbb{P}}
\newcommand{\Qq}{\mathbb{Q}}
\newcommand{\Rr}{\mathbb{R}}
\newcommand{\Zz}{\mathbb{Z}}

\newcommand{\rhodef}{\rho^{\phantom{*}}_{{\rm def}}}
\newcommand{\rhotriv}{\rho^{\phantom{*}}_{{\rm triv}}}
\newcommand{\rhosign}{\rho^{\phantom{*}}_{{\rm sign}}}
\newcommand{\rhoreg}{\rho^{\phantom{*}}_{{\rm reg}}}
\newcommand{\chidef}{\chi^{\phantom{*}}_{{\rm def}}}
\newcommand{\chitriv}{\chi^{\phantom{*}}_{{\rm triv}}}
\newcommand{\chisign}{\chi^{\phantom{*}}_{{\rm sign}}}
\newcommand{\chireg}{\chi^{\phantom{*}}_{{\rm reg}}}
\newcommand{\scp}[2]{\left\langle #1,\:#2\right\rangle_G}
\newcommand{\scpH}[2]{\left\langle #1,\:#2\right\rangle_H}

\newcounter{probno}
\setcounter{probno}{0}
\newcounter{partno}
\setcounter{partno}{0}
%% versions that don't print the number of points
\newcommand{\prob}{
  \vskip10bp%
  \setcounter{partno}{0}%
  \addtocounter{probno}{1}%
  {\bf Problem~\#{\arabic{probno}}}\quad}
\newcommand{\probpart}{%\rule{0in}{0in}\\ \phantom{xxx}
  \addtocounter{partno}{1}%
  {\bf (\#\arabic{probno}\alph{partno})}\ \ }
\newcommand{\probcont}{%
  {\bf Problem~\#{\arabic{probno}}}~(\emph{continued})}
\newcommand{\probo}{
  \setcounter{partno}{0}%
  \addtocounter{probno}{1}%
  {\bf (\#\arabic{probno})}\ \ }
%% versions that do print the number of points
\newcommand{\Prob}[1]{
  \vskip10bp%
  \setcounter{partno}{0}%
  \addtocounter{probno}{1}%
  {\bf Problem~\#{\arabic{probno}}~[{#1}~pts]}\quad}
\newcommand{\Probpart}[1]{%\rule{0in}{0in}\\ \phantom{xxx}
  \addtocounter{partno}{1}%
  {\bf (\#\arabic{probno}\alph{partno})~[{#1}~pts]}\ \ }

%\renewcommand{\thefootnote}{\fnsymbol{footnote}}

\begin{document}
\thispagestyle{empty}
{\bf Math 821, Spring 2014\\
Solution Set \#1\\
Due date: Friday, January 31}
\bigskip

\yellprob{Let $X$ be a path-connected topological space.
Prove that $X$ is connected.  (Recall that the
converse is not true --- the topologists' sine curve
is a counterexample.)}

\soln We need to show that $X$ has no subset $U$ that is ``clopen'' (i.e., both closed
and open) other than $\0$ or $X$.  Suppose $U$ is such a clopen subset,
so that its complement $V=X\sm U$ is also clopen.
Suppose that $p\in U$ and that $q\in X\sm U$.
Let $P$ be a $p,q$-path in $X$ parametrized by $f:I=[0,1]\to X$.
Then $P=(P\cap U)\dju(P\cap V)$.  Let
$A=f^{-1}(P\cap U)$ and $B=f^{-1}(P\cap V)$.  Then
\begin{itemize}
\item $A\cup B=I$ (because $f(I)=P$);
\item $A\cap B=\0$ (because $f(A)\cap f(B)=\0$);
\item $A,B$ are open in $I$ (by continuity);
\item $\0\subsetneq A,B\subsetneq I$ (because $0\in A\sm B$
and $1\in B\sm A$).
\end{itemize}
But this is a contradiction because $I$ is connected.
Therefore, no such pair $p,q$ can exist, which says that one 
of $U,X\sm U$ is empty.  It follows that $X$ is connected.
(In principle, this argument uses the continuity of~$f$
to reduce the ``path-connected implies connected'' statement
about an arbitrary topological space to the same statement about
the familiar topological space~$I$.)\hfill\qed

\altsoln Fix $x\in X$.
For every $y\in X$, choose a path $f_y:I\to X$ with
$f_y(0)=x$ and $f_y(1)=y$, and let $P_y=f_y(I)\subseteq X$.
Then each $P_y$ is connected (because it is the continuous
image of the connected space $I$) and $\bigcap_{y\in X}P_y\neq\0$
(because it contains $x$), so $\bigcup_{y\in X}P_y=X$ is connected.

%------------------------------------------------------------
\yellprob{Let $\Gamma$ be a finite graph.  Prove that if $\Gamma$
is connected, then it is path-connected.}

\soln I'll prove something more general: every space $X$ that is both
connected and locally path-connected is path-connected.
(``Locally path-connected'' means that every point $x$ has a connected open
neighborhood $U_x$.)
A graph is locally path-connected because every vertex has a neighborhood that
looks like the vertex itself plus $d$ rays sticking out (where $d$ is the degree
of the vertex --- the number of edges attached to it, possibly infinite, counting a loop as two edges)
and every point on the interior of an edge has a neighborhood that looks like
an open interval.

Let $x\in X$ and let $Y$ be the set of all points that are joined
to $x$ by a path. For every $y\in Y$ and $q\in U_y$, we
have an $x,y$-path and a $y,q$-path; concatenating them produces
an $x,q$-path.  Therefore $Y=\bigcup_{y\in Y} U_y$ is open.

On the other hand, let $Z=X\sm Y$.
For every $z\in Z$ and $q\in U_z$, we
cannot have an $q,x$-path, since we certainly have
a $z,q$-path and concatenating the two would produce an
$z,x$-path, which cannot exist.  Therefore,
$Z=\bigcup_{z\in Z} U_z$ is open.

We have constructed a clopen decomposition $X=Y\dju Z$
with $Y\neq\0$ (because $x\in Y$).
Since $X$ is connected, we must have $Y=X$ and $Z=\0$.
This is precisely the statement that $X$ is path-connected.

In fact, we don't even need the assumption of
finiteness --- any cell complex will work.
What this tells us, among other things, is that the topologists' sine curve cannot be
realized as a cell complex.

\pagebreak
%------------------------------------------------------------
\yellprob{Let $X$ and $Y$ be topological spaces and
let $f:X\surj Y$ be a continuous function that is onto.
Prove that if $X$ is compact, then so is $Y$.}

\soln Let $\{U_\alpha\st\alpha\in A\}$ be an open cover
of $Y$.  Then $\{f^{-1}(U_\alpha)\st\alpha\in A\}$ is an open cover
of $X$ (because every point in $X$ gets mapped to
a point in at least one $U_\alpha$, hence belongs to $f^{-1}(U_\alpha)$).
By compactness, it has a finite subcover:
$\{f^{-1}(U_\alpha)\st\alpha\in A'\}$, where $A'\subseteq A$ is finite.
I.e., $X=\bigcup_{\alpha\in A'}f^{-1}(U_\alpha)$.
This implies set-theoretically that $Y=\bigcup_{\alpha\in A'}U_\alpha$,
so there we have the desired finite subcover.

%------------------------------------------------------------
\yellprob{[Hatcher p.18 \#1] Construct an explicit deformation retraction of the torus with one point deleted onto a graph consisting of two circles intersecting in a point.}

Solution: Draw a square $Q$ from which the center point $c$ has been deleted.  For each point $p\in Q\sm c$, draw the ray $\ray{cp}$ and let $f(p)$ be the point where that ray hits the boundary $\bd Q$.  Define $F:Q\x I\to Q$ by
\[F(p,t) = (1-t)p + t f(p)\]
(where the arithmetic is vector arithmetic in $\Rr^2$).  Note that $F(0,p)=p$ and $F(1,p)=f(p)$.  Moreover, if $p\in\bd Q$ then $F(p,t)=p$ for all $t$.  So $F$ is a deformation retraction.

\includefigure{3in}{3in}{torusDR}

If we pass from $Q$ to the torus $T$ by identifying opposite sides, the map $F$ is still well-defined and is a deformation retraction.  Note that $\bd Q$ maps onto the union of two circles that meet in a point --- one longitudinal and one meridional circle.

{\bf Note:} Many of you found the explicit formula for $f$.  Specifically, if $p=(x,y)$ in Cartesian coordinates, then
\[f(p) = \frac{1}{\max(|x|,|y|}p\]
which is well-defined precisely because $p\neq(0,0)$.  However, the geometric description above is sufficient.

One solver (Billy) found another way to say this: build the torus by starting with the closed unit \textbf{disk}, partitioning its boundary circle into four $90^\circ$ arcs (say, the intersections with the four quadrants in $\Rr^2$), and identifying them.  Then the deformation retraction can be expressed very naturally in polar coordinates: $F((r,\theta),t)=((1-t)r+t,\theta)$.



\pagebreak
%------------------------------------------------------------
\yellprob{[Hatcher p.18 \#3, more or less]  (a) Show that homotopy equivalence of spaces is an equivalence relation.}

For reflexivity, the identity map is a homotopy equivalence, and symmetry is immediate from the definition.  For transitivity,  suppose we have maps as shown that are all homotopy equivalences.
\[\xymatrix{
X \ar@/^/[r]^f & Y\ar@/^/[l]^g\ar@/^/[r]^h & Z \ar@/^/[l]^k
}\]
Let $p=g\circ k\circ h\circ f$.  We need to construct a homotopy $\alpha\htop\IdMap_X$.  By hypothesis, suppose we have a homotopy
\[q_t:Y\to Y, \qquad q_0=\IdMap_Y,\quad q_1=k\circ h.\]
Then $g\circ q_t\circ f$ is a homotopy
with  $g\circ q_1\circ f=p$ and $g\circ q_t\circ f=g\circ f$.  Therefore
\[p =  g\circ k\circ h\circ f \htop g\circ f \htop \IdMap_X.\]

{\bf (b) Fix spaces $X,Y$ and let $f,g$ be maps $X\to Y$.  Show that the relation ``$f$ is homotopic to $g$'' is an equivalence relation.}

Reflexivity: $f\htop f$ by the homotopy $F(x,t)=f(x)$.

Symmetry: If $F(x,t)$ is a homotopy between $f$ and $g$ then $F(x,1-t)$ is a homotopy between $g$ and $f$.

Transitivity: If $F(x,t),G(x,t)$ realize homotopies $f\htop g$ and $g\htop h$, then define $H(x,t)=F(x,2t)$ for $0\leq t\leq 1/2$ and $H(x,t)=G(x,2t-1)$ for $1/2\leq t\leq 1$.

{\bf (c) Show that a map homotopic to a homotopy equivalence is a homotopy equivalence.}

Lemma: Let $f_0,f_1$ be homotopic maps $X\to Y$.  Let $a:W\to X$ and $b:Y\to Z$ be maps.  Then $f_0\circ a\htop f_1\circ a$
and $b\circ f_0\htop b\circ f_1$.

Proof: If $F:X\to I$ is a homotopy between $f_0$ and $f_1$, then  $F\circ(a\x\IdMap)$ is a homotopy between $f_0\circ a$ and $f_1\circ a$ and $b\circ F$ is a homotopy between $b\circ f_0$ and $b\circ f_1$.

Corollary: If $f_0\htop f_1$ and $g$ is a homotopy inverse for $f_0$, then by the lemma we have
$f_1\circ g\htop f_0\circ g\htop\IdMap_Y$ and $g\circ f_1\htop g\circ f_1\htop\IdMap_X$.  So something even stronger is true: if $g$ is a homotopy inverse for $f$ then it is a homotopy inverse for any map to which $F$ is homotopic.

\pagebreak
%------------------------------------------------------------
\yellprob{[Hatcher p.19 \#14] Given nonnegative integers $v,e,f$ with $v-e+f=2$ (and $v,f>0$), construct a cell structure on $S^2$ having $v$ 0-cells, $e$ 1-cells, and $f$ 2-cells.  (Do not use any facts about spanning trees or Euler characteristic.)}

\soln First, a terminological note.  The following phrases are all synonyms:
\begin{itemize}
\item ``cell structure on $S^2$'';
\item ``cell complex homeomorphic to $S^2$'';
\item ``cellular 2-sphere'';
\item ``cellular $S^2$'';
\item ``cellulation of $S^2$''.
\end{itemize}

There is only one cellular $S^2$ with $(v,e,f)=(1,0,1)$: take a 2-cell and squash its boundary to a point.  Equivalently, this is the one-point compactification of $\Rr^2$.

For the case $e>0$, there are several constructions; here is one.  Draw a sphere with an equator, and put $v$ vertices on the equator, making $v$ edges between them.  (If $e=1$ this means that the equator is a loop; that's okay.)
Then pick one of the vertices and draw $f-2$ nested loops at it, all reaching into the southern hemisphere.  We end up with a cell structure with $v$ vertices and $v+f-2=e$ edges.  The number of faces is $f$ because the equator separates the globe into two 2-cells (the northern and southern hemispheres), and each loop adds one more to the count of 2-cells.  For example, here is a picture with $(v,e,f)=(5,7,4)$:

\includefigure{3in}{3in}{vef-sphere}

Note 1:  In order to have a \emph{regular} cellular 2-sphere, I believe it is necessary and sufficient to have at least two cells of each dimension.

Note 2: Many of you broke the problem into several cases depending on the value of $e$.  This is OK as a means of solving the problem, but when you write it up, you should see if you can find a simpler solution that does not require case analysis.

\end{document}
