\documentclass{amsart}
\usepackage{amssymb,amsmath,amsthm,mathrsfs,graphics,hyperref,ifthen,framed,cancel,fullpage,color,ytableau,tcolorbox,bm,tikz}
\usepackage[enableskew]{youngtab}
\raggedbottom
\parskip=10bp
\parindent=0bp
\raggedbottom

%%%%%%%%%%%%%%%%   Colors for TikZ and text  %%%%%%%%%%%%%%%%%

\definecolor{light}{gray}{.75}
\definecolor{med}{gray}{.5}
\definecolor{dark}{gray}{.25}
\newcommand{\Red}[1]{{\color{red}{#1}}}
\newcommand{\RED}[1]{{\color{red}{\boldmath\textbf{#1}\unboldmath}}}
\newcommand{\Blue}[1]{{\color{blue}{#1}}}
\newcommand{\BLUE}[1]{{\color{blue}{\boldmath\textbf{#1}\unboldmath}}}

% Hyperlinks
\hypersetup{colorlinks, citecolor=red, filecolor=black, linkcolor=blue, urlcolor=blue}
\newcommand{\hreftext}[2]{\href{#1}{\Blue{#2}}} % for text anchors
\newcommand{\hrefurl}[2]{\href{#1}{\Blue{\tt #2}}} % if you want to actually write out the URL in text

% Spacing and commands
\newcommand{\bigpad}{\rule[-14mm]{0mm}{30mm}}
\newcommand{\smallpad}{\rule[-1.5mm]{0mm}{5mm}}
\newcommand{\pad}{\rule[-3mm]{0mm}{8mm}}
\newcommand{\padup}{\rule{0mm}{5mm}}
\newcommand{\paddown}{\rule[-3mm]{0mm}{2mm}}
\newcommand{\blank}{\rule{1.25in}{0.25mm}}
\newcommand{\yell}[1]{\fbox{\rule[-1mm]{0mm}{4mm} \large\bf #1 }}
\newcommand{\indnt}{\phantom{.}\qquad}
\newcommand{\littleline}{\begin{center}\rule{4in}{0.5bp}\end{center}}

% macros for inserting figures
\newcommand{\includefigure}[3]{\begin{center}\resizebox{#1}{#2}{\includegraphics{#3}}\end{center}}
\newcommand{\includefigurewithinmath}[3]{\resizebox{#1}{#2}{\includegraphics{#3}}}
% E.g., to insert a standalone figure of width 3" and height 1.5", use
% \includefigure{3in}{1.5in}{foo.pdf}

% math operators
\DeclareMathOperator{\Comp}{Comp}
\DeclareMathOperator{\Fix}{Fix}
\DeclareMathOperator{\id}{id}
\DeclareMathOperator{\im}{im}
\DeclareMathOperator{\lcm}{lcm}
\DeclareMathOperator{\Par}{Par}
\DeclareMathOperator{\rank}{rank}
\DeclareMathOperator{\sh}{sh}
\DeclareMathOperator{\tr}{tr}
\DeclareMathOperator{\wt}{wt}

% theorem environments, automatically numbered
\newtheorem{theorem}{Theorem}[section]
\newtheorem{proposition}[theorem]{Proposition}
\newtheorem{lemma}[theorem]{Lemma}
\newtheorem{corollary}[theorem]{Corollary}
\theoremstyle{definition}
\newtheorem{definition}[theorem]{Definition}
\newtheorem{example}[theorem]{Example}
\newtheorem{remark}[theorem]{Remark}
\newtheorem{problem}[theorem]{Problem}

\newcommand{\skpr}{\emph{Sketch of proof: }}
\newcommand{\soln}{\textit{Solution:\ }}

% Generally useful macros
\newcommand{\excise}[1]{} % useful for commenting out large chunks
\newcommand{\0}{\emptyset}
\newcommand{\compn}{\models} % compositions
\newcommand{\dju}{\mathaccent\cdot\cup} % disjoint union
\newcommand{\dsum}{\displaystyle\sum}
\newcommand{\fallfac}[2]{{#1}^{\underline{#2}}} % falling factorial
\newcommand{\isom}{\cong} % isomorphism symbol
\newcommand{\partn}{\vdash} % partition symbol
\newcommand{\qqandqq}{\qquad\text{and}\qquad}
\newcommand{\qandq}{\quad\text{and}\quad}
\newcommand{\qand}{\quad\text{and}}
\newcommand{\qbin}[2]{{\begin{bmatrix}#1\\#2\end{bmatrix}_q}} % q-binomial coefficient
\newcommand{\risefac}[2]{{#1}^{\overline{#2}}} % rising factorial
\newcommand{\sd}{\triangle} % symmetric difference
\newcommand{\sm}{\setminus} % don't use a minus sign for this
\newcommand{\st}{\colon} % "such that"
\newcommand{\surj}{\twoheadrightarrow}
\newcommand{\x}{\times}

% blackboard bold fonts for sets of numbers
\newcommand{\Cc}{\mathbb{C}} % complex numbers
\newcommand{\Ff}{\mathbb{F}} % finite field
\newcommand{\Nn}{\mathbb{N}} % natural numbers
\newcommand{\Qq}{\mathbb{Q}}
\newcommand{\Rr}{\mathbb{R}}
\newcommand{\Zz}{\mathbb{Z}}

% miscellaneous
\newcommand{\TwoCases}[4]{\begin{cases}{#1}&\text{ #2}\\{#3}&\text{ #4}\end{cases}}
\newcommand{\ThreeCases}[6]{\begin{cases}{#1}&\text{ #2}\\{#3}&\text{ #4}\\{#5}&\text{ #6}\end{cases}}
\newcommand{\bridgehand}[4]{\spadesuit\ {\textsf{#1}}\ \ \heartsuit\ {\textsf{#2}}\ \ \diamondsuit\ {\textsf{#3}}\ \ \clubsuit\ {\textsf{#4}}}

% macros for automatic problem numbering --- students don't have to use these
\newcounter{probno}
\setcounter{probno}{0}
\newcounter{partno}
\setcounter{partno}{0}
%% versions that don't print the number of points
\newcommand{\prob}{
  \vskip10bp%
  \setcounter{partno}{0}%
  \addtocounter{probno}{1}%
  {\bf Problem~\#{\arabic{probno}}}\quad}

% No initial whitespace to make framing look nicer
\newcommand{\probns}{
  \setcounter{partno}{0}%
  \addtocounter{probno}{1}%
  {\bf Problem~\#{\arabic{probno}}}\quad}

\newcommand{\probpart}{%
  \addtocounter{partno}{1}%
  {\bf (\#\arabic{probno}\alph{partno})}\ \ }
\newcommand{\probcont}{%
  {\bf Problem~\#{\arabic{probno}}}~(\emph{continued})}
\newcommand{\probo}{
  \setcounter{partno}{0}%
  \addtocounter{probno}{1}%
  {\bf (\#\arabic{probno}}\ \ }

%% versions that do print the number of points
\newcommand{\Prob}[1]{
  \vskip10bp%
  \setcounter{partno}{0}%
  \addtocounter{probno}{1}%
  {\bf Problem~\#{\arabic{probno}}~[{#1}~pts]}\quad}

%no initial whitespace
\newcommand{\Probns}[1]{
  \setcounter{partno}{0}%
  \addtocounter{probno}{1}%
  {\bf Problem~\#{\arabic{probno}}~[{#1}~pts]}\quad}
\newcommand{\Probpart}[1]{%\rule{0in}{0in}\\ \phantom{xxx}
  \addtocounter{partno}{1}%
  {\bf (\#\arabic{probno}\alph{partno})~[{#1}~pts]}\ \ }

\newboolean{answers}
\newcommand{\Answer}[1]{\ifthenelse{\boolean{answers}}{{\bf Answer:}\ #1}{}\bigskip}

\begin{document}
\thispagestyle{empty}
\begin{center}
{\bf\LARGE Simplicial Homology Computations Using Macaulay2}
\end{center}
\bigskip

{\bf Macaulay2} is a free software system for computations in commutative algebra.  To use it, do one of the following:
\begin{enumerate}
\item Download and install it from the website:
\item[] \href{http://www.math.uiuc.edu/Macaulay2}{\fbox{\tt http://www.math.uiuc.edu/Macaulay2/}}
\item[]
\item Use the online interface:
\item[] \href{http://habanero.math.cornell.edu:3690}{\fbox{\tt http://habanero.math.cornell.edu:3690}}
\item[]
\item Log into your math account, open a terminal window, and type ``M2''.
\end{enumerate}
\bigskip

When you start Macaulay2, you'll see something like this:
\begin{framed}
\begin{verbatim}
Macaulay2, version 1.6
with packages: ConwayPolynomials, Elimination, IntegralClosure, LLLBases,
               PrimaryDecomposition, ReesAlgebra, TangentCone
i1 : 
\end{verbatim}
\end{framed}

You can compute the kernel of a matrix directly:
\begin{framed}
\begin{verbatim}
i1 : A=matrix{{1,2,3},{4,5,6}}

o1 = | 1 2 3 |
     | 4 5 6 |

              2        3
o1 : Matrix ZZ  <--- ZZ

i2 : ker A

o2 = image | -1 |
           | 2  |
           | -1 |

                               3
o2 : ZZ-module, submodule of ZZ
\end{verbatim}
\end{framed}

By default, M2 works over $\Zz$ (that's what \texttt{ZZ} means).
Here I have typed in a $2\x3$ matrix $A$ corresponding
to a linear transformation $\Zz^3\to\Zz^2$ (for various reasons,
Macaulay writes the arrows right to left) and computed
its kernel, which Macaulay has described as the image (i.e.,
column space) of the matrix $B=\begin{bmatrix}-1\\2\\-1\end{bmatrix}$.

\pagebreak

If you have a $\Zz$-module and you just want to know what it is
up to isomorphism, you can use the \texttt{prune} command:
\begin{framed}
\begin{verbatim}
i3 : prune ker A

       1
o3 = ZZ

o3 : ZZ-module, free

i4 : M = matrix{{4,2,6},{2,6,4},{6,2,4}}; prune coker M

              3        3
o4 : Matrix ZZ  <--- ZZ

o5 = cokernel | 24 0 0 |
              | 0  2 0 |
              | 0  0 2 |

                              3
o5 : ZZ-module, quotient of ZZ

i6 : N = matrix{{4,-4},{-4,4}}; prune coker N

              2        2
o6 : Matrix ZZ  <--- ZZ

\end{verbatim}
\end{framed}

In these last two computations, Macaulay is actually producing the Smith normal form.  So $\coker M\isom\Zz_{24}\oplus\Zz_2\oplus\Zz_2$ and $\coker N\isom\Zz\oplus\Zz_4$.  (The zero row corresponds to the $\Zz$ summand.)

Macaulay also has a package to work with simplicial complexes (although not $\Delta$-complexes, so far as I know).  First you have to define a polynomial ring in variables corresponding to vertices; then you can specify the complex by its facets.  Macaulay has built-in commands to compute the simplicial chain complex and homology groups.

\begin{framed}
\begin{verbatim}
i7: load "SimplicialComplexes.m2";

i8 : R = ZZ[a..d]; -- make a polynomial ring with variables a,b,c,d

i9 : X1 = simplicialComplex{a*b, a*c, b*c};  -- the 1-skeleton of the 2-simplex

i10 : X2 = simplicialComplex{a*b*c}; -- the full 2-simplex

i11 : X3 = simplicialComplex{a*b*c, a*b*d};  -- two 2-simplices joined at an edge

i12 : X4 = simplicialComplex{a*b*c, a*b*d, a*c*d, b*c*d}; -- a hollow tetrahedron

i13 : C = chainComplex X3 -- compute the simplicial chain complex

        1       4       5       2
o13 = ZZ  <-- ZZ  <-- ZZ  <-- ZZ

      -1      0       1       2

o13 : ChainComplex
\end{verbatim}
\end{framed}


The numbers on the bottom are dimensions.  So $X_3$ has one (-1)-simplex (of course),
four 0-simplices, five 1-simplices and two 2-simplices.  Notice that the arrows point \emph{to the left}.
Why this is a good idea is beyond the scope of Math 821; just be aware that Macaulay2 does use this convention.

\begin{framed}
\begin{verbatim}
i14 : C.dd_2 -- extract the boundary map d2 (from 2-chains to 1-chains)

o14 = | -1 -1 |
      | 1  0  |
      | 0  1  |
      | -1 0  |
      | 0  -1 |

               5        2
o14 : Matrix ZZ  <--- ZZ

i15 : prune HH_2 X4 -- compute just one homology group

        1
o15 = ZZ

o15 : ZZ-module, free

i16 : prune HH X4 -- compute all the homology groups

o16 = -1 : 0

       0 : 0

       1 : 0

             1
       2 : ZZ

o16 : GradedModule

i17 : C.dd_1 * C.dd_2 -- verify that boundary-squared = 0

o17 = 0

               4        2
o17 : Matrix ZZ  <--- ZZ
\end{verbatim}
\end{framed}

\end{document}