\documentclass{amsart}
\usepackage{amssymb,amsmath,amsthm,mathrsfs,graphics,hyperref,stmaryrd,psfrag,arcs,xypic}
\usepackage[enableskew]{youngtab}
\numberwithin{equation}{section}
\raggedbottom
\oddsidemargin=0in
\evensidemargin=0in
\textwidth=6.5in
\textheight=8.8in
\topmargin=0.25in
\headheight=0in
\headsep=0.2in
\footskip=0in
\parskip=10bp
\parindent=0bp

\newcommand{\excise}[1]{}
\newcommand{\Latex}[1]{\textbackslash\texttt{#1}}

\newcommand{\bigpad}{\rule[-14mm]{0mm}{30mm}}
\newcommand{\smallpad}{\rule[-1.5mm]{0mm}{5mm}}
\newcommand{\pad}{\rule[-3mm]{0mm}{8mm}}
\newcommand{\padup}{\rule{0mm}{5mm}}
\newcommand{\paddown}{\rule[-3mm]{0mm}{2mm}}
\newcommand{\blank}{\rule{1.25in}{0.25mm}}
\newcommand{\commentout}[1]{}
\newcommand{\yell}[1]{\fbox{\rule[-1mm]{0mm}{4mm} \large\bf #1 }}
\newcommand{\bang}{$\bullet$\quad}
\newcommand{\indnt}{\phantom{.}\qquad}
\newcommand{\littleline}{\begin{center}\rule{4in}{0.5bp}\end{center}}

\newcommand{\includefigure}[3]{{
  \begin{center}
  \resizebox{#1}{#2}{\includegraphics{{figs/#3}}}
  \end{center}}}
\newcommand{\includefigurewithinmath}[3]{{
  \resizebox{#1}{#2}{\includegraphics{{figs/#3}}}}}

%\newcommand{\defterm}[1]{\underline{\textbf{#1}}}
\newcommand{\defterm}[1]{\textbf{#1}}

\DeclareMathOperator{\ch}{\mathbf{ch}}
\DeclareMathOperator{\colspace}{colspace}
\DeclareMathOperator{\corank}{corank}
\DeclareMathOperator{\CST}{CST}
\DeclareMathOperator{\deln}{del}
\DeclareMathOperator{\diag}{diag}
\DeclareMathOperator{\ess}{ess}
\DeclareMathOperator{\Gr}{Gr}
\DeclareMathOperator{\Hom}{Hom}
\DeclareMathOperator{\im}{im}
\DeclareMathOperator{\Irr}{Irr}
\DeclareMathOperator{\Ind}{Ind}
\DeclareMathOperator{\Int}{Int}
\DeclareMathOperator{\lcm}{lcm}
\DeclareMathOperator{\link}{lk}
\DeclareMathOperator{\nullity}{nullity}
\DeclareMathOperator{\nullspace}{nullspace}
\DeclareMathOperator{\Poin}{Poin}
\DeclareMathOperator{\proj}{proj}
\DeclareMathOperator{\rank}{rank}
\DeclareMathOperator{\Res}{Res}
\DeclareMathOperator{\Span}{span}
\DeclareMathOperator{\supp}{supp}
\DeclareMathOperator{\row}{row}
\DeclareMathOperator{\rowspace}{rowspace}
\DeclareMathOperator{\sh}{sh}
\DeclareMathOperator{\tr}{tr}
\DeclareMathOperator{\wt}{wt}

\newtheorem{theorem}{Theorem}[section]
\newtheorem{proposition}[theorem]{Proposition}
\newtheorem{lemma}[theorem]{Lemma}
\newtheorem{corollary}[theorem]{Corollary}
\theoremstyle{definition}
\newtheorem{definition}[theorem]{Definition}
\newtheorem{example}[theorem]{Example}
\newtheorem{remark}[theorem]{Remark}
\newtheorem{problem}[theorem]{Problem}


\newcommand{\cor}{{\bf Corollary: }}
\newcommand{\defn}{{\bf Definition: }}
\newcommand{\defns}{{\bf Definitions: }}
\newcommand{\exa}{{\bf Example: }}
\newcommand{\fact}{{\bf Fact: }}
\newcommand{\lem}{{\bf Lemma: }}
\newcommand{\notn}{{\bf Notation: }}
\newcommand{\obs}{{\bf Observation: }}
\newcommand{\note}{{\bf Note: }}
\newcommand{\prop}{{\bf Proposition: }}
\newcommand{\rmk}{{\bf Remark: }}
\newcommand{\thm}{{\bf Theorem: }}

\newcommand{\basecase}{\emph{Base case: }}
\newcommand{\indstep}{\emph{Inductive step: }}
\newcommand{\skpr}{\emph{Sketch of proof: }}

\newcommand{\0}{\emptyset}
\newcommand{\Alt}{\mathfrak{A}}
\newcommand{\Braid}{Br}
\newcommand{\CHI}{\chi^{\phantom{*}}}
\newcommand{\Cl}{C\ell}
\newcommand{\covers}{\gtrdot}
\newcommand{\coveredby}{\lessdot}
\newcommand{\dedge}[1]{\overrightarrow{{#1}}}
\newcommand{\dom}{\rhd}
\newcommand{\domeq}{\unrhd}
\newcommand{\domby}{\lhd}
\newcommand{\dombyeq}{\unlhd}
\newcommand{\Fspan}{\Ff\text{-span}}
\newcommand{\isom}{\cong}
\newcommand{\join}{\vee}
\renewcommand{\Join}{\bigvee}
\newcommand{\Laff}{L^{\text{aff}}}
\newcommand{\lin}[1]{\overleftrightarrow{{#1}}}
\newcommand{\meet}{\wedge}
\newcommand{\Meet}{\bigwedge}
\newcommand{\ov}[1]{\overline{{#1}}}
\newcommand{\partn}{\vdash}
\newcommand{\qqandqq}{\qquad\text{and}\qquad}
\newcommand{\qandq}{\quad\text{and}\quad}
\newcommand{\qand}{\quad\text{and}}
\newcommand{\qbin}[2]{{\begin{bmatrix}#1\\#2\end{bmatrix}_q}}
\newcommand{\sd}{\triangle} % symmetric difference
\newcommand{\simK}{\underset{K}{\sim}} % Knuth equivalence
\newcommand{\simJ}{\underset{J}{\sim}} % jeu de taquin equivalence
\newcommand{\sm}{\setminus}
\newcommand{\st}{~|~}
\newcommand{\soln}{\textit{Solution:\ }}
\newcommand{\Sym}{\mathfrak{S}}
\newcommand{\un}[1]{\underset{*}{#1}}
\newcommand{\unA}{\un{A}}
\newcommand{\unB}{\un{B}}
\newcommand{\unw}{\un{w}}
\newcommand{\unx}{\un{x}}
\newcommand{\uny}{\un{y}}
\newcommand{\unz}{\un{z}}
\newcommand{\x}{\times}

\renewcommand{\aa}{\mathbf{a}}
\newcommand{\bb}{\mathbf{b}}
\newcommand{\nn}{\mathbf{n}}
\newcommand{\pp}{\mathbf{p}}
\newcommand{\qq}{\mathbf{q}}
\newcommand{\xx}{\mathbf{x}}
\newcommand{\yy}{\mathbf{y}}
\newcommand{\zz}{\mathbf{z}}
 
\newcommand{\A}{\mathcal{A}}
\newcommand{\B}{\mathcal{B}}
\newcommand{\C}{\mathcal{C}}
\newcommand{\M}{\mathcal{M}}
\renewcommand{\P}{\mathcal{P}}

\newcommand{\BB}{\mathscr{B}}  %% use these for fancy script fonts -- requires mathrsfs package
\newcommand{\CC}{\mathscr{C}}
\newcommand{\FF}{\mathscr{F}}
\newcommand{\II}{\mathscr{I}}
\newcommand{\LL}{\mathscr{L}}
\newcommand{\PP}{\mathscr{P}}
\renewcommand{\SS}{\mathscr{S}}
\newcommand{\XX}{\mathscr{X}}

\newcommand{\TT}{\tilde{T}}

\newcommand{\Aa}{\mathbb{A}}
\newcommand{\Cc}{\mathbb{C}}
\newcommand{\Ff}{\mathbb{F}}
\newcommand{\Nn}{\mathbb{N}}
\newcommand{\Pp}{\mathbb{P}}
\newcommand{\Qq}{\mathbb{Q}}
\newcommand{\Rr}{\mathbb{R}}
\newcommand{\Zz}{\mathbb{Z}}

\newcommand{\rhodef}{\rho^{\phantom{*}}_{{\rm def}}}
\newcommand{\rhotriv}{\rho^{\phantom{*}}_{{\rm triv}}}
\newcommand{\rhosign}{\rho^{\phantom{*}}_{{\rm sign}}}
\newcommand{\rhoreg}{\rho^{\phantom{*}}_{{\rm reg}}}
\newcommand{\chidef}{\chi^{\phantom{*}}_{{\rm def}}}
\newcommand{\chitriv}{\chi^{\phantom{*}}_{{\rm triv}}}
\newcommand{\chisign}{\chi^{\phantom{*}}_{{\rm sign}}}
\newcommand{\chireg}{\chi^{\phantom{*}}_{{\rm reg}}}
\newcommand{\scp}[2]{\left\langle #1,\:#2\right\rangle_G}
\newcommand{\scpH}[2]{\left\langle #1,\:#2\right\rangle_H}

\newcounter{probno}
\setcounter{probno}{0}
\newcounter{partno}
\setcounter{partno}{0}
%% versions that don't print the number of points
\newcommand{\prob}{
  \vskip10bp%
  \setcounter{partno}{0}%
  \addtocounter{probno}{1}%
  {\bf Problem~\#{\arabic{probno}}}\quad}
\newcommand{\probpart}{%\rule{0in}{0in}\\ \phantom{xxx}
  \addtocounter{partno}{1}%
  {\bf (\#\arabic{probno}\alph{partno})}\ \ }
\newcommand{\probcont}{%
  {\bf Problem~\#{\arabic{probno}}}~(\emph{continued})}
\newcommand{\probo}{
  \setcounter{partno}{0}%
  \addtocounter{probno}{1}%
  {\bf (\#\arabic{probno})}\ \ }
%% versions that do print the number of points
\newcommand{\Prob}[1]{
  \vskip10bp%
  \setcounter{partno}{0}%
  \addtocounter{probno}{1}%
  {\bf Problem~\#{\arabic{probno}}~[{#1}~pts]}\quad}
\newcommand{\Probpart}[1]{%\rule{0in}{0in}\\ \phantom{xxx}
  \addtocounter{partno}{1}%
  {\bf (\#\arabic{probno}\alph{partno})~[{#1}~pts]}\ \ }

%\renewcommand{\thefootnote}{\fnsymbol{footnote}}

\usepackage{youngtab}
\begin{document}
\thispagestyle{empty}
{\bf Math 821, Spring 2014\\
Problem Set \#5\\
Due date: Friday, April 18}

\prob [Hatcher p.131 \#11] Show that if $A$ is a retract of $X$
then the map $H_n(A)\to H_n(X)$ induced by the inclusion $A\subset X$ is
injective.

\prob (a) [Hatcher p.132 \#15] Homological algebra warmup: Prove that if $A\xrightarrow{f}B\xrightarrow{g}C\xrightarrow{h}D\xrightarrow{j}E$ is exact with $f$ surjective and $j$ injective, then $C=0$.

(b) Prove the \emph{Snake Lemma}: if the commutative diagram
\[\xymatrix{
0\ar@{-->}[r] & A\rto^d\dto_f & B\rto^e\dto_g & C\rto\dto_h & 0\\
0\rto & A'\rto^{d'} & B'\rto^{e'} & C'\ar@{-->}[r] & 0
}\]
of abelian groups has exact rows, then there is an exact sequence
\[0\dashrightarrow \ker f\xrightarrow{\alpha} \ker g\xrightarrow{\beta}
  \ker h\xrightarrow{\gamma} \coker f\xrightarrow{\delta}
 \coker g\xrightarrow{\varepsilon}\coker h\dashrightarrow 0.\]

(The dashed arrows can be either included or omitted from both
 diagrams.  Both versions of the result are commonly referred to as
 the Snake Lemma.  In your solution, prove the version without the
 dashed arrows and then observe what happens if the arrows are
 included.)

\excise{
(b) Prove the \emph{snake lemma}: if the commutative diagram
\[\xymatrix{
& A\rto^d\dto_f & B\rto^e\dto_g & C\rto\dto_h & 0\\
0\rto & A'\rto^{d'} & B'\rto^{e'} & C'
}\]
has exact rows, then there is an exact sequence
\[0\to \ker f\to \ker g\to \ker h\to \coker f\to\coker g\to\coker h\to 0.\]
}

\prob Recall that the \emph{torsion subgroup} $T(G)$ of an abelian group is the subgroup consisting of all elements of finite order.  Let $0\to A\xrightarrow{f} B\xrightarrow{g} C\to 0$ be a short exact sequence of finitely generated $\Zz$-modules.

(a) Show that if $C$ is free abelian, then $T(A)\isom T(B)$.

(b) Show that $A$ free abelian does not necessarily imply that $T(B)=T(C)$.

\bigskip

\bigskip

\yell{\bf\boldmath In the following problems, you may use the result of Proposition 2.22, namely that $H_n(X,A)\isom\HH_n(X/A)$ for all $n$ and all good pairs $(X,A)$.\unboldmath}

\bigskip

\prob [Hatcher p.132 \#17] (a) Compute the homology groups $H_n(X,A)$ when $X$ is $S^2$ or $S^1\x S^1$, and~$A$ is a set of $k$ points in $X$ with $k<\infty$.  You may use the computation of the homology groups of $X$ from \S2.1.

(b) Compute the groups $H_n(X,A)$ and $H_n(X,B)$, where $X$ is a closed orientable surface of genus two with $A$ and $B$ the circles shown.  (What are $X/A$ and $X/B$?)
\includefigure{2in}{0.75in}{Hatcher-p132-prob17.pdf}

\pagebreak

\prob [Hatcher p.132 \#20] The \textbf{suspension} $SX$ of a space $X$ is obtained by taking two copies of the cone
$CX=X\x[0,1]/X\x\{1\}$ and attaching them along their bases.  Equivalently, take a prism over $X$ and successively contract each of the top and bottom faces to points:
\[SX = X \x [0,1] ~/~ X\x\{0\} ~/~ X\x\{1\}.\]
For example, the suspension of $S^n$ is $S^{n+1}$.

Prove that $\HH_n(SX)\isom \HH_{n-1}(X)$ for all $n>0$.  More generally, for any integer $k$, compute the reduced homology groups of the union of $k$ copies of $CX$ with their bases identified.  (The suspension is the case $k=2$.)

\prob Let $n\leq d\geq 0$ and let $X=\Delta^{n,d}$ denote the $d$-skeleton of the $n$-dimensional simplex (whose vertices are $v_0,v_1,\dots,v_n$).  Most of you conjectured last time that the reduced homology groups of $X$ are given by
\[\HH_k(X) = \begin{cases}
\Zz^{\raisebox{3.5bp}{$\binom{n}{d+1}$}} & \text{ if $k=d$,}\\
0 & \text{ if $k<d$}.
\end{cases}\]
This conjecture is correct.  Prove it without writing down any explicit simplicial boundary matrices.

\vfill\hrule\vfill

\textbf{\Large Appendix: Making commutative diagrams in LaTeX}

The \texttt{xypic} package provides a way to typeset commutative diagrams
in LaTeX.  For instance, consider the following
diagram, which arises in the proof of Theorem 2.10 in Hatcher:
$$
\xymatrix{
\cdots\ar[r] & C_{n+1}(X) \ar[r]^{\bd}\ar[d]^{i_\#} & C_{n}(X) \ar[r]^{\bd} \ar[d]^{i_\#} \ar[dl]^{P} & C_{n-1}(X) \ar[r] \ar[d]^{i_\#} \ar[dl]^{P} &\cdots\\
\cdots\ar[r] & C_{n+1}(Y) \ar[r]_{\bd} & C_{n}(Y) \ar[r]_{\bd} & C_{n-1}(Y)\ar[r] &\cdots\\
}$$
It can be typeset as follows:
\begin{verbatim}
$$\xymatrix{
\cdots\ar[r]
  & C_{n+1}(X) \ar[r]^{\bd} \ar[d]^{i_\#}
  & C_{n}(X)   \ar[r]^{\bd} \ar[d]^{i_\#} \ar[dl]^{P}
  & C_{n-1}(X) \ar[r]       \ar[d]^{i_\#} \ar[dl]^{P}
  &\cdots\\
\cdots\ar[r]
  & C_{n+1}(Y) \ar[r]_{\bd}
  & C_{n}(Y)   \ar[r]_{\bd}
  & C_{n-1}(Y) \ar[r]
  &\cdots}$$
\end{verbatim}
This is like a \texttt{tabular} or \texttt{array} environment:
the \& symbols are delimiters between columns.  The $\backslash$\texttt{ar}
commands create arrows emanating from the current cell in the table, with
the code in [square brackets] specifying where the arrow should point;
e.g., \texttt{$\backslash$ar[dl]} makes an arrow pointing towards the
cell one row down and one column left of the current cell.

For more details, here are hyperlinks to the \fbox{\href{www.jlmartin.faculty.ku.edu/math821/xyguide.pdf}{XY User's Guide}} and the \fbox{\href{http://xy-pic.sourceforge.net/}{website.}}
\end{document}

