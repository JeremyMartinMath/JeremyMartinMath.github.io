\documentclass{amsart}
\usepackage{amssymb,amsmath,amsthm,mathrsfs,graphics,hyperref,ifthen,framed,cancel,fullpage,color,ytableau,tcolorbox,bm,tikz}
\usepackage[enableskew]{youngtab}
\raggedbottom
\parskip=10bp
\parindent=0bp
\raggedbottom

%%%%%%%%%%%%%%%%   Colors for TikZ and text  %%%%%%%%%%%%%%%%%

\definecolor{light}{gray}{.75}
\definecolor{med}{gray}{.5}
\definecolor{dark}{gray}{.25}
\newcommand{\Red}[1]{{\color{red}{#1}}}
\newcommand{\RED}[1]{{\color{red}{\boldmath\textbf{#1}\unboldmath}}}
\newcommand{\Blue}[1]{{\color{blue}{#1}}}
\newcommand{\BLUE}[1]{{\color{blue}{\boldmath\textbf{#1}\unboldmath}}}

% Hyperlinks
\hypersetup{colorlinks, citecolor=red, filecolor=black, linkcolor=blue, urlcolor=blue}
\newcommand{\hreftext}[2]{\href{#1}{\Blue{#2}}} % for text anchors
\newcommand{\hrefurl}[2]{\href{#1}{\Blue{\tt #2}}} % if you want to actually write out the URL in text

% Spacing and commands
\newcommand{\bigpad}{\rule[-14mm]{0mm}{30mm}}
\newcommand{\smallpad}{\rule[-1.5mm]{0mm}{5mm}}
\newcommand{\pad}{\rule[-3mm]{0mm}{8mm}}
\newcommand{\padup}{\rule{0mm}{5mm}}
\newcommand{\paddown}{\rule[-3mm]{0mm}{2mm}}
\newcommand{\blank}{\rule{1.25in}{0.25mm}}
\newcommand{\yell}[1]{\fbox{\rule[-1mm]{0mm}{4mm} \large\bf #1 }}
\newcommand{\indnt}{\phantom{.}\qquad}
\newcommand{\littleline}{\begin{center}\rule{4in}{0.5bp}\end{center}}

% macros for inserting figures
\newcommand{\includefigure}[3]{\begin{center}\resizebox{#1}{#2}{\includegraphics{#3}}\end{center}}
\newcommand{\includefigurewithinmath}[3]{\resizebox{#1}{#2}{\includegraphics{#3}}}
% E.g., to insert a standalone figure of width 3" and height 1.5", use
% \includefigure{3in}{1.5in}{foo.pdf}

% math operators
\DeclareMathOperator{\Comp}{Comp}
\DeclareMathOperator{\Fix}{Fix}
\DeclareMathOperator{\id}{id}
\DeclareMathOperator{\im}{im}
\DeclareMathOperator{\lcm}{lcm}
\DeclareMathOperator{\Par}{Par}
\DeclareMathOperator{\rank}{rank}
\DeclareMathOperator{\sh}{sh}
\DeclareMathOperator{\tr}{tr}
\DeclareMathOperator{\wt}{wt}

% theorem environments, automatically numbered
\newtheorem{theorem}{Theorem}[section]
\newtheorem{proposition}[theorem]{Proposition}
\newtheorem{lemma}[theorem]{Lemma}
\newtheorem{corollary}[theorem]{Corollary}
\theoremstyle{definition}
\newtheorem{definition}[theorem]{Definition}
\newtheorem{example}[theorem]{Example}
\newtheorem{remark}[theorem]{Remark}
\newtheorem{problem}[theorem]{Problem}

\newcommand{\skpr}{\emph{Sketch of proof: }}
\newcommand{\soln}{\textit{Solution:\ }}

% Generally useful macros
\newcommand{\excise}[1]{} % useful for commenting out large chunks
\newcommand{\0}{\emptyset}
\newcommand{\compn}{\models} % compositions
\newcommand{\dju}{\mathaccent\cdot\cup} % disjoint union
\newcommand{\dsum}{\displaystyle\sum}
\newcommand{\fallfac}[2]{{#1}^{\underline{#2}}} % falling factorial
\newcommand{\isom}{\cong} % isomorphism symbol
\newcommand{\partn}{\vdash} % partition symbol
\newcommand{\qqandqq}{\qquad\text{and}\qquad}
\newcommand{\qandq}{\quad\text{and}\quad}
\newcommand{\qand}{\quad\text{and}}
\newcommand{\qbin}[2]{{\begin{bmatrix}#1\\#2\end{bmatrix}_q}} % q-binomial coefficient
\newcommand{\risefac}[2]{{#1}^{\overline{#2}}} % rising factorial
\newcommand{\sd}{\triangle} % symmetric difference
\newcommand{\sm}{\setminus} % don't use a minus sign for this
\newcommand{\st}{\colon} % "such that"
\newcommand{\surj}{\twoheadrightarrow}
\newcommand{\x}{\times}

% blackboard bold fonts for sets of numbers
\newcommand{\Cc}{\mathbb{C}} % complex numbers
\newcommand{\Ff}{\mathbb{F}} % finite field
\newcommand{\Nn}{\mathbb{N}} % natural numbers
\newcommand{\Qq}{\mathbb{Q}}
\newcommand{\Rr}{\mathbb{R}}
\newcommand{\Zz}{\mathbb{Z}}

% miscellaneous
\newcommand{\TwoCases}[4]{\begin{cases}{#1}&\text{ #2}\\{#3}&\text{ #4}\end{cases}}
\newcommand{\ThreeCases}[6]{\begin{cases}{#1}&\text{ #2}\\{#3}&\text{ #4}\\{#5}&\text{ #6}\end{cases}}
\newcommand{\bridgehand}[4]{\spadesuit\ {\textsf{#1}}\ \ \heartsuit\ {\textsf{#2}}\ \ \diamondsuit\ {\textsf{#3}}\ \ \clubsuit\ {\textsf{#4}}}

% macros for automatic problem numbering --- students don't have to use these
\newcounter{probno}
\setcounter{probno}{0}
\newcounter{partno}
\setcounter{partno}{0}
%% versions that don't print the number of points
\newcommand{\prob}{
  \vskip10bp%
  \setcounter{partno}{0}%
  \addtocounter{probno}{1}%
  {\bf Problem~\#{\arabic{probno}}}\quad}

% No initial whitespace to make framing look nicer
\newcommand{\probns}{
  \setcounter{partno}{0}%
  \addtocounter{probno}{1}%
  {\bf Problem~\#{\arabic{probno}}}\quad}

\newcommand{\probpart}{%
  \addtocounter{partno}{1}%
  {\bf (\#\arabic{probno}\alph{partno})}\ \ }
\newcommand{\probcont}{%
  {\bf Problem~\#{\arabic{probno}}}~(\emph{continued})}
\newcommand{\probo}{
  \setcounter{partno}{0}%
  \addtocounter{probno}{1}%
  {\bf (\#\arabic{probno}}\ \ }

%% versions that do print the number of points
\newcommand{\Prob}[1]{
  \vskip10bp%
  \setcounter{partno}{0}%
  \addtocounter{probno}{1}%
  {\bf Problem~\#{\arabic{probno}}~[{#1}~pts]}\quad}

%no initial whitespace
\newcommand{\Probns}[1]{
  \setcounter{partno}{0}%
  \addtocounter{probno}{1}%
  {\bf Problem~\#{\arabic{probno}}~[{#1}~pts]}\quad}
\newcommand{\Probpart}[1]{%\rule{0in}{0in}\\ \phantom{xxx}
  \addtocounter{partno}{1}%
  {\bf (\#\arabic{probno}\alph{partno})~[{#1}~pts]}\ \ }

\newboolean{answers}
\newcommand{\Answer}[1]{\ifthenelse{\boolean{answers}}{{\bf Answer:}\ #1}{}\bigskip}

\begin{document}
\thispagestyle{empty}
{\bf Math 725, Spring 2016 \hfill Problem Set \#6}
\smallskip\hrule

{\bf Instructions:} Do all problems and typeset them in \LaTeX.  E-mail the PDF file to Jeremy at \hrefurl{mailto:jlmartin@ku.edu}{\tt jlmartin@ku.edu} under the filename {\tt your-last-name.pdf} by {\bf Friday, April 22, 5:00pm.}
You are encouraged to use the \hreftext{http://www.jlmartin.faculty.ku.edu/math725/header.tex}{LaTeX header file} and to refer to Jeremy's \hreftext{http://www.jlmartin.faculty.ku.edu/math725/notesnew.pdf}{lecture notes}.

\smallskip\hrule

\prob Complete the proof of Theorem 7.8 (connecting the deletion/contraction recurrence and the closed form of the Tutte polynomial) by showing that $\tilde T(G)=x\cdot T(G/e)$ if $e$ is a bridge of $G$.  (Hint: Mimic the proof for the loop case.)

\prob Prove that $T(C_n;x,y)=x^{n-1}+\cdots+x^2+x+y$ in two different ways:\\
\probpart using the corank/nullity generating function;\\
\probpart using the deletion/contraction recurrence and induction.

Then, convince yourself that the numbers of spanning trees, acyclic orientations and strong orientations of $C_n$, as well as its chromatic polynomial, are indeed given by the Tutte polynomial specializations we discussed in class.  (For the chromatic polynomial, see Problem Set \#5, Problem \#3.)
 
\prob Denote by $G^{(p)}$ the \emph{$p^{th}$ parallel extension of $G$}, i.e., the graph formed from $G$ by replacing every edge of $G$ with $p$ parallel copies of itself.  Find a formula for the Tutte polynomial of $G$ in terms of that of $G$.  (For example, if $G=K_2$, then $G^{(p)}\isom C_p^*$, the graph with two vertices and $p$ parallel edges between them, so $T(G^{(p)};x,y)=x+y+y^2+\cdots+y^{p-1}$.)  {\bf Hint:} For $A\subseteq E(G^{(p)}$, let $\tilde A$ be the set of edges of $G$ with at least one copy in $A$.  How do $r(\tilde A)$ and $r(A)$ compare?  Write down the corank/nullity generating function for $T(G^{(p)};x,y)$, then break it into pieces depending on what $\tilde A$ is.

\prob Let $\Gamma$ be a plane graph, which you can assume is connected.

\probpart Prove that $(\Gamma-e)^*=\Gamma^*/e^*$ (provided that $e$ is not a bridge) and that $(\Gamma/e)^*=\Gamma^*-e^*$ (provided that $e$ is not a loop).\\
\probpart We proved in class that $T(\Gamma;\,x,y)=T(\Gamma^*;\,y,x)$ by comparing the rank functions in $\Gamma$ and $\Gamma^*$.  Give another proof, using part (a) and the deletion/contraction recurrence for the Tutte polynomial.

\prob Let $\Gamma$ be a connected plane graph with no loops and let $\Delta$ be an orientation of $\Gamma$.  The \defterm{dual orientation} $\Delta^*$ of $\Gamma^*$ is defined as follows: if you stand at the tail vertex of $e\in E(G)$ and look toward the head, then the dual edge $e^*$ points to the right.
\includefigure{2.6in}{1.8in}{oriented-dual} % 3.25 x 2.25
Prove that $\Delta$ is an acyclic orientation of $\Gamma$ if and only $\Delta^*$ is a strong orientation of $\Gamma^*$.  (This explains why the numbers of acyclic and strong orientations are given by $T_G(2,0)$ and $T_G(0,2)$, respectively.)

\end{document}