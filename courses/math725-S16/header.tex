\documentclass[10pt]{amsart}
%\documentclass[10pt]{article}
\usepackage{amsthm,amsmath,amssymb,mathrsfs,setspace,graphicx,mathabx,tcolorbox,color,tikz,enumitem,hyperref,cancel,graphics,bm,fullpage,esvect}
\usepackage[normalem]{ulem}
%\usepackage[color,curve]{xypic}
\parskip=10bp
\parindent=0bp
\raggedbottom

%%%%%%%%%%%%%%%%%%%%   Essentials   %%%%%%%%%%%%%%%%%%%%

\newcommand{\excise}[1]{}
\newcommand{\pagehead}[1]{\pagebreak\Large\begin{center}{\bf #1}\paddown\end{center}\normalsize\hrule\bigskip}
\newcommand{\defterm}[1]{\boldmath\textbf{#1}\unboldmath}
\newcommand{\littleline}{\begin{center}\rule{4in}{0.5bp}\end{center}}

%%%%%%%%%%%%%%%%   Colors for TikZ and text  %%%%%%%%%%%%%%%%%

\definecolor{light}{gray}{.75}
\definecolor{med}{gray}{.5}
\definecolor{dark}{gray}{.25}
\newcommand{\Red}[1]{{\color{red}{#1}}}
\newcommand{\RED}[1]{{\color{red}{\boldmath\textbf{#1}\unboldmath}}}
\newcommand{\Blue}[1]{{\color{blue}{#1}}}
\newcommand{\BLUE}[1]{{\color{blue}{\boldmath\textbf{#1}\unboldmath}}}
\newcommand{\Green}[1]{{\color{green}{#1}}}
\newcommand{\GREEN}[1]{{\color{green}{\boldmath\textbf{#1}\unboldmath}}}

%%%%%%%%%%%%%%%%%%%%   Hyperlinks   %%%%%%%%%%%%%%%%%%%%

\newcommand{\hreftext}[2]{\href{#1}{\Blue{#2}}}
\newcommand{\hrefurl}[2]{\href{#1}{\Blue{\tt #2}}}

%%%%%%%%%%%%%%%%%%%%   Figures   %%%%%%%%%%%%%%%%%%%%

%% (Note: I keep my figure files in a separate folder.  If you keep your figure files in the same folder as your .tex files, remove the "figs/" from these macros.)
\newcommand{\includefigure}[3]{\begin{center}\resizebox{#1}{#2}{\includegraphics{figs/#3}}\end{center}}
\newcommand{\includefigurewithinmath}[3]{\resizebox{#1}{#2}{\includegraphics{figs/#3}}}
\newcommand{\insertfigure}[3]{\resizebox{#1}{#2}{\includegraphics{figs/#3}}}

%%%%%%%%%%%%%%%%%%%%   Padding   %%%%%%%%%%%%%%%%%%%%

\newcommand{\xxxx}{\phantom{XXXX}}
\newcommand{\indnt}{\rule{0.5cm}{0cm}}
\newcommand{\doubleindnt}{\rule{1cm}{0cm}}
\newcommand{\pad}{\rule[-3mm]{0mm}{8mm}}
\newcommand{\padup}{\rule{0mm}{5mm}}
\newcommand{\paddown}{\rule[-3mm]{0mm}{3mm}}
\newcommand{\bpad}{\rule[-5mm]{0mm}{12mm}}
\newcommand{\bpadup}{\rule{0mm}{7mm}}
\newcommand{\bpaddown}{\rule[-5mm]{0mm}{5mm}}
\newcommand{\spad}{\rule[-2mm]{0mm}{6mm}}
\newcommand{\spadup}{\rule{0mm}{5mm}}
\newcommand{\spaddown}{\rule[-1mm]{0mm}{1mm}}

%%%%%%%%%%%%%%%%%%%%   Bullets and such   %%%%%%%%%%%%%%%%%%%%

\newcommand{\bang}{\noindent $\bullet$ \quad}
\newcommand{\pbang}{\noindent\phantom{$\bullet$}\quad}
\newcommand{\ibang}{\phantom{xxxxx} $\bullet$ \quad}
\newcommand{\subbang}{\noindent {}${}\quad$ $\circ$ \quad}
\newcommand{\shout}[1]{\fbox{\parbox{6.3in}{{#1}}}}
\newcommand{\yell}[1]{\fbox{\rule[-1.5mm]{0mm}{5mm}\bf #1 }}
\newcommand{\yellpar}[1]{\fbox{\parbox{6.3in}{{#1}}}}
\newcommand{\skpr}{\textit{Sketch of proof:\ }}

%%%%%%%%%%%%%%%%%%%%   Theorem-like environments, invoked with   %%%%%%%%%%%%%%%%%%%%
%%%%%%%%%%%%%%%%%%%%   \begin{theorem} ... \end{theorem}         %%%%%%%%%%%%%%%%%%%%
%%%%%%%%%%%%%%%%%%%%   and numbered automatically by LaTeX       %%%%%%%%%%%%%%%%%%%%

\newtheorem{theorem}{Theorem}
\numberwithin{theorem}{section}
\newtheorem{proposition}[theorem]{Proposition}
\newtheorem{defprop}[theorem]{Definition/Proposition}
\newtheorem{corollary}[theorem]{Corollary}
\newtheorem{lemma}[theorem]{Lemma}
\newtheorem{conjecture}[theorem]{Conjecture}
\newtheorem{question}[theorem]{Question}
\theoremstyle{definition}
\newtheorem{definition}[theorem]{Definition}
\newtheorem{remark}[theorem]{Remark}
\newtheorem{example}[theorem]{Example}
\newcommand{\exa}{\textbf{Example:} \ \ } %% unnumbered -- use discouraged
\newcommand{\warning}{\RED{Warning:}\ }

%%%%%%%%%%%%%%%%%%%%   Logical connectors in the   %%%%%%%%%%%%%%%%%%%%
%%%%%%%%%%%%%%%%%%%%    middle of display math     %%%%%%%%%%%%%%%%%%%%

\newcommand{\caseif}{&\text{ if }}
\newcommand{\caseelse}{&\text{ otherwise.}}
\newcommand{\IF}{\text{ if }}
\newcommand{\AND}{\text{ and }}
\newcommand{\OR}{\text{ or }}
\newcommand{\qandq}{\quad\text{and}\quad}
\newcommand{\qorq}{\quad\text{or}\quad}
\newcommand{\qergoq}{\quad\therefore\quad}
\newcommand{\qqandqq}{\qquad\text{and}\qquad}
\newcommand{\qqorqq}{\qquad\text{or}\qquad}
\newcommand{\qqergoqq}{\qquad\therefore\qquad}
\newcommand{\qand}{\quad\text{and}}

%%%%%%%%%%%%%%%%%%%%   Fancy letters    %%%%%%%%%%%%%%%%%%%%

%% Sets of numbers
\newcommand{\Cc}{{\mathbb C}}
\newcommand{\Nn}{{\mathbb N}}
\newcommand{\Qq}{{\mathbb Q}}
\newcommand{\Rr}{{\mathbb R}}
\newcommand{\Zz}{{\mathbb Z}}

\renewcommand{\AA}{\mathcal{A}}
\newcommand{\FF}{\mathcal{F}}
\newcommand{\PP}{\mathcal{P}}
\newcommand{\II}{\mathcal{I}}
\newcommand{\TT}{\mathscr{T}}
\newcommand{\VV}{\mathcal{V}}
\newcommand{\ZZ}{\mathscr{Z}}

\newcommand{\xx}{\mathbf{x}}

%%%%%%%%%%%%%%%%%%%%   Math symbols    %%%%%%%%%%%%%%%%%%%%

\newcommand{\0}{\emptyset}				%% empty set symbol
\newcommand{\adj}{\leftrightarrow}			%% "adjacent to"
\newcommand{\bd}{\partial}
\newcommand{\bij}{\longleftrightarrow}			%% bijection
\newcommand{\dju}{\ensuremath{\mathaccent\cdot\cup}}	%% disjoint union
\newcommand{\dedge}[1]{\vv{#1}}				%% directed edge -- much better with esvect package!
\newcommand{\bedge}[1]{\reflectbox{\ensuremath{\vv{\reflectbox{\ensuremath{#1}}}}}}
	%% backward directed edge
\newcommand{\ep}{\varepsilon}
\newcommand{\eqdef}{\overset{{\rm def}}{=}}		%% "defined to be equal to"
\newcommand{\fin}{f_{\mathrm{in}}}			%% inflow
\newcommand{\fout}{f_{\mathrm{out}}}			%% outflow
\newcommand{\isom}{\cong}				%% "isomorphic to"
\newcommand{\jn}{\vee}					%% join
\newcommand{\sd}{\triangle}				%% symmetric difference
\newcommand{\sm}{\setminus}				%% backslash for difference of two sets
\newcommand{\Sptr}{{\mathscr T}}			%% set of spanning trees
\newcommand{\st}{~|~}					%% vertical bar for "such that"
\newcommand{\x}{\times}

%%%%%%%%%%%%%%%%%%%%   Math operators    %%%%%%%%%%%%%%%%%%%%

\DeclareMathOperator{\Aut}{Aut}				%% automorphism group
\DeclareMathOperator{\ch}{ch}
\DeclareMathOperator{\co}{co}
\DeclareMathOperator{\df}{def}
\DeclareMathOperator{\diam}{diam} 
\DeclareMathOperator{\dir}{dir}
\DeclareMathOperator{\girth}{girth} 
\DeclareMathOperator{\head}{head}
\DeclareMathOperator{\indeg}{indeg}
\DeclareMathOperator{\Inv}{Inv} 
\DeclareMathOperator{\length}{length}
\DeclareMathOperator{\nul}{nul}
\DeclareMathOperator{\outdeg}{outdeg}
\DeclareMathOperator{\rad}{rad}
\DeclareMathOperator{\rank}{rank}
\DeclareMathOperator{\rt}{root}
\DeclareMathOperator{\sign}{sign} 
\DeclareMathOperator{\tail}{tail}
\DeclareMathOperator{\ter}{ter}
\DeclareMathOperator{\val}{val}
\DeclareMathOperator{\wt}{wt}

%%%%%%%%%%%%%%   Automatic numbering of HW/test problems    %%%%%%%%%%%%%%

\newcounter{probno}
\setcounter{probno}{0}
\newcounter{partno}
\setcounter{partno}{0}
%% versions that don't print the number of points
\newcommand{\solprob}{\setcounter{partno}{0}\addtocounter{probno}{1}{\bf Problem~\#{\arabic{probno}}}\quad}
\newcommand{\prob}{\vskip10bp\setcounter{partno}{0}\addtocounter{probno}{1}{\bf Problem~\#{\arabic{probno}}}\quad}
\newcommand{\probpart}{\addtocounter{partno}{1}{\bf (\#\arabic{probno}\alph{partno})}\ \ }
\newcommand{\probcont}{{\bf Problem~\#{\arabic{probno}}}~(\emph{continued})}
%% versions that do print the number of points
\newcommand{\Prob}[1]{\vskip10bp\setcounter{partno}{0}\addtocounter{probno}{1}{\bf Problem~\#{\arabic{probno}}~[{#1}~pts]}\quad}
\newcommand{\Probnospace}[1]{\setcounter{partno}{0}\addtocounter{probno}{1}{\bf Problem~\#{\arabic{probno}}~[{#1}~pts]}\quad}
\newcommand{\Probpart}[1]{\addtocounter{partno}{1}{\bf (\#\arabic{probno}\alph{partno})~[{#1}~pts]}\ \ }
