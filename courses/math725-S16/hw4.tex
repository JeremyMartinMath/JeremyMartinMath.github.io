\documentclass{amsart}
\usepackage{amssymb,amsmath,amsthm,mathrsfs,graphics,hyperref,ifthen,framed,cancel,fullpage,color,ytableau,tcolorbox,bm,tikz}
\usepackage[enableskew]{youngtab}
\raggedbottom
\parskip=10bp
\parindent=0bp
\raggedbottom

%%%%%%%%%%%%%%%%   Colors for TikZ and text  %%%%%%%%%%%%%%%%%

\definecolor{light}{gray}{.75}
\definecolor{med}{gray}{.5}
\definecolor{dark}{gray}{.25}
\newcommand{\Red}[1]{{\color{red}{#1}}}
\newcommand{\RED}[1]{{\color{red}{\boldmath\textbf{#1}\unboldmath}}}
\newcommand{\Blue}[1]{{\color{blue}{#1}}}
\newcommand{\BLUE}[1]{{\color{blue}{\boldmath\textbf{#1}\unboldmath}}}

% Hyperlinks
\hypersetup{colorlinks, citecolor=red, filecolor=black, linkcolor=blue, urlcolor=blue}
\newcommand{\hreftext}[2]{\href{#1}{\Blue{#2}}} % for text anchors
\newcommand{\hrefurl}[2]{\href{#1}{\Blue{\tt #2}}} % if you want to actually write out the URL in text

% Spacing and commands
\newcommand{\bigpad}{\rule[-14mm]{0mm}{30mm}}
\newcommand{\smallpad}{\rule[-1.5mm]{0mm}{5mm}}
\newcommand{\pad}{\rule[-3mm]{0mm}{8mm}}
\newcommand{\padup}{\rule{0mm}{5mm}}
\newcommand{\paddown}{\rule[-3mm]{0mm}{2mm}}
\newcommand{\blank}{\rule{1.25in}{0.25mm}}
\newcommand{\yell}[1]{\fbox{\rule[-1mm]{0mm}{4mm} \large\bf #1 }}
\newcommand{\indnt}{\phantom{.}\qquad}
\newcommand{\littleline}{\begin{center}\rule{4in}{0.5bp}\end{center}}

% macros for inserting figures
\newcommand{\includefigure}[3]{\begin{center}\resizebox{#1}{#2}{\includegraphics{#3}}\end{center}}
\newcommand{\includefigurewithinmath}[3]{\resizebox{#1}{#2}{\includegraphics{#3}}}
% E.g., to insert a standalone figure of width 3" and height 1.5", use
% \includefigure{3in}{1.5in}{foo.pdf}

% math operators
\DeclareMathOperator{\Comp}{Comp}
\DeclareMathOperator{\Fix}{Fix}
\DeclareMathOperator{\id}{id}
\DeclareMathOperator{\im}{im}
\DeclareMathOperator{\lcm}{lcm}
\DeclareMathOperator{\Par}{Par}
\DeclareMathOperator{\rank}{rank}
\DeclareMathOperator{\sh}{sh}
\DeclareMathOperator{\tr}{tr}
\DeclareMathOperator{\wt}{wt}

% theorem environments, automatically numbered
\newtheorem{theorem}{Theorem}[section]
\newtheorem{proposition}[theorem]{Proposition}
\newtheorem{lemma}[theorem]{Lemma}
\newtheorem{corollary}[theorem]{Corollary}
\theoremstyle{definition}
\newtheorem{definition}[theorem]{Definition}
\newtheorem{example}[theorem]{Example}
\newtheorem{remark}[theorem]{Remark}
\newtheorem{problem}[theorem]{Problem}

\newcommand{\skpr}{\emph{Sketch of proof: }}
\newcommand{\soln}{\textit{Solution:\ }}

% Generally useful macros
\newcommand{\excise}[1]{} % useful for commenting out large chunks
\newcommand{\0}{\emptyset}
\newcommand{\compn}{\models} % compositions
\newcommand{\dju}{\mathaccent\cdot\cup} % disjoint union
\newcommand{\dsum}{\displaystyle\sum}
\newcommand{\fallfac}[2]{{#1}^{\underline{#2}}} % falling factorial
\newcommand{\isom}{\cong} % isomorphism symbol
\newcommand{\partn}{\vdash} % partition symbol
\newcommand{\qqandqq}{\qquad\text{and}\qquad}
\newcommand{\qandq}{\quad\text{and}\quad}
\newcommand{\qand}{\quad\text{and}}
\newcommand{\qbin}[2]{{\begin{bmatrix}#1\\#2\end{bmatrix}_q}} % q-binomial coefficient
\newcommand{\risefac}[2]{{#1}^{\overline{#2}}} % rising factorial
\newcommand{\sd}{\triangle} % symmetric difference
\newcommand{\sm}{\setminus} % don't use a minus sign for this
\newcommand{\st}{\colon} % "such that"
\newcommand{\surj}{\twoheadrightarrow}
\newcommand{\x}{\times}

% blackboard bold fonts for sets of numbers
\newcommand{\Cc}{\mathbb{C}} % complex numbers
\newcommand{\Ff}{\mathbb{F}} % finite field
\newcommand{\Nn}{\mathbb{N}} % natural numbers
\newcommand{\Qq}{\mathbb{Q}}
\newcommand{\Rr}{\mathbb{R}}
\newcommand{\Zz}{\mathbb{Z}}

% miscellaneous
\newcommand{\TwoCases}[4]{\begin{cases}{#1}&\text{ #2}\\{#3}&\text{ #4}\end{cases}}
\newcommand{\ThreeCases}[6]{\begin{cases}{#1}&\text{ #2}\\{#3}&\text{ #4}\\{#5}&\text{ #6}\end{cases}}
\newcommand{\bridgehand}[4]{\spadesuit\ {\textsf{#1}}\ \ \heartsuit\ {\textsf{#2}}\ \ \diamondsuit\ {\textsf{#3}}\ \ \clubsuit\ {\textsf{#4}}}

% macros for automatic problem numbering --- students don't have to use these
\newcounter{probno}
\setcounter{probno}{0}
\newcounter{partno}
\setcounter{partno}{0}
%% versions that don't print the number of points
\newcommand{\prob}{
  \vskip10bp%
  \setcounter{partno}{0}%
  \addtocounter{probno}{1}%
  {\bf Problem~\#{\arabic{probno}}}\quad}

% No initial whitespace to make framing look nicer
\newcommand{\probns}{
  \setcounter{partno}{0}%
  \addtocounter{probno}{1}%
  {\bf Problem~\#{\arabic{probno}}}\quad}

\newcommand{\probpart}{%
  \addtocounter{partno}{1}%
  {\bf (\#\arabic{probno}\alph{partno})}\ \ }
\newcommand{\probcont}{%
  {\bf Problem~\#{\arabic{probno}}}~(\emph{continued})}
\newcommand{\probo}{
  \setcounter{partno}{0}%
  \addtocounter{probno}{1}%
  {\bf (\#\arabic{probno}}\ \ }

%% versions that do print the number of points
\newcommand{\Prob}[1]{
  \vskip10bp%
  \setcounter{partno}{0}%
  \addtocounter{probno}{1}%
  {\bf Problem~\#{\arabic{probno}}~[{#1}~pts]}\quad}

%no initial whitespace
\newcommand{\Probns}[1]{
  \setcounter{partno}{0}%
  \addtocounter{probno}{1}%
  {\bf Problem~\#{\arabic{probno}}~[{#1}~pts]}\quad}
\newcommand{\Probpart}[1]{%\rule{0in}{0in}\\ \phantom{xxx}
  \addtocounter{partno}{1}%
  {\bf (\#\arabic{probno}\alph{partno})~[{#1}~pts]}\ \ }

\newboolean{answers}
\newcommand{\Answer}[1]{\ifthenelse{\boolean{answers}}{{\bf Answer:}\ #1}{}\bigskip}

\begin{document}
\thispagestyle{empty}
{\bf Math 725, Spring 2016 \hfill Problem Set \#4}
\smallskip\hrule

{\bf Instructions:} Do all problems and typeset them in \LaTeX.  E-mail the PDF file to Jeremy at \hrefurl{mailto:jlmartin@ku.edu}{\tt jlmartin@ku.edu} under the filename {\tt your-last-name.pdf} by {\bf Friday, March 11, 5:00pm.}
You are encouraged to use the \hreftext{http://www.jlmartin.faculty.ku.edu/math725/header.tex}{LaTeX header file} and to refer to Jeremy's \hreftext{http://www.jlmartin.faculty.ku.edu/math725/notesnew.pdf}{lecture notes}.

\smallskip\hrule

\prob Let $G$ be a $k$-regular graph of even order that is $(k-1)$-edge-connected (i.e., $G-A$ is connected for any edge set $A$ of size $k-2$ or less). Prove that $G$ has a perfect matching.  (Hint: Adapt the proof of Petersen's theorem, Thm.~3.26 in the lecture notes.)

\prob Let $G$ be a simple graph.\\
\probpart Let $C,C'\subseteq G$ be cycles.  Prove that $C\sd C'$ is the (edge-)disjoint union of cycles.\\
\probpart Let $B,B'\subseteq G$ be cuts.  Prove that $B\sd B'$ is the (edge-)disjoint union of bonds.

\prob Let $G$ be a connected simple graph, and let $F\subseteq E(G)$ be a nonempty set of edges.  Prove that $F$ is an edge cut if and only if $|F\cap C|$ is even for every cycle $C\subseteq G$.

\prob Derive the K\"onig-Egerv\'ary Theorem from the Max-Flow/Min-Cut Theorem.  (Hint: Given an arbitrary $X,Y$-bipartite graph $G$, transform it into a source-sink network $N$ in such a way that flows and cuts in $N$ correspond respectively to matchings and vertex covers in $G$.)

\prob Prove Dilworth's Theorem by the following steps:

\probpart Prove that if $\PP=\{P_1,\dots,P_n\}$ is any path cover of a digraph $D$ and $I=\{v_1,\dots,v_m\}$ is an independent set in $D$, then $n\geq m$.  (This is weak duality, and it is the easier part.)

\probpart Given a digraph $D$, define its \emph{split} as the undirected, bipartite graph $G=S(D)$ with two vertices $v^+,v^-$
for each vertex $v\in V(D)$, and an edge $u^+v^-$ for each edge $\dedge{uv}$ in $D$.
(This is a similar construction to the one used to derive vertex-Menger from MFMC.)

\includefigure{6in}{2in}{split1} % If  you use this .tex file as a template, you don't have to include the figure - just comment this line out.

Now apply the K\"onig-Egerv\'ary Theorem to $S(D)$  and translate the result into Dilworth's Theorem for $D$.

\end{document}
