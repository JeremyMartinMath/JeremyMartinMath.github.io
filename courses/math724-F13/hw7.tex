\documentclass{amsart}
\usepackage{amssymb,amsmath,amsthm,mathrsfs,graphics,hyperref,ifthen,framed,cancel,fullpage,color,ytableau,tcolorbox,bm,tikz}
\usepackage[enableskew]{youngtab}
\raggedbottom
\parskip=10bp
\parindent=0bp
\raggedbottom

%%%%%%%%%%%%%%%%   Colors for TikZ and text  %%%%%%%%%%%%%%%%%

\definecolor{light}{gray}{.75}
\definecolor{med}{gray}{.5}
\definecolor{dark}{gray}{.25}
\newcommand{\Red}[1]{{\color{red}{#1}}}
\newcommand{\RED}[1]{{\color{red}{\boldmath\textbf{#1}\unboldmath}}}
\newcommand{\Blue}[1]{{\color{blue}{#1}}}
\newcommand{\BLUE}[1]{{\color{blue}{\boldmath\textbf{#1}\unboldmath}}}

% Hyperlinks
\hypersetup{colorlinks, citecolor=red, filecolor=black, linkcolor=blue, urlcolor=blue}
\newcommand{\hreftext}[2]{\href{#1}{\Blue{#2}}} % for text anchors
\newcommand{\hrefurl}[2]{\href{#1}{\Blue{\tt #2}}} % if you want to actually write out the URL in text

% Spacing and commands
\newcommand{\bigpad}{\rule[-14mm]{0mm}{30mm}}
\newcommand{\smallpad}{\rule[-1.5mm]{0mm}{5mm}}
\newcommand{\pad}{\rule[-3mm]{0mm}{8mm}}
\newcommand{\padup}{\rule{0mm}{5mm}}
\newcommand{\paddown}{\rule[-3mm]{0mm}{2mm}}
\newcommand{\blank}{\rule{1.25in}{0.25mm}}
\newcommand{\yell}[1]{\fbox{\rule[-1mm]{0mm}{4mm} \large\bf #1 }}
\newcommand{\indnt}{\phantom{.}\qquad}
\newcommand{\littleline}{\begin{center}\rule{4in}{0.5bp}\end{center}}

% macros for inserting figures
\newcommand{\includefigure}[3]{\begin{center}\resizebox{#1}{#2}{\includegraphics{#3}}\end{center}}
\newcommand{\includefigurewithinmath}[3]{\resizebox{#1}{#2}{\includegraphics{#3}}}
% E.g., to insert a standalone figure of width 3" and height 1.5", use
% \includefigure{3in}{1.5in}{foo.pdf}

% math operators
\DeclareMathOperator{\Comp}{Comp}
\DeclareMathOperator{\Fix}{Fix}
\DeclareMathOperator{\id}{id}
\DeclareMathOperator{\im}{im}
\DeclareMathOperator{\lcm}{lcm}
\DeclareMathOperator{\Par}{Par}
\DeclareMathOperator{\rank}{rank}
\DeclareMathOperator{\sh}{sh}
\DeclareMathOperator{\tr}{tr}
\DeclareMathOperator{\wt}{wt}

% theorem environments, automatically numbered
\newtheorem{theorem}{Theorem}[section]
\newtheorem{proposition}[theorem]{Proposition}
\newtheorem{lemma}[theorem]{Lemma}
\newtheorem{corollary}[theorem]{Corollary}
\theoremstyle{definition}
\newtheorem{definition}[theorem]{Definition}
\newtheorem{example}[theorem]{Example}
\newtheorem{remark}[theorem]{Remark}
\newtheorem{problem}[theorem]{Problem}

\newcommand{\skpr}{\emph{Sketch of proof: }}
\newcommand{\soln}{\textit{Solution:\ }}

% Generally useful macros
\newcommand{\excise}[1]{} % useful for commenting out large chunks
\newcommand{\0}{\emptyset}
\newcommand{\compn}{\models} % compositions
\newcommand{\dju}{\mathaccent\cdot\cup} % disjoint union
\newcommand{\dsum}{\displaystyle\sum}
\newcommand{\fallfac}[2]{{#1}^{\underline{#2}}} % falling factorial
\newcommand{\isom}{\cong} % isomorphism symbol
\newcommand{\partn}{\vdash} % partition symbol
\newcommand{\qqandqq}{\qquad\text{and}\qquad}
\newcommand{\qandq}{\quad\text{and}\quad}
\newcommand{\qand}{\quad\text{and}}
\newcommand{\qbin}[2]{{\begin{bmatrix}#1\\#2\end{bmatrix}_q}} % q-binomial coefficient
\newcommand{\risefac}[2]{{#1}^{\overline{#2}}} % rising factorial
\newcommand{\sd}{\triangle} % symmetric difference
\newcommand{\sm}{\setminus} % don't use a minus sign for this
\newcommand{\st}{\colon} % "such that"
\newcommand{\surj}{\twoheadrightarrow}
\newcommand{\x}{\times}

% blackboard bold fonts for sets of numbers
\newcommand{\Cc}{\mathbb{C}} % complex numbers
\newcommand{\Ff}{\mathbb{F}} % finite field
\newcommand{\Nn}{\mathbb{N}} % natural numbers
\newcommand{\Qq}{\mathbb{Q}}
\newcommand{\Rr}{\mathbb{R}}
\newcommand{\Zz}{\mathbb{Z}}

% miscellaneous
\newcommand{\TwoCases}[4]{\begin{cases}{#1}&\text{ #2}\\{#3}&\text{ #4}\end{cases}}
\newcommand{\ThreeCases}[6]{\begin{cases}{#1}&\text{ #2}\\{#3}&\text{ #4}\\{#5}&\text{ #6}\end{cases}}
\newcommand{\bridgehand}[4]{\spadesuit\ {\textsf{#1}}\ \ \heartsuit\ {\textsf{#2}}\ \ \diamondsuit\ {\textsf{#3}}\ \ \clubsuit\ {\textsf{#4}}}

% macros for automatic problem numbering --- students don't have to use these
\newcounter{probno}
\setcounter{probno}{0}
\newcounter{partno}
\setcounter{partno}{0}
%% versions that don't print the number of points
\newcommand{\prob}{
  \vskip10bp%
  \setcounter{partno}{0}%
  \addtocounter{probno}{1}%
  {\bf Problem~\#{\arabic{probno}}}\quad}

% No initial whitespace to make framing look nicer
\newcommand{\probns}{
  \setcounter{partno}{0}%
  \addtocounter{probno}{1}%
  {\bf Problem~\#{\arabic{probno}}}\quad}

\newcommand{\probpart}{%
  \addtocounter{partno}{1}%
  {\bf (\#\arabic{probno}\alph{partno})}\ \ }
\newcommand{\probcont}{%
  {\bf Problem~\#{\arabic{probno}}}~(\emph{continued})}
\newcommand{\probo}{
  \setcounter{partno}{0}%
  \addtocounter{probno}{1}%
  {\bf (\#\arabic{probno}}\ \ }

%% versions that do print the number of points
\newcommand{\Prob}[1]{
  \vskip10bp%
  \setcounter{partno}{0}%
  \addtocounter{probno}{1}%
  {\bf Problem~\#{\arabic{probno}}~[{#1}~pts]}\quad}

%no initial whitespace
\newcommand{\Probns}[1]{
  \setcounter{partno}{0}%
  \addtocounter{probno}{1}%
  {\bf Problem~\#{\arabic{probno}}~[{#1}~pts]}\quad}
\newcommand{\Probpart}[1]{%\rule{0in}{0in}\\ \phantom{xxx}
  \addtocounter{partno}{1}%
  {\bf (\#\arabic{probno}\alph{partno})~[{#1}~pts]}\ \ }

\newboolean{answers}
\newcommand{\Answer}[1]{\ifthenelse{\boolean{answers}}{{\bf Answer:}\ #1}{}\bigskip}

\setboolean{answers}{false}
\begin{document}

{\bf Math 724, Fall 2013\\
Homework \#7

Instructions:} Write up your solutions in LaTeX and hand in a hard copy in class on {\bf Monday, December~9.}  Collaboration is allowed (and in fact encouraged), but each student must write up his or her solutions independently and acknowledge all collaborators.

\prob Bogart, Chapter 5, Supplementary Problem \#2.

\prob Bogart, Chapter 5, Supplementary Problem \#5.

\prob Bogart, Chapter 5, Supplementary Problem \#8.

\prob Let $C_n$ be the cycle graph of length $n$ --- that is, the undirected graph with vertices $v_1,v_2,\dots,v_n$ and edges $v_1v_2,v_2v_3,\dots,v_{n-1}v_n,v_nv_1$.
Find a formula for the chromatic polynomial of $C_n$.

\prob Let $Y_n$ be the graph $K_n$ with one edge removed.  Find the chromatic polynomial of $Y_n$.

\prob \emph{Orienting} a graph means assigning a direction to each edge.  We can think of a directed edge as an arrow with a head and a tail.  There are two possible orientations for each edge (even loops --- you can think of the orientations as ``clockwise'' and ``counterclockwise''), so the total number of orientations of a graph with $e$ edges is $2^e$.

An orientation is \emph{acyclic} if there is no way to walk from any vertex back to itself by following one or more arrows.  The left-hand orientation is acyclic, but the right-hand orientation is not.
\includefigure{2.4in}{0.75in}{orientations} % 4 x 1.25

Let $A(G)$ denote the set of acyclic orientations of $G$, and let $\alpha(G)=|A(G)|$.

(a) What is $\alpha(G)$ if $G$ is a forest?  What is $\alpha(K_n)$?  What is $\alpha(C_n)$?  (Here $K_n$ is the complete graph with $n$ vertices, and
$C_n$ is the cycle graph of length~$n$.)
To get at these counting problems (particularly for $K_n$), observe the following.  Any labeling of the vertices of $G$ by distinct real numbers (say $1,\dots,n$, where $n$ is the number of vertices) gives rise to an orientation by pointing every edge toward its larger endpoint.  This orientation is always acyclic (why?)  Do different labelings always induce different orientations?  Does every orientation of $G$ arise in this way?

(b) Show that $\alpha(G)=\alpha(G-e)+\alpha(G/e)$ for any edge $e$.

Hint: Consider the map $\pi:A(G)\to A(G-e)$ given by forgetting the orientation of $e$.  First, show that $\pi$ is onto.  Second, figure out which acyclic orientations of $A(G-e)$ arise from one acyclic orientation of $G$ and which from two.  This should give you a combinatorial interpretation for the ``overcount'' $\alpha(G)-\alpha(G-e)$.  Use a bijection to show that this overcount is exactly $\alpha(G/e)$.

(c) Find a formula for $\alpha(G)$ in terms of the chromatic polynomial $\chi_G(k)$.
This should \emph{definitely} knock your socks off.

\prob
Let $V=[n]$ and let $G$ be a graph with vertex set $V$.
The \emph{chromatic symmetric function} $X(G)$ of $G$ is the formal power series
defined by
\[X(G) = \sum_f \prod_{i=1}^n x_{f(i)}\]
where the sum ranges over all proper colorings $f$.  Note that this is a power series in infinitely many variables.  An equivalent definition is
\[X(G) = \sum_f \prod_{j=1}^\infty x_j^{c_j(f)}\]
where $c_j(f)$ is the number of vertices that are assigned color~$j$ by the coloring~$f$.  So $X(G)$ keeps track not only of how many colorings are proper, but also of how many proper colorings use a specified distribution of colors.

If you don't want to write out all these power series, there's a simpler way of describing $X(G)$.  Given a partition $\lambda\partn n$, let $c_\lambda(G)$ be the number of proper colorings with $\lambda_1$ parts of color 1, $\lambda_2$ parts of color 2, etc.  Then $c_\lambda(G)$ is the coefficient in $X(G)$ of any monomial of the form $x_{i_1}^{\lambda_1}x_{i_2}^{\lambda_2}\cdots$ (where $i_1,i_2,\dots$ are all distinct).  Therefore, in order to specify $X(G)$, it is sufficient to say what the numbers $c_\lambda(G)$ are for all $\lambda\partn n$.

For example, let $G$ be the path with two edges (graph (a) below).  There are
6 ways to color one vertex red, one vertex blue and one vertex green (since every such coloring is proper, and there are $3!=6$ such colorings), so $c_{111}=6$.
There is 1 way to color one vertex red and two vertices blue, and it is impossible to color all three vertices magenta.  So $X(G)$ is completely specified by the values
\[c_{111}(G)=6, \quad c_{21}(G)=1,\quad c_{3}(G)=0.\]

\includefigure{3.3in}{0.45in}{smalltrees} % 5.5 x 0.75

(a) What is $X(G)$ if $G$ has no edges? 

(b) What is $X(G)$ if $G=K_n$?

(c) Let $k$ be a positive integer.  Explain how to derive the number of proper $k$-colorings from $X(G)$.  (Be careful: there is not (to my knowledge) an easy algebraic specialization of $X(G)$ that yields the polynomial $\chi_G(k)$ --- but for any \emph{number} $k$, it is possible to obtain the \emph{number} $\chi_G(k)$ from $X(G)$.  Therefore, all information about $G$ that can be obtained from $\chi_G$ can in principle be obtained from $X(G)$.

(d) There are two trees on four vertices, as shown in (b) and (c) above.  Show that they do \emph{not} have the same chromatic symmetric function, even though they have the same chromatic polynomial.  (It is unknown whether two nonisomorphic trees have the same chromatic symmetric function.  This is an open problem that has caused Jeremy many sleepless nights.)

\vfill

{\bf Extra credit}  Let $n$ be a positive integer.  For $1\leq i<j\leq n$, define a hyperplane $H_{ij}\subset\Rr^n$ by
\[H_{ij} = \{(x_1,x_2\dots,x_n)\in\Rr^n \st x_i=x_j\}.\]
Now let $G$ be a simple graph with vertex set $[n]$ and edge set $E$, and let
\[\mathcal{A}_G = \bigcup_{ij\in E} H_{ij}.\]
Thus $\mathcal{A}_G$ is a subset of $\Rr^n$ (it is called a \emph{graphical hyperplane arrangement}).  In terms of $G$,
how many connected components does $\Rr^n\sm\mathcal{A}_G$ have?

\end{document}
