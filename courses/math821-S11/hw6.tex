\documentclass{amsart}
\usepackage{amssymb,amsmath,amsthm,mathrsfs,graphics,hyperref,stmaryrd,psfrag,arcs,xypic}
\usepackage[enableskew]{youngtab}
\numberwithin{equation}{section}
\raggedbottom
\oddsidemargin=0in
\evensidemargin=0in
\textwidth=6.5in
\textheight=8.8in
\topmargin=0.25in
\headheight=0in
\headsep=0.2in
\footskip=0in
\parskip=10bp
\parindent=0bp

\newcommand{\excise}[1]{}
\newcommand{\Latex}[1]{\textbackslash\texttt{#1}}

\newcommand{\bigpad}{\rule[-14mm]{0mm}{30mm}}
\newcommand{\smallpad}{\rule[-1.5mm]{0mm}{5mm}}
\newcommand{\pad}{\rule[-3mm]{0mm}{8mm}}
\newcommand{\padup}{\rule{0mm}{5mm}}
\newcommand{\paddown}{\rule[-3mm]{0mm}{2mm}}
\newcommand{\blank}{\rule{1.25in}{0.25mm}}
\newcommand{\commentout}[1]{}
\newcommand{\yell}[1]{\fbox{\rule[-1mm]{0mm}{4mm} \large\bf #1 }}
\newcommand{\bang}{$\bullet$\quad}
\newcommand{\indnt}{\phantom{.}\qquad}
\newcommand{\littleline}{\begin{center}\rule{4in}{0.5bp}\end{center}}

\newcommand{\includefigure}[3]{{
  \begin{center}
  \resizebox{#1}{#2}{\includegraphics{{figs/#3}}}
  \end{center}}}
\newcommand{\includefigurewithinmath}[3]{{
  \resizebox{#1}{#2}{\includegraphics{{figs/#3}}}}}

%\newcommand{\defterm}[1]{\underline{\textbf{#1}}}
\newcommand{\defterm}[1]{\textbf{#1}}

\DeclareMathOperator{\ch}{\mathbf{ch}}
\DeclareMathOperator{\colspace}{colspace}
\DeclareMathOperator{\corank}{corank}
\DeclareMathOperator{\CST}{CST}
\DeclareMathOperator{\deln}{del}
\DeclareMathOperator{\diag}{diag}
\DeclareMathOperator{\ess}{ess}
\DeclareMathOperator{\Gr}{Gr}
\DeclareMathOperator{\Hom}{Hom}
\DeclareMathOperator{\im}{im}
\DeclareMathOperator{\Irr}{Irr}
\DeclareMathOperator{\Ind}{Ind}
\DeclareMathOperator{\Int}{Int}
\DeclareMathOperator{\lcm}{lcm}
\DeclareMathOperator{\link}{lk}
\DeclareMathOperator{\nullity}{nullity}
\DeclareMathOperator{\nullspace}{nullspace}
\DeclareMathOperator{\Poin}{Poin}
\DeclareMathOperator{\proj}{proj}
\DeclareMathOperator{\rank}{rank}
\DeclareMathOperator{\Res}{Res}
\DeclareMathOperator{\Span}{span}
\DeclareMathOperator{\supp}{supp}
\DeclareMathOperator{\row}{row}
\DeclareMathOperator{\rowspace}{rowspace}
\DeclareMathOperator{\sh}{sh}
\DeclareMathOperator{\tr}{tr}
\DeclareMathOperator{\wt}{wt}

\newtheorem{theorem}{Theorem}[section]
\newtheorem{proposition}[theorem]{Proposition}
\newtheorem{lemma}[theorem]{Lemma}
\newtheorem{corollary}[theorem]{Corollary}
\theoremstyle{definition}
\newtheorem{definition}[theorem]{Definition}
\newtheorem{example}[theorem]{Example}
\newtheorem{remark}[theorem]{Remark}
\newtheorem{problem}[theorem]{Problem}


\newcommand{\cor}{{\bf Corollary: }}
\newcommand{\defn}{{\bf Definition: }}
\newcommand{\defns}{{\bf Definitions: }}
\newcommand{\exa}{{\bf Example: }}
\newcommand{\fact}{{\bf Fact: }}
\newcommand{\lem}{{\bf Lemma: }}
\newcommand{\notn}{{\bf Notation: }}
\newcommand{\obs}{{\bf Observation: }}
\newcommand{\note}{{\bf Note: }}
\newcommand{\prop}{{\bf Proposition: }}
\newcommand{\rmk}{{\bf Remark: }}
\newcommand{\thm}{{\bf Theorem: }}

\newcommand{\basecase}{\emph{Base case: }}
\newcommand{\indstep}{\emph{Inductive step: }}
\newcommand{\skpr}{\emph{Sketch of proof: }}

\newcommand{\0}{\emptyset}
\newcommand{\Alt}{\mathfrak{A}}
\newcommand{\Braid}{Br}
\newcommand{\CHI}{\chi^{\phantom{*}}}
\newcommand{\Cl}{C\ell}
\newcommand{\covers}{\gtrdot}
\newcommand{\coveredby}{\lessdot}
\newcommand{\dedge}[1]{\overrightarrow{{#1}}}
\newcommand{\dom}{\rhd}
\newcommand{\domeq}{\unrhd}
\newcommand{\domby}{\lhd}
\newcommand{\dombyeq}{\unlhd}
\newcommand{\Fspan}{\Ff\text{-span}}
\newcommand{\isom}{\cong}
\newcommand{\join}{\vee}
\renewcommand{\Join}{\bigvee}
\newcommand{\Laff}{L^{\text{aff}}}
\newcommand{\lin}[1]{\overleftrightarrow{{#1}}}
\newcommand{\meet}{\wedge}
\newcommand{\Meet}{\bigwedge}
\newcommand{\ov}[1]{\overline{{#1}}}
\newcommand{\partn}{\vdash}
\newcommand{\qqandqq}{\qquad\text{and}\qquad}
\newcommand{\qandq}{\quad\text{and}\quad}
\newcommand{\qand}{\quad\text{and}}
\newcommand{\qbin}[2]{{\begin{bmatrix}#1\\#2\end{bmatrix}_q}}
\newcommand{\sd}{\triangle} % symmetric difference
\newcommand{\simK}{\underset{K}{\sim}} % Knuth equivalence
\newcommand{\simJ}{\underset{J}{\sim}} % jeu de taquin equivalence
\newcommand{\sm}{\setminus}
\newcommand{\st}{~|~}
\newcommand{\soln}{\textit{Solution:\ }}
\newcommand{\Sym}{\mathfrak{S}}
\newcommand{\un}[1]{\underset{*}{#1}}
\newcommand{\unA}{\un{A}}
\newcommand{\unB}{\un{B}}
\newcommand{\unw}{\un{w}}
\newcommand{\unx}{\un{x}}
\newcommand{\uny}{\un{y}}
\newcommand{\unz}{\un{z}}
\newcommand{\x}{\times}

\renewcommand{\aa}{\mathbf{a}}
\newcommand{\bb}{\mathbf{b}}
\newcommand{\nn}{\mathbf{n}}
\newcommand{\pp}{\mathbf{p}}
\newcommand{\qq}{\mathbf{q}}
\newcommand{\xx}{\mathbf{x}}
\newcommand{\yy}{\mathbf{y}}
\newcommand{\zz}{\mathbf{z}}
 
\newcommand{\A}{\mathcal{A}}
\newcommand{\B}{\mathcal{B}}
\newcommand{\C}{\mathcal{C}}
\newcommand{\M}{\mathcal{M}}
\renewcommand{\P}{\mathcal{P}}

\newcommand{\BB}{\mathscr{B}}  %% use these for fancy script fonts -- requires mathrsfs package
\newcommand{\CC}{\mathscr{C}}
\newcommand{\FF}{\mathscr{F}}
\newcommand{\II}{\mathscr{I}}
\newcommand{\LL}{\mathscr{L}}
\newcommand{\PP}{\mathscr{P}}
\renewcommand{\SS}{\mathscr{S}}
\newcommand{\XX}{\mathscr{X}}

\newcommand{\TT}{\tilde{T}}

\newcommand{\Aa}{\mathbb{A}}
\newcommand{\Cc}{\mathbb{C}}
\newcommand{\Ff}{\mathbb{F}}
\newcommand{\Nn}{\mathbb{N}}
\newcommand{\Pp}{\mathbb{P}}
\newcommand{\Qq}{\mathbb{Q}}
\newcommand{\Rr}{\mathbb{R}}
\newcommand{\Zz}{\mathbb{Z}}

\newcommand{\rhodef}{\rho^{\phantom{*}}_{{\rm def}}}
\newcommand{\rhotriv}{\rho^{\phantom{*}}_{{\rm triv}}}
\newcommand{\rhosign}{\rho^{\phantom{*}}_{{\rm sign}}}
\newcommand{\rhoreg}{\rho^{\phantom{*}}_{{\rm reg}}}
\newcommand{\chidef}{\chi^{\phantom{*}}_{{\rm def}}}
\newcommand{\chitriv}{\chi^{\phantom{*}}_{{\rm triv}}}
\newcommand{\chisign}{\chi^{\phantom{*}}_{{\rm sign}}}
\newcommand{\chireg}{\chi^{\phantom{*}}_{{\rm reg}}}
\newcommand{\scp}[2]{\left\langle #1,\:#2\right\rangle_G}
\newcommand{\scpH}[2]{\left\langle #1,\:#2\right\rangle_H}

\newcounter{probno}
\setcounter{probno}{0}
\newcounter{partno}
\setcounter{partno}{0}
%% versions that don't print the number of points
\newcommand{\prob}{
  \vskip10bp%
  \setcounter{partno}{0}%
  \addtocounter{probno}{1}%
  {\bf Problem~\#{\arabic{probno}}}\quad}
\newcommand{\probpart}{%\rule{0in}{0in}\\ \phantom{xxx}
  \addtocounter{partno}{1}%
  {\bf (\#\arabic{probno}\alph{partno})}\ \ }
\newcommand{\probcont}{%
  {\bf Problem~\#{\arabic{probno}}}~(\emph{continued})}
\newcommand{\probo}{
  \setcounter{partno}{0}%
  \addtocounter{probno}{1}%
  {\bf (\#\arabic{probno})}\ \ }
%% versions that do print the number of points
\newcommand{\Prob}[1]{
  \vskip10bp%
  \setcounter{partno}{0}%
  \addtocounter{probno}{1}%
  {\bf Problem~\#{\arabic{probno}}~[{#1}~pts]}\quad}
\newcommand{\Probpart}[1]{%\rule{0in}{0in}\\ \phantom{xxx}
  \addtocounter{partno}{1}%
  {\bf (\#\arabic{probno}\alph{partno})~[{#1}~pts]}\ \ }

%\renewcommand{\thefootnote}{\fnsymbol{footnote}}

\usepackage{youngtab}
\begin{document}
\thispagestyle{empty}
{\bf Math 821 Problem Set \#6\\
Posted: Friday 4/15/11\\
Due date: Monday 4/25/11}

\prob [Hatcher p.131 \#4] Compute the simplicial homology groups of
the ``triangular parachute'' obtained from the standard 2-simplex $\Delta^2$
by identifying its three vertices to a single point.


\vfill

\prob [Hatcher p.131 \#8]
Construct a 3-dimensional $\Delta$-complex $X$ from $n$ tetrahedra
$T_1,\dots,T_n$ by the following two steps.

First, arrange the tetrahedra in a cyclic pattern as in the figure (see p.~131)
so that each $T_i$ shares a common vertical face with its two neighbors.
For consistent notation, call the top and bottom vertices
$x$ and $y$ respectively, and call the side vertices
$v_1,\dots,v_n$, so the tetrahedra are
$$T_1=[x,v_1,v_2,y],\ 
T_2=[x,v_2,v_3,y],\ \dots,
T_{n-1}=[x,v_{n-1},v_{n},y],\ 
T_n=[x,v_n,v_1,y].$$
Second, identify the bottom face of $T_i$ with the top face
of $T_{i+1}$ for all $i$, that is, $[v_i,v_{i+1},y]=[v_{i+1},v_{i+2},y]$.

Show that the simplicial homology groups of $X$ in dimensions 0, 1, 2, 3
are $\Zz$, $\Zz_n$ (=$\Zz/n\Zz$), $0$, $\Zz$ respectively.
(Start by making a complete census of the oriented simplices, including a record of which ones have been identified --- for example, $[x,v_1,v_2]=[y,v_n,v_1]$ is a triangle in $X$.)

\vfill

\prob [Hatcher p.131 \#11] Show that if $A$ is a retract of $X$
then the map $H_n(A)\to H_n(X)$ induced by the inclusion $A\subset X$ is
injective.

\vfill


\prob
A [finite] \emph{partially ordered set} or \emph{poset}
is a finite set $P$ with an order relation $\leq$ such that
for all $x,y,z\in P$:
(1) $x\leq x$;
(2) if $x\leq y$ and $y\leq x$, then $x=y$; and
(3) if $x\leq y$ and $y\leq z$, then $x\leq z$.
Of course, $x<z$ means that $z\leq z$ and $x\neq z$.
If $x\leq z$ or $z\leq x$, we say that $x,z$ are \emph{comparable}.
A \emph{chain} in $P$ (not to be confused with a simplicial or singular chain!)
is a subset in which every two elements are comparable.

\probpart Prove that the set $\Delta(P)$ of chains in $P$ is a simplicial complex.
(This is called the \emph{order complex} of~$P$.)

\probpart Suppose that $P$ has a unique maximal element.  Prove that $\Delta(P)$ is contractible.

\probpart For each $n\geq 1$, construct a poset for which $\Delta(P)$ is homeomorphic to an
$n$-sphere.

\probpart The \emph{M\"obius function} $\mu$ of $P$ is defined as follows.
\begin{enumerate}
\item Adjoin two new elements $\hat0,\hat1$ to $P$ to obtain a poset
$\hat P$, in which $\hat0<x<\hat1$ for every $x\in P$.
\item Define $\mu$ recursively as follows: First, if $x$ is minimal
(i.e., there exists no $y$ such that $x>y$) then $\mu(x)=-1$.
Second, if $\mu(y)$ has already been defined for all $y<x$, then define
  $$\mu(x)=-\sum_{y<x}\mu(y).$$
(So you can work out the values of $\mu$ on all elements of $P$
by starting at the bottom and working your way up.)
\end{enumerate}
Make a conjecture as to how the Euler characteristic
of $\Delta(P)$ can be obtained from the M\"obius function of~$P$.





\pagebreak

\textbf{\Large Some LaTeX tips}

\textbf{1. Matrices with borders}

The $\backslash$\texttt{bordermatrix} command
can be used for matrices whose columns and rows
you want to label.  This can be useful for bookkeeping
in a simplicial homology calculation.
For example, the boundary map $\bd_2$ of the standard 3-simplex
is
$$
\bordermatrix{
   & 123 & 124 & 134 & 234 \cr
12 &  1  &  1  &  0  &  0\cr
13 & -1  &  0  &  1  &  0\cr
14 &  0  & -1  & -1  &  0\cr
23 &  1  &  0  &  0  &  1\cr
24 &  0  &  1  &  0  & -1\cr
34 &  0  &  0  &  1  &  1
}
$$
which can be produced as follows:
\begin{verbatim}
$$\bordermatrix{
   & 123 & 124 & 134 & 234 \cr
12 &  1  &  1  &  0  &  0\cr
13 & -1  &  0  &  1  &  0\cr
14 &  0  & -1  & -1  &  0\cr
23 &  1  &  0  &  0  &  1\cr
24 &  0  &  1  &  0  & -1\cr
34 &  0  &  0  &  1  &  1}$$
\end{verbatim}

\textbf{2. Commutative diagrams}

The \texttt{xypic} package provides a way to typeset commutative diagrams
in LaTeX.  For instance, consider the following
diagram, which arises in the proof of Theorem 2.10 in Hatcher:
$$
\xymatrix{
\cdots\ar[r] & C_{n+1}(X) \ar[r]^{\bd}\ar[d]^{i_\#} & C_{n}(X) \ar[r]^{\bd} \ar[d]^{i_\#} \ar[dl]^{P} & C_{n-1}(X) \ar[r] \ar[d]^{i_\#} \ar[dl]^{P} &\cdots\\
\cdots\ar[r] & C_{n+1}(Y) \ar[r]_{\bd} & C_{n}(Y) \ar[r]_{\bd} & C_{n-1}(Y)\ar[r] &\cdots\\
}$$
It can be typeset as follows:
\begin{verbatim}
$$\xymatrix{
\cdots\ar[r]
  & C_{n+1}(X) \ar[r]^{\bd} \ar[d]^{i_\#}
  & C_{n}(X)   \ar[r]^{\bd} \ar[d]^{i_\#} \ar[dl]^{P}
  & C_{n-1}(X) \ar[r]       \ar[d]^{i_\#} \ar[dl]^{P}
  &\cdots\\
\cdots\ar[r]
  & C_{n+1}(Y) \ar[r]_{\bd}
  & C_{n}(Y)   \ar[r]_{\bd}
  & C_{n-1}(Y) \ar[r]
  &\cdots}$$
\end{verbatim}
This is like a \texttt{tabular} or \texttt{array} environment:
the \& symbols are delimiters between columns.  The $\backslash$\texttt{ar}
commands create arrows emanating from the current cell in the table, with
the code in [square brackets] specifying where the arrow should point;
e.g., \texttt{$\backslash$ar[dl]} makes an arrow pointing towards the
cell one row down and one column left of the current cell.

\end{document}

