\documentclass{amsart}
\usepackage{amssymb,amsmath,amsthm,mathrsfs,graphics,hyperref,ifthen,framed,cancel,fullpage,color,ytableau,tcolorbox,bm,tikz}
\usepackage[enableskew]{youngtab}
\raggedbottom
\parskip=10bp
\parindent=0bp
\raggedbottom

%%%%%%%%%%%%%%%%   Colors for TikZ and text  %%%%%%%%%%%%%%%%%

\definecolor{light}{gray}{.75}
\definecolor{med}{gray}{.5}
\definecolor{dark}{gray}{.25}
\newcommand{\Red}[1]{{\color{red}{#1}}}
\newcommand{\RED}[1]{{\color{red}{\boldmath\textbf{#1}\unboldmath}}}
\newcommand{\Blue}[1]{{\color{blue}{#1}}}
\newcommand{\BLUE}[1]{{\color{blue}{\boldmath\textbf{#1}\unboldmath}}}

% Hyperlinks
\hypersetup{colorlinks, citecolor=red, filecolor=black, linkcolor=blue, urlcolor=blue}
\newcommand{\hreftext}[2]{\href{#1}{\Blue{#2}}} % for text anchors
\newcommand{\hrefurl}[2]{\href{#1}{\Blue{\tt #2}}} % if you want to actually write out the URL in text

% Spacing and commands
\newcommand{\bigpad}{\rule[-14mm]{0mm}{30mm}}
\newcommand{\smallpad}{\rule[-1.5mm]{0mm}{5mm}}
\newcommand{\pad}{\rule[-3mm]{0mm}{8mm}}
\newcommand{\padup}{\rule{0mm}{5mm}}
\newcommand{\paddown}{\rule[-3mm]{0mm}{2mm}}
\newcommand{\blank}{\rule{1.25in}{0.25mm}}
\newcommand{\yell}[1]{\fbox{\rule[-1mm]{0mm}{4mm} \large\bf #1 }}
\newcommand{\indnt}{\phantom{.}\qquad}
\newcommand{\littleline}{\begin{center}\rule{4in}{0.5bp}\end{center}}

% macros for inserting figures
\newcommand{\includefigure}[3]{\begin{center}\resizebox{#1}{#2}{\includegraphics{#3}}\end{center}}
\newcommand{\includefigurewithinmath}[3]{\resizebox{#1}{#2}{\includegraphics{#3}}}
% E.g., to insert a standalone figure of width 3" and height 1.5", use
% \includefigure{3in}{1.5in}{foo.pdf}

% math operators
\DeclareMathOperator{\Comp}{Comp}
\DeclareMathOperator{\Fix}{Fix}
\DeclareMathOperator{\id}{id}
\DeclareMathOperator{\im}{im}
\DeclareMathOperator{\lcm}{lcm}
\DeclareMathOperator{\Par}{Par}
\DeclareMathOperator{\rank}{rank}
\DeclareMathOperator{\sh}{sh}
\DeclareMathOperator{\tr}{tr}
\DeclareMathOperator{\wt}{wt}

% theorem environments, automatically numbered
\newtheorem{theorem}{Theorem}[section]
\newtheorem{proposition}[theorem]{Proposition}
\newtheorem{lemma}[theorem]{Lemma}
\newtheorem{corollary}[theorem]{Corollary}
\theoremstyle{definition}
\newtheorem{definition}[theorem]{Definition}
\newtheorem{example}[theorem]{Example}
\newtheorem{remark}[theorem]{Remark}
\newtheorem{problem}[theorem]{Problem}

\newcommand{\skpr}{\emph{Sketch of proof: }}
\newcommand{\soln}{\textit{Solution:\ }}

% Generally useful macros
\newcommand{\excise}[1]{} % useful for commenting out large chunks
\newcommand{\0}{\emptyset}
\newcommand{\compn}{\models} % compositions
\newcommand{\dju}{\mathaccent\cdot\cup} % disjoint union
\newcommand{\dsum}{\displaystyle\sum}
\newcommand{\fallfac}[2]{{#1}^{\underline{#2}}} % falling factorial
\newcommand{\isom}{\cong} % isomorphism symbol
\newcommand{\partn}{\vdash} % partition symbol
\newcommand{\qqandqq}{\qquad\text{and}\qquad}
\newcommand{\qandq}{\quad\text{and}\quad}
\newcommand{\qand}{\quad\text{and}}
\newcommand{\qbin}[2]{{\begin{bmatrix}#1\\#2\end{bmatrix}_q}} % q-binomial coefficient
\newcommand{\risefac}[2]{{#1}^{\overline{#2}}} % rising factorial
\newcommand{\sd}{\triangle} % symmetric difference
\newcommand{\sm}{\setminus} % don't use a minus sign for this
\newcommand{\st}{\colon} % "such that"
\newcommand{\surj}{\twoheadrightarrow}
\newcommand{\x}{\times}

% blackboard bold fonts for sets of numbers
\newcommand{\Cc}{\mathbb{C}} % complex numbers
\newcommand{\Ff}{\mathbb{F}} % finite field
\newcommand{\Nn}{\mathbb{N}} % natural numbers
\newcommand{\Qq}{\mathbb{Q}}
\newcommand{\Rr}{\mathbb{R}}
\newcommand{\Zz}{\mathbb{Z}}

% miscellaneous
\newcommand{\TwoCases}[4]{\begin{cases}{#1}&\text{ #2}\\{#3}&\text{ #4}\end{cases}}
\newcommand{\ThreeCases}[6]{\begin{cases}{#1}&\text{ #2}\\{#3}&\text{ #4}\\{#5}&\text{ #6}\end{cases}}
\newcommand{\bridgehand}[4]{\spadesuit\ {\textsf{#1}}\ \ \heartsuit\ {\textsf{#2}}\ \ \diamondsuit\ {\textsf{#3}}\ \ \clubsuit\ {\textsf{#4}}}

% macros for automatic problem numbering --- students don't have to use these
\newcounter{probno}
\setcounter{probno}{0}
\newcounter{partno}
\setcounter{partno}{0}
%% versions that don't print the number of points
\newcommand{\prob}{
  \vskip10bp%
  \setcounter{partno}{0}%
  \addtocounter{probno}{1}%
  {\bf Problem~\#{\arabic{probno}}}\quad}

% No initial whitespace to make framing look nicer
\newcommand{\probns}{
  \setcounter{partno}{0}%
  \addtocounter{probno}{1}%
  {\bf Problem~\#{\arabic{probno}}}\quad}

\newcommand{\probpart}{%
  \addtocounter{partno}{1}%
  {\bf (\#\arabic{probno}\alph{partno})}\ \ }
\newcommand{\probcont}{%
  {\bf Problem~\#{\arabic{probno}}}~(\emph{continued})}
\newcommand{\probo}{
  \setcounter{partno}{0}%
  \addtocounter{probno}{1}%
  {\bf (\#\arabic{probno}}\ \ }

%% versions that do print the number of points
\newcommand{\Prob}[1]{
  \vskip10bp%
  \setcounter{partno}{0}%
  \addtocounter{probno}{1}%
  {\bf Problem~\#{\arabic{probno}}~[{#1}~pts]}\quad}

%no initial whitespace
\newcommand{\Probns}[1]{
  \setcounter{partno}{0}%
  \addtocounter{probno}{1}%
  {\bf Problem~\#{\arabic{probno}}~[{#1}~pts]}\quad}
\newcommand{\Probpart}[1]{%\rule{0in}{0in}\\ \phantom{xxx}
  \addtocounter{partno}{1}%
  {\bf (\#\arabic{probno}\alph{partno})~[{#1}~pts]}\ \ }

\newboolean{answers}
\newcommand{\Answer}[1]{\ifthenelse{\boolean{answers}}{{\bf Answer:}\ #1}{}\bigskip}

\usepackage{youngtab}
\begin{document}
\thispagestyle{empty}
{\bf Math 821 Problem Set \#6\\
Posted: Friday 4/15/11\\
Due date: Monday 4/25/11}

\prob [Hatcher p.131 \#4] Compute the simplicial homology groups of
the ``triangular parachute'' obtained from the standard 2-simplex $\Delta^2$
by identifying its three vertices to a single point.


\vfill

\prob [Hatcher p.131 \#8]
Construct a 3-dimensional $\Delta$-complex $X$ from $n$ tetrahedra
$T_1,\dots,T_n$ by the following two steps.

First, arrange the tetrahedra in a cyclic pattern as in the figure (see p.~131)
so that each $T_i$ shares a common vertical face with its two neighbors.
For consistent notation, call the top and bottom vertices
$x$ and $y$ respectively, and call the side vertices
$v_1,\dots,v_n$, so the tetrahedra are
$$T_1=[x,v_1,v_2,y],\ 
T_2=[x,v_2,v_3,y],\ \dots,
T_{n-1}=[x,v_{n-1},v_{n},y],\ 
T_n=[x,v_n,v_1,y].$$
Second, identify the bottom face of $T_i$ with the top face
of $T_{i+1}$ for all $i$, that is, $[v_i,v_{i+1},y]=[v_{i+1},v_{i+2},y]$.

Show that the simplicial homology groups of $X$ in dimensions 0, 1, 2, 3
are $\Zz$, $\Zz_n$ (=$\Zz/n\Zz$), $0$, $\Zz$ respectively.
(Start by making a complete census of the oriented simplices, including a record of which ones have been identified --- for example, $[x,v_1,v_2]=[y,v_n,v_1]$ is a triangle in $X$.)

\vfill

\prob [Hatcher p.131 \#11] Show that if $A$ is a retract of $X$
then the map $H_n(A)\to H_n(X)$ induced by the inclusion $A\subset X$ is
injective.

\vfill


\prob
A [finite] \emph{partially ordered set} or \emph{poset}
is a finite set $P$ with an order relation $\leq$ such that
for all $x,y,z\in P$:
(1) $x\leq x$;
(2) if $x\leq y$ and $y\leq x$, then $x=y$; and
(3) if $x\leq y$ and $y\leq z$, then $x\leq z$.
Of course, $x<z$ means that $z\leq z$ and $x\neq z$.
If $x\leq z$ or $z\leq x$, we say that $x,z$ are \emph{comparable}.
A \emph{chain} in $P$ (not to be confused with a simplicial or singular chain!)
is a subset in which every two elements are comparable.

\probpart Prove that the set $\Delta(P)$ of chains in $P$ is a simplicial complex.
(This is called the \emph{order complex} of~$P$.)

\probpart Suppose that $P$ has a unique maximal element.  Prove that $\Delta(P)$ is contractible.

\probpart For each $n\geq 1$, construct a poset for which $\Delta(P)$ is homeomorphic to an
$n$-sphere.

\probpart The \emph{M\"obius function} $\mu$ of $P$ is defined as follows.
\begin{enumerate}
\item Adjoin two new elements $\hat0,\hat1$ to $P$ to obtain a poset
$\hat P$, in which $\hat0<x<\hat1$ for every $x\in P$.
\item Define $\mu$ recursively as follows: First, if $x$ is minimal
(i.e., there exists no $y$ such that $x>y$) then $\mu(x)=-1$.
Second, if $\mu(y)$ has already been defined for all $y<x$, then define
  $$\mu(x)=-\sum_{y<x}\mu(y).$$
(So you can work out the values of $\mu$ on all elements of $P$
by starting at the bottom and working your way up.)
\end{enumerate}
Make a conjecture as to how the Euler characteristic
of $\Delta(P)$ can be obtained from the M\"obius function of~$P$.





\pagebreak

\textbf{\Large Some LaTeX tips}

\textbf{1. Matrices with borders}

The $\backslash$\texttt{bordermatrix} command
can be used for matrices whose columns and rows
you want to label.  This can be useful for bookkeeping
in a simplicial homology calculation.
For example, the boundary map $\bd_2$ of the standard 3-simplex
is
$$
\bordermatrix{
   & 123 & 124 & 134 & 234 \cr
12 &  1  &  1  &  0  &  0\cr
13 & -1  &  0  &  1  &  0\cr
14 &  0  & -1  & -1  &  0\cr
23 &  1  &  0  &  0  &  1\cr
24 &  0  &  1  &  0  & -1\cr
34 &  0  &  0  &  1  &  1
}
$$
which can be produced as follows:
\begin{verbatim}
$$\bordermatrix{
   & 123 & 124 & 134 & 234 \cr
12 &  1  &  1  &  0  &  0\cr
13 & -1  &  0  &  1  &  0\cr
14 &  0  & -1  & -1  &  0\cr
23 &  1  &  0  &  0  &  1\cr
24 &  0  &  1  &  0  & -1\cr
34 &  0  &  0  &  1  &  1}$$
\end{verbatim}

\textbf{2. Commutative diagrams}

The \texttt{xypic} package provides a way to typeset commutative diagrams
in LaTeX.  For instance, consider the following
diagram, which arises in the proof of Theorem 2.10 in Hatcher:
$$
\xymatrix{
\cdots\ar[r] & C_{n+1}(X) \ar[r]^{\bd}\ar[d]^{i_\#} & C_{n}(X) \ar[r]^{\bd} \ar[d]^{i_\#} \ar[dl]^{P} & C_{n-1}(X) \ar[r] \ar[d]^{i_\#} \ar[dl]^{P} &\cdots\\
\cdots\ar[r] & C_{n+1}(Y) \ar[r]_{\bd} & C_{n}(Y) \ar[r]_{\bd} & C_{n-1}(Y)\ar[r] &\cdots\\
}$$
It can be typeset as follows:
\begin{verbatim}
$$\xymatrix{
\cdots\ar[r]
  & C_{n+1}(X) \ar[r]^{\bd} \ar[d]^{i_\#}
  & C_{n}(X)   \ar[r]^{\bd} \ar[d]^{i_\#} \ar[dl]^{P}
  & C_{n-1}(X) \ar[r]       \ar[d]^{i_\#} \ar[dl]^{P}
  &\cdots\\
\cdots\ar[r]
  & C_{n+1}(Y) \ar[r]_{\bd}
  & C_{n}(Y)   \ar[r]_{\bd}
  & C_{n-1}(Y) \ar[r]
  &\cdots}$$
\end{verbatim}
This is like a \texttt{tabular} or \texttt{array} environment:
the \& symbols are delimiters between columns.  The $\backslash$\texttt{ar}
commands create arrows emanating from the current cell in the table, with
the code in [square brackets] specifying where the arrow should point;
e.g., \texttt{$\backslash$ar[dl]} makes an arrow pointing towards the
cell one row down and one column left of the current cell.

\end{document}

