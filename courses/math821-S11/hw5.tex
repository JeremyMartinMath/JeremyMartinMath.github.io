\documentclass{amsart}
\usepackage{amssymb,amsmath,amsthm,mathrsfs,graphics,hyperref,ifthen,framed,cancel,fullpage,color,ytableau,tcolorbox,bm,tikz}
\usepackage[enableskew]{youngtab}
\raggedbottom
\parskip=10bp
\parindent=0bp
\raggedbottom

%%%%%%%%%%%%%%%%   Colors for TikZ and text  %%%%%%%%%%%%%%%%%

\definecolor{light}{gray}{.75}
\definecolor{med}{gray}{.5}
\definecolor{dark}{gray}{.25}
\newcommand{\Red}[1]{{\color{red}{#1}}}
\newcommand{\RED}[1]{{\color{red}{\boldmath\textbf{#1}\unboldmath}}}
\newcommand{\Blue}[1]{{\color{blue}{#1}}}
\newcommand{\BLUE}[1]{{\color{blue}{\boldmath\textbf{#1}\unboldmath}}}

% Hyperlinks
\hypersetup{colorlinks, citecolor=red, filecolor=black, linkcolor=blue, urlcolor=blue}
\newcommand{\hreftext}[2]{\href{#1}{\Blue{#2}}} % for text anchors
\newcommand{\hrefurl}[2]{\href{#1}{\Blue{\tt #2}}} % if you want to actually write out the URL in text

% Spacing and commands
\newcommand{\bigpad}{\rule[-14mm]{0mm}{30mm}}
\newcommand{\smallpad}{\rule[-1.5mm]{0mm}{5mm}}
\newcommand{\pad}{\rule[-3mm]{0mm}{8mm}}
\newcommand{\padup}{\rule{0mm}{5mm}}
\newcommand{\paddown}{\rule[-3mm]{0mm}{2mm}}
\newcommand{\blank}{\rule{1.25in}{0.25mm}}
\newcommand{\yell}[1]{\fbox{\rule[-1mm]{0mm}{4mm} \large\bf #1 }}
\newcommand{\indnt}{\phantom{.}\qquad}
\newcommand{\littleline}{\begin{center}\rule{4in}{0.5bp}\end{center}}

% macros for inserting figures
\newcommand{\includefigure}[3]{\begin{center}\resizebox{#1}{#2}{\includegraphics{#3}}\end{center}}
\newcommand{\includefigurewithinmath}[3]{\resizebox{#1}{#2}{\includegraphics{#3}}}
% E.g., to insert a standalone figure of width 3" and height 1.5", use
% \includefigure{3in}{1.5in}{foo.pdf}

% math operators
\DeclareMathOperator{\Comp}{Comp}
\DeclareMathOperator{\Fix}{Fix}
\DeclareMathOperator{\id}{id}
\DeclareMathOperator{\im}{im}
\DeclareMathOperator{\lcm}{lcm}
\DeclareMathOperator{\Par}{Par}
\DeclareMathOperator{\rank}{rank}
\DeclareMathOperator{\sh}{sh}
\DeclareMathOperator{\tr}{tr}
\DeclareMathOperator{\wt}{wt}

% theorem environments, automatically numbered
\newtheorem{theorem}{Theorem}[section]
\newtheorem{proposition}[theorem]{Proposition}
\newtheorem{lemma}[theorem]{Lemma}
\newtheorem{corollary}[theorem]{Corollary}
\theoremstyle{definition}
\newtheorem{definition}[theorem]{Definition}
\newtheorem{example}[theorem]{Example}
\newtheorem{remark}[theorem]{Remark}
\newtheorem{problem}[theorem]{Problem}

\newcommand{\skpr}{\emph{Sketch of proof: }}
\newcommand{\soln}{\textit{Solution:\ }}

% Generally useful macros
\newcommand{\excise}[1]{} % useful for commenting out large chunks
\newcommand{\0}{\emptyset}
\newcommand{\compn}{\models} % compositions
\newcommand{\dju}{\mathaccent\cdot\cup} % disjoint union
\newcommand{\dsum}{\displaystyle\sum}
\newcommand{\fallfac}[2]{{#1}^{\underline{#2}}} % falling factorial
\newcommand{\isom}{\cong} % isomorphism symbol
\newcommand{\partn}{\vdash} % partition symbol
\newcommand{\qqandqq}{\qquad\text{and}\qquad}
\newcommand{\qandq}{\quad\text{and}\quad}
\newcommand{\qand}{\quad\text{and}}
\newcommand{\qbin}[2]{{\begin{bmatrix}#1\\#2\end{bmatrix}_q}} % q-binomial coefficient
\newcommand{\risefac}[2]{{#1}^{\overline{#2}}} % rising factorial
\newcommand{\sd}{\triangle} % symmetric difference
\newcommand{\sm}{\setminus} % don't use a minus sign for this
\newcommand{\st}{\colon} % "such that"
\newcommand{\surj}{\twoheadrightarrow}
\newcommand{\x}{\times}

% blackboard bold fonts for sets of numbers
\newcommand{\Cc}{\mathbb{C}} % complex numbers
\newcommand{\Ff}{\mathbb{F}} % finite field
\newcommand{\Nn}{\mathbb{N}} % natural numbers
\newcommand{\Qq}{\mathbb{Q}}
\newcommand{\Rr}{\mathbb{R}}
\newcommand{\Zz}{\mathbb{Z}}

% miscellaneous
\newcommand{\TwoCases}[4]{\begin{cases}{#1}&\text{ #2}\\{#3}&\text{ #4}\end{cases}}
\newcommand{\ThreeCases}[6]{\begin{cases}{#1}&\text{ #2}\\{#3}&\text{ #4}\\{#5}&\text{ #6}\end{cases}}
\newcommand{\bridgehand}[4]{\spadesuit\ {\textsf{#1}}\ \ \heartsuit\ {\textsf{#2}}\ \ \diamondsuit\ {\textsf{#3}}\ \ \clubsuit\ {\textsf{#4}}}

% macros for automatic problem numbering --- students don't have to use these
\newcounter{probno}
\setcounter{probno}{0}
\newcounter{partno}
\setcounter{partno}{0}
%% versions that don't print the number of points
\newcommand{\prob}{
  \vskip10bp%
  \setcounter{partno}{0}%
  \addtocounter{probno}{1}%
  {\bf Problem~\#{\arabic{probno}}}\quad}

% No initial whitespace to make framing look nicer
\newcommand{\probns}{
  \setcounter{partno}{0}%
  \addtocounter{probno}{1}%
  {\bf Problem~\#{\arabic{probno}}}\quad}

\newcommand{\probpart}{%
  \addtocounter{partno}{1}%
  {\bf (\#\arabic{probno}\alph{partno})}\ \ }
\newcommand{\probcont}{%
  {\bf Problem~\#{\arabic{probno}}}~(\emph{continued})}
\newcommand{\probo}{
  \setcounter{partno}{0}%
  \addtocounter{probno}{1}%
  {\bf (\#\arabic{probno}}\ \ }

%% versions that do print the number of points
\newcommand{\Prob}[1]{
  \vskip10bp%
  \setcounter{partno}{0}%
  \addtocounter{probno}{1}%
  {\bf Problem~\#{\arabic{probno}}~[{#1}~pts]}\quad}

%no initial whitespace
\newcommand{\Probns}[1]{
  \setcounter{partno}{0}%
  \addtocounter{probno}{1}%
  {\bf Problem~\#{\arabic{probno}}~[{#1}~pts]}\quad}
\newcommand{\Probpart}[1]{%\rule{0in}{0in}\\ \phantom{xxx}
  \addtocounter{partno}{1}%
  {\bf (\#\arabic{probno}\alph{partno})~[{#1}~pts]}\ \ }

\newboolean{answers}
\newcommand{\Answer}[1]{\ifthenelse{\boolean{answers}}{{\bf Answer:}\ #1}{}\bigskip}

\usepackage{youngtab}
\begin{document}
\thispagestyle{empty}
{\bf Math 821 Problem Set \#5\\
Posted: Friday 4/1/11\\
Due date: Wednesday 4/13/11}

\prob In class on Friday, I asserted that if $G$ is a graph with
$n$ vertices and $c$ connected components, and $M$ is the signed
vertex-edge incidence matrix of $G$, then $\rank M=n-c$.  Prove
this statement (over any ground field).

\prob Fix a ground field $\Ff$ and a nonnegative integer $n$.
Let $V_k$ be the vector space with basis $\{\sigma_A\}$, where
$A$ ranges over all $k$-element subsets of $\{1,2,\dots,n\}$.
Define a linear transformation $\bd_k:V_k\to V_{k-1}$ as follows:
if $A=\{a_1,\dots,a_k\}$ with $a_1<\cdots<a_k$, then
$$\bd_k(\sigma_A)=\sum_{i=1}^k(-1)^{i+1}\sigma_{A\sm\{a_i\}}.$$
(Having defined $\bd_k$ on the basis elements, it extends
uniquely to all of $V_k$ by linearity.)

\probpart Prove that $d_k\circ d_{k+1}=0$ for all $k$.  (Note: I know this
calculation is done explicitly in Hatcher,
but it is so important that everyone should do it for themselves at least once!)
Conclude that
$$\im\bd_k\subseteq\ker\bd_{k+1}.$$

\probpart For $n=3$, write out the maps $\bd_i$ as explicit
matrices.

\probpart Prove that for every $k$, the set
$\{\bd_k(\sigma_A):\ 1\in A\}$
is a basis for the vector space $\im\bd_k$.

\probpart Use (3) to prove that in fact
$\im\bd_k=\ker\bd_{k+1}.$
(Hint: By (1), all you have to show is that these vector spaces
have the same dimension.)

\prob Consider the matrix
$$A=\begin{bmatrix} 1&1&0\\1&0&1\\0&1&1\end{bmatrix}.$$
Describe $\coker A$
(i) if $A$ is regarded as a linear transformation over $\Qq$;
(ii) if $A$ is regarded as a linear transformation over $\Zz$;
(iii) if $A$ is regarded as a linear transformation over $\Ff_q$
(the finite field with $q$ elements).

\prob Let $R=\Ff[x_1,\dots,x_n]$ be the ring of polynomials in $n$ 
variables over a field $\Ff$. A \emph{squarefree monomial} in $R$ is a 
product of distinct indeterminates (e.g., $x_1x_4x_5$, but not 
$x_1x_5^2$). Let $I$ be an ideal generated by squarefree monomials of 
degree $\geq 2$.

\probpart  Show that the set
  $$\Delta=\{\sigma\subset[n] \st \prod_{i\in\sigma} x_i\not\in I\}$$
is an abstract simplicial complex on $n$ vertices.

(This is called the
\emph{Stanley-Reisner complex} of $I$ --- or, alternately, $I$ is the 
Stanley-Reisner ideal of $\Delta$.)

\probpart  Describe  $\Delta$
looks like in the case 
that $I$ is (i) the zero ideal; (ii) generated by a single monomial
of degree~$d$; (iii) generated by all monomials of degree~$d$ for some $k\leq n$;
(iv) (assuming $n=2m$ is even) generated by the degree-2 monomials $x_1x_2$,
$x_3x_4$, \dots, $x_{2m-1}x_{2m}$.

\end{document}
\prob Find an explicit triangulation $\Delta$ of $\Rr P^2$ with
6 vertices, 15 edges and 10 faces.  (Hint: Draw the cell complex
picture and start subdividing.)

Compute the simplicial homology groups $H_k(\Delta;\Qq)$ of
\bang $\Delta$;
\bang $\Delta\sm\{\sigma\}$, where $\sigma$ is a single 2-simplex (does 
it matter which?);
\bang $\Delta\sm\{\sigma,\tau\}$, where $\sigma,\tau$ are (different) 
2-simplices (does it matter which?);
\bang $\Delta\cup\{\rho\}$, where $\rho$ is a 2-simplex not belonging to $\Delta$.
(By ``compute'', I mean ``determine up to isomorphism''.

You can do this in a computer algebra system such as Macaulay2.

\pagebreak

Macaulay2 is a free software system to do computations in commutative algebra.

To run Macaulay2, log into your math account,
open a terminal window, and type ``M2''.  You'll see something like this:
\begin{verbatim}
Macaulay2, version 1.4
with packages: ConwayPolynomials, Elimination, IntegralClosure, LLLBases,
               PrimaryDecomposition, ReesAlgebra, TangentCone
i1 : 
\end{verbatim}
You can compute the kernel of a matrix directly:
\begin{verbatim}
i1 : A=matrix{{1,2,3},{4,5,6}}

o1 = | 1 2 3 |
     | 4 5 6 |

              2        3
o1 : Matrix ZZ  <--- ZZ

i2 : ker A

o2 = image | -1 |
           | 2  |
           | -1 |

                               3
o2 : ZZ-module, submodule of ZZ

i3 : 
\end{verbatim}
By default, M2 works over $\Zz$ (that's what \texttt{ZZ} means).
Here I have typed in a $2\x3$ matrix $A$ corresponding
to a linear transformation $\Zz^3\to\Zz^2$ (for various reasons,
Macaulay writes the arrows right to left) and computed
its kernel, which Macaulay has described as the image (i.e.,
column space) of the matrix $B=\begin{bmatrix}-1\\2\\-1\end{bmatrix}$.

If you have a $\Zz$-module and you just want to know what it is
up to isomorphism, you can use the \texttt{prune} command:
\begin{verbatim}
i3 : prune ker A

       1
o3 = ZZ

o3 : ZZ-module, free
\end{verbatim}
Here is a computation of the homology of the complete graph
on 3 vertices (with 

In order to 

\end{document}
