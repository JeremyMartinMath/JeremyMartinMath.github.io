\documentclass{amsart}
\usepackage{amssymb,amsmath,amsthm,mathrsfs,graphics,hyperref,stmaryrd,psfrag,arcs,xypic}
\usepackage[enableskew]{youngtab}
\numberwithin{equation}{section}
\raggedbottom
\oddsidemargin=0in
\evensidemargin=0in
\textwidth=6.5in
\textheight=8.8in
\topmargin=0.25in
\headheight=0in
\headsep=0.2in
\footskip=0in
\parskip=10bp
\parindent=0bp

\newcommand{\excise}[1]{}
\newcommand{\Latex}[1]{\textbackslash\texttt{#1}}

\newcommand{\bigpad}{\rule[-14mm]{0mm}{30mm}}
\newcommand{\smallpad}{\rule[-1.5mm]{0mm}{5mm}}
\newcommand{\pad}{\rule[-3mm]{0mm}{8mm}}
\newcommand{\padup}{\rule{0mm}{5mm}}
\newcommand{\paddown}{\rule[-3mm]{0mm}{2mm}}
\newcommand{\blank}{\rule{1.25in}{0.25mm}}
\newcommand{\commentout}[1]{}
\newcommand{\yell}[1]{\fbox{\rule[-1mm]{0mm}{4mm} \large\bf #1 }}
\newcommand{\bang}{$\bullet$\quad}
\newcommand{\indnt}{\phantom{.}\qquad}
\newcommand{\littleline}{\begin{center}\rule{4in}{0.5bp}\end{center}}

\newcommand{\includefigure}[3]{{
  \begin{center}
  \resizebox{#1}{#2}{\includegraphics{{figs/#3}}}
  \end{center}}}
\newcommand{\includefigurewithinmath}[3]{{
  \resizebox{#1}{#2}{\includegraphics{{figs/#3}}}}}

%\newcommand{\defterm}[1]{\underline{\textbf{#1}}}
\newcommand{\defterm}[1]{\textbf{#1}}

\DeclareMathOperator{\ch}{\mathbf{ch}}
\DeclareMathOperator{\colspace}{colspace}
\DeclareMathOperator{\corank}{corank}
\DeclareMathOperator{\CST}{CST}
\DeclareMathOperator{\deln}{del}
\DeclareMathOperator{\diag}{diag}
\DeclareMathOperator{\ess}{ess}
\DeclareMathOperator{\Gr}{Gr}
\DeclareMathOperator{\Hom}{Hom}
\DeclareMathOperator{\im}{im}
\DeclareMathOperator{\Irr}{Irr}
\DeclareMathOperator{\Ind}{Ind}
\DeclareMathOperator{\Int}{Int}
\DeclareMathOperator{\lcm}{lcm}
\DeclareMathOperator{\link}{lk}
\DeclareMathOperator{\nullity}{nullity}
\DeclareMathOperator{\nullspace}{nullspace}
\DeclareMathOperator{\Poin}{Poin}
\DeclareMathOperator{\proj}{proj}
\DeclareMathOperator{\rank}{rank}
\DeclareMathOperator{\Res}{Res}
\DeclareMathOperator{\Span}{span}
\DeclareMathOperator{\supp}{supp}
\DeclareMathOperator{\row}{row}
\DeclareMathOperator{\rowspace}{rowspace}
\DeclareMathOperator{\sh}{sh}
\DeclareMathOperator{\tr}{tr}
\DeclareMathOperator{\wt}{wt}

\newtheorem{theorem}{Theorem}[section]
\newtheorem{proposition}[theorem]{Proposition}
\newtheorem{lemma}[theorem]{Lemma}
\newtheorem{corollary}[theorem]{Corollary}
\theoremstyle{definition}
\newtheorem{definition}[theorem]{Definition}
\newtheorem{example}[theorem]{Example}
\newtheorem{remark}[theorem]{Remark}
\newtheorem{problem}[theorem]{Problem}


\newcommand{\cor}{{\bf Corollary: }}
\newcommand{\defn}{{\bf Definition: }}
\newcommand{\defns}{{\bf Definitions: }}
\newcommand{\exa}{{\bf Example: }}
\newcommand{\fact}{{\bf Fact: }}
\newcommand{\lem}{{\bf Lemma: }}
\newcommand{\notn}{{\bf Notation: }}
\newcommand{\obs}{{\bf Observation: }}
\newcommand{\note}{{\bf Note: }}
\newcommand{\prop}{{\bf Proposition: }}
\newcommand{\rmk}{{\bf Remark: }}
\newcommand{\thm}{{\bf Theorem: }}

\newcommand{\basecase}{\emph{Base case: }}
\newcommand{\indstep}{\emph{Inductive step: }}
\newcommand{\skpr}{\emph{Sketch of proof: }}

\newcommand{\0}{\emptyset}
\newcommand{\Alt}{\mathfrak{A}}
\newcommand{\Braid}{Br}
\newcommand{\CHI}{\chi^{\phantom{*}}}
\newcommand{\Cl}{C\ell}
\newcommand{\covers}{\gtrdot}
\newcommand{\coveredby}{\lessdot}
\newcommand{\dedge}[1]{\overrightarrow{{#1}}}
\newcommand{\dom}{\rhd}
\newcommand{\domeq}{\unrhd}
\newcommand{\domby}{\lhd}
\newcommand{\dombyeq}{\unlhd}
\newcommand{\Fspan}{\Ff\text{-span}}
\newcommand{\isom}{\cong}
\newcommand{\join}{\vee}
\renewcommand{\Join}{\bigvee}
\newcommand{\Laff}{L^{\text{aff}}}
\newcommand{\lin}[1]{\overleftrightarrow{{#1}}}
\newcommand{\meet}{\wedge}
\newcommand{\Meet}{\bigwedge}
\newcommand{\ov}[1]{\overline{{#1}}}
\newcommand{\partn}{\vdash}
\newcommand{\qqandqq}{\qquad\text{and}\qquad}
\newcommand{\qandq}{\quad\text{and}\quad}
\newcommand{\qand}{\quad\text{and}}
\newcommand{\qbin}[2]{{\begin{bmatrix}#1\\#2\end{bmatrix}_q}}
\newcommand{\sd}{\triangle} % symmetric difference
\newcommand{\simK}{\underset{K}{\sim}} % Knuth equivalence
\newcommand{\simJ}{\underset{J}{\sim}} % jeu de taquin equivalence
\newcommand{\sm}{\setminus}
\newcommand{\st}{~|~}
\newcommand{\soln}{\textit{Solution:\ }}
\newcommand{\Sym}{\mathfrak{S}}
\newcommand{\un}[1]{\underset{*}{#1}}
\newcommand{\unA}{\un{A}}
\newcommand{\unB}{\un{B}}
\newcommand{\unw}{\un{w}}
\newcommand{\unx}{\un{x}}
\newcommand{\uny}{\un{y}}
\newcommand{\unz}{\un{z}}
\newcommand{\x}{\times}

\renewcommand{\aa}{\mathbf{a}}
\newcommand{\bb}{\mathbf{b}}
\newcommand{\nn}{\mathbf{n}}
\newcommand{\pp}{\mathbf{p}}
\newcommand{\qq}{\mathbf{q}}
\newcommand{\xx}{\mathbf{x}}
\newcommand{\yy}{\mathbf{y}}
\newcommand{\zz}{\mathbf{z}}
 
\newcommand{\A}{\mathcal{A}}
\newcommand{\B}{\mathcal{B}}
\newcommand{\C}{\mathcal{C}}
\newcommand{\M}{\mathcal{M}}
\renewcommand{\P}{\mathcal{P}}

\newcommand{\BB}{\mathscr{B}}  %% use these for fancy script fonts -- requires mathrsfs package
\newcommand{\CC}{\mathscr{C}}
\newcommand{\FF}{\mathscr{F}}
\newcommand{\II}{\mathscr{I}}
\newcommand{\LL}{\mathscr{L}}
\newcommand{\PP}{\mathscr{P}}
\renewcommand{\SS}{\mathscr{S}}
\newcommand{\XX}{\mathscr{X}}

\newcommand{\TT}{\tilde{T}}

\newcommand{\Aa}{\mathbb{A}}
\newcommand{\Cc}{\mathbb{C}}
\newcommand{\Ff}{\mathbb{F}}
\newcommand{\Nn}{\mathbb{N}}
\newcommand{\Pp}{\mathbb{P}}
\newcommand{\Qq}{\mathbb{Q}}
\newcommand{\Rr}{\mathbb{R}}
\newcommand{\Zz}{\mathbb{Z}}

\newcommand{\rhodef}{\rho^{\phantom{*}}_{{\rm def}}}
\newcommand{\rhotriv}{\rho^{\phantom{*}}_{{\rm triv}}}
\newcommand{\rhosign}{\rho^{\phantom{*}}_{{\rm sign}}}
\newcommand{\rhoreg}{\rho^{\phantom{*}}_{{\rm reg}}}
\newcommand{\chidef}{\chi^{\phantom{*}}_{{\rm def}}}
\newcommand{\chitriv}{\chi^{\phantom{*}}_{{\rm triv}}}
\newcommand{\chisign}{\chi^{\phantom{*}}_{{\rm sign}}}
\newcommand{\chireg}{\chi^{\phantom{*}}_{{\rm reg}}}
\newcommand{\scp}[2]{\left\langle #1,\:#2\right\rangle_G}
\newcommand{\scpH}[2]{\left\langle #1,\:#2\right\rangle_H}

\newcounter{probno}
\setcounter{probno}{0}
\newcounter{partno}
\setcounter{partno}{0}
%% versions that don't print the number of points
\newcommand{\prob}{
  \vskip10bp%
  \setcounter{partno}{0}%
  \addtocounter{probno}{1}%
  {\bf Problem~\#{\arabic{probno}}}\quad}
\newcommand{\probpart}{%\rule{0in}{0in}\\ \phantom{xxx}
  \addtocounter{partno}{1}%
  {\bf (\#\arabic{probno}\alph{partno})}\ \ }
\newcommand{\probcont}{%
  {\bf Problem~\#{\arabic{probno}}}~(\emph{continued})}
\newcommand{\probo}{
  \setcounter{partno}{0}%
  \addtocounter{probno}{1}%
  {\bf (\#\arabic{probno})}\ \ }
%% versions that do print the number of points
\newcommand{\Prob}[1]{
  \vskip10bp%
  \setcounter{partno}{0}%
  \addtocounter{probno}{1}%
  {\bf Problem~\#{\arabic{probno}}~[{#1}~pts]}\quad}
\newcommand{\Probpart}[1]{%\rule{0in}{0in}\\ \phantom{xxx}
  \addtocounter{partno}{1}%
  {\bf (\#\arabic{probno}\alph{partno})~[{#1}~pts]}\ \ }

%\renewcommand{\thefootnote}{\fnsymbol{footnote}}

\usepackage{youngtab}
\begin{document}
\thispagestyle{empty}
{\bf Math 821 Problem Set \#5\\
Posted: Friday 4/1/11\\
Due date: Wednesday 4/13/11}

\prob In class on Friday, I asserted that if $G$ is a graph with
$n$ vertices and $c$ connected components, and $M$ is the signed
vertex-edge incidence matrix of $G$, then $\rank M=n-c$.  Prove
this statement (over any ground field).

\prob Fix a ground field $\Ff$ and a nonnegative integer $n$.
Let $V_k$ be the vector space with basis $\{\sigma_A\}$, where
$A$ ranges over all $k$-element subsets of $\{1,2,\dots,n\}$.
Define a linear transformation $\bd_k:V_k\to V_{k-1}$ as follows:
if $A=\{a_1,\dots,a_k\}$ with $a_1<\cdots<a_k$, then
$$\bd_k(\sigma_A)=\sum_{i=1}^k(-1)^{i+1}\sigma_{A\sm\{a_i\}}.$$
(Having defined $\bd_k$ on the basis elements, it extends
uniquely to all of $V_k$ by linearity.)

\probpart Prove that $d_k\circ d_{k+1}=0$ for all $k$.  (Note: I know this
calculation is done explicitly in Hatcher,
but it is so important that everyone should do it for themselves at least once!)
Conclude that
$$\im\bd_k\subseteq\ker\bd_{k+1}.$$

\probpart For $n=3$, write out the maps $\bd_i$ as explicit
matrices.

\probpart Prove that for every $k$, the set
$\{\bd_k(\sigma_A):\ 1\in A\}$
is a basis for the vector space $\im\bd_k$.

\probpart Use (3) to prove that in fact
$\im\bd_k=\ker\bd_{k+1}.$
(Hint: By (1), all you have to show is that these vector spaces
have the same dimension.)

\prob Consider the matrix
$$A=\begin{bmatrix} 1&1&0\\1&0&1\\0&1&1\end{bmatrix}.$$
Describe $\coker A$
(i) if $A$ is regarded as a linear transformation over $\Qq$;
(ii) if $A$ is regarded as a linear transformation over $\Zz$;
(iii) if $A$ is regarded as a linear transformation over $\Ff_q$
(the finite field with $q$ elements).

\prob Let $R=\Ff[x_1,\dots,x_n]$ be the ring of polynomials in $n$ 
variables over a field $\Ff$. A \emph{squarefree monomial} in $R$ is a 
product of distinct indeterminates (e.g., $x_1x_4x_5$, but not 
$x_1x_5^2$). Let $I$ be an ideal generated by squarefree monomials of 
degree $\geq 2$.

\probpart  Show that the set
  $$\Delta=\{\sigma\subset[n] \st \prod_{i\in\sigma} x_i\not\in I\}$$
is an abstract simplicial complex on $n$ vertices.

(This is called the
\emph{Stanley-Reisner complex} of $I$ --- or, alternately, $I$ is the 
Stanley-Reisner ideal of $\Delta$.)

\probpart  Describe  $\Delta$
looks like in the case 
that $I$ is (i) the zero ideal; (ii) generated by a single monomial
of degree~$d$; (iii) generated by all monomials of degree~$d$ for some $k\leq n$;
(iv) (assuming $n=2m$ is even) generated by the degree-2 monomials $x_1x_2$,
$x_3x_4$, \dots, $x_{2m-1}x_{2m}$.

\end{document}
\prob Find an explicit triangulation $\Delta$ of $\Rr P^2$ with
6 vertices, 15 edges and 10 faces.  (Hint: Draw the cell complex
picture and start subdividing.)

Compute the simplicial homology groups $H_k(\Delta;\Qq)$ of
\bang $\Delta$;
\bang $\Delta\sm\{\sigma\}$, where $\sigma$ is a single 2-simplex (does 
it matter which?);
\bang $\Delta\sm\{\sigma,\tau\}$, where $\sigma,\tau$ are (different) 
2-simplices (does it matter which?);
\bang $\Delta\cup\{\rho\}$, where $\rho$ is a 2-simplex not belonging to $\Delta$.
(By ``compute'', I mean ``determine up to isomorphism''.

You can do this in a computer algebra system such as Macaulay2.

\pagebreak

Macaulay2 is a free software system to do computations in commutative algebra.

To run Macaulay2, log into your math account,
open a terminal window, and type ``M2''.  You'll see something like this:
\begin{verbatim}
Macaulay2, version 1.4
with packages: ConwayPolynomials, Elimination, IntegralClosure, LLLBases,
               PrimaryDecomposition, ReesAlgebra, TangentCone
i1 : 
\end{verbatim}
You can compute the kernel of a matrix directly:
\begin{verbatim}
i1 : A=matrix{{1,2,3},{4,5,6}}

o1 = | 1 2 3 |
     | 4 5 6 |

              2        3
o1 : Matrix ZZ  <--- ZZ

i2 : ker A

o2 = image | -1 |
           | 2  |
           | -1 |

                               3
o2 : ZZ-module, submodule of ZZ

i3 : 
\end{verbatim}
By default, M2 works over $\Zz$ (that's what \texttt{ZZ} means).
Here I have typed in a $2\x3$ matrix $A$ corresponding
to a linear transformation $\Zz^3\to\Zz^2$ (for various reasons,
Macaulay writes the arrows right to left) and computed
its kernel, which Macaulay has described as the image (i.e.,
column space) of the matrix $B=\begin{bmatrix}-1\\2\\-1\end{bmatrix}$.

If you have a $\Zz$-module and you just want to know what it is
up to isomorphism, you can use the \texttt{prune} command:
\begin{verbatim}
i3 : prune ker A

       1
o3 = ZZ

o3 : ZZ-module, free
\end{verbatim}
Here is a computation of the homology of the complete graph
on 3 vertices (with 

In order to 

\end{document}
