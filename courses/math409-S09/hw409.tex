\documentclass{amsart}
\usepackage{amssymb,amsmath,amsthm,mathrsfs,graphics,hyperref,stmaryrd,psfrag,arcs,xypic}
\usepackage[enableskew]{youngtab}
\numberwithin{equation}{section}
\raggedbottom
\oddsidemargin=0in
\evensidemargin=0in
\textwidth=6.5in
\textheight=8.8in
\topmargin=0.25in
\headheight=0in
\headsep=0.2in
\footskip=0in
\parskip=10bp
\parindent=0bp

\newcommand{\excise}[1]{}
\newcommand{\Latex}[1]{\textbackslash\texttt{#1}}

\newcommand{\bigpad}{\rule[-14mm]{0mm}{30mm}}
\newcommand{\smallpad}{\rule[-1.5mm]{0mm}{5mm}}
\newcommand{\pad}{\rule[-3mm]{0mm}{8mm}}
\newcommand{\padup}{\rule{0mm}{5mm}}
\newcommand{\paddown}{\rule[-3mm]{0mm}{2mm}}
\newcommand{\blank}{\rule{1.25in}{0.25mm}}
\newcommand{\commentout}[1]{}
\newcommand{\yell}[1]{\fbox{\rule[-1mm]{0mm}{4mm} \large\bf #1 }}
\newcommand{\bang}{$\bullet$\quad}
\newcommand{\indnt}{\phantom{.}\qquad}
\newcommand{\littleline}{\begin{center}\rule{4in}{0.5bp}\end{center}}

\newcommand{\includefigure}[3]{{
  \begin{center}
  \resizebox{#1}{#2}{\includegraphics{{figs/#3}}}
  \end{center}}}
\newcommand{\includefigurewithinmath}[3]{{
  \resizebox{#1}{#2}{\includegraphics{{figs/#3}}}}}

%\newcommand{\defterm}[1]{\underline{\textbf{#1}}}
\newcommand{\defterm}[1]{\textbf{#1}}

\DeclareMathOperator{\ch}{\mathbf{ch}}
\DeclareMathOperator{\colspace}{colspace}
\DeclareMathOperator{\corank}{corank}
\DeclareMathOperator{\CST}{CST}
\DeclareMathOperator{\deln}{del}
\DeclareMathOperator{\diag}{diag}
\DeclareMathOperator{\ess}{ess}
\DeclareMathOperator{\Gr}{Gr}
\DeclareMathOperator{\Hom}{Hom}
\DeclareMathOperator{\im}{im}
\DeclareMathOperator{\Irr}{Irr}
\DeclareMathOperator{\Ind}{Ind}
\DeclareMathOperator{\Int}{Int}
\DeclareMathOperator{\lcm}{lcm}
\DeclareMathOperator{\link}{lk}
\DeclareMathOperator{\nullity}{nullity}
\DeclareMathOperator{\nullspace}{nullspace}
\DeclareMathOperator{\Poin}{Poin}
\DeclareMathOperator{\proj}{proj}
\DeclareMathOperator{\rank}{rank}
\DeclareMathOperator{\Res}{Res}
\DeclareMathOperator{\Span}{span}
\DeclareMathOperator{\supp}{supp}
\DeclareMathOperator{\row}{row}
\DeclareMathOperator{\rowspace}{rowspace}
\DeclareMathOperator{\sh}{sh}
\DeclareMathOperator{\tr}{tr}
\DeclareMathOperator{\wt}{wt}

\newtheorem{theorem}{Theorem}[section]
\newtheorem{proposition}[theorem]{Proposition}
\newtheorem{lemma}[theorem]{Lemma}
\newtheorem{corollary}[theorem]{Corollary}
\theoremstyle{definition}
\newtheorem{definition}[theorem]{Definition}
\newtheorem{example}[theorem]{Example}
\newtheorem{remark}[theorem]{Remark}
\newtheorem{problem}[theorem]{Problem}


\newcommand{\cor}{{\bf Corollary: }}
\newcommand{\defn}{{\bf Definition: }}
\newcommand{\defns}{{\bf Definitions: }}
\newcommand{\exa}{{\bf Example: }}
\newcommand{\fact}{{\bf Fact: }}
\newcommand{\lem}{{\bf Lemma: }}
\newcommand{\notn}{{\bf Notation: }}
\newcommand{\obs}{{\bf Observation: }}
\newcommand{\note}{{\bf Note: }}
\newcommand{\prop}{{\bf Proposition: }}
\newcommand{\rmk}{{\bf Remark: }}
\newcommand{\thm}{{\bf Theorem: }}

\newcommand{\basecase}{\emph{Base case: }}
\newcommand{\indstep}{\emph{Inductive step: }}
\newcommand{\skpr}{\emph{Sketch of proof: }}

\newcommand{\0}{\emptyset}
\newcommand{\Alt}{\mathfrak{A}}
\newcommand{\Braid}{Br}
\newcommand{\CHI}{\chi^{\phantom{*}}}
\newcommand{\Cl}{C\ell}
\newcommand{\covers}{\gtrdot}
\newcommand{\coveredby}{\lessdot}
\newcommand{\dedge}[1]{\overrightarrow{{#1}}}
\newcommand{\dom}{\rhd}
\newcommand{\domeq}{\unrhd}
\newcommand{\domby}{\lhd}
\newcommand{\dombyeq}{\unlhd}
\newcommand{\Fspan}{\Ff\text{-span}}
\newcommand{\isom}{\cong}
\newcommand{\join}{\vee}
\renewcommand{\Join}{\bigvee}
\newcommand{\Laff}{L^{\text{aff}}}
\newcommand{\lin}[1]{\overleftrightarrow{{#1}}}
\newcommand{\meet}{\wedge}
\newcommand{\Meet}{\bigwedge}
\newcommand{\ov}[1]{\overline{{#1}}}
\newcommand{\partn}{\vdash}
\newcommand{\qqandqq}{\qquad\text{and}\qquad}
\newcommand{\qandq}{\quad\text{and}\quad}
\newcommand{\qand}{\quad\text{and}}
\newcommand{\qbin}[2]{{\begin{bmatrix}#1\\#2\end{bmatrix}_q}}
\newcommand{\sd}{\triangle} % symmetric difference
\newcommand{\simK}{\underset{K}{\sim}} % Knuth equivalence
\newcommand{\simJ}{\underset{J}{\sim}} % jeu de taquin equivalence
\newcommand{\sm}{\setminus}
\newcommand{\st}{~|~}
\newcommand{\soln}{\textit{Solution:\ }}
\newcommand{\Sym}{\mathfrak{S}}
\newcommand{\un}[1]{\underset{*}{#1}}
\newcommand{\unA}{\un{A}}
\newcommand{\unB}{\un{B}}
\newcommand{\unw}{\un{w}}
\newcommand{\unx}{\un{x}}
\newcommand{\uny}{\un{y}}
\newcommand{\unz}{\un{z}}
\newcommand{\x}{\times}

\renewcommand{\aa}{\mathbf{a}}
\newcommand{\bb}{\mathbf{b}}
\newcommand{\nn}{\mathbf{n}}
\newcommand{\pp}{\mathbf{p}}
\newcommand{\qq}{\mathbf{q}}
\newcommand{\xx}{\mathbf{x}}
\newcommand{\yy}{\mathbf{y}}
\newcommand{\zz}{\mathbf{z}}
 
\newcommand{\A}{\mathcal{A}}
\newcommand{\B}{\mathcal{B}}
\newcommand{\C}{\mathcal{C}}
\newcommand{\M}{\mathcal{M}}
\renewcommand{\P}{\mathcal{P}}

\newcommand{\BB}{\mathscr{B}}  %% use these for fancy script fonts -- requires mathrsfs package
\newcommand{\CC}{\mathscr{C}}
\newcommand{\FF}{\mathscr{F}}
\newcommand{\II}{\mathscr{I}}
\newcommand{\LL}{\mathscr{L}}
\newcommand{\PP}{\mathscr{P}}
\renewcommand{\SS}{\mathscr{S}}
\newcommand{\XX}{\mathscr{X}}

\newcommand{\TT}{\tilde{T}}

\newcommand{\Aa}{\mathbb{A}}
\newcommand{\Cc}{\mathbb{C}}
\newcommand{\Ff}{\mathbb{F}}
\newcommand{\Nn}{\mathbb{N}}
\newcommand{\Pp}{\mathbb{P}}
\newcommand{\Qq}{\mathbb{Q}}
\newcommand{\Rr}{\mathbb{R}}
\newcommand{\Zz}{\mathbb{Z}}

\newcommand{\rhodef}{\rho^{\phantom{*}}_{{\rm def}}}
\newcommand{\rhotriv}{\rho^{\phantom{*}}_{{\rm triv}}}
\newcommand{\rhosign}{\rho^{\phantom{*}}_{{\rm sign}}}
\newcommand{\rhoreg}{\rho^{\phantom{*}}_{{\rm reg}}}
\newcommand{\chidef}{\chi^{\phantom{*}}_{{\rm def}}}
\newcommand{\chitriv}{\chi^{\phantom{*}}_{{\rm triv}}}
\newcommand{\chisign}{\chi^{\phantom{*}}_{{\rm sign}}}
\newcommand{\chireg}{\chi^{\phantom{*}}_{{\rm reg}}}
\newcommand{\scp}[2]{\left\langle #1,\:#2\right\rangle_G}
\newcommand{\scpH}[2]{\left\langle #1,\:#2\right\rangle_H}

\newcounter{probno}
\setcounter{probno}{0}
\newcounter{partno}
\setcounter{partno}{0}
%% versions that don't print the number of points
\newcommand{\prob}{
  \vskip10bp%
  \setcounter{partno}{0}%
  \addtocounter{probno}{1}%
  {\bf Problem~\#{\arabic{probno}}}\quad}
\newcommand{\probpart}{%\rule{0in}{0in}\\ \phantom{xxx}
  \addtocounter{partno}{1}%
  {\bf (\#\arabic{probno}\alph{partno})}\ \ }
\newcommand{\probcont}{%
  {\bf Problem~\#{\arabic{probno}}}~(\emph{continued})}
\newcommand{\probo}{
  \setcounter{partno}{0}%
  \addtocounter{probno}{1}%
  {\bf (\#\arabic{probno})}\ \ }
%% versions that do print the number of points
\newcommand{\Prob}[1]{
  \vskip10bp%
  \setcounter{partno}{0}%
  \addtocounter{probno}{1}%
  {\bf Problem~\#{\arabic{probno}}~[{#1}~pts]}\quad}
\newcommand{\Probpart}[1]{%\rule{0in}{0in}\\ \phantom{xxx}
  \addtocounter{partno}{1}%
  {\bf (\#\arabic{probno}\alph{partno})~[{#1}~pts]}\ \ }

%\renewcommand{\thefootnote}{\fnsymbol{footnote}}

\begin{document}

\begin{center}\large Math 409, Spring 2009\\
Homework/In-Class Problems
(updated 2/24/09)
\end{center}

\section{Euclidean geometry}

Unless otherwise specified, do each construction on its own page.
(It is okay to use the front and back of the same piece of paper for different constructions.)
Always make your constructions big enough that all the parts of the drawing are easily readable.

\begin{EG}
Draw a straight line segment $\seg{AB}$.  Draw a point~$P$ somewhere else 
on the page.  Now construct a line segment of exactly the same length as $\seg{AB}$ 
with~$P$ as one endpoint.  Label the new line segment~$PQ$.
\end{EG}

\begin{EG}
Draw a triangle $\tri{ABC}$.  Now copy it exactly somewhere else on the 
page.  Label the new copy $\tri{PQR}$.
\end{EG}

\begin{EG}
Construct an equilateral triangle.  Label its vertices~$A,B,C$.
\end{EG}

\begin{EG}
Draw an angle~$\ang{ABC}$.  Draw a point~$P$ somewhere else on the page.
Now construct an angle of exactly the same measure as $\ang{ABC}$, with~$P$ as
its vertex.
\end{EG}

\begin{EG}
Draw an angle $\ang{ABC}$.  Now bisect it.
\end{EG}

\begin{EG}\label{perp-bisect}
Draw a straight line segment $\seg{AB}$.  Now construct a line
perpendicular to $\seg{AB}$ that bisects its length.  Label the line~$L$.
\end{EG}

\begin{EG}\label{perp-line}
Draw a line~$L$ and a point~$P$ that does not lie on~$L$.
Now construct a line through~$P$ that is perpendicular to~$L$.
Label this line~$M$.  (Hint: Use EG~\ref{perp-bisect}.)
\end{EG}

\begin{EG}\label{parallel-line}
Draw a line~$L$ and a point~$P$ that does not lie on~$L$.
Now construct a line through~$P$ that is parallel to~$L$.
Label this line~$N$.  (Hint: Use EG~\ref{perp-line}.  If you want, you can use the same 
sketch for EG~\ref{perp-line} and EG~\ref{parallel-line}, provided that both~$M$ and~$N$ are clearly
labelled.)
\end{EG}

\begin{EG}\label{square}
Construct a square.  Label its vertices~$A,B,C,D$.
\end{EG}

\begin{EG}\label{hex}
Construct a regular hexagon.  Label its vertices~$A,B,C,D,E,F$.
\end{EG}

\begin{EG}
Let $\tri{ABC}$ be an equilateral triangle.  Let $X$, $Y$, and~$Z$ be the midpoints of the segments $\seg{AB}$, $\seg{AC}$, and $\seg{BC}$ respectively.
Prove that $\tri{XYZ}$ is equilateral.  \textbf{Bonus:} State
and prove a conjecture about what happens with if $\tri{ABC}$ is
only assumed to be isosceles, rather than equilateral.
\end{EG}

\begin{EG}\label{angle-bisect}
Draw an angle.  Bisect it.
\end{EG}

\begin{EG}
Draw a line segment $\seg{AB}$.  Trisect it.  That is, find a point $C$
on $\seg{AB}$ such that $AC=\frac13AB$.  Can you extend your method to
$n$-section?
\end{EG}

\begin{EG}
\begin{enumerate}
\item[(a)] Explain why your construction of a perpendicular bisector in EG~\ref{perp-bisect} works.
\item[(b)] Explain why your construction of an angle bisector in EG~\ref{angle-bisect} works.
\end{enumerate}
\end{EG}

\begin{EG}
Prove that the angles of a triangle sum to $180^\circ$.
\end{EG}

\begin{EG}
Prove your observation about the angles in SA~\ref{double-angle}.
\end{EG}

\begin{EG}\label{ITT}
Prove that the base angles of an isosceles triangle are congruent.
That is, if $\tri{ABC}$ is a triangle and $AB=AC$,
then prove that $m\ang{ABC}=\ang{ACB}$.
\end{EG}

\begin{EG}\label{ITT-converse}
Let $\tri{ABC}$ be a triangle such that $m\ang{BAC}=m\ang{ABC}$.
Prove that $AC=BC$.
\end{EG}

\begin{EG} \label{circumcircle-explain}
In SA~\ref{circumcircle}, you started with an arbitrary triangle
$\tri{ABC}$ and constructed its circumscribed circle.
Explain why your construction works.
\end{EG}

\begin{EG} \label{inscribed-rhombus}
Suppose that~$R$ is a rhombus inscribed in a circle
(that is, it is possible to draw a circle containing all four
points of the rhombus).  Prove that~$R$ is in fact not just
a rhombus, but a square.
\end{EG}

\begin{EG} \label{VAT}
Prove the Vertical Angle Theorem: when two lines meet and form
four angles, each pair of angles not sharing a common side has
equal measure.
\end{EG}

\begin{EG}\label{ray}
Write a precise definition for the term ``ray''.  Your definition
may (in fact, should) use other axioms and definitions as needed.
\end{EG}

\begin{EG}
Prove that the interior angles of a quadrilateral add up to $360^\circ$.
\end{EG}

\begin{EG}
%% requires SAS
\label{point-from-line}
Define the \emph{distance of a point~$p$ from a line $\ell$} to be the length of the line segment from~$p$ to a point on~$\ell$.  Prove that a point is equidistant from the sides of an angle iff it is on the angle bisector.
\end{EG}

\begin{EG}
Use EG 11 to show that the three angle bisectors of a triangle meet in one point.
\end{EG}

\begin{EG}
Explain why your Sketchpad construction of a ``quintasection'' in
SA~\ref{SA:trisection} works.  (Hint: Use similar triangles.
There aren't many triangles in the construction as it is,
so you'll have to draw one more line.)
\end{EG}

%% Delete the second part of the hint, or make it more open-ended:
%% ``You may need to draw one or more new lines, or to extend the lines that are already in the figure.''
%% Also, have them do the trisection instead of the quintasection,
%% and to adopt the notation of SA~\ref{SA:trisection}, so as to
%% save Jeremy a lot of trouble next time!

\begin{EG}
Let $PQRS$ be a parallelogram, that is, a quadrilateral such that
$\lin{PQ}$ is parallel to $\lin{RS}$, and 
$\lin{PS}$ is parallel to $\lin{QR}$, as in the following figure.
\includefigure{4in}{1in}{pgram}
\begin{itemize}
\item[a.] Prove that opposite sides of the parallelogram have equal length,
i.e., that $PQ=RS$ and $PS=QR$.  (Hint: Use ASA.)
\item[b.] Prove that opposite angles are equal, i.e., that
$m\ang{PQR}=m\ang{RSP}$ and $m\ang{QRS}=m\ang{SPQ}$.  (Hint: Use part (a).)
\end{itemize}
\end{EG}

\begin{EG}
Prove that the diagonals of a parallelogram meet at a right angle
iff the parallelogram is a rhombus.  (Remember that ``iff''
means ``if and only if''---so you need to prove two separate things here:
(1) \emph{if} the diagonals of a parallelogram meet at a right angle
\emph{then} the parallelogram is a rhombus, and 
(2) \emph{if} a parallelogram is a rhombus
\emph{then} its diagonals meet at a right angle.)
\end{EG}

\begin{EG}
Prove that the diagonals of a parallelogram are congruent to each other
iff the parallelogram is a rectangle.
\end{EG}

\begin{EG}
Three circles in the plane, all of the same radius $r$, pass through
a common point $O$.  Show that their other points of intersection
$X,Y,Z$ lie on a circle of radius $r$.
\end{EG}

%----------------------------------------


%============================================================
%============================================================
%============================================================
%============================================================
%============================================================
\section{\textit{Sketchpad} assignments}

Do each problem in a fresh sketch and send it to me (jmartin@math.ku.edu) as an 
e-mail attachment.  Be sure to name each sketch exactly as specified, and be sure 
to include your last name and the problem numbers in the subject line of your 
e-mail.  For example, if I were handing in problems 11 and 15, the subject line 
should read ``Martin SA~11, 15''.

In these problems, ``draw'' means to make an arbitrary figure,
while ``construct'' means to build a figure based on previous objects.
For example, if the instructions say ``draw a triangle,'' you should
just put three independent points in your sketch and connect them
with three line segments --- you should not specify any relationship
between the points.

\begin{SA}
Construct an equilateral triangle with vertices labeled~$A,B,C$.  Send it 
to me in a sketch named EQUILATERAL.
\end{SA}

\begin{SA}
Construct a regular hexagon with vertices labeled~$A,B,C,D,E,F$.  Send it 
to me in a sketch named HEX.
\end{SA}

\begin{SA}
Construct a square with vertices labeled~$A,B,C,D$.  Send it to me in a 
sketch named SQUARE.
\end{SA}

\begin{SA}\label{double-angle}
Draw a line segment~$AB$.  Construct its midpoint and label it~$C$.
Construct a circle centered at~$C$ that has~$AB$ as a 
diameter.  Draw a point~$D$ on the circle.
Measure the angles $\ang{ADB}$, $\ang{CAD}$ and $\ang{BCD}$.  Now move the point~$D$ around.  Send it to me in a sketch named SA4.
Tell me what you observe about $m\ang{ADB}$.
Tell me what you observe about $m\ang{CAD}$ and $m\ang{BCD}$
relative to each other.
\end{SA}

\begin{SA}
Start with the construction of SA~\ref{double-angle}.
Now draw a point~$E$ anywhere on your worksheet, and measure $\ang{AEB}$.
Move the point~$E$ around inside the circle and tell me what you observe about
$m\ang{AEB}$.  Now move~$E$ around outside the circle and tell me what you 
observe about $m\ang{AEB}$.  Make a conjecture about how $m\ang{AEB}$
depends on the position of~$E$.
\end{SA}

\begin{SA} \label{triangle-centers}
Start by drawing a triangle $\tri{ABC}$.
\begin{itemize}
  \item[a.] In blue, construct the perpendicular bisectors of each
side of the triangle.  Move the points $A$, $B$, $C$ around.
Tell me on your sketch whether the blue lines meet in one, two or three points.
  \item[b.] In green, construct the lines bisecting each angle of the
triangle.  Move the points $A$, $B$, $C$ around.
Tell me on your sketch whether the green lines meet in one, two or three points.
  \item[c.] In red, draw lines connecting each vertex of the triangle to the midpoint of the opposite side.  Move the points $A$, $B$, $C$ around.
Tell me on your sketch whether the red lines meet in one, two or three points.
\end{itemize}
Send your sketch to me in a file named TRIANGLE.
\end{SA}

\begin{SA} \label{circumcircle}
Draw a triangle $\tri{ABC}$.  Construct its circumscribed circle,
that is, a circle which passes through
all three points $A,B,C$.  (Do not use any of the circle or
arc constructions from the Construct menu.)
Move the triangle around to make sure that your
construction works.  Send your sketch to me in a sketch named
CIRCUMCIRCLE.
\end{SA}

\begin{SA}
Start with the construction of SA~\ref{circumcircle}.
Label the center of the circle $Z$.  Measure the angle
$\ang{ABC}$ and move the triangle around some more, so that $Z$ will
move around too.  Tell me what the measure of the angle is when $Z$ is
(a) inside $\tri{ABC}$; (b) on segment $\seg{AC}$; (c) outside
$\tri{ABC}$.
\end{SA}

\begin{SA}
Draw a rectangle $Q$.  Construct a new quadrilateral~$R$ whose vertices
are the midpoints of the sides of~$Q$.  Tell me what kind of
quadrilateral~$R$ is (as specifically as possible).  Give a proof
that your answer is correct.
\end{SA}

\begin{SA}
Draw a rhombus $Q$.  Construct a new quadrilateral~$R$ whose vertices
are the midpoints of the sides of~$Q$.  Tell me what kind of
quadrilateral~$R$ is (as specifically as possible).  Give a proof
that your answer is correct.
\end{SA}

\begin{SA}
Draw an arbitrary quadrilateral $Q$.  Construct a new quadrilateral~$R$
whose vertices are the midpoints of the sides of~$Q$.  Tell me what kind of
quadrilateral~$R$ is (as specifically as possible).  Give a proof
that your answer is correct.
\end{SA}

\begin{SA} Start by drawing a circle~$O$.
Put four points $A,B,C,D$ on $O$ and connect them
to form a quadrilateral. Use \textit{Sketchpad} to calculate $m\ang{ABC}+m\ang{ADC}$.
What happens when you move the points around?
\end{SA}

\begin{SA}
Draw a circle and put a point on it.  Construct a line which is
tangent to the circle at the point you drew.
\end{SA}

\begin{SA} \label{triangle-same-center}
Draw a triangle $\tri{ABC}$ and use Sketchpad to measure the side
lengths.  Construct the angle bisector of $\ang{BAC}$,
and color it green.   Find the midpoint $X$ of $\seg{BC}$, construct
the segment $\seg{AX}$, and color it red.  Move
the points around and tell me what happens when the red and green lines
(or line segments) overlap.
\end{SA}

\begin{SA} \label{incircle}
Draw a triangle $\tri{ABC}$.  Now construct its inscribed circle,
that is, a circle which touches every side of the triangle but
does not cross any side.  Move the triangle around to make sure that your
construction works.  Send your sketch to me in a sketch named
INCIRCLE.
\end{SA}

\begin{SA} \label{rhombus}
Construct a rhombus $ABCD$.  (A rhombus is a quadrilateral with
four sides of equal length.)  To earn full credit, you should
be able to drag the vertices around so that the lengths and angles
can vary freely.  For example, you should be able to make your figure
look like any possible rhombus (such as all of the following):

\includefigure{6in}{1in}{rhombi}

Send your construction to me in a sketch named RHOMBUS.  Before you 
send it,
hide everything except the four points and four sides of the rhombus.

Note: One idea you may have is to draw a rectangle and connect the midpoints.
However, I don't think you'll be able to get all possible rhombi
this way (although you should feel free to try and prove me wrong).
\end{SA}

\begin{SA}
Draw a circle, label its center $O$, and draw a chord of the circle
(that is, a line segment whose endpoints lie on the circle).  Call those
endpoints $A$ and $B$.  Now place two points $C$ and $D$ on the circle, 
one on each side of the chord.  Measure the angles $\ang{ACB}$, 
$\ang{ADB}$, and $\ang{AOB}$.  Then answer the following 
questions.

\begin{enumerate}

\item[a.] Move $A$ around the circle while keeping the other 
points fixed, and while keeping $C$ and $D$ on different sides of the 
chord $\seg{AB}$.  Tell me how the angles change as you do so.

\item[b.] Move $C$ and $D$ around, while keeping them on different sides 
of the chord $\seg{AB}$.  Tell me how the angles change as you do so.

\item[c.] How do you know which of $\ang{ACB}$ and $\ang{ADB}$ is bigger?
When are they equal?

\item[d.] Tell me an equation that you think relates
$\ang{ACB}$ and $\ang{ADB}$.

\item[e.] Tell me an equation that you think relates
$\ang{AOB}$ to the \emph{smaller} of $\ang{ACB}$ and $\ang{ADB}$.

\end{enumerate}

Write your answers in a text box in your sketch.  Send the sketch
to me under the name CIRCLEANGLES.
\end{SA}

\begin{SA}
\label{SA:trisection}
\textit{Trisection of a line segment.}
Draw a line segment $\seg{AE}$.  Construct a new line $\ell$ somewhere 
else in your sketch, and construct points $W, X, Y, Z$ appearing in that order on $\ell$,
so that $WX=XY=YZ$.  (It doesn't matter what length these segments have
as long as they're all equal.)  Then
find a point $D$ so that (i) $AE=AD$ and (ii) $\lin{AW}$
is parallel to $\lin{DZ}$.

To find the points $W,X,Y,Z,D$, you'll probably
have to draw a bunch of circles, which you should
hide after you finish constructing the points.

Now, construct a line $m$ parallel to $\lin{AW}$ containing $X$,
and let $B$ be the point where $m$ meets $\lin{AD}$.  Then
construct a line $n$ parallel to $\lin{AW}$ containing $Y$,
and let $C$ be the point where $n$ meets $\lin{AD}$.

Next, use Sketchpad to convince yourself that $AB=BC=CD=\frac13AD$. (Move 
the points around, and do some measurement.)  If you want to trisect 
$\seg{AE}$ itself, you should be able at this point to find a point $T$ on 
$\seg{AE}$ such that $AT=\frac13AE$.) Do \emph{not} send me this sketch.

% Add a problem in which the students have to fill in the proof
% that the construction works.  Probably write it out for them and
% have them fill in the reasons.

Finally, construct a new sketch with a ``quintasection'',
in which you draw an arbitrary line segment
$AF$ and divide it into five equal-length segments $AB,BC,CD,DE,EF$,
adapting the method you've just learned.  Send me this sketch
under the name FIVE.  (This sketch is the only thing you have to hand in for
this problem.)
\end{SA}

\begin{SA}
\newcommand{\MM}{{\mathcal M}}
\newcommand{\TT}{{\mathcal T}}
Draw an arbitrary triangle $\TT$.  Construct the midpoints
of the sides of $\mathcal{T}$ and connect them to form a new triangle
$\MM(\TT)$.  Construct the medians\footnote{A \emph{median} of a triangle
is a segment from a vertex of the triangle to the midpoint of the opposite side.} of $\TT$ and the midpoints of the sides of $\MM(\TT)$.
\begin{itemize}
\item[a.] Send me this sketch under the name MID.
\item[b.] Tell me the relationship between the midpoints of $\MM(\TT)$ and the
medians of $\TT$.
\item[c.] Tell me the relationship between the medians of $\MM(\TT)$ and the
medians of $\TT$.
\item[d.] Tell me an equation that relates the area of $\MM(\TT)$ and the
area of $\TT$.  (To measure the area of a triangle in Sketchpad, use the arrow tool to select its points.  Then select \textbf{Interior} from the \textbf{Construct} menu (which will make the triangle change color) and select \textbf{Area} from the \textbf{Measure} menu.)
\end{itemize}
Repeat this construction five more times to make triangles $\TT_0 = \TT$; $\TT_1 =\MM(\TT_0)$; $\TT_2 =\MM(\TT_1)$; $\TT_3 =\MM(\TT_2)$; $\TT_4 =\MM(\TT_3)$; $\TT_5 =\MM(\TT_4)$.  (Don't send me a sketch of this.)
\begin{itemize}
\item[e.] Tell me the relationship between the medians of all these triangles.
\item[f.] Tell me an equation that relates the area of $\MM(\TT_n)$ and the area of $\TT$.
\end{itemize}
\end{SA}

\begin{SA}
Use Sketchpad to convince yourself that there is no ``SSA congruence theorem''.
That is, construct two triangles $\tri{ABC}$, $\tri{A'B'C'}$ such that
  $$AB=A'B',\quad BC=B'C', \quad\text{and}\quad m\ang{BAC}=m\ang{B'A'C'},
  \qquad\text{but}\qquad
  \mathbf{\tri{ABC}\not\cong\tri{A'B'C'}.}$$

(Hints: It is okay if $A=A'$ and $B=B'$; in fact, it will make 
the construction easier to understand.  So you need to construct a segment
$AB$ and two points $C,C'$ such that $BC=BC'$ and $m\ang{BAC}=m\ang{BAC'}$,
but $\tri{ABC}\not\cong\tri{ABC'}$.)

Do this at least twice, once if $m\ang{BAC}$ is an acute angle (i.e.,
$0^\circ<m\ang{BAC}<90^\circ$), and once if $m\ang{BAC}$
is an obtuse angle (i.e., $90^\circ<m\ang{BAC}<180^\circ$).

What happens if $m\ang{BAC}=90^\circ$?
\end{SA}



%----------------------------------------
\begin{SA} \textit{The Euler line.}
Draw a triangle.  Find the circumcenter (where the perpendicular
bisectors of the sides meet; see SA~6a) and color it blue.
Find the centroid (where the medians meet; see SA~6c) and color
it green.  Find the orthocenter (where the altitudes\footnote{%
  An \emph{altitude} of a triangle is a line through one of the
  vertices that is perpendicular to the opposite side of the triangle.}
meet) and color it red.  Draw a line segment connecting the
circumcenter to the orthocenter.  What do you notice?  State your observation
as a conjecture about all triangles.  Deform the triangle to convince
yourself that your conjecture works.  Send me the sketch under the name
EULER.  All I should see is the original triangle, the colored points,
and the line segment.
\end{SA}

\begin{SA} \textit{Pappus' theorem.}
Draw two lines $\ell,\ell'$ and color them black.  Put three points $A,B,C$ on $\ell$
and three points $A',B',C'$ on $\ell'$.  Construct the lines
  $$\lin{AB'},\ \ \lin{BA'},\quad
    \lin{AC'},\ \ \lin{CA'},\quad
    \lin{BC'},\ \ \lin{CB'}$$
and color them red (to make it easier to distinguish them from $\ell$ and $\ell'$).
Then construct the points\footnote{%
  The symbol $\cap$ means ``intersection''.  So $X$ is the unique
  point where $\lin{AB'}$ meets $\lin{BA'}$, etc.}
  $$X = \lin{AB'}\cap\lin{BA'},\qquad
    Y = \lin{AC'}\cap\lin{CA'},\qquad
    Z = \lin{BC'}\cap\lin{CB'}.$$
\begin{itemize}
\item[a.] What do you notice about the points $X$, $Y$ and $Z$?
State your conclusion as a conjecture.
\item[b.]  What happens if you move the points $A,B,C,A',B',C'$ around?
In particular, does your conjecture still remain true if you change
which of $A,B,C$ is between the others?
\item[c.] What happens to your sketch when $A=B$?  (In particular,
what happens to the points $X,Y,Z$?)
\end{itemize}
Send me the sketch under the name PAPPUS, with your answers included in a
text box.
\end{SA}

\begin{SA} \textit{Ptolemy's theorem.}
Draw a circle and inscribe a convex\footnote{%
  ``Convex'' just means that if you walk around the circle, you
  should arrive at $A$, $B$, $C$, and $D$ in that order.  So the
  sides of the quadrilateral are $\seg{AB}$, $\seg{BC}$, $\seg{CD}$
  and $\seg{DA}$, and its diagonals are $\seg{AC}$ and $\seg{BD}$.}
quadrilateral ABCD in it. Measure
  $$AB\cdot CD+AD\cdot BC \qandq AC\cdot BD.$$
\begin{itemize}
\item[a.] What do you notice? State your conclusion as a conjecture.
\item[b.] Deform the quadrilateral to convince yourself this is a plausible
hypothesis. Which deformations maximize the expression 
$AB\cdot CD + AD\cdot BC$? What kind of quadrilateral gives the maximum value?
\item[c.] Is your conjecture still true for an arbitrary quadrilateral (that is,
a quadrilateral whose points do not all lie on a common circle)?
\end{itemize}
Wait, there's more!  Measure the area of the quadrilateral $ABCD$,
and calculate
  $$\Omega = \frac{AC\cdot BD}{\text{area of } ABCD}.$$
Deform the quadrilateral some more and see what happens to the
number~$\Omega$
\begin{itemize}
\item[d.] How can you make $\Omega$ as small as possible?  How small can it get?
\item[e.] How can you make $\Omega$ as large as possible?  How large can it get?
\end{itemize}
Send me the sketch under the name PTOLEMY, with your answers included in a
text box.
\end{SA}
\pagebreak

\section{Transformational geometry}

\begin{TG}
What is the complete set of symmetries of\dots
\begin{itemize}
\item[(a)] \dots a line segment?
\item[(b)] \dots an equilateral triangle?
\item[(c)] \dots an isosceles triangle that is not equilateral?
\item[(d)] \dots a scalene triangle?  (That is, a triangle with
three unequal sides.)
\item[(e)] \dots a circle?
\end{itemize}
\end{TG}
%% Note for next time: Make this pictorial.
%% And specify that we're only talking about isometries,
%% or perhaps similarities.
%% I.e., draw a picture of an equilateral triangle and ask
%% them what its symmetries are.

\bigskip

\begin{TG}
Consider the following figure.
\includefigure{4in}{2in}{TG1}
What is the complete set of symmetries of\dots
\begin{itemize}
\item[(a)] \dots point $A$?
\item[(b)] \dots the two-point set $\{A,B\}$?  (That is, the transformations
that either fix both $A$ and $B$, or that interchange them.)
\item[(c)] \dots the line $L$?
\item[(d)] \dots both line $L$ and point $A$?
\item[(e)] \dots line $L$ and both points?
\end{itemize}
\end{TG}

\bigskip

\begin{TG}
Let $\ell$ be the line with equation $y=0$ and let $m$ be the line
with equation $y=4$.  Describe the transformations
$r_\ell\circ r_m$ and $r_m\circ r_\ell$ as simply as possible.
\end{TG}

\bigskip

\begin{TG}
Let $\ell$ be the line with equation $y=0$ and let $m$ be the line
with equation $x=0$.  Describe the transformations
$r_\ell\circ r_m$ and $r_m\circ r_\ell$ as simply as possible.
\end{TG}

\bigskip

\begin{TG} \phantom{j}
\begin{itemize}
\item[(a)] Does the set of all translations of the plane form a group?  
(That is, the set of all transformations $\tau_{\vec v}$, where $\vec v$
is any vector.)  Why or why not?
\item[(b)] Does the set of all reflections of the plane form a group?
(That is, the set of all transformations $r_\ell$, where $\ell$
is any line.)  Why or why not?
\end{itemize}
\end{TG}

\begin{TG}
Consider the transformation of $\Rr^2$ that takes $(x,y)$ to $(x,2y)$.
Prove that it's affine, but not a similarity and not an isometry.
\end{TG}

\bigskip

\begin{TG}
Let $m,n$ be distinct parallel lines. Show that $r_m\circ r_n$ is a
translation. What translation is it?  That is, describe a vector
$\vec{v}$ such that $r_m\circ r_n=\tau_{\vec{v}}$.
(Hint: Pick three non-collinear points that are easy to work with,
and use the Three-Point Theorem.)
\end{TG}

%% Next time: First have them do a concrete example or two ---
%% maybe with vertical lines and then with lines of slope 1 ---
%% and only then ask them to describe the general case.  The
%% problem as it stands is too abstract for them.

\bigskip

\begin{TG} \label{ref-rot-commute}
Let $\ell$ be a line and let $A$ be a point on $\ell$.

\begin{itemize}
\item[(a)] Find a line $m$ such that $r_\ell\circ\rho_{\lpadup A,90}=r_m$.
(Hint: Carry out the composition $r_\ell\circ\rho_{\lpadup A,90}$ using a piece of 
paper and/or a transparency.  Use your example to make an educated guess
about what $m$ is.  Then prove that you are correct using the
Three-Point Theorem.)

\item[(b)] Do the same thing for these other three compositions (that
is, express each one of them as reflection across an appropriate line):
  $$r_\ell\circ\rho_{\lpadup A,270}, \qquad \rho_{\lpadup A,90}\circ r_\ell, \qquad \rho_{\lpadup A,270}\circ r_\ell.$$
%% change the names of the lines --- they called all four of the
%% result lines $m$, which led to confusion in (c) 


\item[(c)] What do you notice about these four transformations?
%% Not specific enough.  Actually ask them which ones
%% equal each other.  ``You've just described four transformations
%% in four different ways.  Are they all different?  Or are some
%% of them equal to each other?  Which ones?''


\end{itemize}
\end{TG}

\bigskip

\begin{TG} \label{threerefs}
Start by downloading the Geometer's Sketchpad worksheet from\\
\fbox{\href{http://www.jlmartin.faculty.ku.edu/math409/congruent-triangles.gsp}{\tt http://www.math.ku.edu/$\sim$jmartin/math409/congruent-triangles.gsp}.}

When you open the worksheet, you will see two congruent triangles $T=\tri{ABC}$ and $T^*=\tri{A^*B^*C^*}$.
Try moving one of the vertices around, and observe that the other vertices get dragged around
so that the two triangles always remain congruent.

The Three-Point Theorem says that there exists an isometry $\psi$ that takes $A$ to $A^*$, $B$ to $B^*$,
and $C$ to $C^*$. The Three-Reflection Theorem says that $\psi$ can be realized as the composition
of at most three reflections, i.e., either
  $$\psi=\id,\qquad \psi=r_\ell,\qquad \psi=r_\ell\circ r_m,\qquad \text{or}\qquad \psi=r_\ell\circ r_m\circ r_n$$
where $\ell,m,n$ are appropriately chosen lines.

Before you go any further, think about how you could perform three reflections in a row that
would take $A,B,C$ to $A^*,B^*,C^*$ respectively in the sketch you've been given.  If you see how to do it,
then work through the steps below and try to connect the procedure (and the notation) with your intuition.
If you don't see how to do it, then work through the steps below and try to understand why they work.


\underline{\bf Step 1:} come up with a line $\ell$ such that $r_\ell(A)=A^*$.  Construct this line
in Sketchpad (by using an appropriate tool from the \textbf{Construct} menu).
Then, construct the triangle $T_1=r_\ell[T]$; that is, actually reflect $T$ across~$\ell$ in your
sketch.\footnote{To
perform a reflection in \textit{Sketchpad}, first highlight (only) the line across which
you want to reflect --- in this case, $\ell$ --- and select ``Mark Mirror'' from the
\textbf{Transform} menu.  Then highlight the object(s) you want to reflect across the
mirror and select ``Reflect''.}  To help keep track of which object is which, color both
the line $\ell$ and the triangle $T_1$ red.

Try dragging some of the points around.
If you've done the first step correctly, then the reflection line $\ell$ and the
triangle $T_1$ will change, but $r_\ell(A)$ should always coincide with $A^*$.

If $T_1=T^*$. then $r_\ell=\psi$ (by the Three-Point Theorem), so you'd be all done.
Unless you drag the points around very carefully, this probably won't be the case
In other words, most isometries are not single reflections.

Okay, now you've gotten $A$ moved to $A^*$. Let $B_1 = r_\ell(B)$, $C_1 = r_\ell(C)$.

\underline{\bf Step 2:}  Come up with a line~$m$ such that
$r_m(A^*)=r_m(A^*)$ and $r_m(B_1)=B^*$.  As before, construct this line
and the triangle $T_2=r_m[T_1]$ in Sketchpad.  (Color them blue so you can
keep track of them.)  Notice that the composition
$r_m\circ r_\ell$ moves $A$ and $B$ to $A^*$ and $B^*$ respectively: that is,
  $$(r_m\circ r_\ell)(A) = r_m(r_\ell(A)) = r_m(A^*) = A^* \qqandqq
    (r_m\circ r_\ell)(B) = r_m(r_\ell(B)) = r_m(B_1) = B^*.$$
In other words, if you first apply $r_\ell$ and then apply $r_m$, the two points
$A$ and $B$ will get moved to where they're supposed to be, namely to $A^*$ and $B^*$
respectively.

Again, try dragging some of the points around, and make sure that the equalities
$(r_m\circ r_\ell)(A)=A^*$ and $(r_m\circ r_\ell)(B)=B^*$ remain true even while
$m$ and $T_2$ move around.

If $\psi=r_m\circ r_\ell$ then you're done.  In the example sketch,
it doesn't so you're not.  At this point we've moved $A$ to $A^*$ and $B$ to 
$B^*$, but $C$ has moved to a point $C_2=r_m(C_1)=(r_m\circ r_\ell)(C)$, which need not
equal $C^*$.

\underline{\bf Step 3:} Come up with a line~$n$ such that
  $$r_n(A^*)=A^*, \qquad r_n(B^*)=B^*, \qquad r_n(C_2)=C^*.$$
That is,
  $$(r_n\circ r_m\circ r_\ell)(A)=A^*, \qquad
    (r_n\circ r_m\circ r_\ell)(B)=C^*, \qquad
    (r_n\circ r_m\circ r_\ell)(B)=C^*,$$
Construct this line $n$ (in green) on your sketch.  Without actually performing
the reflection, you should be able to see that $r_n[T_2]=T^*$, which is
equivalent to the preceding three equations.

Now you're done --- you've constructed a transformation $r_n\circ r_m\circ r_\ell$
that takes $A,B,C$ to $A^*,B^*,C^*$.  By the Three-Point Theorem,
it must be the case that $r_n\circ r_m\circ r_\ell=\psi$.

Send me the sketch under the name REFLECT.  All I should see is: the original
triangles $T=\tri{ABC}$ and $T^*=\tri{A^*B^*C^*}$; the lines $\ell$, $m$ and
$n$ that you've constructed; and the triangles $T_1$ and $T_2$.
\end{TG}

\bigskip

\begin{TG}
In TG~\ref{threerefs}, you needed three reflections to transform $T$ to $T^*$.  What kinds of triangles
require only two reflections?  In other words, suppose I tell you that
$\tri{ABC}=\tri{A^*B^*C^*}$, and that you can't get one triangle from the other
by fewer than two reflections.  How can you tell whether two or three reflections
are required?  (Hint: In your sketch for TG~\ref{threerefs}, move $A,B,C$ around and observe
the different kinds of triangles that occur as $T_2$ and $T^*$.)
\end{TG}

\bigskip

\begin{TG}
Let $\ell$ be a line, let $Q$ be a point on $\ell$, and let
$\theta$ be any angle.
Express $r_\ell\circ\rho_{\lpadup Q,\theta}\circ r_\ell$ as a single transformation.
Express $\rho_{\lpadup Q,\theta}\circ r_\ell\circ\rho_{\lpadup Q,\theta}$ as a single transformation.
\end{TG}

\bigskip

\begin{TG}
Let $X$ be a point in the plane.  Let $G$ be the following set (actually,
group) of transformations:
  $$G ~=~ \{\id,\ \rho_{\lpadup X,120},\ \rho_{\lpadup X,240}\}.$$
Construct a figure whose symmetries are exactly the
transformations in $G$.  (Be careful---you need to make sure
that the figure has no reflection symmetries.  For example,
an equilateral triangle will not work.)
\end{TG}

\bigskip

\begin{TG}
Suppose that lines $\ell$ and $m$ meet at an angle of $40^\circ$.
Let $\phi$ be any transformation that can be obtained by repeatedly
composing $r_\ell$ and $r_m$; for example,
  $$\phi=r_m \qorq \phi=r_\ell\circ r_m\circ r_\ell \qorq \phi=r_\ell\circ r_m\circ r_\ell\circ r_m\circ r_\ell\circ r_m\circ r_\ell\circ r_m.$$
How many different possibilities are there for $\phi$?  (The answer
is a finite number.)
\end{TG}

\bigskip

\begin{TG}
Recall that the group of symmetries of a square $S$ is
  $$\Sym(S) ~=~ \left\{\id,\ \rho_{\lpadup Z,90},\ \rho_{\lpadup Z,180},\ 
    \rho_{\lpadup Z,270},\ r_k,\ r_\ell,\ r_m,\ r_n\right\}$$
where $Z$ is the center of the square, $k$ and $\ell$
are the two diagonals, and $m$ and $n$ are the lines joining
the midpoints of opposite sides, as shown.
\includefigure{2in}{2in}{square-syms}
(a) Write out the multiplication table for the group of 
isometries of the plane that are symmetries of $S$.

(b) A \emph{subgroup} of $\Sym(S)$ is defined as a subset of $\Sym(S)$
that is itself a group.  One example of a subgroup
is the set of rotations in $S$, namely
  $$\left\{\id,\ \rho_{\lpadup Z,90},\ \rho_{\lpadup Z,180},\ \rho_{\lpadup Z,270}\right\}.$$
Another (silly-looking) example of a subgroup is the set
  $$\{\id\}$$
(which is, after all, a group, albeit not a very exciting one).
Explain how you can use the multiplication table to find other subgroups,
and find as many as you can.  (In fact, $\Sym(S)$ has a grand total of
ten subgroups, including $\Sym(S)$ itself and the two examples above.)
%% Next time: Have them compare the rows and columns of a table
%% for (a) a subgroup; (b) a subset that is not a subgroup.

\end{TG}

\bigskip

\begin{TG}
Let $H=ABCDEF$ be a hexagon in which all internal angles are $120^\circ$,
and $AB=CD=EF$ and $BC=DE=FA$, but $AB\neq BC$, as shown.
\includefigure{2in}{2in}{semireghex}

(a) List all the elements of the group $\Sym(H)$.  (You don't
have to write out the multiplication table.)

(b) How does $\Sym(H)$ compare to the group of symmetries of
an equilateral triangle?  Explain your answer by drawing a picture.
\end{TG}

\bigskip

\begin{TG}
Let $Q$ be a cube in $\Rr^3$.
How many isometries of $\Rr^3$
are symmetries of~$Q$?  (Hint: Mimic the argument
in the notes for counting the isometries of $\Rr^2$
that are symmetries of a square.)
\end{TG}

\bigskip

\begin{TG}
Let $\R=WXYZ$ be a rhombus that is not a square.
\includefigure{2.25in}{1.5in}{rhombus}
We already know that $\R$ has exactly four symmetries
(see \S6.4 of the notes on transformations).

(a) Write out the permutation words for all four symmetries of $\R$.

%% This problem is wrong.  They're not the same lists.  Duh.
%% It would be better to have them compare the multiplication
%% tables.
(b) Compare your results with the list of permutation words for
the symmetries of a rectangle (see Example~4 in the notes).
What do you notice?  Can you explain what is going on?
\end{TG}

\bigskip

\begin{TG}
Let $n\geq 3$, and let $\P$ be a polygon with $n$ vertices.
Why can't $\P$ have more than $2n$ symmetries?
(You may be tempted to say, ``Regular polygons have the most
possible symmetries, and they have only $2n$ symmetries,
so the answer is no.''  However, that argument is tautological ---
how do you know that a regular $n$-gon has the most symmetries
of any $n$-sided polygon?  Instead, try counting all possible symmetries
of $\P$.)
\end{TG}

\bigskip

%% Next time: Have them first do this for some particular $n$, like 5.
%% *Then* see if they can generalize.  Not all of them are familiar
%% enough with notation to do it in general anyway.

\begin{TG}
Let $\P$ be a regular $n$-sided polygon.

(a.) Describe the permutation words for all $2n$ symmetries of $\R$.
(You don't have to write out the entire list, because the number
of items in the list depends on $n$.  Instead, describe what
kinds of permutation words show up in general.)
%% Point them to an example in the notes.

(b.) Which ones are 
rotations and which ones are reflections?  (To put it another way,
how can you tell
whether a symmetry of $\P$ is a rotation or a reflection just by
looking at its permutation word?)
\end{TG}

\pagebreak

\begin{TG}
Let $\P$ be a rectangular prism in $\Rr^3$ with 
length $a$, width $b$, and height $c$.

(a) Suppose that $a,b,c,$  are all
different (as in the left-hand figure below).  How many
symmetries does $\P$ have?

(b) Suppose that $a=b$, but $a\neq c$ and $b\neq c$
(as in the right-hand figure below).  How many
symmetries does $\P$ have?
\includefigure{3.3in}{1.8in}{prisms}
\end{TG}

%------------------------------------------------------------

\begin{TG}
Let $\F$ be a figure obtained by superimposing a regular $n$-sided
polygon and a regular $m$-sided polygon with the same center.
For example, if $n=4$ and $m=6$, then $\F$ might look like this:
\includefigure{2in}{2in}{squarehex}
Describe the symmetry group of $\F$.
\end{TG}

\section{Polyhedra}

\begin{PH}
Determine $v$, $e$ and $f$ (the numbers of vertices, edges and faces)
for (a) a pyramid whose
base is an $n$-sided polygon (``$n$-gon''); (b) a bipyramid whose base is an $n$-gon; (c) a prism whose base is an $n$-gon.  Verify that each of
these polyhedra satisfies Euler's formula.
\end{PH}

\begin{PH}
Consider a square prism with
a hole drilled through the middle (a ``big square donut''), as pictured.
\includefigure{3in}{3in}{rectangular-torus}
(a) Determine $v$, $e$ and $f$.  Do these numbers
satisfy Euler's formula?  If not, why not?

(b) Suppose that you drill lots of holes --- say $h$ of them.
(In the figure below, $h=2$.)
What's the relationship between $v$, $e$, $f$ and $h$?
\includefigure{6in}{3in}{rectangular-torus2}
\end{PH}

\begin{PH}
Let $f(n,k)$ be the number of $k$-dimensional faces of the $n$-dimensional
cube $Q_n$.  (For example, a 3-dimensional cube has 12 edges and 6 faces,
so $f(3,1)=12$ and $f(3,2)=6$.)  We saw some of the following
values of $f(n,k)$ in class:

$$
\begin{array}{l|cccccc}
    & k=0 & k=1 & k=2 & k=3 & k=4 & k=5\\ \hline
n=0 & 1   & 0   & 0   & 0   & 0   & 0\\
n=1 & 2   & 1   & 0   & 0   & 0   & 0\\
n=2 & 4   & 4   & 1   & 0   & 0   & 0\\ 
n=3 & 8   & 12  & 6   & 1   & 0   & 0\\
n=4 & 16  & 32  & 24  & 8   & 1   & 0\\
n=5 & 32  & 80  & 80  & 40  & 10  & 1
\end{array}
$$

(a) What's the relationship between these numbers
and the polynomial $f(x)=2x+1$?  (Hint: Look at $f(x)^2$,
then go from there.)

(b) How many constituent pieces (i.e., vertices, edges, faces, etc.)
do you think
an $n$-dimensional cube has?  (For instance, the square $Q_2$
has 4 vertices + 4 edges + 1 square = 9 pieces total.)

(c) [Extra credit] What's the connection between your answers
to (a) and to (b)?
\end{PH}
\bigskip
\begin{PH}
In class and in the notes, we proved that there are only
five Platonic (regular) solids in 3-space.  However, you might
be bothered by an omission in the proof: we didn't
actually prove that the sum of angles meeting at a vertex
in a polyhedron must be less than $360^\circ$.  Here's a way to get
around that problem that only uses counting.

Suppose that $\P$ is a Platonic solid with $v$ vertices,
$e$ edges and $f$ faces.  We know that every face is a regular
polygon with the same number of sides; as in the notes, we'll
call this number $s$.  Also, let's let $d$ be the degree of
each vertex (remember, this means the number of edges\footnote{%
In fact this is the same as the number of \emph{faces} containing
each vertex, but for this argument it's more helpful to think
in terms of \emph{edges} containing a vertex.} incident
to it; since $\P$ is regular, this is the same for all vertices).

By Theorem~1 from the notes on polyhedra, we know that 
  \begin{subequations}
  \begin{equation} \label{sd:1}
  s\geq 3 \qqandqq d\geq 3.
  \end{equation}
On the other hand, Theorem~2 tells us that
  \begin{equation} \label{sd:2}
  s\leq 5 \qqandqq d\leq 5.
  \end{equation}
  \end{subequations}

So we have five parameters to work with: $v,e,f,s,d$.  We don't know
anything in advance about how large or small $v$, $e$ and $f$ can be,
but we do have bounds on $s$ and $d$ that we can use.

(a) Prove that $dv=fs=2e$.

(b) Use Euler's formula and the equations you just proved
to solve for $v$ in terms of $d$ and $s$.

(c) Based on equations \eqref{sd:1} and \eqref{sd:2}, there are
nine possibilities for $s$ and $d$.  Which of them produce 
sensible values for $v$?  Which ones are impossible?
\end{PH}
\vfill
\end{document}
