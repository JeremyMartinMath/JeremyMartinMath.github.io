\documentclass{amsart}
\usepackage{amssymb,amsmath,amsthm,mathrsfs,graphics,hyperref,stmaryrd,psfrag,arcs,xypic}
\usepackage[enableskew]{youngtab}
\numberwithin{equation}{section}
\raggedbottom
\oddsidemargin=0in
\evensidemargin=0in
\textwidth=6.5in
\textheight=8.8in
\topmargin=0.25in
\headheight=0in
\headsep=0.2in
\footskip=0in
\parskip=10bp
\parindent=0bp

\newcommand{\excise}[1]{}
\newcommand{\Latex}[1]{\textbackslash\texttt{#1}}

\newcommand{\bigpad}{\rule[-14mm]{0mm}{30mm}}
\newcommand{\smallpad}{\rule[-1.5mm]{0mm}{5mm}}
\newcommand{\pad}{\rule[-3mm]{0mm}{8mm}}
\newcommand{\padup}{\rule{0mm}{5mm}}
\newcommand{\paddown}{\rule[-3mm]{0mm}{2mm}}
\newcommand{\blank}{\rule{1.25in}{0.25mm}}
\newcommand{\commentout}[1]{}
\newcommand{\yell}[1]{\fbox{\rule[-1mm]{0mm}{4mm} \large\bf #1 }}
\newcommand{\bang}{$\bullet$\quad}
\newcommand{\indnt}{\phantom{.}\qquad}
\newcommand{\littleline}{\begin{center}\rule{4in}{0.5bp}\end{center}}

\newcommand{\includefigure}[3]{{
  \begin{center}
  \resizebox{#1}{#2}{\includegraphics{{figs/#3}}}
  \end{center}}}
\newcommand{\includefigurewithinmath}[3]{{
  \resizebox{#1}{#2}{\includegraphics{{figs/#3}}}}}

%\newcommand{\defterm}[1]{\underline{\textbf{#1}}}
\newcommand{\defterm}[1]{\textbf{#1}}

\DeclareMathOperator{\ch}{\mathbf{ch}}
\DeclareMathOperator{\colspace}{colspace}
\DeclareMathOperator{\corank}{corank}
\DeclareMathOperator{\CST}{CST}
\DeclareMathOperator{\deln}{del}
\DeclareMathOperator{\diag}{diag}
\DeclareMathOperator{\ess}{ess}
\DeclareMathOperator{\Gr}{Gr}
\DeclareMathOperator{\Hom}{Hom}
\DeclareMathOperator{\im}{im}
\DeclareMathOperator{\Irr}{Irr}
\DeclareMathOperator{\Ind}{Ind}
\DeclareMathOperator{\Int}{Int}
\DeclareMathOperator{\lcm}{lcm}
\DeclareMathOperator{\link}{lk}
\DeclareMathOperator{\nullity}{nullity}
\DeclareMathOperator{\nullspace}{nullspace}
\DeclareMathOperator{\Poin}{Poin}
\DeclareMathOperator{\proj}{proj}
\DeclareMathOperator{\rank}{rank}
\DeclareMathOperator{\Res}{Res}
\DeclareMathOperator{\Span}{span}
\DeclareMathOperator{\supp}{supp}
\DeclareMathOperator{\row}{row}
\DeclareMathOperator{\rowspace}{rowspace}
\DeclareMathOperator{\sh}{sh}
\DeclareMathOperator{\tr}{tr}
\DeclareMathOperator{\wt}{wt}

\newtheorem{theorem}{Theorem}[section]
\newtheorem{proposition}[theorem]{Proposition}
\newtheorem{lemma}[theorem]{Lemma}
\newtheorem{corollary}[theorem]{Corollary}
\theoremstyle{definition}
\newtheorem{definition}[theorem]{Definition}
\newtheorem{example}[theorem]{Example}
\newtheorem{remark}[theorem]{Remark}
\newtheorem{problem}[theorem]{Problem}


\newcommand{\cor}{{\bf Corollary: }}
\newcommand{\defn}{{\bf Definition: }}
\newcommand{\defns}{{\bf Definitions: }}
\newcommand{\exa}{{\bf Example: }}
\newcommand{\fact}{{\bf Fact: }}
\newcommand{\lem}{{\bf Lemma: }}
\newcommand{\notn}{{\bf Notation: }}
\newcommand{\obs}{{\bf Observation: }}
\newcommand{\note}{{\bf Note: }}
\newcommand{\prop}{{\bf Proposition: }}
\newcommand{\rmk}{{\bf Remark: }}
\newcommand{\thm}{{\bf Theorem: }}

\newcommand{\basecase}{\emph{Base case: }}
\newcommand{\indstep}{\emph{Inductive step: }}
\newcommand{\skpr}{\emph{Sketch of proof: }}

\newcommand{\0}{\emptyset}
\newcommand{\Alt}{\mathfrak{A}}
\newcommand{\Braid}{Br}
\newcommand{\CHI}{\chi^{\phantom{*}}}
\newcommand{\Cl}{C\ell}
\newcommand{\covers}{\gtrdot}
\newcommand{\coveredby}{\lessdot}
\newcommand{\dedge}[1]{\overrightarrow{{#1}}}
\newcommand{\dom}{\rhd}
\newcommand{\domeq}{\unrhd}
\newcommand{\domby}{\lhd}
\newcommand{\dombyeq}{\unlhd}
\newcommand{\Fspan}{\Ff\text{-span}}
\newcommand{\isom}{\cong}
\newcommand{\join}{\vee}
\renewcommand{\Join}{\bigvee}
\newcommand{\Laff}{L^{\text{aff}}}
\newcommand{\lin}[1]{\overleftrightarrow{{#1}}}
\newcommand{\meet}{\wedge}
\newcommand{\Meet}{\bigwedge}
\newcommand{\ov}[1]{\overline{{#1}}}
\newcommand{\partn}{\vdash}
\newcommand{\qqandqq}{\qquad\text{and}\qquad}
\newcommand{\qandq}{\quad\text{and}\quad}
\newcommand{\qand}{\quad\text{and}}
\newcommand{\qbin}[2]{{\begin{bmatrix}#1\\#2\end{bmatrix}_q}}
\newcommand{\sd}{\triangle} % symmetric difference
\newcommand{\simK}{\underset{K}{\sim}} % Knuth equivalence
\newcommand{\simJ}{\underset{J}{\sim}} % jeu de taquin equivalence
\newcommand{\sm}{\setminus}
\newcommand{\st}{~|~}
\newcommand{\soln}{\textit{Solution:\ }}
\newcommand{\Sym}{\mathfrak{S}}
\newcommand{\un}[1]{\underset{*}{#1}}
\newcommand{\unA}{\un{A}}
\newcommand{\unB}{\un{B}}
\newcommand{\unw}{\un{w}}
\newcommand{\unx}{\un{x}}
\newcommand{\uny}{\un{y}}
\newcommand{\unz}{\un{z}}
\newcommand{\x}{\times}

\renewcommand{\aa}{\mathbf{a}}
\newcommand{\bb}{\mathbf{b}}
\newcommand{\nn}{\mathbf{n}}
\newcommand{\pp}{\mathbf{p}}
\newcommand{\qq}{\mathbf{q}}
\newcommand{\xx}{\mathbf{x}}
\newcommand{\yy}{\mathbf{y}}
\newcommand{\zz}{\mathbf{z}}
 
\newcommand{\A}{\mathcal{A}}
\newcommand{\B}{\mathcal{B}}
\newcommand{\C}{\mathcal{C}}
\newcommand{\M}{\mathcal{M}}
\renewcommand{\P}{\mathcal{P}}

\newcommand{\BB}{\mathscr{B}}  %% use these for fancy script fonts -- requires mathrsfs package
\newcommand{\CC}{\mathscr{C}}
\newcommand{\FF}{\mathscr{F}}
\newcommand{\II}{\mathscr{I}}
\newcommand{\LL}{\mathscr{L}}
\newcommand{\PP}{\mathscr{P}}
\renewcommand{\SS}{\mathscr{S}}
\newcommand{\XX}{\mathscr{X}}

\newcommand{\TT}{\tilde{T}}

\newcommand{\Aa}{\mathbb{A}}
\newcommand{\Cc}{\mathbb{C}}
\newcommand{\Ff}{\mathbb{F}}
\newcommand{\Nn}{\mathbb{N}}
\newcommand{\Pp}{\mathbb{P}}
\newcommand{\Qq}{\mathbb{Q}}
\newcommand{\Rr}{\mathbb{R}}
\newcommand{\Zz}{\mathbb{Z}}

\newcommand{\rhodef}{\rho^{\phantom{*}}_{{\rm def}}}
\newcommand{\rhotriv}{\rho^{\phantom{*}}_{{\rm triv}}}
\newcommand{\rhosign}{\rho^{\phantom{*}}_{{\rm sign}}}
\newcommand{\rhoreg}{\rho^{\phantom{*}}_{{\rm reg}}}
\newcommand{\chidef}{\chi^{\phantom{*}}_{{\rm def}}}
\newcommand{\chitriv}{\chi^{\phantom{*}}_{{\rm triv}}}
\newcommand{\chisign}{\chi^{\phantom{*}}_{{\rm sign}}}
\newcommand{\chireg}{\chi^{\phantom{*}}_{{\rm reg}}}
\newcommand{\scp}[2]{\left\langle #1,\:#2\right\rangle_G}
\newcommand{\scpH}[2]{\left\langle #1,\:#2\right\rangle_H}

\newcounter{probno}
\setcounter{probno}{0}
\newcounter{partno}
\setcounter{partno}{0}
%% versions that don't print the number of points
\newcommand{\prob}{
  \vskip10bp%
  \setcounter{partno}{0}%
  \addtocounter{probno}{1}%
  {\bf Problem~\#{\arabic{probno}}}\quad}
\newcommand{\probpart}{%\rule{0in}{0in}\\ \phantom{xxx}
  \addtocounter{partno}{1}%
  {\bf (\#\arabic{probno}\alph{partno})}\ \ }
\newcommand{\probcont}{%
  {\bf Problem~\#{\arabic{probno}}}~(\emph{continued})}
\newcommand{\probo}{
  \setcounter{partno}{0}%
  \addtocounter{probno}{1}%
  {\bf (\#\arabic{probno})}\ \ }
%% versions that do print the number of points
\newcommand{\Prob}[1]{
  \vskip10bp%
  \setcounter{partno}{0}%
  \addtocounter{probno}{1}%
  {\bf Problem~\#{\arabic{probno}}~[{#1}~pts]}\quad}
\newcommand{\Probpart}[1]{%\rule{0in}{0in}\\ \phantom{xxx}
  \addtocounter{partno}{1}%
  {\bf (\#\arabic{probno}\alph{partno})~[{#1}~pts]}\ \ }

%\renewcommand{\thefootnote}{\fnsymbol{footnote}}

\begin{document}
\thispagestyle{empty}
{\bf Math 821, Spring 2018\\
Problem Set \#5\\
Deadline: Thursday, May 3, 11:59pm}

\textbf{Instructions:} Typeset your solutions in LaTeX.  You are encouraged to use the \hreftext{http://jlmartin.faculty.ku.edu/math821/header.tex}{Math 821 header file}. Email your solutions to Jeremy (jlmartin@ku.edu) as a PDF file named with your last name and the problem set number (e.g., \texttt{Mirzakhani5.pdf}).  Collaboration is encouraged, but each student must write up his or her solutions independently and acknowledge all collaborators.
\medskip\hrule

\prob Let $n\leq d\geq 0$ and let $X=\Delta^{n,d}$ denote the $d$-skeleton of the $n$-dimensional simplex (with $n+1$ vertices $v_0,v_1,\dots,v_n$).  
\emph{Without writing down any explicit simplicial boundary matrices}, prove that the reduced homology groups of $X$ are given by
\[\HH_k(X) = \begin{cases}
\Zz^{\binom{n}{d+1}} & \text{ if $k=d$,}\\
0 & \text{ if $k<d$}.
\end{cases}\]
(You may want to first use a computer to convince yourself that the result is correct.)

\vfill\prob [Hatcher p.156 \#9bc]  Compute the homology groups of the following spaces:

\probpart $\Ss^1\x(\Ss^1\vee \Ss^1)$.\\
\padup\probpart The space obtained from $D^2$ by first deleting the interiors of two disjoint subdisks, then identifying all three resulting circles together via homomorphisms preserving clockwise orientations of the circles.

\vfill\prob [Hatcher p.156 \#11] Let $K$ be the 3-dimensional cell complex obtained from the cube $I^3$ by identifying each pair of opposite faces via a one-quarter twist.  (See exercise \#14 on p.54.)  Compute the homology groups $\HH_n(K;\Zz)$ and $\HH_n(K;\Zz_2)$ for $n>0$.

\vfill\prob [Hatcher p.157 \#28(a), modified] (a) Use a Mayer-Vietoris sequence to compute the homology groups of the space $X$ obtained from a torus $T=\Ss^1\x \Ss^1$ by attaching a M\"obius band $M$ via a homeomorphism from the boundary circle $C$ of $M$ to the circle $\Ss^1\x\{x_0\}$ in the torus.

(b) How does the answer change if $C$ is attached to a closed loop that wraps $k$ times around the first circle (i.e., via the path $f:I\to\Ss^1\x\Ss^1\subset\Cc\x\Cc$ given by $f(t)=(e^{2\pi i kt},1)$)?

\vfill\prob [Hatcher p.205 \#8(a)] Many basic homology arguments work equally well for cohomology, with maps in the opposite direction.  In particular, for all coefficient rings $R$ and numbers $i\leq n$, compute $H^i(S^n;R)$ by induction on $n$ in two ways: using the long exact sequence of a pair, and using the Mayer-Vietoris sequence.
(We didn't explicitly talk about these sequences, but they are essentially the same as the homology versions with the arrows reversed; see \S3.1 of Hatcher for details if you are not confident about reversing the arrows yourself.)

\vfill\prob Let $X$ and $Y$ be connected spaces, so that $H^0(X;R)\isom H^0(Y;R)\isom H^0(X\vee Y;R)\isom R$.

\probpart  Show that $H^k(X\vee Y;R)\isom H^k(X;R)\oplus H^k(Y;R)$ for all $k>0$.\\
\probpart Show that if $\alpha\in H^k(X;R)$ and $\beta\in H^\ell(Y;R)$ with $k,\ell>0$, then $\alpha\smile\beta=0$ in $H^*(X\vee Y;R)$.\pad\\
\probpart Conclude that $\Ss^1\x\Ss^1$ is not homeomorphic to $\Ss^2\vee\Ss^1\vee\Ss^1$, even though their homology and cohomology groups are equal in all dimensions.
\vfill\end{document}