\documentclass{amsart}
\usepackage{amssymb,amsmath,amsthm,mathrsfs,graphics,hyperref,ifthen,framed,cancel,fullpage,color,ytableau,tcolorbox,bm,tikz}
\usepackage[enableskew]{youngtab}
\raggedbottom
\parskip=10bp
\parindent=0bp
\raggedbottom

%%%%%%%%%%%%%%%%   Colors for TikZ and text  %%%%%%%%%%%%%%%%%

\definecolor{light}{gray}{.75}
\definecolor{med}{gray}{.5}
\definecolor{dark}{gray}{.25}
\newcommand{\Red}[1]{{\color{red}{#1}}}
\newcommand{\RED}[1]{{\color{red}{\boldmath\textbf{#1}\unboldmath}}}
\newcommand{\Blue}[1]{{\color{blue}{#1}}}
\newcommand{\BLUE}[1]{{\color{blue}{\boldmath\textbf{#1}\unboldmath}}}

% Hyperlinks
\hypersetup{colorlinks, citecolor=red, filecolor=black, linkcolor=blue, urlcolor=blue}
\newcommand{\hreftext}[2]{\href{#1}{\Blue{#2}}} % for text anchors
\newcommand{\hrefurl}[2]{\href{#1}{\Blue{\tt #2}}} % if you want to actually write out the URL in text

% Spacing and commands
\newcommand{\bigpad}{\rule[-14mm]{0mm}{30mm}}
\newcommand{\smallpad}{\rule[-1.5mm]{0mm}{5mm}}
\newcommand{\pad}{\rule[-3mm]{0mm}{8mm}}
\newcommand{\padup}{\rule{0mm}{5mm}}
\newcommand{\paddown}{\rule[-3mm]{0mm}{2mm}}
\newcommand{\blank}{\rule{1.25in}{0.25mm}}
\newcommand{\yell}[1]{\fbox{\rule[-1mm]{0mm}{4mm} \large\bf #1 }}
\newcommand{\indnt}{\phantom{.}\qquad}
\newcommand{\littleline}{\begin{center}\rule{4in}{0.5bp}\end{center}}

% macros for inserting figures
\newcommand{\includefigure}[3]{\begin{center}\resizebox{#1}{#2}{\includegraphics{#3}}\end{center}}
\newcommand{\includefigurewithinmath}[3]{\resizebox{#1}{#2}{\includegraphics{#3}}}
% E.g., to insert a standalone figure of width 3" and height 1.5", use
% \includefigure{3in}{1.5in}{foo.pdf}

% math operators
\DeclareMathOperator{\Comp}{Comp}
\DeclareMathOperator{\Fix}{Fix}
\DeclareMathOperator{\id}{id}
\DeclareMathOperator{\im}{im}
\DeclareMathOperator{\lcm}{lcm}
\DeclareMathOperator{\Par}{Par}
\DeclareMathOperator{\rank}{rank}
\DeclareMathOperator{\sh}{sh}
\DeclareMathOperator{\tr}{tr}
\DeclareMathOperator{\wt}{wt}

% theorem environments, automatically numbered
\newtheorem{theorem}{Theorem}[section]
\newtheorem{proposition}[theorem]{Proposition}
\newtheorem{lemma}[theorem]{Lemma}
\newtheorem{corollary}[theorem]{Corollary}
\theoremstyle{definition}
\newtheorem{definition}[theorem]{Definition}
\newtheorem{example}[theorem]{Example}
\newtheorem{remark}[theorem]{Remark}
\newtheorem{problem}[theorem]{Problem}

\newcommand{\skpr}{\emph{Sketch of proof: }}
\newcommand{\soln}{\textit{Solution:\ }}

% Generally useful macros
\newcommand{\excise}[1]{} % useful for commenting out large chunks
\newcommand{\0}{\emptyset}
\newcommand{\compn}{\models} % compositions
\newcommand{\dju}{\mathaccent\cdot\cup} % disjoint union
\newcommand{\dsum}{\displaystyle\sum}
\newcommand{\fallfac}[2]{{#1}^{\underline{#2}}} % falling factorial
\newcommand{\isom}{\cong} % isomorphism symbol
\newcommand{\partn}{\vdash} % partition symbol
\newcommand{\qqandqq}{\qquad\text{and}\qquad}
\newcommand{\qandq}{\quad\text{and}\quad}
\newcommand{\qand}{\quad\text{and}}
\newcommand{\qbin}[2]{{\begin{bmatrix}#1\\#2\end{bmatrix}_q}} % q-binomial coefficient
\newcommand{\risefac}[2]{{#1}^{\overline{#2}}} % rising factorial
\newcommand{\sd}{\triangle} % symmetric difference
\newcommand{\sm}{\setminus} % don't use a minus sign for this
\newcommand{\st}{\colon} % "such that"
\newcommand{\surj}{\twoheadrightarrow}
\newcommand{\x}{\times}

% blackboard bold fonts for sets of numbers
\newcommand{\Cc}{\mathbb{C}} % complex numbers
\newcommand{\Ff}{\mathbb{F}} % finite field
\newcommand{\Nn}{\mathbb{N}} % natural numbers
\newcommand{\Qq}{\mathbb{Q}}
\newcommand{\Rr}{\mathbb{R}}
\newcommand{\Zz}{\mathbb{Z}}

% miscellaneous
\newcommand{\TwoCases}[4]{\begin{cases}{#1}&\text{ #2}\\{#3}&\text{ #4}\end{cases}}
\newcommand{\ThreeCases}[6]{\begin{cases}{#1}&\text{ #2}\\{#3}&\text{ #4}\\{#5}&\text{ #6}\end{cases}}
\newcommand{\bridgehand}[4]{\spadesuit\ {\textsf{#1}}\ \ \heartsuit\ {\textsf{#2}}\ \ \diamondsuit\ {\textsf{#3}}\ \ \clubsuit\ {\textsf{#4}}}

% macros for automatic problem numbering --- students don't have to use these
\newcounter{probno}
\setcounter{probno}{0}
\newcounter{partno}
\setcounter{partno}{0}
%% versions that don't print the number of points
\newcommand{\prob}{
  \vskip10bp%
  \setcounter{partno}{0}%
  \addtocounter{probno}{1}%
  {\bf Problem~\#{\arabic{probno}}}\quad}

% No initial whitespace to make framing look nicer
\newcommand{\probns}{
  \setcounter{partno}{0}%
  \addtocounter{probno}{1}%
  {\bf Problem~\#{\arabic{probno}}}\quad}

\newcommand{\probpart}{%
  \addtocounter{partno}{1}%
  {\bf (\#\arabic{probno}\alph{partno})}\ \ }
\newcommand{\probcont}{%
  {\bf Problem~\#{\arabic{probno}}}~(\emph{continued})}
\newcommand{\probo}{
  \setcounter{partno}{0}%
  \addtocounter{probno}{1}%
  {\bf (\#\arabic{probno}}\ \ }

%% versions that do print the number of points
\newcommand{\Prob}[1]{
  \vskip10bp%
  \setcounter{partno}{0}%
  \addtocounter{probno}{1}%
  {\bf Problem~\#{\arabic{probno}}~[{#1}~pts]}\quad}

%no initial whitespace
\newcommand{\Probns}[1]{
  \setcounter{partno}{0}%
  \addtocounter{probno}{1}%
  {\bf Problem~\#{\arabic{probno}}~[{#1}~pts]}\quad}
\newcommand{\Probpart}[1]{%\rule{0in}{0in}\\ \phantom{xxx}
  \addtocounter{partno}{1}%
  {\bf (\#\arabic{probno}\alph{partno})~[{#1}~pts]}\ \ }

\newboolean{answers}
\newcommand{\Answer}[1]{\ifthenelse{\boolean{answers}}{{\bf Answer:}\ #1}{}\bigskip}

\begin{document}
\thispagestyle{empty}
{\bf Math 821, Spring 2018\\
Problem Set \#2\\
Deadline: Friday, February 16, 5:00pm}

\textbf{Instructions:} Typeset your solutions in LaTeX.  You are encouraged to use the \hreftext{http://jlmartin.faculty.ku.edu/math821/header.tex}{Math 821 header file}. Email your solutions to Jeremy (jlmartin@ku.edu) as a PDF file named with your last name and the problem set number (e.g., \texttt{Riemann2.pdf}).  Collaboration is encouraged, but each student must write up his or her solutions independently and acknowledge all collaborators.
\medskip
\hrule

\prob (Hatcher, p.20, \#22) Let $X$ be a finite graph lying in a half-plane $P\subset\Rr^3$ and intersecting the edge line $\ell$ of $P$ in a subset of its vertices.  Describe the homotopy type of the ``surface of revolution'' $Z$ obtained by rotating $X$ about $\ell$.  (Examples: If $X$ has two vertices and one edge, then $Z$ is a 2-sphere, a disk or a cylinder according as $X$ has two, one or zero vertices on $\ell$.  If $X$ has one vertex not on $\ell$ with a loop attached, then $Z$ is a torus.)


\prob Recall that for a space $X$ and basepoint $p\in X$, we have defined $\pi_1(X,p)$ to be
the set of homotopy classes of $p$-loops on $X$, or equivalently of continuous functions
$S^1\to X$.  Recall also that $S^0$ consists of two points (let's call them $a$ and $b$) with the discrete topology.
Accordingly, we could define $\pi_0(X,p)$ to be the set of homotopy classes of continuous functions $f:S^0\to X$
such that $f(a)=p$.  Describe the set $\pi_0(X,p)$ intrinsically in terms of $X$.  Is there a natural group structure on $\pi_0(X,p)$?
(Of course every set can be made into a group somehow, so the question is whether you can do it in a way depends only on, and tell you something about, the topology of $X$.)

\prob (Hatcher, p.38, \#2) Show that the change-of-basepoint homomorphism $\beta_h$ (see p.28)
depends only on the homotopy class of the path $h$.

\prob (Hatcher, p.38, \#7) Define $f:S^1\x I\to S^1\x I$ by $f(\theta,s)=(\theta+2\pi s,s)$,
so $f$ restricts to the identity on the two boundary circles of $S^1\x I$.   Show that $f$ is homotopic
to the identity by a homotopy $f_t$ that is stationary on \emph{one} of the boundary circles, but not
by any homotopy $f_t$ that is stationary on \emph{both} boundary circles.

\prob [Hatcher p.38 \#8] Does the Borsuk-Ulam theorem hold for the torus?  In other words, for every continuous $f:S^1\x S^1\to\Rr^2$ must there exist $(x,y)\in S^1\x S^1$ such that $f(x,y)=f(-x,-y)$?  Why or why not?

\prob [Hatcher p.39 \#12] Fix $p\in S^1$.  Show that every homomorphism $\pi_1(S^1,p)\to\pi(S^1,p)$ can be realized as the induced homomorphism $\phi_*$ for some continuous $\phi:S^1\to S^1$.

\prob [Hatcher, p.52, \#1]  Recall that the \defterm{center} of a group $G$ is defined as $Z(G)=\{g\in G:\ gh=hg \ \forall h\in G\}$.

\probpart Show that the free product $G*H$ of nontrivial groups $G$ and $H$ has trivial center.

\probpart Show that the only elements of $G*H$ of finite order are the conjugates of finite-order elements in~$G\cup H$.

\end{document}
