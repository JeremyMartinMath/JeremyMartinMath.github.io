\documentclass{amsart}
\usepackage{amssymb,amsmath,amsthm,mathrsfs,graphics,hyperref,ifthen,framed,cancel,fullpage,color,ytableau,tcolorbox,bm,tikz}
\usepackage[enableskew]{youngtab}
\raggedbottom
\parskip=10bp
\parindent=0bp
\raggedbottom

%%%%%%%%%%%%%%%%   Colors for TikZ and text  %%%%%%%%%%%%%%%%%

\definecolor{light}{gray}{.75}
\definecolor{med}{gray}{.5}
\definecolor{dark}{gray}{.25}
\newcommand{\Red}[1]{{\color{red}{#1}}}
\newcommand{\RED}[1]{{\color{red}{\boldmath\textbf{#1}\unboldmath}}}
\newcommand{\Blue}[1]{{\color{blue}{#1}}}
\newcommand{\BLUE}[1]{{\color{blue}{\boldmath\textbf{#1}\unboldmath}}}

% Hyperlinks
\hypersetup{colorlinks, citecolor=red, filecolor=black, linkcolor=blue, urlcolor=blue}
\newcommand{\hreftext}[2]{\href{#1}{\Blue{#2}}} % for text anchors
\newcommand{\hrefurl}[2]{\href{#1}{\Blue{\tt #2}}} % if you want to actually write out the URL in text

% Spacing and commands
\newcommand{\bigpad}{\rule[-14mm]{0mm}{30mm}}
\newcommand{\smallpad}{\rule[-1.5mm]{0mm}{5mm}}
\newcommand{\pad}{\rule[-3mm]{0mm}{8mm}}
\newcommand{\padup}{\rule{0mm}{5mm}}
\newcommand{\paddown}{\rule[-3mm]{0mm}{2mm}}
\newcommand{\blank}{\rule{1.25in}{0.25mm}}
\newcommand{\yell}[1]{\fbox{\rule[-1mm]{0mm}{4mm} \large\bf #1 }}
\newcommand{\indnt}{\phantom{.}\qquad}
\newcommand{\littleline}{\begin{center}\rule{4in}{0.5bp}\end{center}}

% macros for inserting figures
\newcommand{\includefigure}[3]{\begin{center}\resizebox{#1}{#2}{\includegraphics{#3}}\end{center}}
\newcommand{\includefigurewithinmath}[3]{\resizebox{#1}{#2}{\includegraphics{#3}}}
% E.g., to insert a standalone figure of width 3" and height 1.5", use
% \includefigure{3in}{1.5in}{foo.pdf}

% math operators
\DeclareMathOperator{\Comp}{Comp}
\DeclareMathOperator{\Fix}{Fix}
\DeclareMathOperator{\id}{id}
\DeclareMathOperator{\im}{im}
\DeclareMathOperator{\lcm}{lcm}
\DeclareMathOperator{\Par}{Par}
\DeclareMathOperator{\rank}{rank}
\DeclareMathOperator{\sh}{sh}
\DeclareMathOperator{\tr}{tr}
\DeclareMathOperator{\wt}{wt}

% theorem environments, automatically numbered
\newtheorem{theorem}{Theorem}[section]
\newtheorem{proposition}[theorem]{Proposition}
\newtheorem{lemma}[theorem]{Lemma}
\newtheorem{corollary}[theorem]{Corollary}
\theoremstyle{definition}
\newtheorem{definition}[theorem]{Definition}
\newtheorem{example}[theorem]{Example}
\newtheorem{remark}[theorem]{Remark}
\newtheorem{problem}[theorem]{Problem}

\newcommand{\skpr}{\emph{Sketch of proof: }}
\newcommand{\soln}{\textit{Solution:\ }}

% Generally useful macros
\newcommand{\excise}[1]{} % useful for commenting out large chunks
\newcommand{\0}{\emptyset}
\newcommand{\compn}{\models} % compositions
\newcommand{\dju}{\mathaccent\cdot\cup} % disjoint union
\newcommand{\dsum}{\displaystyle\sum}
\newcommand{\fallfac}[2]{{#1}^{\underline{#2}}} % falling factorial
\newcommand{\isom}{\cong} % isomorphism symbol
\newcommand{\partn}{\vdash} % partition symbol
\newcommand{\qqandqq}{\qquad\text{and}\qquad}
\newcommand{\qandq}{\quad\text{and}\quad}
\newcommand{\qand}{\quad\text{and}}
\newcommand{\qbin}[2]{{\begin{bmatrix}#1\\#2\end{bmatrix}_q}} % q-binomial coefficient
\newcommand{\risefac}[2]{{#1}^{\overline{#2}}} % rising factorial
\newcommand{\sd}{\triangle} % symmetric difference
\newcommand{\sm}{\setminus} % don't use a minus sign for this
\newcommand{\st}{\colon} % "such that"
\newcommand{\surj}{\twoheadrightarrow}
\newcommand{\x}{\times}

% blackboard bold fonts for sets of numbers
\newcommand{\Cc}{\mathbb{C}} % complex numbers
\newcommand{\Ff}{\mathbb{F}} % finite field
\newcommand{\Nn}{\mathbb{N}} % natural numbers
\newcommand{\Qq}{\mathbb{Q}}
\newcommand{\Rr}{\mathbb{R}}
\newcommand{\Zz}{\mathbb{Z}}

% miscellaneous
\newcommand{\TwoCases}[4]{\begin{cases}{#1}&\text{ #2}\\{#3}&\text{ #4}\end{cases}}
\newcommand{\ThreeCases}[6]{\begin{cases}{#1}&\text{ #2}\\{#3}&\text{ #4}\\{#5}&\text{ #6}\end{cases}}
\newcommand{\bridgehand}[4]{\spadesuit\ {\textsf{#1}}\ \ \heartsuit\ {\textsf{#2}}\ \ \diamondsuit\ {\textsf{#3}}\ \ \clubsuit\ {\textsf{#4}}}

% macros for automatic problem numbering --- students don't have to use these
\newcounter{probno}
\setcounter{probno}{0}
\newcounter{partno}
\setcounter{partno}{0}
%% versions that don't print the number of points
\newcommand{\prob}{
  \vskip10bp%
  \setcounter{partno}{0}%
  \addtocounter{probno}{1}%
  {\bf Problem~\#{\arabic{probno}}}\quad}

% No initial whitespace to make framing look nicer
\newcommand{\probns}{
  \setcounter{partno}{0}%
  \addtocounter{probno}{1}%
  {\bf Problem~\#{\arabic{probno}}}\quad}

\newcommand{\probpart}{%
  \addtocounter{partno}{1}%
  {\bf (\#\arabic{probno}\alph{partno})}\ \ }
\newcommand{\probcont}{%
  {\bf Problem~\#{\arabic{probno}}}~(\emph{continued})}
\newcommand{\probo}{
  \setcounter{partno}{0}%
  \addtocounter{probno}{1}%
  {\bf (\#\arabic{probno}}\ \ }

%% versions that do print the number of points
\newcommand{\Prob}[1]{
  \vskip10bp%
  \setcounter{partno}{0}%
  \addtocounter{probno}{1}%
  {\bf Problem~\#{\arabic{probno}}~[{#1}~pts]}\quad}

%no initial whitespace
\newcommand{\Probns}[1]{
  \setcounter{partno}{0}%
  \addtocounter{probno}{1}%
  {\bf Problem~\#{\arabic{probno}}~[{#1}~pts]}\quad}
\newcommand{\Probpart}[1]{%\rule{0in}{0in}\\ \phantom{xxx}
  \addtocounter{partno}{1}%
  {\bf (\#\arabic{probno}\alph{partno})~[{#1}~pts]}\ \ }

\newboolean{answers}
\newcommand{\Answer}[1]{\ifthenelse{\boolean{answers}}{{\bf Answer:}\ #1}{}\bigskip}

\begin{document}
\thispagestyle{empty}
\textheight=9in
\parskip=9bp
\bf Math 824, Fall 2012\\
Problem Set \#4\rm

{\bf Instructions:} Type up your solutions using LaTeX; there is a
\href{http://www.jlmartin.faculty.ku.edu/math824/header.tex}{header file}
on the course website with macros that may be useful.
E-mail me (\texttt{jmartin@math.ku.edu}) the PDF file under the name \textsl{$\{$your-name$\}$4.pdf}.  Deadline: {\bf 3:00 PM on Friday, October 26.}
\smallskip\hrule

\prob (Stanley, EC1, 2nd ed., 3.119)  Prove the \emph{$q$-binomial theorem}:
  $$\prod_{k=0}^{n-1}(x-q^k) ~=~
    \sum_{k=0}^n \binom{\bf n}{\bf k} (-1)^k q^{\binom{k}{2}} x^{n-k}.$$
Here $\binom{\bf n}{\bf k}$ denotes the \emph{$q$-binomial coefficient}:
  $$\binom{\bf n}{\bf k} ~=~ \frac{(q^n-1)(q^n-q)\cdots(q^n-q^{k-1})}
    {(q^k-1)(q^k-q)\cdots(q^k-q^{k-1})}.$$
You may, with appropriate citation,
use identities such as those on p.~55 of Stanley's EC1.
(Hint: Let $V=\Ff_q^n$ and let $X$ be a vector
space over $\Ff_q$ with $x$ elements.  Count the number of one-to-one
linear transformations $V\to X$ in two ways.)  Derive the ordinary binomial
theorem as a corollary.

\prob (Stanley, EC1, 3.129) Here is a cute application of combinatorics to
elementary number theory.  Let $P$ be a finite poset,
and let $\mu$ be the M\"obius function of $\hat P=P\cup\{\hat0,\hat1\}$.
Suppose that $P$ has a fixed-point-free automorphism $\sigma:P\to P$
of prime order $p$; that is, $\sigma(x)\neq x$
and $\sigma^p(x)=x$  for all $x\in P$.  Prove
that $\mu_{\hat P}(\hat0,\hat1)\cong-1\pmod p$.  What does this say
in the case that $\hat P=\Pi_p$?

\prob (Stanley, HA, 2.5) Let $G$ be a graph on
$n$ vertices, let $\A_G$ be its graphic arrangement in $\Rr^n$,
and let $\B_G=\BB_n\cup\A_G$.  (That is, $\B$ consists of the
coordinate hyperplanes $x_i=0$ in $\Rr^n$ together with the hyperplanes
$x_i=x_j$ for all edges $ij$ of $G$.)  Calculate $\chi_{\B_G}(q)$
in terms of $\chi_{\A_G}(q)$.

\prob (Stanley, EC2, 3.115(d)]
Calculate the characteristic polynomial and the number of regions
of the \emph{type-B braid arrangement} $\mathcal{O}_n\subseteq\Rr^n$,
with $n(n-1)$ hyperplanes $x_i=x_j$, $x_i=-x_j$ for $1\leq i<j\leq n$.

\prob
Recall that each permutation $w=(w_1,\dots,w_n)\in\Sym_n$ corresponds
to a region of the braid arrangement $Br_n$, namely the open cone
$C_w=\{(x_1,\dots,x_n)\in\Rr^n\st x_{w_1}<x_{w_2}<\dots<x_{w_n}\}$.
Denote its closure by $\overline{C_w}$.  For any set
$W\subseteq\Sym_n$, consider the closed fan
  \[F(W) = \bigcup_{w\in W} \overline{C_w} = \{(x_1,\dots,x_n)\in\Rr^n\st
  x_{w_1}\leq\dots\leq x_{w_n}\ \text{ for some } w\in W\}.\]
Prove that $F(W)$ is a convex set if and only if
$W$ is the set of linear extensions of some poset $P$ on $[n]$.
(A \emph{linear extension} of $P$ is a total
ordering $\prec$ consistent with the ordering of $P$, i.e., if $x<_Py$
then $x\prec y$.)

\prob The runners in a
race are seeded $1,\dots,n$ (stronger runners are assigned
higher numbers).  To even the playing field, the rules specify
that you earn one point for each higher-ranked opponent you beat, and
one point for each lower-ranked opponent you beat \emph{by at least
one second}.  (If a higher-ranked runner beats a lower-ranked
runner by less than 1 second, no one gets a the point for that
matchup.)  Let $s_i$ be the number of points scored by the $i^{th}$
player and let $s=(s_1,\dots,s_n)$ be the \emph{score vector}.

\probpart Show that the possible score vectors are in
bijection with the regions of the Shi arrangement.

\probpart Work out all possible score vectors in the cases of 2 and 3 players.
Conjecture a necessary and sufficient condition for $(s_1,\dots,s_n)$
to be a possible score vector for $n$ players.  Prove it if you can.
\end{document}
