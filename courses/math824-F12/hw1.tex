\documentclass{amsart}
\usepackage{amssymb,amsmath,amsthm,mathrsfs,graphics,hyperref,ifthen,framed,cancel,fullpage,color,ytableau,tcolorbox,bm,tikz}
\usepackage[enableskew]{youngtab}
\raggedbottom
\parskip=10bp
\parindent=0bp
\raggedbottom

%%%%%%%%%%%%%%%%   Colors for TikZ and text  %%%%%%%%%%%%%%%%%

\definecolor{light}{gray}{.75}
\definecolor{med}{gray}{.5}
\definecolor{dark}{gray}{.25}
\newcommand{\Red}[1]{{\color{red}{#1}}}
\newcommand{\RED}[1]{{\color{red}{\boldmath\textbf{#1}\unboldmath}}}
\newcommand{\Blue}[1]{{\color{blue}{#1}}}
\newcommand{\BLUE}[1]{{\color{blue}{\boldmath\textbf{#1}\unboldmath}}}

% Hyperlinks
\hypersetup{colorlinks, citecolor=red, filecolor=black, linkcolor=blue, urlcolor=blue}
\newcommand{\hreftext}[2]{\href{#1}{\Blue{#2}}} % for text anchors
\newcommand{\hrefurl}[2]{\href{#1}{\Blue{\tt #2}}} % if you want to actually write out the URL in text

% Spacing and commands
\newcommand{\bigpad}{\rule[-14mm]{0mm}{30mm}}
\newcommand{\smallpad}{\rule[-1.5mm]{0mm}{5mm}}
\newcommand{\pad}{\rule[-3mm]{0mm}{8mm}}
\newcommand{\padup}{\rule{0mm}{5mm}}
\newcommand{\paddown}{\rule[-3mm]{0mm}{2mm}}
\newcommand{\blank}{\rule{1.25in}{0.25mm}}
\newcommand{\yell}[1]{\fbox{\rule[-1mm]{0mm}{4mm} \large\bf #1 }}
\newcommand{\indnt}{\phantom{.}\qquad}
\newcommand{\littleline}{\begin{center}\rule{4in}{0.5bp}\end{center}}

% macros for inserting figures
\newcommand{\includefigure}[3]{\begin{center}\resizebox{#1}{#2}{\includegraphics{#3}}\end{center}}
\newcommand{\includefigurewithinmath}[3]{\resizebox{#1}{#2}{\includegraphics{#3}}}
% E.g., to insert a standalone figure of width 3" and height 1.5", use
% \includefigure{3in}{1.5in}{foo.pdf}

% math operators
\DeclareMathOperator{\Comp}{Comp}
\DeclareMathOperator{\Fix}{Fix}
\DeclareMathOperator{\id}{id}
\DeclareMathOperator{\im}{im}
\DeclareMathOperator{\lcm}{lcm}
\DeclareMathOperator{\Par}{Par}
\DeclareMathOperator{\rank}{rank}
\DeclareMathOperator{\sh}{sh}
\DeclareMathOperator{\tr}{tr}
\DeclareMathOperator{\wt}{wt}

% theorem environments, automatically numbered
\newtheorem{theorem}{Theorem}[section]
\newtheorem{proposition}[theorem]{Proposition}
\newtheorem{lemma}[theorem]{Lemma}
\newtheorem{corollary}[theorem]{Corollary}
\theoremstyle{definition}
\newtheorem{definition}[theorem]{Definition}
\newtheorem{example}[theorem]{Example}
\newtheorem{remark}[theorem]{Remark}
\newtheorem{problem}[theorem]{Problem}

\newcommand{\skpr}{\emph{Sketch of proof: }}
\newcommand{\soln}{\textit{Solution:\ }}

% Generally useful macros
\newcommand{\excise}[1]{} % useful for commenting out large chunks
\newcommand{\0}{\emptyset}
\newcommand{\compn}{\models} % compositions
\newcommand{\dju}{\mathaccent\cdot\cup} % disjoint union
\newcommand{\dsum}{\displaystyle\sum}
\newcommand{\fallfac}[2]{{#1}^{\underline{#2}}} % falling factorial
\newcommand{\isom}{\cong} % isomorphism symbol
\newcommand{\partn}{\vdash} % partition symbol
\newcommand{\qqandqq}{\qquad\text{and}\qquad}
\newcommand{\qandq}{\quad\text{and}\quad}
\newcommand{\qand}{\quad\text{and}}
\newcommand{\qbin}[2]{{\begin{bmatrix}#1\\#2\end{bmatrix}_q}} % q-binomial coefficient
\newcommand{\risefac}[2]{{#1}^{\overline{#2}}} % rising factorial
\newcommand{\sd}{\triangle} % symmetric difference
\newcommand{\sm}{\setminus} % don't use a minus sign for this
\newcommand{\st}{\colon} % "such that"
\newcommand{\surj}{\twoheadrightarrow}
\newcommand{\x}{\times}

% blackboard bold fonts for sets of numbers
\newcommand{\Cc}{\mathbb{C}} % complex numbers
\newcommand{\Ff}{\mathbb{F}} % finite field
\newcommand{\Nn}{\mathbb{N}} % natural numbers
\newcommand{\Qq}{\mathbb{Q}}
\newcommand{\Rr}{\mathbb{R}}
\newcommand{\Zz}{\mathbb{Z}}

% miscellaneous
\newcommand{\TwoCases}[4]{\begin{cases}{#1}&\text{ #2}\\{#3}&\text{ #4}\end{cases}}
\newcommand{\ThreeCases}[6]{\begin{cases}{#1}&\text{ #2}\\{#3}&\text{ #4}\\{#5}&\text{ #6}\end{cases}}
\newcommand{\bridgehand}[4]{\spadesuit\ {\textsf{#1}}\ \ \heartsuit\ {\textsf{#2}}\ \ \diamondsuit\ {\textsf{#3}}\ \ \clubsuit\ {\textsf{#4}}}

% macros for automatic problem numbering --- students don't have to use these
\newcounter{probno}
\setcounter{probno}{0}
\newcounter{partno}
\setcounter{partno}{0}
%% versions that don't print the number of points
\newcommand{\prob}{
  \vskip10bp%
  \setcounter{partno}{0}%
  \addtocounter{probno}{1}%
  {\bf Problem~\#{\arabic{probno}}}\quad}

% No initial whitespace to make framing look nicer
\newcommand{\probns}{
  \setcounter{partno}{0}%
  \addtocounter{probno}{1}%
  {\bf Problem~\#{\arabic{probno}}}\quad}

\newcommand{\probpart}{%
  \addtocounter{partno}{1}%
  {\bf (\#\arabic{probno}\alph{partno})}\ \ }
\newcommand{\probcont}{%
  {\bf Problem~\#{\arabic{probno}}}~(\emph{continued})}
\newcommand{\probo}{
  \setcounter{partno}{0}%
  \addtocounter{probno}{1}%
  {\bf (\#\arabic{probno}}\ \ }

%% versions that do print the number of points
\newcommand{\Prob}[1]{
  \vskip10bp%
  \setcounter{partno}{0}%
  \addtocounter{probno}{1}%
  {\bf Problem~\#{\arabic{probno}}~[{#1}~pts]}\quad}

%no initial whitespace
\newcommand{\Probns}[1]{
  \setcounter{partno}{0}%
  \addtocounter{probno}{1}%
  {\bf Problem~\#{\arabic{probno}}~[{#1}~pts]}\quad}
\newcommand{\Probpart}[1]{%\rule{0in}{0in}\\ \phantom{xxx}
  \addtocounter{partno}{1}%
  {\bf (\#\arabic{probno}\alph{partno})~[{#1}~pts]}\ \ }

\newboolean{answers}
\newcommand{\Answer}[1]{\ifthenelse{\boolean{answers}}{{\bf Answer:}\ #1}{}\bigskip}

\begin{document}
\bf Math 824, Fall 2012\\
Problem Set \#1\rm

{\bf Instructions:} Type up your solutions using LaTeX.  There is a
header file at\\
\href{http://www.jlmartin.faculty.ku.edu/math824/header.tex}{\tt http://www.jlmartin.faculty.ku.edu/math824/header.tex} with macros that may be useful.
E-mail me (\texttt{jmartin@math.ku.edu}) the PDF file under the name \textit{$\{$your-name$\}$1.pdf}.

Deadline: {\bf 5:00 PM on Wednesday, September 5.}
\medskip\hrule

\prob
A \emph{directed acyclic graph} or DAG, is a pair $G=(V,E)$, where $V$ 
is a finite set of \emph{vertices}; $E$ is a finite set of \emph{edges}, 
each of which is an ordered pair of distinct vertices; and $E$ contains 
no directed cycles, i.e., no subsets of the form
  $$\{(v_1,v_2),\: (v_2,v_3),\: \ldots,\: (v_{n-1},v_n),\: (v_n,v_1)\}$$
for any $v_1,\dots,v_n\in V$.

\probpart Let $P$ be a poset with order relation $<$.
Let $E=\{(v,w) \st v,w \in P, v<w\}$.  Prove that the pair $(P,E)$ is a 
DAG.

\probpart Let $G=(V,E)$ be a DAG.  Define a relation $<$ on $V$ by 
setting 
$v<w$ iff there is some directed path from $v$ to $w$ in $G$, i.e., iff 
$E$ has a subset of the form $\{(v_1,v_2),\: (v_2,v_3),\: \ldots,\: 
(v_{n-1},v_n)\}$ with $v=v_1$ and $w=v_n$.  Prove that this relation 
makes $V$ into a poset.

(This problem is purely a technical exercise, but it does show that 
posets and DAGs are essentially the same thing.)

\prob
Let $n$ be a positive integer.
Let $D_n$ be the set of all positive-integer divisors of $n$ (including $n$
itself), partially ordered by divisibility.

\probpart Prove that $D_n$ is a ranked poset, and describe the rank function.

\probpart For which values of $n$ is $D_n$ (i) a chain; (ii) a Boolean 
algebra?  For which values of $n,m$ is it the case that $D_n\isom D_m$?

\probpart Prove that $D_n$ is a distributive lattice.
  Describe its meet and join operations and its join-irreducible elements.

\probpart Prove that $D_n$ is \emph{self-dual}, i.e., there is a bijection $f:D_n\to D_n$
    such that $f(x)\leq f(y)$ if and only if $x\geq y$.

\prob
Prove that if $L$ is a lattice, then
  $$x\meet(y\join z)=(x\meet y)\join(x\meet z) \qquad \forall x,y,z\in L$$
if and only if
  $$x\join(y\meet z)=(x\join y)\meet(x\join z) \qquad \forall x,y,z\in L.$$
(A consequence is that $L$ is distributive if and only if $L^*$ is;
that is, distributivity is a self-dual condition.)

\prob 
\probpart Describe the join-irreducible elements of Young's lattice~$Y$.

\probpart Let $\lambda=(\lambda_1,\dots,\lambda_\ell)$ be a partition,
  and let $\lambda=\mu_1\join\mu_2\join\cdots\join\mu_k$ be the
  unique minimal decomposition of $\lambda$ into join-irreducibles.
  Explain how to find $k$ from the Ferrers diagram of $\lambda$.

\prob
\probpart Count the maximal chains in $L_n(q)$. (Recall that this is
the lattice of vector subspaces of $V=(\Ff_q)^n$, where $\Ff_q$
is the finite field with $q$ elements).

\probpart Count the maximal chains in the interval $[\0,\lambda]\subset Y$
if the Ferrers diagram of $\lambda$ is a $2\x n$ rectangle.

\probpart Ditto if $\lambda$ is a hook shape (i.e., $\lambda=(n+1,1,1,\dots,1)$,
with a total of $m$ copies of 1).

\prob Prove that the rank-generating function of Bruhat order on $\Sym_n$ is
\[\sum_{\sigma\in\Sym_n} q^{r(\sigma)} = \prod_{i=1}^n \frac{1-q^i}{1-q}\]
where $r(\sigma)=\#\{\{i,j\} \st i<j~\text{and}~\sigma_i>\sigma_j\}$.
(Hint: Induct on $n$, and use one-line notation
for permutations, not cycle notation,.)

\prob
Fill in the details in the proof of Birkhoff's theorem by showing the following facts.

\probpart For a finite distributive lattice~$L$, show that the map $\phi:L\to J(\Irr(L))$
given by
  $$\phi(x) ~=~ \langle p \st p\in\Irr(L),\; p\leq x \rangle$$
is indeed a lattice isomorphism.

\probpart For a finite poset~$P$, show that an order ideal in $P$ is join-irreducible in
$J(P)$ if and only if it is principal (i.e., generated by a single element).

\end{document}
