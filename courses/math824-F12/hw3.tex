\documentclass{amsart}
\usepackage{amssymb,amsmath,amsthm,mathrsfs,graphics,hyperref,stmaryrd,psfrag,arcs,xypic}
\usepackage[enableskew]{youngtab}
\numberwithin{equation}{section}
\raggedbottom
\oddsidemargin=0in
\evensidemargin=0in
\textwidth=6.5in
\textheight=8.8in
\topmargin=0.25in
\headheight=0in
\headsep=0.2in
\footskip=0in
\parskip=10bp
\parindent=0bp

\newcommand{\excise}[1]{}
\newcommand{\Latex}[1]{\textbackslash\texttt{#1}}

\newcommand{\bigpad}{\rule[-14mm]{0mm}{30mm}}
\newcommand{\smallpad}{\rule[-1.5mm]{0mm}{5mm}}
\newcommand{\pad}{\rule[-3mm]{0mm}{8mm}}
\newcommand{\padup}{\rule{0mm}{5mm}}
\newcommand{\paddown}{\rule[-3mm]{0mm}{2mm}}
\newcommand{\blank}{\rule{1.25in}{0.25mm}}
\newcommand{\commentout}[1]{}
\newcommand{\yell}[1]{\fbox{\rule[-1mm]{0mm}{4mm} \large\bf #1 }}
\newcommand{\bang}{$\bullet$\quad}
\newcommand{\indnt}{\phantom{.}\qquad}
\newcommand{\littleline}{\begin{center}\rule{4in}{0.5bp}\end{center}}

\newcommand{\includefigure}[3]{{
  \begin{center}
  \resizebox{#1}{#2}{\includegraphics{{figs/#3}}}
  \end{center}}}
\newcommand{\includefigurewithinmath}[3]{{
  \resizebox{#1}{#2}{\includegraphics{{figs/#3}}}}}

%\newcommand{\defterm}[1]{\underline{\textbf{#1}}}
\newcommand{\defterm}[1]{\textbf{#1}}

\DeclareMathOperator{\ch}{\mathbf{ch}}
\DeclareMathOperator{\colspace}{colspace}
\DeclareMathOperator{\corank}{corank}
\DeclareMathOperator{\CST}{CST}
\DeclareMathOperator{\deln}{del}
\DeclareMathOperator{\diag}{diag}
\DeclareMathOperator{\ess}{ess}
\DeclareMathOperator{\Gr}{Gr}
\DeclareMathOperator{\Hom}{Hom}
\DeclareMathOperator{\im}{im}
\DeclareMathOperator{\Irr}{Irr}
\DeclareMathOperator{\Ind}{Ind}
\DeclareMathOperator{\Int}{Int}
\DeclareMathOperator{\lcm}{lcm}
\DeclareMathOperator{\link}{lk}
\DeclareMathOperator{\nullity}{nullity}
\DeclareMathOperator{\nullspace}{nullspace}
\DeclareMathOperator{\Poin}{Poin}
\DeclareMathOperator{\proj}{proj}
\DeclareMathOperator{\rank}{rank}
\DeclareMathOperator{\Res}{Res}
\DeclareMathOperator{\Span}{span}
\DeclareMathOperator{\supp}{supp}
\DeclareMathOperator{\row}{row}
\DeclareMathOperator{\rowspace}{rowspace}
\DeclareMathOperator{\sh}{sh}
\DeclareMathOperator{\tr}{tr}
\DeclareMathOperator{\wt}{wt}

\newtheorem{theorem}{Theorem}[section]
\newtheorem{proposition}[theorem]{Proposition}
\newtheorem{lemma}[theorem]{Lemma}
\newtheorem{corollary}[theorem]{Corollary}
\theoremstyle{definition}
\newtheorem{definition}[theorem]{Definition}
\newtheorem{example}[theorem]{Example}
\newtheorem{remark}[theorem]{Remark}
\newtheorem{problem}[theorem]{Problem}


\newcommand{\cor}{{\bf Corollary: }}
\newcommand{\defn}{{\bf Definition: }}
\newcommand{\defns}{{\bf Definitions: }}
\newcommand{\exa}{{\bf Example: }}
\newcommand{\fact}{{\bf Fact: }}
\newcommand{\lem}{{\bf Lemma: }}
\newcommand{\notn}{{\bf Notation: }}
\newcommand{\obs}{{\bf Observation: }}
\newcommand{\note}{{\bf Note: }}
\newcommand{\prop}{{\bf Proposition: }}
\newcommand{\rmk}{{\bf Remark: }}
\newcommand{\thm}{{\bf Theorem: }}

\newcommand{\basecase}{\emph{Base case: }}
\newcommand{\indstep}{\emph{Inductive step: }}
\newcommand{\skpr}{\emph{Sketch of proof: }}

\newcommand{\0}{\emptyset}
\newcommand{\Alt}{\mathfrak{A}}
\newcommand{\Braid}{Br}
\newcommand{\CHI}{\chi^{\phantom{*}}}
\newcommand{\Cl}{C\ell}
\newcommand{\covers}{\gtrdot}
\newcommand{\coveredby}{\lessdot}
\newcommand{\dedge}[1]{\overrightarrow{{#1}}}
\newcommand{\dom}{\rhd}
\newcommand{\domeq}{\unrhd}
\newcommand{\domby}{\lhd}
\newcommand{\dombyeq}{\unlhd}
\newcommand{\Fspan}{\Ff\text{-span}}
\newcommand{\isom}{\cong}
\newcommand{\join}{\vee}
\renewcommand{\Join}{\bigvee}
\newcommand{\Laff}{L^{\text{aff}}}
\newcommand{\lin}[1]{\overleftrightarrow{{#1}}}
\newcommand{\meet}{\wedge}
\newcommand{\Meet}{\bigwedge}
\newcommand{\ov}[1]{\overline{{#1}}}
\newcommand{\partn}{\vdash}
\newcommand{\qqandqq}{\qquad\text{and}\qquad}
\newcommand{\qandq}{\quad\text{and}\quad}
\newcommand{\qand}{\quad\text{and}}
\newcommand{\qbin}[2]{{\begin{bmatrix}#1\\#2\end{bmatrix}_q}}
\newcommand{\sd}{\triangle} % symmetric difference
\newcommand{\simK}{\underset{K}{\sim}} % Knuth equivalence
\newcommand{\simJ}{\underset{J}{\sim}} % jeu de taquin equivalence
\newcommand{\sm}{\setminus}
\newcommand{\st}{~|~}
\newcommand{\soln}{\textit{Solution:\ }}
\newcommand{\Sym}{\mathfrak{S}}
\newcommand{\un}[1]{\underset{*}{#1}}
\newcommand{\unA}{\un{A}}
\newcommand{\unB}{\un{B}}
\newcommand{\unw}{\un{w}}
\newcommand{\unx}{\un{x}}
\newcommand{\uny}{\un{y}}
\newcommand{\unz}{\un{z}}
\newcommand{\x}{\times}

\renewcommand{\aa}{\mathbf{a}}
\newcommand{\bb}{\mathbf{b}}
\newcommand{\nn}{\mathbf{n}}
\newcommand{\pp}{\mathbf{p}}
\newcommand{\qq}{\mathbf{q}}
\newcommand{\xx}{\mathbf{x}}
\newcommand{\yy}{\mathbf{y}}
\newcommand{\zz}{\mathbf{z}}
 
\newcommand{\A}{\mathcal{A}}
\newcommand{\B}{\mathcal{B}}
\newcommand{\C}{\mathcal{C}}
\newcommand{\M}{\mathcal{M}}
\renewcommand{\P}{\mathcal{P}}

\newcommand{\BB}{\mathscr{B}}  %% use these for fancy script fonts -- requires mathrsfs package
\newcommand{\CC}{\mathscr{C}}
\newcommand{\FF}{\mathscr{F}}
\newcommand{\II}{\mathscr{I}}
\newcommand{\LL}{\mathscr{L}}
\newcommand{\PP}{\mathscr{P}}
\renewcommand{\SS}{\mathscr{S}}
\newcommand{\XX}{\mathscr{X}}

\newcommand{\TT}{\tilde{T}}

\newcommand{\Aa}{\mathbb{A}}
\newcommand{\Cc}{\mathbb{C}}
\newcommand{\Ff}{\mathbb{F}}
\newcommand{\Nn}{\mathbb{N}}
\newcommand{\Pp}{\mathbb{P}}
\newcommand{\Qq}{\mathbb{Q}}
\newcommand{\Rr}{\mathbb{R}}
\newcommand{\Zz}{\mathbb{Z}}

\newcommand{\rhodef}{\rho^{\phantom{*}}_{{\rm def}}}
\newcommand{\rhotriv}{\rho^{\phantom{*}}_{{\rm triv}}}
\newcommand{\rhosign}{\rho^{\phantom{*}}_{{\rm sign}}}
\newcommand{\rhoreg}{\rho^{\phantom{*}}_{{\rm reg}}}
\newcommand{\chidef}{\chi^{\phantom{*}}_{{\rm def}}}
\newcommand{\chitriv}{\chi^{\phantom{*}}_{{\rm triv}}}
\newcommand{\chisign}{\chi^{\phantom{*}}_{{\rm sign}}}
\newcommand{\chireg}{\chi^{\phantom{*}}_{{\rm reg}}}
\newcommand{\scp}[2]{\left\langle #1,\:#2\right\rangle_G}
\newcommand{\scpH}[2]{\left\langle #1,\:#2\right\rangle_H}

\newcounter{probno}
\setcounter{probno}{0}
\newcounter{partno}
\setcounter{partno}{0}
%% versions that don't print the number of points
\newcommand{\prob}{
  \vskip10bp%
  \setcounter{partno}{0}%
  \addtocounter{probno}{1}%
  {\bf Problem~\#{\arabic{probno}}}\quad}
\newcommand{\probpart}{%\rule{0in}{0in}\\ \phantom{xxx}
  \addtocounter{partno}{1}%
  {\bf (\#\arabic{probno}\alph{partno})}\ \ }
\newcommand{\probcont}{%
  {\bf Problem~\#{\arabic{probno}}}~(\emph{continued})}
\newcommand{\probo}{
  \setcounter{partno}{0}%
  \addtocounter{probno}{1}%
  {\bf (\#\arabic{probno})}\ \ }
%% versions that do print the number of points
\newcommand{\Prob}[1]{
  \vskip10bp%
  \setcounter{partno}{0}%
  \addtocounter{probno}{1}%
  {\bf Problem~\#{\arabic{probno}}~[{#1}~pts]}\quad}
\newcommand{\Probpart}[1]{%\rule{0in}{0in}\\ \phantom{xxx}
  \addtocounter{partno}{1}%
  {\bf (\#\arabic{probno}\alph{partno})~[{#1}~pts]}\ \ }

%\renewcommand{\thefootnote}{\fnsymbol{footnote}}

\begin{document}
\bf Math 824, Fall 2012\\
Problem Set \#3\rm

{\bf Instructions:} Type up your solutions using LaTeX.  There is a
header file at\\
\href{http://www.jlmartin.faculty.ku.edu/math824/header.tex}{\tt http://www.math.ku.edu/$\sim$jmartin/math824/header.tex} with macros that may be useful.
E-mail me (\texttt{jmartin@math.ku.edu}) the PDF file under the name \textsl{$\{$your-name$\}$3.pdf}.

Deadline: {\bf 3:00 PM on Friday, October 5.}
\smallskip\hrule


\prob Determine, with proof, all pairs of integers $k\leq n$ such
that there exists a graph $G$ with $M(G)\isom U_k(n)$.  (Recall
that $U_k(n)$ is the matroid on $E=[n]$ such that every
subset of~$E$ of cardinality~$k$ is a basis.)

\prob Let $X$ and $Y$ be disjoint sets of vertices, and let $B$ be an $X,Y$-bipartite
graph: that is, every edge of $B$ has one endpoint in each of $X$ and $Y$.
For $V=\{x_1,\dots,x_n\}\subset X$, a \emph{transversal} of $V$ is a set $W=\{y_1,\dots,y_n\}\subset Y$
such that $x_iy_i$ is an edge of $B$.  (The set of all edges $x_iy_i$ is called a \emph{matching}.)
Let $\II$ be the family of all subsets of $X$
that have a transversal; it is immediate that $\II$ is a simplicial complex.

Prove that $\II$ is in fact a matroid independence system by verifying that the donation condition
holds.  (Suggestion: Write down an example
or two of a pair of independent sets $I,J$ with $|I|<|J|$, and use the corresponding matchings
to find a systematic way of choosing a vertex that $J$ can donate to $I$.)
These matroids are called \emph{transversal matroids}; along with linear and graphic matroids,
they are the other ``classical'' examples of matroids in combinatorics.)

\prob Let $G=(V,E)$ be a graph with $n$ vertices and $c$ components.
For a vertex coloring $f:V\to\Nn$, let $i(f)$ denote the number of ``improper'' edges,
i.e., whose endpoints are assigned the same color.
\emph{Crapo's coboundary polynomial} of $G$ is
  $$\bar\chi_G(q;t) ~=~ q^{-c}\sum_{f:V\to[q]} t^{i(f)}.$$
This is evidently a stronger invariant than the chromatic polynomial of $G$, which can be obtained as $q\bar\chi_G(q,0)$.
In fact, the coboundary polynomial provides the same information as the Tutte polynomial.

Prove that 
  $$\bar\chi_G(q;t) ~=~ (t-1)^{n-c} T_G\left(\frac{q+t-1}{t-1},t\right)$$
by finding a deletion/contraction recurrence for the coboundary polynomial.

\prob
Let $P$ be a chain-finite poset.
The \emph{kappa function} of $P$ is the element
of the incidence algebra $I(P)$ defined by $\kappa(x,y) = 1$
if $x\coveredby y$, $\kappa(x,y) = 0$ otherwise.

\probpart Give a condition on $\kappa$ that is equivalent   
to $P$ being ranked.

\probpart Give combinatorial interpretations of
$\kappa*\zeta$ and $\zeta*\kappa$.

(See next page for Problem \#5.)
\pagebreak

\prob
Let~$\Pi_n$ be the lattice of set partitions of~$[n]$.  Recall that the
order relation on~$\Pi_n$ is given as follows: if~$\pi,\sigma\in\Pi_n$,
then $\pi\leq\sigma$
if every block of~$\pi$ is contained in some block of~$\sigma$ (for short,
``$\pi$ refines~$\sigma$'').
In this problem,
you're going to calculate the number $\mu_n:=\mu_{\Pi_n}(\hat0,\hat1)$.

\probpart Calculate~$\mu_n$ by brute force for~$n=1,2,3,4$.
Make a conjecture about the value of~$\mu_n$ in general.    

\probpart Define a function~$f:\Pi_n\to\Qq[x]$ as follows: if~$X$
is a finite set of cardinality~$x$, then
  $$f(\pi) = \#\big\{h:[n]\to X \quad\big\vert\quad h(s)=h(s') \iff s,s'
    \text{ belong to the same block of }\pi\big\}.$$
For example, if~$\pi=\hat 1=\{\{1,2,\dots,n\}\}$ is the one-block
partition, then~$f(\pi)$ counts the constant functions from~$[n]$ to~$X$,
so~$f(\pi)=x$.  Find a formula for $f(\pi)$ in general.

\probpart Let~$g(\pi)=\sum_{\sigma\geq\pi} f(\sigma)$.  Prove that
$g(\pi) = x^{|\pi|}$ for all~$\pi\in\Pi_n$.  (Hint: What kinds of
functions are counted by the sum?)

\probpart Apply M\"obius inversion and
an appropriate substitution for~$x$ to calculate~$\mu_n$.

\end{document}
