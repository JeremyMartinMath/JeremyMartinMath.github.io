\documentclass{amsart}
\usepackage{amssymb,amsmath,amsthm,mathrsfs,graphics,hyperref,ifthen,framed,cancel,fullpage,color,ytableau,tcolorbox,bm,tikz}
\usepackage[enableskew]{youngtab}
\raggedbottom
\parskip=10bp
\parindent=0bp
\raggedbottom

%%%%%%%%%%%%%%%%   Colors for TikZ and text  %%%%%%%%%%%%%%%%%

\definecolor{light}{gray}{.75}
\definecolor{med}{gray}{.5}
\definecolor{dark}{gray}{.25}
\newcommand{\Red}[1]{{\color{red}{#1}}}
\newcommand{\RED}[1]{{\color{red}{\boldmath\textbf{#1}\unboldmath}}}
\newcommand{\Blue}[1]{{\color{blue}{#1}}}
\newcommand{\BLUE}[1]{{\color{blue}{\boldmath\textbf{#1}\unboldmath}}}

% Hyperlinks
\hypersetup{colorlinks, citecolor=red, filecolor=black, linkcolor=blue, urlcolor=blue}
\newcommand{\hreftext}[2]{\href{#1}{\Blue{#2}}} % for text anchors
\newcommand{\hrefurl}[2]{\href{#1}{\Blue{\tt #2}}} % if you want to actually write out the URL in text

% Spacing and commands
\newcommand{\bigpad}{\rule[-14mm]{0mm}{30mm}}
\newcommand{\smallpad}{\rule[-1.5mm]{0mm}{5mm}}
\newcommand{\pad}{\rule[-3mm]{0mm}{8mm}}
\newcommand{\padup}{\rule{0mm}{5mm}}
\newcommand{\paddown}{\rule[-3mm]{0mm}{2mm}}
\newcommand{\blank}{\rule{1.25in}{0.25mm}}
\newcommand{\yell}[1]{\fbox{\rule[-1mm]{0mm}{4mm} \large\bf #1 }}
\newcommand{\indnt}{\phantom{.}\qquad}
\newcommand{\littleline}{\begin{center}\rule{4in}{0.5bp}\end{center}}

% macros for inserting figures
\newcommand{\includefigure}[3]{\begin{center}\resizebox{#1}{#2}{\includegraphics{#3}}\end{center}}
\newcommand{\includefigurewithinmath}[3]{\resizebox{#1}{#2}{\includegraphics{#3}}}
% E.g., to insert a standalone figure of width 3" and height 1.5", use
% \includefigure{3in}{1.5in}{foo.pdf}

% math operators
\DeclareMathOperator{\Comp}{Comp}
\DeclareMathOperator{\Fix}{Fix}
\DeclareMathOperator{\id}{id}
\DeclareMathOperator{\im}{im}
\DeclareMathOperator{\lcm}{lcm}
\DeclareMathOperator{\Par}{Par}
\DeclareMathOperator{\rank}{rank}
\DeclareMathOperator{\sh}{sh}
\DeclareMathOperator{\tr}{tr}
\DeclareMathOperator{\wt}{wt}

% theorem environments, automatically numbered
\newtheorem{theorem}{Theorem}[section]
\newtheorem{proposition}[theorem]{Proposition}
\newtheorem{lemma}[theorem]{Lemma}
\newtheorem{corollary}[theorem]{Corollary}
\theoremstyle{definition}
\newtheorem{definition}[theorem]{Definition}
\newtheorem{example}[theorem]{Example}
\newtheorem{remark}[theorem]{Remark}
\newtheorem{problem}[theorem]{Problem}

\newcommand{\skpr}{\emph{Sketch of proof: }}
\newcommand{\soln}{\textit{Solution:\ }}

% Generally useful macros
\newcommand{\excise}[1]{} % useful for commenting out large chunks
\newcommand{\0}{\emptyset}
\newcommand{\compn}{\models} % compositions
\newcommand{\dju}{\mathaccent\cdot\cup} % disjoint union
\newcommand{\dsum}{\displaystyle\sum}
\newcommand{\fallfac}[2]{{#1}^{\underline{#2}}} % falling factorial
\newcommand{\isom}{\cong} % isomorphism symbol
\newcommand{\partn}{\vdash} % partition symbol
\newcommand{\qqandqq}{\qquad\text{and}\qquad}
\newcommand{\qandq}{\quad\text{and}\quad}
\newcommand{\qand}{\quad\text{and}}
\newcommand{\qbin}[2]{{\begin{bmatrix}#1\\#2\end{bmatrix}_q}} % q-binomial coefficient
\newcommand{\risefac}[2]{{#1}^{\overline{#2}}} % rising factorial
\newcommand{\sd}{\triangle} % symmetric difference
\newcommand{\sm}{\setminus} % don't use a minus sign for this
\newcommand{\st}{\colon} % "such that"
\newcommand{\surj}{\twoheadrightarrow}
\newcommand{\x}{\times}

% blackboard bold fonts for sets of numbers
\newcommand{\Cc}{\mathbb{C}} % complex numbers
\newcommand{\Ff}{\mathbb{F}} % finite field
\newcommand{\Nn}{\mathbb{N}} % natural numbers
\newcommand{\Qq}{\mathbb{Q}}
\newcommand{\Rr}{\mathbb{R}}
\newcommand{\Zz}{\mathbb{Z}}

% miscellaneous
\newcommand{\TwoCases}[4]{\begin{cases}{#1}&\text{ #2}\\{#3}&\text{ #4}\end{cases}}
\newcommand{\ThreeCases}[6]{\begin{cases}{#1}&\text{ #2}\\{#3}&\text{ #4}\\{#5}&\text{ #6}\end{cases}}
\newcommand{\bridgehand}[4]{\spadesuit\ {\textsf{#1}}\ \ \heartsuit\ {\textsf{#2}}\ \ \diamondsuit\ {\textsf{#3}}\ \ \clubsuit\ {\textsf{#4}}}

% macros for automatic problem numbering --- students don't have to use these
\newcounter{probno}
\setcounter{probno}{0}
\newcounter{partno}
\setcounter{partno}{0}
%% versions that don't print the number of points
\newcommand{\prob}{
  \vskip10bp%
  \setcounter{partno}{0}%
  \addtocounter{probno}{1}%
  {\bf Problem~\#{\arabic{probno}}}\quad}

% No initial whitespace to make framing look nicer
\newcommand{\probns}{
  \setcounter{partno}{0}%
  \addtocounter{probno}{1}%
  {\bf Problem~\#{\arabic{probno}}}\quad}

\newcommand{\probpart}{%
  \addtocounter{partno}{1}%
  {\bf (\#\arabic{probno}\alph{partno})}\ \ }
\newcommand{\probcont}{%
  {\bf Problem~\#{\arabic{probno}}}~(\emph{continued})}
\newcommand{\probo}{
  \setcounter{partno}{0}%
  \addtocounter{probno}{1}%
  {\bf (\#\arabic{probno}}\ \ }

%% versions that do print the number of points
\newcommand{\Prob}[1]{
  \vskip10bp%
  \setcounter{partno}{0}%
  \addtocounter{probno}{1}%
  {\bf Problem~\#{\arabic{probno}}~[{#1}~pts]}\quad}

%no initial whitespace
\newcommand{\Probns}[1]{
  \setcounter{partno}{0}%
  \addtocounter{probno}{1}%
  {\bf Problem~\#{\arabic{probno}}~[{#1}~pts]}\quad}
\newcommand{\Probpart}[1]{%\rule{0in}{0in}\\ \phantom{xxx}
  \addtocounter{partno}{1}%
  {\bf (\#\arabic{probno}\alph{partno})~[{#1}~pts]}\ \ }

\newboolean{answers}
\newcommand{\Answer}[1]{\ifthenelse{\boolean{answers}}{{\bf Answer:}\ #1}{}\bigskip}

\begin{document}
\bf Math 824, Fall 2012\\
Problem Set \#3\rm

{\bf Instructions:} Type up your solutions using LaTeX.  There is a
header file at\\
\href{http://www.jlmartin.faculty.ku.edu/math824/header.tex}{\tt http://www.math.ku.edu/$\sim$jmartin/math824/header.tex} with macros that may be useful.
E-mail me (\texttt{jmartin@math.ku.edu}) the PDF file under the name \textsl{$\{$your-name$\}$3.pdf}.

Deadline: {\bf 3:00 PM on Friday, October 5.}
\smallskip\hrule


\prob Determine, with proof, all pairs of integers $k\leq n$ such
that there exists a graph $G$ with $M(G)\isom U_k(n)$.  (Recall
that $U_k(n)$ is the matroid on $E=[n]$ such that every
subset of~$E$ of cardinality~$k$ is a basis.)

\prob Let $X$ and $Y$ be disjoint sets of vertices, and let $B$ be an $X,Y$-bipartite
graph: that is, every edge of $B$ has one endpoint in each of $X$ and $Y$.
For $V=\{x_1,\dots,x_n\}\subset X$, a \emph{transversal} of $V$ is a set $W=\{y_1,\dots,y_n\}\subset Y$
such that $x_iy_i$ is an edge of $B$.  (The set of all edges $x_iy_i$ is called a \emph{matching}.)
Let $\II$ be the family of all subsets of $X$
that have a transversal; it is immediate that $\II$ is a simplicial complex.

Prove that $\II$ is in fact a matroid independence system by verifying that the donation condition
holds.  (Suggestion: Write down an example
or two of a pair of independent sets $I,J$ with $|I|<|J|$, and use the corresponding matchings
to find a systematic way of choosing a vertex that $J$ can donate to $I$.)
These matroids are called \emph{transversal matroids}; along with linear and graphic matroids,
they are the other ``classical'' examples of matroids in combinatorics.)

\prob Let $G=(V,E)$ be a graph with $n$ vertices and $c$ components.
For a vertex coloring $f:V\to\Nn$, let $i(f)$ denote the number of ``improper'' edges,
i.e., whose endpoints are assigned the same color.
\emph{Crapo's coboundary polynomial} of $G$ is
  $$\bar\chi_G(q;t) ~=~ q^{-c}\sum_{f:V\to[q]} t^{i(f)}.$$
This is evidently a stronger invariant than the chromatic polynomial of $G$, which can be obtained as $q\bar\chi_G(q,0)$.
In fact, the coboundary polynomial provides the same information as the Tutte polynomial.

Prove that 
  $$\bar\chi_G(q;t) ~=~ (t-1)^{n-c} T_G\left(\frac{q+t-1}{t-1},t\right)$$
by finding a deletion/contraction recurrence for the coboundary polynomial.

\prob
Let $P$ be a chain-finite poset.
The \emph{kappa function} of $P$ is the element
of the incidence algebra $I(P)$ defined by $\kappa(x,y) = 1$
if $x\coveredby y$, $\kappa(x,y) = 0$ otherwise.

\probpart Give a condition on $\kappa$ that is equivalent   
to $P$ being ranked.

\probpart Give combinatorial interpretations of
$\kappa*\zeta$ and $\zeta*\kappa$.

(See next page for Problem \#5.)
\pagebreak

\prob
Let~$\Pi_n$ be the lattice of set partitions of~$[n]$.  Recall that the
order relation on~$\Pi_n$ is given as follows: if~$\pi,\sigma\in\Pi_n$,
then $\pi\leq\sigma$
if every block of~$\pi$ is contained in some block of~$\sigma$ (for short,
``$\pi$ refines~$\sigma$'').
In this problem,
you're going to calculate the number $\mu_n:=\mu_{\Pi_n}(\hat0,\hat1)$.

\probpart Calculate~$\mu_n$ by brute force for~$n=1,2,3,4$.
Make a conjecture about the value of~$\mu_n$ in general.    

\probpart Define a function~$f:\Pi_n\to\Qq[x]$ as follows: if~$X$
is a finite set of cardinality~$x$, then
  $$f(\pi) = \#\big\{h:[n]\to X \quad\big\vert\quad h(s)=h(s') \iff s,s'
    \text{ belong to the same block of }\pi\big\}.$$
For example, if~$\pi=\hat 1=\{\{1,2,\dots,n\}\}$ is the one-block
partition, then~$f(\pi)$ counts the constant functions from~$[n]$ to~$X$,
so~$f(\pi)=x$.  Find a formula for $f(\pi)$ in general.

\probpart Let~$g(\pi)=\sum_{\sigma\geq\pi} f(\sigma)$.  Prove that
$g(\pi) = x^{|\pi|}$ for all~$\pi\in\Pi_n$.  (Hint: What kinds of
functions are counted by the sum?)

\probpart Apply M\"obius inversion and
an appropriate substitution for~$x$ to calculate~$\mu_n$.

\end{document}
