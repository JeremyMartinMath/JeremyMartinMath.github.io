\documentclass{amsart}
\usepackage{amssymb,amsmath,amsthm,mathrsfs,graphics,hyperref,ifthen,framed,cancel,fullpage,color,ytableau,tcolorbox,bm,tikz}
\usepackage[enableskew]{youngtab}
\raggedbottom
\parskip=10bp
\parindent=0bp
\raggedbottom

%%%%%%%%%%%%%%%%   Colors for TikZ and text  %%%%%%%%%%%%%%%%%

\definecolor{light}{gray}{.75}
\definecolor{med}{gray}{.5}
\definecolor{dark}{gray}{.25}
\newcommand{\Red}[1]{{\color{red}{#1}}}
\newcommand{\RED}[1]{{\color{red}{\boldmath\textbf{#1}\unboldmath}}}
\newcommand{\Blue}[1]{{\color{blue}{#1}}}
\newcommand{\BLUE}[1]{{\color{blue}{\boldmath\textbf{#1}\unboldmath}}}

% Hyperlinks
\hypersetup{colorlinks, citecolor=red, filecolor=black, linkcolor=blue, urlcolor=blue}
\newcommand{\hreftext}[2]{\href{#1}{\Blue{#2}}} % for text anchors
\newcommand{\hrefurl}[2]{\href{#1}{\Blue{\tt #2}}} % if you want to actually write out the URL in text

% Spacing and commands
\newcommand{\bigpad}{\rule[-14mm]{0mm}{30mm}}
\newcommand{\smallpad}{\rule[-1.5mm]{0mm}{5mm}}
\newcommand{\pad}{\rule[-3mm]{0mm}{8mm}}
\newcommand{\padup}{\rule{0mm}{5mm}}
\newcommand{\paddown}{\rule[-3mm]{0mm}{2mm}}
\newcommand{\blank}{\rule{1.25in}{0.25mm}}
\newcommand{\yell}[1]{\fbox{\rule[-1mm]{0mm}{4mm} \large\bf #1 }}
\newcommand{\indnt}{\phantom{.}\qquad}
\newcommand{\littleline}{\begin{center}\rule{4in}{0.5bp}\end{center}}

% macros for inserting figures
\newcommand{\includefigure}[3]{\begin{center}\resizebox{#1}{#2}{\includegraphics{#3}}\end{center}}
\newcommand{\includefigurewithinmath}[3]{\resizebox{#1}{#2}{\includegraphics{#3}}}
% E.g., to insert a standalone figure of width 3" and height 1.5", use
% \includefigure{3in}{1.5in}{foo.pdf}

% math operators
\DeclareMathOperator{\Comp}{Comp}
\DeclareMathOperator{\Fix}{Fix}
\DeclareMathOperator{\id}{id}
\DeclareMathOperator{\im}{im}
\DeclareMathOperator{\lcm}{lcm}
\DeclareMathOperator{\Par}{Par}
\DeclareMathOperator{\rank}{rank}
\DeclareMathOperator{\sh}{sh}
\DeclareMathOperator{\tr}{tr}
\DeclareMathOperator{\wt}{wt}

% theorem environments, automatically numbered
\newtheorem{theorem}{Theorem}[section]
\newtheorem{proposition}[theorem]{Proposition}
\newtheorem{lemma}[theorem]{Lemma}
\newtheorem{corollary}[theorem]{Corollary}
\theoremstyle{definition}
\newtheorem{definition}[theorem]{Definition}
\newtheorem{example}[theorem]{Example}
\newtheorem{remark}[theorem]{Remark}
\newtheorem{problem}[theorem]{Problem}

\newcommand{\skpr}{\emph{Sketch of proof: }}
\newcommand{\soln}{\textit{Solution:\ }}

% Generally useful macros
\newcommand{\excise}[1]{} % useful for commenting out large chunks
\newcommand{\0}{\emptyset}
\newcommand{\compn}{\models} % compositions
\newcommand{\dju}{\mathaccent\cdot\cup} % disjoint union
\newcommand{\dsum}{\displaystyle\sum}
\newcommand{\fallfac}[2]{{#1}^{\underline{#2}}} % falling factorial
\newcommand{\isom}{\cong} % isomorphism symbol
\newcommand{\partn}{\vdash} % partition symbol
\newcommand{\qqandqq}{\qquad\text{and}\qquad}
\newcommand{\qandq}{\quad\text{and}\quad}
\newcommand{\qand}{\quad\text{and}}
\newcommand{\qbin}[2]{{\begin{bmatrix}#1\\#2\end{bmatrix}_q}} % q-binomial coefficient
\newcommand{\risefac}[2]{{#1}^{\overline{#2}}} % rising factorial
\newcommand{\sd}{\triangle} % symmetric difference
\newcommand{\sm}{\setminus} % don't use a minus sign for this
\newcommand{\st}{\colon} % "such that"
\newcommand{\surj}{\twoheadrightarrow}
\newcommand{\x}{\times}

% blackboard bold fonts for sets of numbers
\newcommand{\Cc}{\mathbb{C}} % complex numbers
\newcommand{\Ff}{\mathbb{F}} % finite field
\newcommand{\Nn}{\mathbb{N}} % natural numbers
\newcommand{\Qq}{\mathbb{Q}}
\newcommand{\Rr}{\mathbb{R}}
\newcommand{\Zz}{\mathbb{Z}}

% miscellaneous
\newcommand{\TwoCases}[4]{\begin{cases}{#1}&\text{ #2}\\{#3}&\text{ #4}\end{cases}}
\newcommand{\ThreeCases}[6]{\begin{cases}{#1}&\text{ #2}\\{#3}&\text{ #4}\\{#5}&\text{ #6}\end{cases}}
\newcommand{\bridgehand}[4]{\spadesuit\ {\textsf{#1}}\ \ \heartsuit\ {\textsf{#2}}\ \ \diamondsuit\ {\textsf{#3}}\ \ \clubsuit\ {\textsf{#4}}}

% macros for automatic problem numbering --- students don't have to use these
\newcounter{probno}
\setcounter{probno}{0}
\newcounter{partno}
\setcounter{partno}{0}
%% versions that don't print the number of points
\newcommand{\prob}{
  \vskip10bp%
  \setcounter{partno}{0}%
  \addtocounter{probno}{1}%
  {\bf Problem~\#{\arabic{probno}}}\quad}

% No initial whitespace to make framing look nicer
\newcommand{\probns}{
  \setcounter{partno}{0}%
  \addtocounter{probno}{1}%
  {\bf Problem~\#{\arabic{probno}}}\quad}

\newcommand{\probpart}{%
  \addtocounter{partno}{1}%
  {\bf (\#\arabic{probno}\alph{partno})}\ \ }
\newcommand{\probcont}{%
  {\bf Problem~\#{\arabic{probno}}}~(\emph{continued})}
\newcommand{\probo}{
  \setcounter{partno}{0}%
  \addtocounter{probno}{1}%
  {\bf (\#\arabic{probno}}\ \ }

%% versions that do print the number of points
\newcommand{\Prob}[1]{
  \vskip10bp%
  \setcounter{partno}{0}%
  \addtocounter{probno}{1}%
  {\bf Problem~\#{\arabic{probno}}~[{#1}~pts]}\quad}

%no initial whitespace
\newcommand{\Probns}[1]{
  \setcounter{partno}{0}%
  \addtocounter{probno}{1}%
  {\bf Problem~\#{\arabic{probno}}~[{#1}~pts]}\quad}
\newcommand{\Probpart}[1]{%\rule{0in}{0in}\\ \phantom{xxx}
  \addtocounter{partno}{1}%
  {\bf (\#\arabic{probno}\alph{partno})~[{#1}~pts]}\ \ }

\newboolean{answers}
\newcommand{\Answer}[1]{\ifthenelse{\boolean{answers}}{{\bf Answer:}\ #1}{}\bigskip}

\begin{document}
\thispagestyle{empty}
\bf Math 824, Fall 2012\\
Problem Set \#5\rm

{\bf Instructions:} Type up your solutions using LaTeX; there is a
\href{http://www.jlmartin.faculty.ku.edu/math824/header.tex}{header file}
on the course website with macros that may be useful.
E-mail me (\texttt{jmartin@math.ku.edu}) the PDF file under the name \textsl{$\{$your-name$\}$4.pdf}.  Deadline: {\bf 5:00 PM on Monday, December 3.}
\smallskip\hrule

%--------------------

\prob Consider the permutation action of the symmetric group $\Sym_4$
on the vertices of the complete graph $K_4$, whose corresponding
representation is the defining representation $\rhodef$ (let's say
over $\Cc$).  Let $\sigma$ be the 3-dimensional representation
corresponding to the action of $\Sym_4$ on pairs of opposite edges of
$K_4$.

\probpart Compute the character of $\sigma$.

\probpart Explicitly describe all $G$-equivariant linear transformations $\phi:\rhodef\to\sigma$.
(Hint: Schur's lemma should be useful.)

\bigskip%--------------------

\prob Recall that the \emph{alternating group} $\Alt_n$ consists of the $n!/2$
even permutations in $\Sym_n$, that is, those with an even number of even-length
cycles.

\probpart Show that the conjugacy classes in $\Alt_4$ are not simply the conjugacy classes
in $\Sym_4$.  (Hint: Consider the possibilities for the dimensions of the irreducible characters
of $\Alt_4$.)

\probpart Determine the conjugacy classes in $\Alt_4$, and the complete list of irreducible characters.

\probpart Use this information to determine $[\Alt_4,\Alt_4]$   
without actually computing any commutators.

\bigskip%--------------------

\prob Supply the proofs for the identities (9.12) on p.108 of the lecture notes:
$$\prod_{i,j\geq 1} (1+x_iy_j) ~=~ \sum_\lambda e_\lambda(\xx) m_\lambda(\yy)
~=~ \sum_\lambda\varepsilon_\lambda \frac{p_\lambda(\xx) p_\lambda(\yy)}{z_\lambda}.$$

\bigskip%--------------------

\prob Prove parts (v) and (vi) of Theorem 9.25 on p.118 of the lecture notes,
namely that $\ch(\Ind_{\Sym_\lambda}^{\Sym_n}\chitriv)=h_\lambda$
and $\ch(\Ind_{\Sym_\lambda}^{\Sym_n}\chisign)=e_\lambda$.

\bigskip%--------------------

\prob
\probpart For $w\in\Sym_n$, let $(P(w),Q(w))$ be the pair of tableaux produced by
the RSK algorithm from $w$.  Denote by $w^*$ the reversal of $w$ in one-line notation
(for instance, if $w=57214836$ then $w^*=63841275$).  Prove that $P(w^*)=P(w)^T$ (where ${}^T$ means transpose).
Hint: Figure out how to describe the rows and columns of $P(w)$ in terms of
subsequences of $w$.

\probpart \emph{(Open problem; optional)} For which permutations does  $Q(w^*)=Q(w)$?
Maple computation indicates that the number of such permutations is
  $$\begin{cases}
  \dfrac{2^{(n-1)/2}(n-1)!}{((n-1)/2)!^2} & \text{ if $n$ is odd,}\\
  0 & \text{ if $n$ is even,}
  \end{cases}$$
but I don't know a combinatorial (or even an algebraic) reason.

\probpart \emph{(Open problem; optional)} For which permutations does  $Q(w^*)=Q(w)^T$?  I have no idea what
the answer is. The sequence $(q_1,q_2,\dots)=(1,2,2,12,24,136,344,2872,7108,\dots)$, where
$q_n=\#\{w\in\Sym_n\st Q(w^*)=Q(w)^T\}$,
does not seem to appear in the Online Encyclopedia of Integer Sequences.

\prob (\emph{Open problem; optional}) Let $n\geq 2$ and for $\sigma\in\Sym_n$, let $f(\sigma)$ denote the number of fixed points.
As a warmup, prove that $\sum_{\sigma\in\Sym_n} f(\sigma)^2=2\cdot n!$.
Open problem (to the best of my knowledge): Prove that for any $n,k$,
the number $\frac{1}{n!}\sum_{\sigma\in\Sym_n} f(\sigma)^k$ is an integer.
It appears to be A203647 in OEIS.
Find a formula and/or a representation-theoretic interpretation.

\end{document}
