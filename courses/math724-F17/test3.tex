\documentclass{amsart}
\usepackage{amssymb,amsmath,amsthm,mathrsfs,graphics,hyperref,stmaryrd,psfrag,arcs,xypic}
\usepackage[enableskew]{youngtab}
\numberwithin{equation}{section}
\raggedbottom
\oddsidemargin=0in
\evensidemargin=0in
\textwidth=6.5in
\textheight=8.8in
\topmargin=0.25in
\headheight=0in
\headsep=0.2in
\footskip=0in
\parskip=10bp
\parindent=0bp

\newcommand{\excise}[1]{}
\newcommand{\Latex}[1]{\textbackslash\texttt{#1}}

\newcommand{\bigpad}{\rule[-14mm]{0mm}{30mm}}
\newcommand{\smallpad}{\rule[-1.5mm]{0mm}{5mm}}
\newcommand{\pad}{\rule[-3mm]{0mm}{8mm}}
\newcommand{\padup}{\rule{0mm}{5mm}}
\newcommand{\paddown}{\rule[-3mm]{0mm}{2mm}}
\newcommand{\blank}{\rule{1.25in}{0.25mm}}
\newcommand{\commentout}[1]{}
\newcommand{\yell}[1]{\fbox{\rule[-1mm]{0mm}{4mm} \large\bf #1 }}
\newcommand{\bang}{$\bullet$\quad}
\newcommand{\indnt}{\phantom{.}\qquad}
\newcommand{\littleline}{\begin{center}\rule{4in}{0.5bp}\end{center}}

\newcommand{\includefigure}[3]{{
  \begin{center}
  \resizebox{#1}{#2}{\includegraphics{{figs/#3}}}
  \end{center}}}
\newcommand{\includefigurewithinmath}[3]{{
  \resizebox{#1}{#2}{\includegraphics{{figs/#3}}}}}

%\newcommand{\defterm}[1]{\underline{\textbf{#1}}}
\newcommand{\defterm}[1]{\textbf{#1}}

\DeclareMathOperator{\ch}{\mathbf{ch}}
\DeclareMathOperator{\colspace}{colspace}
\DeclareMathOperator{\corank}{corank}
\DeclareMathOperator{\CST}{CST}
\DeclareMathOperator{\deln}{del}
\DeclareMathOperator{\diag}{diag}
\DeclareMathOperator{\ess}{ess}
\DeclareMathOperator{\Gr}{Gr}
\DeclareMathOperator{\Hom}{Hom}
\DeclareMathOperator{\im}{im}
\DeclareMathOperator{\Irr}{Irr}
\DeclareMathOperator{\Ind}{Ind}
\DeclareMathOperator{\Int}{Int}
\DeclareMathOperator{\lcm}{lcm}
\DeclareMathOperator{\link}{lk}
\DeclareMathOperator{\nullity}{nullity}
\DeclareMathOperator{\nullspace}{nullspace}
\DeclareMathOperator{\Poin}{Poin}
\DeclareMathOperator{\proj}{proj}
\DeclareMathOperator{\rank}{rank}
\DeclareMathOperator{\Res}{Res}
\DeclareMathOperator{\Span}{span}
\DeclareMathOperator{\supp}{supp}
\DeclareMathOperator{\row}{row}
\DeclareMathOperator{\rowspace}{rowspace}
\DeclareMathOperator{\sh}{sh}
\DeclareMathOperator{\tr}{tr}
\DeclareMathOperator{\wt}{wt}

\newtheorem{theorem}{Theorem}[section]
\newtheorem{proposition}[theorem]{Proposition}
\newtheorem{lemma}[theorem]{Lemma}
\newtheorem{corollary}[theorem]{Corollary}
\theoremstyle{definition}
\newtheorem{definition}[theorem]{Definition}
\newtheorem{example}[theorem]{Example}
\newtheorem{remark}[theorem]{Remark}
\newtheorem{problem}[theorem]{Problem}


\newcommand{\cor}{{\bf Corollary: }}
\newcommand{\defn}{{\bf Definition: }}
\newcommand{\defns}{{\bf Definitions: }}
\newcommand{\exa}{{\bf Example: }}
\newcommand{\fact}{{\bf Fact: }}
\newcommand{\lem}{{\bf Lemma: }}
\newcommand{\notn}{{\bf Notation: }}
\newcommand{\obs}{{\bf Observation: }}
\newcommand{\note}{{\bf Note: }}
\newcommand{\prop}{{\bf Proposition: }}
\newcommand{\rmk}{{\bf Remark: }}
\newcommand{\thm}{{\bf Theorem: }}

\newcommand{\basecase}{\emph{Base case: }}
\newcommand{\indstep}{\emph{Inductive step: }}
\newcommand{\skpr}{\emph{Sketch of proof: }}

\newcommand{\0}{\emptyset}
\newcommand{\Alt}{\mathfrak{A}}
\newcommand{\Braid}{Br}
\newcommand{\CHI}{\chi^{\phantom{*}}}
\newcommand{\Cl}{C\ell}
\newcommand{\covers}{\gtrdot}
\newcommand{\coveredby}{\lessdot}
\newcommand{\dedge}[1]{\overrightarrow{{#1}}}
\newcommand{\dom}{\rhd}
\newcommand{\domeq}{\unrhd}
\newcommand{\domby}{\lhd}
\newcommand{\dombyeq}{\unlhd}
\newcommand{\Fspan}{\Ff\text{-span}}
\newcommand{\isom}{\cong}
\newcommand{\join}{\vee}
\renewcommand{\Join}{\bigvee}
\newcommand{\Laff}{L^{\text{aff}}}
\newcommand{\lin}[1]{\overleftrightarrow{{#1}}}
\newcommand{\meet}{\wedge}
\newcommand{\Meet}{\bigwedge}
\newcommand{\ov}[1]{\overline{{#1}}}
\newcommand{\partn}{\vdash}
\newcommand{\qqandqq}{\qquad\text{and}\qquad}
\newcommand{\qandq}{\quad\text{and}\quad}
\newcommand{\qand}{\quad\text{and}}
\newcommand{\qbin}[2]{{\begin{bmatrix}#1\\#2\end{bmatrix}_q}}
\newcommand{\sd}{\triangle} % symmetric difference
\newcommand{\simK}{\underset{K}{\sim}} % Knuth equivalence
\newcommand{\simJ}{\underset{J}{\sim}} % jeu de taquin equivalence
\newcommand{\sm}{\setminus}
\newcommand{\st}{~|~}
\newcommand{\soln}{\textit{Solution:\ }}
\newcommand{\Sym}{\mathfrak{S}}
\newcommand{\un}[1]{\underset{*}{#1}}
\newcommand{\unA}{\un{A}}
\newcommand{\unB}{\un{B}}
\newcommand{\unw}{\un{w}}
\newcommand{\unx}{\un{x}}
\newcommand{\uny}{\un{y}}
\newcommand{\unz}{\un{z}}
\newcommand{\x}{\times}

\renewcommand{\aa}{\mathbf{a}}
\newcommand{\bb}{\mathbf{b}}
\newcommand{\nn}{\mathbf{n}}
\newcommand{\pp}{\mathbf{p}}
\newcommand{\qq}{\mathbf{q}}
\newcommand{\xx}{\mathbf{x}}
\newcommand{\yy}{\mathbf{y}}
\newcommand{\zz}{\mathbf{z}}
 
\newcommand{\A}{\mathcal{A}}
\newcommand{\B}{\mathcal{B}}
\newcommand{\C}{\mathcal{C}}
\newcommand{\M}{\mathcal{M}}
\renewcommand{\P}{\mathcal{P}}

\newcommand{\BB}{\mathscr{B}}  %% use these for fancy script fonts -- requires mathrsfs package
\newcommand{\CC}{\mathscr{C}}
\newcommand{\FF}{\mathscr{F}}
\newcommand{\II}{\mathscr{I}}
\newcommand{\LL}{\mathscr{L}}
\newcommand{\PP}{\mathscr{P}}
\renewcommand{\SS}{\mathscr{S}}
\newcommand{\XX}{\mathscr{X}}

\newcommand{\TT}{\tilde{T}}

\newcommand{\Aa}{\mathbb{A}}
\newcommand{\Cc}{\mathbb{C}}
\newcommand{\Ff}{\mathbb{F}}
\newcommand{\Nn}{\mathbb{N}}
\newcommand{\Pp}{\mathbb{P}}
\newcommand{\Qq}{\mathbb{Q}}
\newcommand{\Rr}{\mathbb{R}}
\newcommand{\Zz}{\mathbb{Z}}

\newcommand{\rhodef}{\rho^{\phantom{*}}_{{\rm def}}}
\newcommand{\rhotriv}{\rho^{\phantom{*}}_{{\rm triv}}}
\newcommand{\rhosign}{\rho^{\phantom{*}}_{{\rm sign}}}
\newcommand{\rhoreg}{\rho^{\phantom{*}}_{{\rm reg}}}
\newcommand{\chidef}{\chi^{\phantom{*}}_{{\rm def}}}
\newcommand{\chitriv}{\chi^{\phantom{*}}_{{\rm triv}}}
\newcommand{\chisign}{\chi^{\phantom{*}}_{{\rm sign}}}
\newcommand{\chireg}{\chi^{\phantom{*}}_{{\rm reg}}}
\newcommand{\scp}[2]{\left\langle #1,\:#2\right\rangle_G}
\newcommand{\scpH}[2]{\left\langle #1,\:#2\right\rangle_H}

\newcounter{probno}
\setcounter{probno}{0}
\newcounter{partno}
\setcounter{partno}{0}
%% versions that don't print the number of points
\newcommand{\prob}{
  \vskip10bp%
  \setcounter{partno}{0}%
  \addtocounter{probno}{1}%
  {\bf Problem~\#{\arabic{probno}}}\quad}
\newcommand{\probpart}{%\rule{0in}{0in}\\ \phantom{xxx}
  \addtocounter{partno}{1}%
  {\bf (\#\arabic{probno}\alph{partno})}\ \ }
\newcommand{\probcont}{%
  {\bf Problem~\#{\arabic{probno}}}~(\emph{continued})}
\newcommand{\probo}{
  \setcounter{partno}{0}%
  \addtocounter{probno}{1}%
  {\bf (\#\arabic{probno})}\ \ }
%% versions that do print the number of points
\newcommand{\Prob}[1]{
  \vskip10bp%
  \setcounter{partno}{0}%
  \addtocounter{probno}{1}%
  {\bf Problem~\#{\arabic{probno}}~[{#1}~pts]}\quad}
\newcommand{\Probpart}[1]{%\rule{0in}{0in}\\ \phantom{xxx}
  \addtocounter{partno}{1}%
  {\bf (\#\arabic{probno}\alph{partno})~[{#1}~pts]}\ \ }

%\renewcommand{\thefootnote}{\fnsymbol{footnote}}

\begin{document}
\thispagestyle{empty}

\textbf{Math 724, Fall 2017\\
Take-Home Test \#3\\
Deadline: Wednesday, December 13, 10:00am}

\textbf{Instructions:} Typeset your solutions in LaTeX.  Email your solutions to Jeremy (jlmartin@ku.edu) as a PDF file named with your last name, e.g., {\tt MirzakhaniTest3.pdf}.  You may refer to the textbook and your class notes, and you may cite the result of any problem from previous assignments, or done in class.  You may also use a computer algebra system such as Sage to carry out calculations and test conjectures.  However, \emph{you are not allowed to collaborate}; you may not consult any external resource or any human other than Jeremy.
\medskip\hrule

\prob
The game of \emph{bridge} is an advanced version of egdirb.  It uses a standard deck with four suits (spades, hearts, diamonds, clubs), each with 13 cards (ace, king, queen, jack, 10, 9, \dots, 2).  There are four players.  Each player is dealt  a hand of 13 cards.

\Probpart{10} Bridge players call a hand ``balanced'' if it contains at least 2 cards in every suit, and no more than 8 cards in any two suits.  How many possible bridge hands are balanced?\\
\Probpart{10} How many possible bridge hands contain at least one ace, at least one king, and at least one queen?

\bigskip\hrule

\Prob{20} How many ways are there of making change for a three-dollar bill with pennies, nickels, dimes, and quarters that use
at least one, but no more than ten, of each kind of coin?
(The answer is {\bf not} ``Zero; there is no such thing as a three-dollar bill.''  Pretend there is.)

\bigskip\hrule

\Prob{20}
For a graph $G=(V,E)$, let $\phi(G)$ be the number of forests in $G$ --- that is, subsets of $E$ that contain no cycles.
For example, if $G$ is itself a forest then $\phi(G)=2^{|E|}$, and if $G$ is a cycle then $\phi(G)=2^{|E|}-1$.
Prove that
\[\phi(G)=\begin{cases}
\phi(G-e) & \text{ if $e$ is a loop},\\
2\phi(G/e) & \text{ if $e$ is a bridge},\\
\phi(G-e)+\phi(G/e)& \text{ otherwise.}
\end{cases}\]
(A \emph{loop} is an edge whose endpoints are equal; a \emph{bridge} is an edge such that deleting it causes one component to split into two.)

\bigskip\hrule

\Prob{10} Give a combinatorial interpretation for the coefficient of $x^nq^k$ in the power series expansion of the infinite product
\[\left(\frac{1}{1-x}\right) \left(\frac{1}{1-qx^2}\right)
\left(\frac{1}{1-x^3}\right) \left(\frac{1}{1-qx^4}\right)
\left(\frac{1}{1-x^5}\right) \left(\frac{1}{1-qx^6}\right)\cdots.\]
In other words, describe a set of combinatorial objects whose cardinality is the coefficient of $x^nq^k$.

\pagebreak

\Prob{20} A \emph{noncrossing matching} consists of $2n$ points arranged in a line, together with $n$ arcs linking the points in pairs, such that no two arcs cross.  For example, the noncrossing matchings for $n=3$ are shown below.

\includefigure{6.375in}{0.375in}{ncm3} % 17 x 1

Prove that for all $n$, the number of noncrossing matchings on~$2n$ points is the Catalan number~$C_n$.  (You can do this either by exhibiting a bijection to a set known to be counted by the Catalan numbers, or by verifying that the Catalan recurrence holds.)

\bigskip\hrule


\prob Let $G(n)$ be the set of directed graphs on vertex set $[n]$ in which every component is either a pair of opposite edges, or a 3-cycle with edges oriented cyclically.  For example, an element of $G(10)$ is shown below.
Let $g(n)=|G(n)|$.  By convention, we will set $g(0)=1$, and $g(n)=0$ for $n<0$.
\includefigure{3in}{1in}{directed23} % original 5.25 x 1.75

\Probpart{10} What are the numbers $g(1),\dots,g(5)$?

\Probpart{10} Find a recurrence for $g(n)$.  (Hint: Consider the cycle containing vertex $n$ --- there are two cases.)

\Probpart{10} Let $y=\sum_{n=0}^\infty g(n) x^n/n!$ be the EGF for $g$.
Translate the recurrence you just found into a differential equation for $y$.

\Probpart{10} Solve the differential equation (it should not be hard) to obtain a nice closed form for $y$.

\Probpart{20} Show that the EGF for the set of directed graphs on vertex set $[n]$ in which every component is an oriented cycle of \emph{odd} length is
\[\sqrt{\frac{1+x}{1-x}}.\]
\end{document}
