%%%%%%%%%% the following setup maximizes number of characters per page---
%%%%%%%%%% good for solution sets, etc.  Don't use for articles.

\documentclass{amsart}
\raggedbottom
\oddsidemargin=0in
\evensidemargin=0in
\textwidth=6.5in
\textheight=9in
\topmargin=0in
\headheight=0in
\headsep=0in
\footskip=0in
\parskip=10bp
\parindent=0bp
%\footheight=0in
\pagestyle{empty}
\usepackage{amsfonts,amssymb}

\newtheorem{defn}{Definition}

\newcommand{\BS}{\blacksquare}
\newcommand{\ini}{\mathop{\rm in}\nolimits}
\newcommand{\Proj}{\mathop{\rm Proj}\nolimits}
\newcommand{\Hilb}{\mathop{\rm Hilb}\nolimits}
\newcommand{\fld}{{\bf k}}
\newcommand{\NN}{{\mathbb N}}
\newcommand{\PP}{{\mathbb P}}
\newcommand{\defterm}[1] {{\it #1\/}}
\newcommand{\puttext}[2] {\put(#1){\makebox(0,0){#2}}}
\newcommand{\putdot}[1]  {\put(#1){\makebox(0,0){$\bullet$}}}
\newcommand{\putline}[3] {\put(#1){\line(#2){#3}}}

\begin{document}

Informal Seminar on Stanley-Reisner Theory, UMN, Fall 2002 \\
17 October 2002

{\bf Introduction and motivation for Stanley-Reisner rings, I} \\
Speaker: Vic Reiner

Scribe notes by Jeremy Martin \\

\hrule

\section{Definitions}

\begin{defn} Let $V$ be a finite set of vertices.  An \defterm{abstract simplicial complex} 
$\Delta$ on $V$ is a subset of the power set $2^V$ which is closed under inclusion, that is,
	$$F \in \Delta, ~ G \subset F \quad\implies\quad G \in \Delta.$$
\end{defn}

The elements of $\Delta$ are called \defterm{faces}.  The \defterm{dimension} of a face, $\dim 
F$, is defined as $|F|-1$.

We can often represent $\Delta$ pictorially.  For instance, if $V = \{a,b,c,d\}$ and $\Delta$
is the abstract simplicial complex
	\begin{equation} \label{ex1}
	\Delta = \left\{ \emptyset, a, b, c, d, ab, bc, cd \right\}
	\end{equation}
(abbreviating the face $\{a\}$ by $a$, $\{a,b\}$ by $ab$, etc.), then the corresponding figure 
is
	\begin{center}
	\begin{picture}(50,50)
	\putdot{10,10}	\puttext{ 2,10}{$a$}
	\putdot{10,40}	\puttext{ 2,40}{$b$}
	\putdot{40,40}	\puttext{48,40}{$c$}
	\putdot{40,10}	\puttext{48,10}{$d$}
	\putline{10,10}{0,1}{30}
	\putline{10,40}{1,0}{30}
	\putline{40,40}{0,-1}{30}
	\end{picture}
	\end{center}
One can think of $a,b,c,d$ as orthonormal basis vectors in $|V|$-space, so that the face $ab$ 
(which has dimension $1$) represents the affine span of the vectors $a,b$ (which is a line 
segment), etc.

Fix a field $\fld$, and let
	$$S = \fld[x_v ~:~ v \in V],$$
the (commutative) polynomial ring in variables corresponding to the vertices.

\begin{defn} The \defterm{Stanley-Reisner ideal} of $\Delta$ is
	$$I_{\Delta} := \left( \prod_{j=1}^r x_{v_{i_j}} ~:~ \left\{v_{i_1}, \dots, 
	v_{i_r}\right\} \not\in \Delta \right).$$
\end{defn}

Note that $I_\Delta$ is a \defterm{monomial ideal} (that is, it is generated by monomials) and
that the generators are \defterm{squarefree} (they are not divisible by the square of any
variable). A minimal set of generators is given by the minimal nonfaces of $\Delta$.

\begin{defn} The \defterm{Stanley-Reisner ring} (or \defterm{face ring}) of $\Delta$ is
	$$\fld[\Delta] := S/I_\Delta.$$
\end{defn}

Note that the set of monomials
	$$\left\{ x_{v_1}^{e_1} \dots x_{v_r}^{e_r} ~:~ \{v_1, \dots, v_r\} \in \Delta, ~~
	e_1, \dots, e_r > 0 \right\}$$
is a basis for $\fld[\Delta]$ as a $\fld$-vector space.  In particular, $\fld[\Delta]$ is a 
graded ring.

For example, if $\Delta$ is the simplicial complex given in (\ref{ex1}), then
	$$I_\Delta = (ac,ad,bd)$$
(the minimal nonfaces of $\Delta$) and $\fld[\Delta]$ is the $k$-linear span of
	\begin{equation} \label{bas}
	\left\{ \begin{array}{ll}
	1, a, a^2, a^3, \dots, & ab, a^2b, ab^2, \dots, \\
	1, b, b^2, b^3, \dots, & bc, b^2c, bc^2, \dots, \\
	1, c, c^2, c^3, \dots, & cd, c^2d, cd^2, \dots, \\
	1, d, d^2, d^3, \dots
	\end{array} \right\}~.
	\end{equation}

This construction actually gives a bijection between simplicial complexes on $V$ and ideals of
$S$ generated by squarefree monomials.  The simplicial complex corresponding to such an ideal
is its \defterm{Stanley-Reisner complex}.

\section{Motivations}

1. In algebraic geometry, one wants to study rings of the form $R=S/I$, where $S$ is a
polynomial ring over a field $\fld$ and $I$ is an ideal of $S$.  That is, $R$ is the
coordinate ring of the affine algebraic variety defined by $I$.  To study $R$ using
Stanley-Reisner rings, we may proceed as follows:

First, ``deform'' $I$ as follows.  Fix some monomial order $<$ on $S$ and compute the
\defterm{initial ideal} $\ini_<(I) \subset S$ (this is equivalent to computing a Gr\"obner
basis of $I$).  By definition, $\ini_<(I)$ is generated by monomials, However, the generators
need not be squarefree, so a second step, \defterm{polarization}, may be required.  The idea
of this step is to get rid of high powers of variables by the following trick: if one of the
generators of $\ini_<(I)$ is, say, $x^2$, then we adjoin a new variable $x'$, replace $x^2$
with $xx'$, and mod out by $x-x'$.  The result is a squarefree monomial ideal of some
polynomial ring $S' \supset S$, which we may regard as the Stanley-Reisner ideal $I_\Delta$ of
a simplicial complex $\Delta$ on the variables of $S'$.  The ideal $\ini_<(I)$ and its
polarization are {\it very\/} closely related, so we don't have to worry too much about this
second step.

The passage from $I$ to $\ini_<(I)$ does not preserve all structure, but it is pretty good (in
the language of algebraic geometry, it is a \defterm{flat deformation}). Lots of
geometric/ring invariants of $R$ are closely related---often equal---to those of $\fld[\Delta]
= S'/I_{\Delta}$.  For instance, the dimension, degree and Hilbert series of $R$ are the same
as for $\fld[\Delta]$, and these can be computed combinatorially from $\Delta$. In addition,
some homological-type properties, such as Cohen-Macaulayness, can only get worse--e.g., if
$\fld[\Delta]$ is Cohen-Macaulay then so is $R$.  (For an example, see the scribe's Ph.D.
thesis.)

Here's an elementary example.  Let $S = \fld[a,b,c,d]$, $I = (ac-b^2, bd-c^2, ad-bc)$, and $R 
= S/I$.  (In fact, $\Proj(R)$ is the twisted cubic, the image of the degree-$3$ Veronese 
embedding $\PP^1 \to \PP^3$ mapping $[s:t]$ to $[s^3 : s^2t : st^2 : t^3]$ in homogeneous 
coordinates.)  The given generators of $I$ form a Gr\"obner basis with respect to
lexicographic order on the monomials of $S$, with $a<b<c<d$, so the initial ideal is
	$$\ini_<(I) = (ac,bd,ad),$$
the Stanley-Reisner ideal of the simplicial complex $\Delta$ given in (\ref{ex1}).  
Geometrically, the twisted cubic is being ``flattened out'':

\begin{center}
\begin{picture}(400,200)
\qbezier(100,200)(50,150)(95,105)
\qbezier(105,95)(140,60)(140,100)
\qbezier(140,100)(140,140)(100,100)
\qbezier(100,100)(50,50)(100,0)
	\puttext{180,100}{\Huge {$\rightsquigarrow$}}
\putline{300,0}{0,1}{200}
\putdot{300,60}
\putdot{300,140}
\putline{330,30}{-1,1}{120}
\putline{330,170}{-1,-1}{65}
\putline{210,50}{1,1}{45}

\end{picture}
\end{center}

We compute the dimension, degree and Hilbert series of the twisted cubic using $\fld[\Delta]$.

{\bf Fact:} $\dim \fld[\Delta] = 1 + \dim \Delta = 1 + \max \{ \dim F ~:~ F \in \Delta\}$.  

In this case, the largest faces have dimension $1$, so $\dim R = 2$.  It seems as though $R$
should have dimension $1$, but it is really the affine coordinate ring of the cone over the
twisted cubic, so $\dim R = 2$ makes sense.

{\bf Fact:} $\deg \fld[\Delta] =$ number of facets (maximal faces) of $\Delta$.

Here, that number is $3$.

Now for the Hilbert series.  By definition, this is
	$$\Hilb(\fld[\Delta],t) := \sum_{m \geq 0} \left( \dim_\fld (\fld[\Delta])_m \right) 
	t^m$$

where $(\fld[\Delta])_m$ denotes the $m$th graded piece of $\fld[\Delta]$ (that is, the
$\fld$-linear span of the monomials of degree $m$.  We have already written down a 
monomial basis (\ref{bas}) for $\fld[\Delta]$.  The minimal elements of the basis
correspond to faces of $\Delta$, and we have
	\begin{eqnarray*}
	\Hilb(\fld[\Delta],t)
	&=& 1 + 4\left(\frac{t}{1-t}\right) + 3\left(\frac{t^2}{(1-t)^2}\right) \\
	&=& \frac{1+2t}{(1-t)^2}.
	\end{eqnarray*}

The dimension and degree can be extracted from the Hilbert series: the dimension is the order 
of the pole at $t=1$ (here $2$), and the degree is the sum of the coefficients in the 
numerator (here $3$).  The Hilbert series has the following combinatorial interpretation in 
terms of $\Delta$.

\begin{defn}
The \defterm{$f$-vector} of $\Delta$ is
	$$f(\Delta) = \left( f_{-1}, f_0, f_1, \dots, f_{\dim\Delta} \right),$$
where $f_i$ is the number of $i$-dimensional faces.  (So $f_{-1}=1$, since $\emptyset$ is the 
unique face of dimension $-1$, and $f_0$ is the number of vertices.)
\end{defn}

\begin{defn}
The \defterm{$h$-vector} of $\Delta$ is
	$$h(\Delta) = \left( h_0, h_1, \dots, h_{\dim\Delta+1} \right),$$
where $h_i$ is defined as follows.  You can use the formula
	$$\sum_i h_it^i = \sum_{i=0}^{\dim\Delta+1} f_{i-1}t^i(1-t)^{\dim\Delta-i+1}$$
(Bruns and Herzog, Lemma 5.1.8).
\end{defn}

A more fun way to compute the $h$-vector is as follows.  First, draw a triangle with the
coefficients of the $f$-vector down the rightmost diagonal, and $1$'s down the leftmost
diagonal:

	$$\begin{array}{ccccccc}
	  &   &     & f_{-1} &     &     & \\ \\
	  &   &   1 &        & f_0 &     & \\ \\
	  & 1 &     & \BS    &     & f_1 & \\ \\
	1 &   & \BS &        & \BS &     & f_2 \\
	  &   &     & \vdots &     &     &
	\end{array}$$

Then, fill in the $\BS$'s down to the row below $f_{\dim\Delta}$ as though you were
constructing Pascal's triangle---but subtracting instead of adding.  That is, replace each
$\BS$ with the number to the northeast minus the number to the northwest.  The bottom row will
be the $h$-vector.

For example, for the complex $\Delta$ we have been working with, we start with

	$$\begin{array}{cccccc}
	  &   &     & 1   &     & \\
	  &   & 1   &     & 4   & \\
	  & 1 &     & \BS &     & 3 \\ \hline
	1 &   & \BS &     & \BS &
	\end{array}$$

Filling in the boxes, we get

	$$\begin{array}{cccccc}
	  &   &   & 1 &   & \\
	  &   & 1 &   & 4 & \\
	  & 1 &   & 3 &   & 3 \\ \hline
	1 &   & 2 &   & 0 & \\
	\end{array}$$

The numbers below the line form the $h$-vector, in this case $(1,2,0)$.  The trailing $0$'s
are frequently dropped, so we would write $h(\Delta)=(1,2)$.

Now back to the Hilbert series.  The connection is the following:
	\begin{eqnarray*}
	\Hilb(\fld[\Delta],t) &:=&
	\sum_{m \geq 0} \left( \dim_\fld (\fld[\Delta])_m \right) t^m \\ \\
	&=& \sum_{i \geq 0} f_i(\Delta) \left(\frac{t}{1-t}\right)^{i+1} \\ \\
	&=& \frac {\displaystyle\sum_{i=0}^{\dim\Delta+1} h_it^i} {(1-t)^{\dim\Delta+1}}.
	\end{eqnarray*}

\end{document}
