%%%%%%%%%% the following setup maximizes number of characters per page---
%%%%%%%%%% good for solution sets, etc.  Don't use for articles.

\documentclass{amsart}
\raggedbottom
\oddsidemargin=0in
\evensidemargin=0in
\textwidth=6.5in
\textheight=9in
\topmargin=0in
\headheight=0in
\headsep=0in
\footskip=0in
\parskip=10bp
\parindent=0bp
%\footheight=0in
\pagestyle{empty}
\usepackage{amsfonts,amssymb}
\usepackage[dvips]{graphicx}

\newtheorem{defn}{Definition}
\newtheorem{thm}[defn]{Theorem}
\newtheorem{prop}[defn]{Proposition}
\newtheorem{cor}[defn]{Corollary}

\newcommand{\BS}{\blacksquare}
\newcommand{\ini}{\mathop{\rm in}\nolimits}
\newcommand{\Proj}{\mathop{\rm Proj}\nolimits}
\newcommand{\Hilb}{\mathop{\rm Hilb}\nolimits}
\newcommand{\link}{\mathop{\rm lk}\nolimits}
\newcommand{\Tor}{\mathop{\rm Tor}\nolimits}
\newcommand{\Sd}{\mathop{\rm Sd}\nolimits}
\newcommand{\supp}{\mathop{\rm supp}\nolimits}
\newcommand{\extpow}[1]{{\textstyle\bigwedge^{#1}\:}}
%\newcommand{\extpow}{\bigwedge\nolimits}

\newcommand{\fld}{{\bf k}}
\newcommand{\isom}{\cong}
\newcommand{\mm}{{\mathfrak m}}
\newcommand{\NN}{{\mathbb N}}
\renewcommand{\SS}{{\mathbb S}}
\newcommand{\ZZ}{{\mathbb Z}}
\newcommand{\PP}{{\mathbb P}}
\newcommand{\xx}{{\bf x}}
\newcommand{\defterm}[1] {{\it #1\/}}
\newcommand{\puttext}[2] {\put(#1){\makebox(0,0){#2}}}
\newcommand{\putdot}[1]  {\put(#1){\makebox(0,0){$\bullet$}}}
\newcommand{\putline}[3] {\put(#1){\line(#2){#3}}}

\begin{document}

Informal Seminar on Stanley-Reisner Theory, UMN, Fall 2002 \\
21 November 2002

{\bf Minimal free resolutions of Stanley-Reisner rings} \\
Speaker: Vic Reiner

Scribe notes by Sandra Di Rocco, LaTeX'ed by Jeremy Martin \\

\begin{tabbing}
{\bf Outline:} \= I. (Minimal) free resolutions and $\Tor^S_\cdot(\fld[\Delta],\fld)$ \\
\> II. Theorem for $\fld[\Delta]$ \\
\> III. Proof \\
\end{tabbing}

\hrule

For more details, see especially section~5.5 of~\cite{BH}.

Let $\Delta$ be the simplicial complex
        \begin{equation} \label{delta-ex}
        \begin{picture}(110,60)
	\putdot{10,10}  \puttext{10, 2}{$1$}
	\putdot{40,50}  \puttext{40,58}{$2$}
	\putdot{70,10}  \puttext{70, 2}{$3$}
	\putdot{100,40}  \puttext{100,32}{$4$}
        \putline{10,10}{3,4}{30}	% 12
        \putline{10,10}{1,0}{60}	% 13
        \putline{70,10}{-3,4}{30}	% 23
        \putline{70,10}{1,1}{30}	% 34
        \end{picture}
        \end{equation}
so that
	$$I_\Delta = (x_1x_2x_3, \; x_1x_4, \; x_2x_4)$$
and
	$$\fld[\Delta] = \underbrace{\fld[x_1,\dots,x_4]}_S / I_\Delta.$$

The $S$-module $\fld[\Delta]$ is $\ZZ^n$-graded; for $\alpha \in \ZZ^n$, the corresponding
graded piece is the linear span of the monomial $\xx^\alpha = x_1^{\alpha_1} \dots
x_n^{\alpha_n}$.  $\fld[\Delta]$ has the following $\ZZ^n$-graded free resolution as an
$S$-module:

	\begin{equation} \label{freeres1}
	0 \to S^1 \xrightarrow[
		\begin{array}{cc} & g \\
			\begin{matrix} f_1 \\ f_2 \\ f_3 \end{matrix} &
			\begin{bmatrix} x_3 \\ 1 \\ -1 \end{bmatrix}
		\end{array}
	]{} S^3 \xrightarrow[
		\begin{array}{cc} & \begin{matrix} f_1 &&& f_2 &&& f_3 \end{matrix} \\
			\begin{matrix} e_1 \\ e_2 \\ e_3 \end{matrix} &
			\begin{bmatrix}
				0    & x_4     & x_4 \\
				x_2  & -x_2x_3 & 0 \\
				-x_1 & 0       & -x_1x_3
			\end{bmatrix}
		\end{array}
	]{} S^3 \xrightarrow[
		\begin{array}{ccc}
			    e_1       & e_2    & e_3    \\
			{[} x_1x_2x_3 & x_1x_4 & x_2x_4 {]} \\
		\end{array}
	]{} S \to S/I_\Delta \to 0.
	\end{equation}

Here the first copy of $S^1$ has basis element $g$ of degree $x_1x_2x_3x_4$; the first copy of
$S^3$ has basis $\{f_1,f_2,f_3\}$ of degrees $x_1x_2x_4$, $x_1x_2x_3x_4$, $x_1x_2x_3x_4$
respectively; and the second copy of $S^3$ has basis $\{e_1,e_2,e_3\}$ of degrees $x_1x_2x_3$,
$x_1x_4$, $x_2x_4$ respectively.

The resolution (\ref{freeres1}) is not minimal.  In a \underline{minimal} free resolution
(henceforth MFR), we want at each stage that the columns give a minimal generating set of the
kernel.  The following is an MFR of $S/I_\Delta$:

	\begin{equation} \label{freeres2}
	0 \to S^2 \xrightarrow[
		\begin{array}{cc} & \begin{matrix} f_1 &&& f_2 \end{matrix} \\
			\begin{matrix} e_1 \\ e_2 \\ e_3 \end{matrix} &
			\begin{bmatrix} 0 & x_4 \\ x_2 & -x_2x_3 \\ -x_1 & 0 \end{bmatrix}
		\end{array}
	]{} S^3 \xrightarrow[
		\begin{array}{ccc}
			    e_1       & e_2    & e_3    \\
			{[} x_1x_2x_3 & x_1x_4 & x_2x_4 {]} \\
		\end{array}
	]{} S \to S/I_\Delta \to 0.
	\end{equation}

\begin{prop} Let $R$ be a graded $\fld$-algebra and $M$ a graded $R$-module.  Then a graded 
$R$-free resolution
	$$\dots \to R^{\beta_1} \to R^{\beta_0} \to M \to 0$$
is minimal if and only if the matrices only contain elements of
	$$\mm := \bigoplus_{n>0} R_n$$
where $R_n$ denotes the $n$th graded piece of $R$.  (Note: $\mm$ is also denoted $R_+$ and 
often called the \defterm{irrelevant ideal} of $R$.)
\end{prop}

For instance, the resolution (\ref{freeres1}) is not minimal because of the $1$ and $-1$ 
appearing in the leftmost map.

Note that in this MFR, the free module $S^2$ has $S$-basis $\{f_1,f_2\}$ of
degrees $x_1 x_2 x_4, x_1 x_2 x_3 x_4$.  It can be shown that the degrees of
the basis elements in the terms of an MFR are uniquely determined, even though
the maps are not unique.

\begin{cor} The number of basis elements of $R^{\beta_i}$ of degree $\xx^\alpha$ in any MFR is
	$$\dim_\fld \Tor^S_i(M,\fld)_{\xx^\alpha},$$
where $\fld = R/\mm = R/R_+$.
\end{cor}

\begin{proof}
To compute $\Tor^S_i(M,\fld)$, we first write down an MFR of $M$ as an $S$-module:
	\begin{equation} \label{mfr1}
	\dots \to R^{\beta_i} \xrightarrow[\phi]{} R^{\beta_{i-1}} \to \dots 
	\to R^{\beta_0} \to M \to 0.
	\end{equation}
Here the matrix $\phi$ has all entries in $\mm$.  Therefore, tensoring the complex 
(\ref{mfr1}) with $\fld$ over $R$ produces
	\begin{equation} \label{mfr2}
	\dots \to \fld^{\beta_i} \xrightarrow{0} R^{\beta_{i-1}} \xrightarrow{0} \dots 
	\xrightarrow{0} R^{\beta_0} \to 0.
	\end{equation}
The $i$th homology of this complex is by definition $\Tor^S_i(M,\fld)$; since all the maps are 
zero, we have $\Tor^S_i(M,\fld) = \fld^{\beta_i}$.  Note that $\fld^{\beta_i}$
still carries a grading, and the $\fld$-basis vectors for
$\fld^{\beta_i}=\Tor^S_i(M,\fld)$ have the same degrees as
the $R$-basis elements for $R^{\beta_i}$.
\end{proof}

Let $\Delta$ be a simplicial complex on vertices $[n]$.  Recall that the \defterm{link} of a
face is defined as
	$$\link_\Delta F := \{G \in \Delta ~:~ G \cup F \in \Delta ~\text{and}~ G \cap F 
	= \emptyset\}.$$

For $S \subset [n]$, we define
	$$\Delta|_S := \{F \in \Delta ~:~ F \subset S\}.$$

The \defterm{dual complex} of $\Delta$ is
	$$\Delta^\vee := \{F \in [n] ~:~ [n]-F \not\in \Delta\}$$

For instance, if $\Delta$ is the simplicial complex of (\ref{delta-ex}), then $\Delta^\vee$ is
        \begin{equation} \label{alex}
        \begin{picture}(110,80)
	\putdot{10,20}  \puttext{10, 2}{$1$}
	\putdot{40,60}  \puttext{40,58}{$2$}
	\putdot{70,20}  \puttext{70, 2}{$3$}
	\putdot{100,50}  \puttext{100,32}{$4$}
        \putline{10,20}{3,4}{30}	% 12
        \putline{10,20}{1,0}{60}	% 13
        \end{picture}
        \end{equation}

One way to picture this is as follows.  Draw the full Boolean algebra on $[4]$ (I'm not going 
to bother to put in all the edges):

        \begin{equation} \label{boolalg}
        \begin{picture}(200,120)
	\puttext{100,100}{$1234$}
	\puttext{ 55,75}{$123$}
	\puttext{ 85,75}{$124$}
	\puttext{115,75}{$134$}
	\puttext{145,75}{$234$}
	\puttext{ 50,50}{\fbox{$12$}}
	\puttext{ 70,50}{\fbox{$13$}}
	\puttext{ 90,50}{$14$}
	\puttext{110,50}{\fbox{$23$}}
	\puttext{130,50}{$24$}
	\puttext{150,50}{\fbox{$34$}}
	\puttext{ 55,25}{\fbox{$1$}}
	\puttext{ 85,25}{\fbox{$2$}}
	\puttext{115,25}{\fbox{$3$}}
	\puttext{145,25}{\fbox{$4$}}
	\puttext{100, 0}{\fbox{$\emptyset$}}
        \end{picture}
        \end{equation}

The boxes indicate faces of $\Delta$.  The unboxed faces are the complements of faces in the
dual complex.

We will also need the notion of the \defterm{barycentric subdivision} $\Sd(\Delta)$ of a
simplicial complex $\Delta$.  Abstractly, this is the simplicial complex whose vertices are
the nonempty faces of $\Delta$ and whose faces are the \defterm{flags} in $\Delta$, that is,
strictly increasing sequences
	$$\emptyset \neq F_0 \subsetneq F_1 \subsetneq \dots \subsetneq F_r$$
where $F_i \in \Delta$ for all $i$.  (For those familiar with posets, the barycentric
subdivision of $\Delta$ is the same thing as the order complex of the poset of
faces of $\Delta$.)

For example, if $\Delta$ is the $2$-dimensional simplex on vertices $\{a,b,c\}$, then 
$\Sd(\Delta)$ has six facets, namely the flags
	\begin{equation*}
	\begin{array}{ccccc}
	\{a\} \subset \{a,b\} \subset \{a,b,c\}, & \qquad &
	\{b\} \subset \{a,b\} \subset \{a,b,c\}, & \qquad &
	\{c\} \subset \{a,c\} \subset \{a,b,c\}, \\
	\{a\} \subset \{a,c\} \subset \{a,b,c\}, & \qquad &
	\{b\} \subset \{b,c\} \subset \{a,b,c\}, & \qquad &
	\{c\} \subset \{b,c\} \subset \{a,b,c\}.
	\end{array}
	\end{equation*}

\begin{center}
\begin{picture}(430,220)

\puttext{80,10}{$\Delta$}
\putdot{  0, 40}  \puttext{  0, 32}{$a$}
\putdot{160, 40}  \puttext{160, 32}{$b$}
\putdot{ 80,200}  \puttext{ 80,208}{$c$}
\putline{  0,40}{1,2}{80}	% a,c
\putline{  0,40}{1,0}{160}	% a,b
\putline{160,40}{-1,2}{80}	% b,c

\puttext{350,10}{$\Sd(\Delta)$}
\putdot{270, 40}  \puttext{270, 32}{$a$}
\putdot{350, 40}  \puttext{350, 32}{$ab$}
\putdot{430, 40}  \puttext{430, 32}{$b$}
\putdot{310,120}  \puttext{300,120}{$ac$}
\putdot{390,120}  \puttext{400,120}{$bc$}
\putdot{350, 93}  \puttext{365, 93}{$abc$}
\putdot{350,200}  \puttext{350,208}{$c$}
\putline{270,40}{3,2}{120}	% a,bc
\putline{270,40}{1,2}{80}	% a,c
\putline{270,40}{1,0}{160}	% a,b
\putline{350,40}{0,1}{160}	% ab,c
\putline{430,40}{-3,2}{120}	% b,ac
\putline{430,40}{-1,2}{80}	% b,c

\end{picture}
\end{center}

\begin{prop}
Each of $\Delta$ and $\Delta^\vee$ is homotopy equivalent to the complement of the other in
$\partial\Delta^{n-1} \isom \SS^{n-2}$.  Consequently, their (co-)homology groups
are related by \defterm{Alexander duality}:
$$
\tilde{H}_i(\Delta^\vee) \cong \tilde{H}^{n-3-i}(\Delta^\vee).
$$
\end{prop}

{\it Idea of proof:} It is possible to embed both barycentric subdivisions
$\Sd(\Delta)$ and $\Sd(\Delta^\vee)$ simultaneously and disjointly inside 
$\Sd(\partial\Delta^{n-1})$.  For $\Sd(\Delta^\vee)$ one must apply the
antipodal map on the barycentric subdivison before embedding it in the usual
way.   (This is unLaTeXable, but try it yourself with the complex 
$\Delta$ of (\ref{delta-ex}).) \qed

For this reason, the complex $\Delta^\vee$ may be called the \defterm{canonical Alexander 
dual} of $\Delta$.


We now state and prove the main result.

\begin{thm} \label{thetheorem} Let $\Delta$ be a simplicial complex on vertices $[n]$, $S = 
\fld[x_1,\dots,x_n]$, and $\alpha \in \NN^n$.  Then
	\begin{eqnarray*}
	\Tor^S_i(\fld[\Delta],\fld)_{\xx^\alpha} &\isom& \begin{cases}
		\tilde H_{|S|-i-1}(\Delta|_S; \; \fld) &
		  \text{if}~ \xx^\alpha = \xx^S ~\text{for some}~ S \subset [n] \\
		0 & \text{otherwise} \end{cases} \\
	&\isom& \begin{cases}
		\tilde H_{i-2}(\link_{\Delta^\vee}(F); \; \fld) &
		  \text{if}~ \xx^\alpha = \xx^{[n]-F} ~\text{for some}~ F \in \Delta^\vee \\
		0 & \text{otherwise.} \end{cases} \\
	\end{eqnarray*}
\end{thm}

The first characterization of Tor is due to Hochster~\cite{Hoc} and the second to Eagon and
Reiner~\cite{ER}.  The equivalence of the two comes from Alexander duality
after one checks that 
$(\Delta|_S)^\vee = \link_{\Delta^\vee}(F)$ if $S = [n]-F$.


\begin{proof} By general homological nonsense we have
	\begin{equation}
	\Tor^S_i(\fld[\Delta],\fld) =
        \Tor^S_i(S/I_\Delta,\fld) ~\isom~ 
        \Tor^S_{i-1}(I_\Delta,\fld) ~\isom~ 
	\Tor^S_{i-1}(\fld,I_\Delta).
	\end{equation}
We compute the last module via the Koszul resolution of $\fld$ as an $S$-module:
	\begin{equation} \label{koszul}
	0 \to S \otimes \extpow{n} \fld^n \to \dots \to S \otimes \extpow{2} \fld^n \to
	S \otimes \extpow{1} \fld^n \to S \to \fld \to 0,
	\end{equation}
where the boundary maps are defined $S$-linearly by
	\begin{equation}
	e_{i_1} \wedge \dots \wedge e_{i_r} \quad\mapsto\quad
	\sum_{j=1}^r (-1)^j x_r e_{i_1} \wedge \dots \wedge \widehat{e_{i_1}} \wedge \dots 
	\wedge e_{i_r}
	\end{equation}
(the hat denoting removal).  Now, we tensor (\ref{koszul}) with $I_\Delta$ over $S$, obtaining
	\begin{equation}
        \label{tensored-complex}
	\dots ~\to~ I_\Delta \otimes \extpow{r} \fld^n ~\to~ \extpow{r-1} \fld^n ~\to~ \dots
	\end{equation}
where the boundary maps are given by
	\begin{equation}
	\xx^\beta \otimes e_{i_1} \wedge \dots \wedge e_{i_r} \quad\mapsto\quad
	\sum_{j=1}^r (-1)^j x_r e_{i_1} \wedge \dots \wedge \widehat{e_{i_1}} \wedge \dots 
	\wedge e_{i_r}.
	\end{equation}
Denoting $e_{i_1} \wedge \dots \wedge e_{i_r}$ by $e_G$ if $G=\{i_1,\ldots,i_r\}$,
we see that a $\fld$-basis in degree $\xx^\alpha$ for the complex 
\eqref{tensored-complex} is
	\begin{equation}
	\left\{ \xx^\beta \otimes e_G ~:~ G \subset [n], ~  \xx^\beta \in I_\Delta, ~ \xx^\beta \xx^G = \xx^\alpha 
	\right\}~.
	\end{equation}
This shows that in degree $\xx^\alpha$, this complex coincides (up to shift
in homological degree by $1$) with the usual augmented chain complex for
the simplicial complex
	\begin{equation}
	\Delta_\alpha := \left\{ G \subset [n] ~:~ \frac{\xx^\alpha}{\xx^G} \in I_\Delta 
	\right\}
        = \left\{ G \subset [n] ~:~ \supp\left( \frac{\xx^\alpha}{\xx^G} \right) 
             \not\in \Delta 
	\right\}.
	\end{equation}
Thus we have 
	\begin{equation}
	\Tor^S_i(\fld[\Delta],\fld)_{\xx^\alpha} ~\isom~
	\Tor^S_{i-1}(\fld,I_\Delta)_{\xx^\alpha} ~\isom~
	\tilde H_{i-2}(\Delta_\alpha; \fld).
	\end{equation}
Note that if $\xx^\alpha$ is divisible by $x_i^2$, then
	$$\supp\left(\frac{\xx^\alpha}{\xx^G}\right) ~=~
	\supp\left(\frac{\xx^\alpha}{\xx^{G\cup\{i\}}}\right)$$
so $i$ will be a cone vertex for $\Delta_\alpha$.  Hence without loss of generality 
$\xx^\alpha = \xx^S$ for some subset $S \subset [n]$.  Let $F = [n]-S$; then
	\begin{eqnarray*}
	\frac{\xx^S}{\xx^G} \in I_\Delta &\iff& G \subset S ~\text{and}~ S-G \not\in \Delta \\
        &\iff& G \cap F = \emptyset ~\text{and}~ [n]-(G \cup F) \not\in \Delta \\
	&\iff& G \in \link_{\Delta^\vee}(F).
	\end{eqnarray*}
Therefore $\Delta_\alpha = \link_{\Delta^\vee}(F)$.
\end{proof}

As an illustration, we can now go back and explain the degrees of the
basis elements for the terms in the MFR:

\vfill

\begin{center}
\begin{tabular}{c|c}
Restricted complex & Homological observations \\ \hline
        \begin{picture}(110,50)
	\puttext{20,25}{$\Delta|_{1234} ~=$}
	\putdot{55,10} \puttext{55, 2}{$1$}
	\putdot{70,30} \puttext{70,38}{$2$}
	\putdot{85,10} \puttext{85, 2}{$3$}
	\putdot{100,25} \puttext{100,17}{$4$}
        \putline{55,10}{3,4}{15}	% 12
        \putline{55,10}{1,0}{30}	% 13
        \putline{85,10}{-3,4}{15}	% 23
        \putline{85,10}{1,1}{15}	% 34
        \end{picture}
	& \raisebox{15pt}[0pt]{$\tilde H_{1} \neq 0$} 
\\ \hline
        \begin{picture}(75,50)
	\puttext{0,25}{$\Delta|_{123} ~\isom$}
	\putdot{35,10}
	\putdot{50,30}
	\putdot{65,10}
        \putline{35,10}{3,4}{15}	% 12
        \putline{35,10}{1,0}{30}	% 13
        \putline{65,10}{-3,4}{15}	% 23
        \end{picture} &\raisebox{15pt}[0pt]{$\tilde H_{1} \neq 0$}
\\ \hline
        \begin{picture}(90,50)
	\puttext{20,25}{$\Delta|_{14} \;\isom\; \Delta|_{23} \;\isom$}
	\putdot{80,10}
	\putdot{80,25}
        \end{picture}
	& \raisebox{15pt}[0pt]{$\tilde H_0 \neq 0$}
\\ \hline
        \begin{picture}(90,50)
	\puttext{0,25}{$\Delta|_{124} ~\isom$}
	\putdot{50,10}
	\putdot{50,30}
	\putdot{70,20}
        \putline{50,10}{0,1}{20}	% 12
        \end{picture}
	& \raisebox{15pt}[0pt]{$\tilde H_0 \neq 0$}
\end{tabular}

\vfill

\begin{tabular}{c|c}
Link & Homological observations \\ \hline
        \begin{picture}(150,50)
	\puttext{40,25}{$\Delta^\vee = \link_{\Delta^\vee}(\emptyset) ~=$}
	\putdot{95,10} \puttext{95, 2}{$1$}
	\putdot{110,30} \puttext{110,38}{$2$}
	\putdot{125,10} \puttext{125, 2}{$3$}
	\putdot{140,25} \puttext{140,17}{$4$}
        \putline{95,10}{1,0}{30}	% 13
        \putline{125,10}{-3,4}{15}	% 23
        \end{picture}
	& \raisebox{15pt}[0pt]{$\tilde H_0 \neq 0$} 
\\ \hline
        \begin{picture}(75,50)
	\puttext{0,25}{$\link_{\Delta^\vee}(3) ~\isom$}
	\putdot{80,10}
	\putdot{80,25}
	\end{picture}
	& \raisebox{15pt}[0pt]{$\tilde H_0 \neq 0$}
\\ \hline
        \begin{picture}(75,50)
	\puttext{0,25}{$\link_{\Delta^\vee}(4),
                        \link_{\Delta^\vee}(23),
                        \link_{\Delta^\vee}(13) = \{ \emptyset \}$}
	\end{picture}
	& \raisebox{15pt}[0pt]{$\tilde H_{-1} \neq 0$}
\end{tabular}
\end{center}

\vfill

\begin{thebibliography}{99}

\bibitem{BH} W. Bruns and J. Herzog.  \textsl{Cohen-Macaulay rings}.  Cambridge U. Press, 1993

\bibitem{ER} J. Eagon and V. Reiner.  \textsl{Resolutions of Stanley-Reisner rings and 
Alexander duality}.  J. of Pure and Applied Algebra {\bf 130} (1998), 265--275.

\bibitem{Hoc} M. Hochster.  \textsl{Cohen-Macaulay rings, combintorics, and simplicial 
complexes}.  In B.R. McDonald and R.A. Morris, eds. \textsl{Ring theory II}, Lecture Notes in 
Pure and Applied Mathematics {\bf 26}, M. Dekker, 1977, pp.~171--223.

\end{thebibliography}
\end{document}

