%%%%%%%%%% the following setup maximizes number of characters per page---
%%%%%%%%%% good for solution sets, etc.  Don't use for articles.

\documentclass{amsart}
\raggedbottom
\oddsidemargin=0in
\evensidemargin=0in
\textwidth=6.5in
\textheight=9in
\topmargin=0in
\headheight=0in
\headsep=0in
\footskip=0in
\parskip=10bp
\parindent=0bp
%\footheight=0in
\pagestyle{empty}
\usepackage{amsfonts,amssymb}

\newtheorem{thm}{Theorem}
\newtheorem{defn}[thm]{Definition}
\newtheorem{lemma}[thm]{Lemma}

\renewcommand{\star}{\mathop{\rm st}\nolimits}
\newcommand{\link}{\mathop{\rm 
lk}\nolimits}
\newcommand{\sm}{\setminus}
\newcommand{\st}{~\mid~}
\newcommand{\0}{\emptyset}
\newcommand{\ini}{\mathop{\rm in}\nolimits}
\newcommand{\CE}{\check{C}}
\newcommand{\Proj}{\mathop{\rm Proj}\nolimits}
\newcommand{\Hilb}{\mathop{\rm Hilb}\nolimits}
\newcommand{\supp}{\mathop{\rm supp}\nolimits}
\newcommand{\depth}{\mathop{\rm depth}\nolimits}
\newcommand{\Ann}{\mathop{\rm Ann}\nolimits}
\newcommand{\Ass}{\mathop{\rm Ass}\nolimits}
\newcommand{\fld}{{\bf k}}
\newcommand{\NN}{{\mathbb N}}
\newcommand{\ZZ}{{\mathbb Z}}
\newcommand{\PP}{{\mathbb P}}
\newcommand{\isom}{\cong}
\newcommand{\defterm}[1] {{\bf #1}}
\newcommand{\puttext}[2] {\put(#1){\makebox(0,0){#2}}}
\newcommand{\putdot}[1]  {\put(#1){\makebox(0,0){$\bullet$}}}
\newcommand{\putline}[3] {\put(#1){\line(#2){#3}}}

\begin{document}

Informal Seminar on Stanley-Reisner Theory, UMN, Fall 2002 \\
5 December 2002

{\bf Local cohomology} \\
Speaker: Anton Leykin

Scribe notes by Jeremy Martin \\

\hrule

We work with the following objects throughout.  Let $\fld$ be a field and $R$ a finitely
generated, $\NN^n$-graded $\fld$-algebra, i.e., $R = \bigoplus_{\alpha \in \NN^n} R_\alpha$
with $R_\alpha R_\beta \subset R_{\alpha+\beta}$.  The motivating example is $R = 
\fld[\Delta]$, the Stanley-Reisner ring of a simplicial complex $\Delta$.  Define
	$$R_+ := \bigoplus_{\alpha \neq 0} R_\alpha$$
the so-called \defterm{irrelevant ideal}.  Finally, $M$ will be a $\ZZ^n$-graded $R$-module, 
i.e., $M = \bigoplus_{\alpha \in \ZZ^n} M_\alpha$ with $R_\alpha M_\beta \subset 
M_{\alpha+\beta}$.

%--------------------------------------------------------------------------------------
\section{Depth and Cohen-Macaulayness}

\begin{defn} An element $a \in R$ is called a \defterm{nonzerodivisor} (or NZD) on $M$ if $m 
\in M$, $am=0$ implies $m=0$.  Equivalently, the map
	$$M \xrightarrow{\cdot a} M$$
given by multiplication by $r$ is one-to-one.
\end{defn}

\begin{defn}
A sequence of homogeneous elements $\theta_1, \dots, \theta_s$ is a \defterm{regular
$M$-sequence}, or \defterm{$M$-sequence} for short, if $\theta_{i+1}$ is a NZD
on $M/(\theta_1, \dots, \theta_i)M$ for $i=0, \dots, s-1$.
\end{defn}

\begin{defn}
The \defterm{dimension} of $M$, denoted $\dim_R M$ or $\dim M$, is the Krull dimension of 
$R/\Ann_R M$.  The \defterm{depth} of $M$, denoted $\depth_R M$ or $\depth M$, is the length 
of a maximal $M$-sequence.  It can be shown that every maximal $M$-sequence has the same
length.
\end{defn}

In general $\depth_R M \leq \dim_R M$ (since any $M$-sequence of length $s$ generates a 
height-$s$ ideal of $R/\Ann M$).  Equality is an important ``niceness'' condition which gets 
its own name:

\begin{defn}
$M$ is \defterm{Cohen-Macaulay} if $\depth_R M = \dim_R M$.
\end{defn}

%--------------------------------------------------------------------------------------
\section{Local Cohomology}

Define the \defterm{torsion functor} $\Gamma$ by
	$$\Gamma(M) := \left\{u \in M \st R_+^n u = 0 ~\text{for}~ n \gg 0 \right\}.$$
It is routine to check that $\Gamma$ is a covariant, left-exact functor.  That is, a map $f: M 
\to N$ of graded $R$-modules induces a map $\Gamma(f): \Gamma(M) \to \Gamma(N)$, and if $f$ is 
injective then $\Gamma(f)$ is injective.

The $i$th local cohomology functor $H^i$ (more precisely, $H^i_{R_+}$) can now be defined as 
the $i$th right derived functor of $\Gamma$:
	$$H^i(M) = R^i \Gamma (M).$$
That is, one may calculate $H^i(M)$ by taking an injective resolution
	$$I^\bullet: \quad 0 ~\to~ M ~\to~ I^0 ~\to~ I^1 ~\to~ \dots,$$
applying $\Gamma$, and defining $H^i(M) := H^i(\Gamma I^\bullet)$.
(See a textbook on homological algebra for more details.)

\begin{lemma} \label{torsion}
For all $i$, the modules $H^i(M)$ are \defterm{$R_+$-torsion}, i.e., they are killed by some 
power of $R_+$.
\end{lemma}

\begin{proof}
By definition of $\Gamma$, every module of the form $\Gamma(N)$ is $R_+$-torsion.  In 
particular, if $I^\bullet$ is an injective resolution, then every $\Gamma(I^i)$ is $R_+$-torsion, 
so the same is true of the cohomology modules of $\Gamma(I^\bullet)$.
\end{proof}

The local cohomology functors are useful because they detect depth and dimension.  
Specifically, we have the following fact.

\begin{thm} Let $e = \depth_R M$ and $d = \dim_R M$.  Then:

\noindent (1) \quad $H^i(M)=0$ unless $e \leq i \leq d$.

\noindent (2) \quad $H^e(M) \neq 0$ and $H^d(M) \neq 0$.

\end{thm}

\begin{proof} We'll prove only (1), which is the case we really need.  We proceed by induction 
on $e$.

If $H^0(M) \neq 0$, then every element of $R_+$ is a zerodivisor on $M$, which is exactly the 
statement that $e=0$.

If $e=0$, then
	$$R_+ \subset \bigcup_{P \in \Ass M} P,$$
so $R_+$ is itself an associated prime.  It is immediate that $\Gamma(M) = H^0(M) \neq 0$ as 
desired.

For the inductive step, assume that (1) is true for all $R$-modules $N$ with $\depth N < e$.  
Let $a \in R_+$ be a homogeneous NZD on $M$ and let $N = M/aM$.  Then $\depth_R N = e-1$, so 
by induction $H^i(N)=0$ for $i<e-1$ and $H^{e-1}(N) \neq 0$.

By general homological nonsense, the short exact sequence of $R$-modules
	\begin{equation}
	0 ~\to~ M ~\xrightarrow{\cdot a}~ M ~\to~ N ~\to~ 0
	\end{equation}
induces a long exact sequence on cohomology 
	\begin{equation} \label{lexact}
	\dots ~\to~ H^{i-1}(M) ~\to~ H^{i-1}(N) ~\to~ H^i(M) ~\xrightarrow{\cdot a}~ H^i(M) 
	~\to~ \dots
	\end{equation}

If $i<e$, then $H^{i-1}(N)=0$, so $a$ is a NZD on $H^i(M)$.  But $H^i(M)$ is $R_+$-torsion and 
so has depth $0$.  It follows that $H^i(M)=0$.

On the other hand, if $i=e$ then the first three terms displayed in (\ref{lexact}) are
	$$0 ~\to~ H^{e-1}(N) ~\to~ H^e(M),$$
and $H^e(M) \neq 0$ since $H^{e-1}(N) \neq 0$.
\end{proof}

The local cohomology functors can be computed using the \defterm{\v{C}ech complex}
$\CE^\bullet(x_1, \dots, x_n ;\; M)$, which is defined as
	\begin{equation} \label{cech1}
	\CE^\bullet(x_1, \dots, x_n ;\; M) := \bigotimes_{i=1}^n \big( (0 \to R \to R_{x_i} 
	\to 0) \otimes M \big).
	\end{equation}
where $R_+ = \sqrt{(x_1, \dots, x_n)}$ and $R_{x_i} = R[x_i^{-1}]$.  The $i$th \v{C}ech module
$\CE^i(x_1, \dots, x_n;\; M)$ may be described explicitly as follows.  For $F \subset [n]$, 
define
	$$x_F = \prod_{i \in F} x_i$$
and let
	$$R_F = R[x_F^{-1}].$$
Then
	\begin{equation}
	\CE^\bullet_i(x_1, \dots, x_n ;\; M) = \bigoplus_{\substack{F \subset [n] \\ |F|=i}} M_F
	\end{equation}
where $M_F = M \otimes R_F$ and the maps between adjacent terms in the \v{C}ech complex are
given by the usual Koszul maps (just like simplicial cohomology.)

%--------------------------------------------------------------------------------------
\section{Hochster's Theorem}

Let $\Delta$ be a simplicial complex on vertices $x_1, \dots, x_n$, and
$R=\fld[\Delta]=\fld[x_1, \dots, x_n]/I_\Delta$ its Stanley-Reisner ring.  With respect to the
obvious $\NN^n$-grading, we have $R_+ = (x_1, \dots, x_n)$.

\begin{defn} Let $F \in \Delta$.  The \defterm{star} of $F$ with respect to $\Delta$ is
	$$\star_\Delta F := \{ G \in \Delta \st G \cup F \in \Delta \}$$
and the \defterm{link} of $F$ with respect to $\Delta$ is
        $$\link_\Delta F := \{ G \in \Delta \st G \cup F \in \Delta, ~ G \cap F = \0\}.$$
We suppress the subscript when possible.
\end{defn}

Note that both $\star F$ and $\link F$ are simplicial complexes, and that $\star F = \langle F
\rangle * \link F$. For instance, if $\Delta = \langle 123, 14, 24 \rangle$ and $F = 12$, then
$\link F = \langle 3 \rangle$ and $\star F = \langle 123 \rangle$.

Let $q_1, \dots, q_n$ be indeterminates.  Denote by $\Hilb(M;\;q)$ the finely graded Hilbert
series of $M$, i.e.,
	$$\Hilb(M;\;q) := \sum_{\alpha \in \ZZ^n} q^\alpha \dim_\fld M_\alpha,$$
where $q^\alpha = q_1^{\alpha_1} \dots q_n^{\alpha_n}$.  Also, let $\tilde H_i(\Delta;\;\fld)$ 
denote the $i$th reduced simplicial homology of a simplicial
complex $\Delta$ with coefficients in $\fld$.

For $\alpha \in \ZZ^n$, define
	\begin{eqnarray*}
	F(\alpha) &=& \{x_i \st \alpha_i < 0\}, \\
	G(\alpha) &=& \{x_i \st \alpha_i > 0\}, \\
	\supp(\alpha) &=& F(\alpha) \;\cup\; G(\alpha) ~=~ \{x_i \st \alpha_i \neq 0\}.
	\end{eqnarray*}

\begin{thm}[Hochster] 
We have
	$$\Hilb(H^i(\fld[\Delta]);\;q) ~=~ \sum_{F \in \Delta} \dim_\fld \tilde
	H_{i-|F|-1}(\link_\Delta F;\;\fld) \prod_{x_i \in F} \frac{q_i^{-1}}{1-q_i^{-1}}.$$
\end{thm}

\begin{proof}
We compute $H^i(R)$ explicitly as the $i$th cohomology of the \v{C}ech complex $\CE^\bullet
= \CE^\bullet(x_1, \dots, x_n ;\; M)$.  If $F \not\in \Delta$, then the ring $R_F$ is zero,
because $x_F=0$ in $R$.  On the other hand, if $F \in \Delta$, then the variables in $F$
become units in $R_F$, and those not in $\star_\Delta F$ get killed (since they annihilate the 
unit
$x_F$).  That is,
	$$R_F ~=~ \fld\left[ \left\{x_i,x_i^{-1} ~:~ i \in F\right\} ~\cup~
	\left\{x_j ~:~ x_j \in \link F \right\} \right] \otimes R.$$
Let $\alpha \in \ZZ^n$.  We will compute the $\alpha$th graded piece $\CE^\bullet_\alpha$ of
the \v{C}ech complex.  If $\supp(\alpha) \not\in \Delta$, then $\CE^\bullet_\alpha = 0$,
because $R_\alpha=0$ and adjoining inverses doesn't change this.  So suppose that
$\supp(\alpha) \in \Delta$.  Let $F = F(\alpha)$, $j=|F|$, and $G = G(\alpha)$.  A priori, we
have
	\begin{equation}
	\CE^r_\alpha ~=~ \left[ \bigoplus_{|F'|=r} R_{F'} \right]_\alpha
	~=~ \bigoplus_{|F'|=r} [R_{F'}]_\alpha.
	\end{equation}
A whole bunch of these summands are zero.  Specifically, for $R_{F'}$ to be nonzero, we must
have $F' \in \Delta$ (as previously noted), $F' \supset F$ (since the variables in $F$ must be
units in the $\alpha$th graded piece of the \v{C}ech complex), and $F' \cup G \in \Delta$ (so
that $x^\alpha$ itself is nonzero).  This is all equivalent to the condition that
$F'' = F' \sm F$ belong to $\link_{\star G} F$, so we may write
	\begin{equation}
	\CE^r_\alpha ~=~ \left[ \bigoplus_{\substack{F'' \in \link_{\star G} F \\ 
	|F''|=r-j}} R_{F \cup F''} \right]_\alpha.
	\end{equation}
The maps in $\CE^r_\alpha$ correspond to the usual coboundary maps of the simplicial 
cochain complex of $\link_{\star G} F$, shifted by $j+1$.  That is,
	\begin{eqnarray}
	\left[H^i(R)\right]_\alpha &=& \tilde H^{i-j-1}(\link_{\star G} F;\; \fld) \\
	&\isom& \tilde H_{i-j-1}(\link_{\star G} F;\; \fld)
	\end{eqnarray}
since this is a finite-dimensional $\fld$-vector space, hence isomorphic to its dual.  (The 
isomorphism is not canonical, but we don't care because we're really only interested in its 
dimension.)

If $G \neq \emptyset$, then $\link_{\star G} F$ is a cone over $G$.  In particular it is 
contractible, so $\tilde H_\bullet(\link_{\star G} F;\; \fld)=0$.  Therefore we only have 
nonzero terms when $G = \emptyset$, so $\link_{\star G} F = \link_\Delta F$ and the last 
equation becomes
	\begin{equation}
	\left[H^i(R)\right]_\alpha ~\isom~ \tilde H_{i-j-1}(\link F;\; \fld).
	\end{equation}
Therefore
	\begin{eqnarray}
	\Hilb(H^i(\fld[\Delta]);\;q) &=& \sum_{F \in \Delta} 
	\sum_{\substack{\alpha: \\ \supp(\alpha)=F(\alpha)=F}}
	\dim_\fld \tilde H_{i-|F|-1}(\link F;\;\fld) q^\alpha \\
	&=& \sum_{F \in \Delta} \dim_\fld \tilde H_{i-|F|-1}(\link F;\;\fld)
	\sum_{\substack{\alpha: \\ \supp(\alpha)=F(\alpha)=F}} q^\alpha \\
	&=& \sum_{F \in \Delta} \dim_\fld \tilde H_{i-|F|-1}(\link F;\;\fld)
	\prod_{x_i \in F} \frac{q_i^{-1}}{1-q_i^{-1}}
	\end{eqnarray}
as desired.
\end{proof}

Here's an example (from Stanley) of computing a \v{C}ech complex.  Let $\Delta$ be the complex
$\langle 12,13,23,4 \rangle =$
        \begin{equation} \label{delta-ex}
        \begin{picture}(110,60)
        \putdot{10,10}  \puttext{10, 2}{$1$}
        \putdot{40,50}  \puttext{40,58}{$2$}
        \putdot{70,10}  \puttext{70, 2}{$3$}
        \putdot{100,40}  \puttext{100,32}{$4$}
        \putline{10,10}{3,4}{30}        % 12
        \putline{10,10}{1,0}{60}        % 13
        \putline{70,10}{-3,4}{30}       % 23
        \end{picture}
        \end{equation}

and $R = \fld[\Delta] = \fld[x_1,x_2,x_3,x_4]/(x_1x_4,x_2x_4,x_3x_4,x_1x_2x_3)$.  Then the 
\v{C}ech complex is
	$$0 ~\to~ \underbrace{R}_{\CE^0} ~\to~
	\underbrace{R_1 \oplus R_2 \oplus R_2 \oplus R_3 \oplus R_4}_{\CE^1} ~\to~
	\underbrace{R_{12} \oplus R_{13} \oplus R_{23}}_{\CE^2} ~\to~ 0,$$
where $R_1 = R[x_1^{-1}]$, $R_{12} = R[x_1^{-1},x_2^{-1}]$, etc.

$\bullet$ \quad For $\alpha=(0,0,0,0)$, so $F(\alpha)=G(\alpha)=\emptyset$, we have
	$$\link_{\star G} F = \link_\Delta \emptyset = \Delta,$$
\quad~ so $\left[ H^i(R) \right]_\alpha = \tilde H_{i-1}(\Delta;\;\fld)$.

$\bullet$ \quad For $\alpha=(-2,3,0,0)$, we have
	$$F = \{x_1\}, \qquad G = \{x_2\}, \qquad \link_{\star G} F = 
	\langle x_2 \rangle ~~ \text{(i.e., a point)}$$
\quad~ so $\left[ H^i(R) \right]_\alpha = \tilde H_{i-2}($point$) = 0$.

%--------------------------------------------------------------------------------------
\section{Reisner's Theorem}

Let $\Delta$ be a simplicial complex and $R=-\fld[Delta]$.  Let $d = \dim R = 1+\dim \Delta$.
We will say that $\Delta$ satisfies \defterm{Reisner's criterion} if for all $F \in \Delta$,
and $i<\dim(\link F)$. we have
	$$\tilde H_i(\link F;\;\fld)=0.$$

\begin{thm}
$\Delta$ is Cohen-Macaulay if and only if it satisfies Reisner's criterion.
\end{thm}

{\bf Remark:} $\Delta$ is \defterm{Gorenstein} (a stronger condition than Cohen-Macaulayness)  
if in addition $\tilde H_i(\link F;\;\fld) \isom \fld$ for $i=\dim(\link F)$.

\begin{proof} First, we show that a Cohen-Macaulay complex is pure (i.e., all maximal faces
have the same dimension).  Indeed, if $\Delta$ is Cohen-Macaulay of dimension $d-1$ and $\dim
F < d-1$, then $\tilde H_{-1}(\link F)=0$ by Hochster's theorem, so $\link F \neq 0$ and $F$
is not maximal.  (This can also be shown without Hochster's theorem; see Bruns and Herzog,
p.~210.)

Next, we show that a complex $\Delta$ satisfying Reisner's criterion is pure.  If $\dim F = 0$
then there is nothing to show.  Otherwise, we induct on dimension.  Reisner's criterion gives
$\tilde H_0(\Delta) = \tilde H_0(\link \emptyset) = 0$, so $\Delta$ is connected. Moreover,
for every vertex $v$, the subcomplex $\link\{v\}$ of $\Delta$ satisfies Reisner's criterion
and has dimension less than that of $\Delta$, so it is pure by induction.  Now, for any
maximal face $F$, let $v,w \in F$; we have
	$$\dim \link\{v\} = |F-v|-1 = |F-w|-1 = \dim \link\{w\};$$
by connectedness all links of vertices must have the same dimension, and the same equation 
implies that $\Delta$ is pure.

By these two observations together, we may assume that $\Delta$ is pure, so $|F|=d$ for all 
maximal faces and $\dim(\link F)=d-|F|-1$ for all faces.  So Cohen-Macaulayness
	\begin{eqnarray*}
	F~\text{is Cohen-Macaulay} &\iff& \fld[\Delta]~\text{is Cohen-Macaulay} \\
	&\iff& \tilde H_{j-|F|-1}(\link F;\;\fld)=0 ~\text{for}~ j<d, ~ F \in \Delta \\
	\end{eqnarray*}
which is exactly Reisner's criterion (set $j = i+|F|+1$).
\end{proof}

By the way, a pure connected simplicial complex $\Delta$ certainly need not be Cohen-Macaulay
(unless $\dim\Delta \leq 1$). The ``minimal'' example is the complex
$\langle 123,345 \rangle =$
	\begin{center}
	\begin{picture}(110,70)
	\putdot{10,10}  \puttext{ 2,10}{$1$}
	\putdot{10,70}  \puttext{ 2,70}{$2$}
	\putdot{50,40}  \puttext{50,32}{$3$}
	\putdot{90,10}  \puttext{98,10}{$4$}
	\putdot{90,70}  \puttext{98,70}{$5$}
	\putline{10,10}{4,3}{80}
	\putline{90,10}{-4,3}{80}
	\putline{10,10}{0,1}{60}
	\putline{90,10}{0,1}{60}
	\putline{14,13}{0,1}{54}
	\putline{18,16}{0,1}{48}
	\putline{22,19}{0,1}{42}
	\putline{26,22}{0,1}{36}
	\putline{30,25}{0,1}{30}
	\putline{34,28}{0,1}{24}
	\putline{38,31}{0,1}{18}
	\putline{42,34}{0,1}{12}
	\putline{46,37}{0,1}{6}

	\putline{90,64}{-1,0}{8}
	\putline{90,58}{-1,0}{16}
	\putline{90,52}{-1,0}{24}
	\putline{90,46}{-1,0}{32}
	\putline{90,40}{-1,0}{40}
	\putline{90,34}{-1,0}{32}
	\putline{90,28}{-1,0}{24}
	\putline{90,22}{-1,0}{16}
	\putline{90,16}{-1,0}{8}

        \end{picture}
        \end{center}

which is not Cohen-Macaulay because $\link(3) = \langle 12,45 \rangle$ is disconnected, so has
$\tilde H_0 \neq 0$.  (Also, the $h$-vector of this complex can be computed as $(1,2,-1)$. In
a Cohen-Macaulay complex, every entry of the $h$-vector is nonnegative.)

\end{document}
